\documentclass[../psets.tex]{subfiles}

\pagestyle{main}
\renewcommand{\leftmark}{Problem Set \thesection}
\setcounter{section}{6}

\begin{document}




\section{Molecular Orbital Theory}
\begin{enumerate}
    \item \marginnote{11/17:}The bonding and antibonding potential energy surfaces of \ce{H2+} were derived in class by applying the linear variational principle to the trial wave function
    \begin{equation*}
        \psi_\text{trial} = c_11s_\text{A}+c_21s_\text{B}
    \end{equation*}
    \begin{enumerate}
        \item Estimate the ground-state potential energy surface by computing the first-order \emph{perturbative} change in the energy where the reference Hamiltonian
        \begin{equation*}
            \hat{H}_0 = -\frac{1}{2}\nabla^2-\frac{1}{r_A}
        \end{equation*}
        is the Hamiltonian of the hydrogen atom at position $A$ and the perturbation
        \begin{equation*}
            \hat{V} = -\frac{1}{r_B}+\frac{1}{R}
        \end{equation*}
        is the interaction of a second proton at position $B$ with the proton at position $A$ where $R$ is the internuclear distance. The reference wave function
        \begin{equation*}
            \psi_0 = 1s_\text{A}
        \end{equation*}
        is the $1s$ orbital of hydrogen centered at position $A$.
        \item Make a sketch of the potential energy surface from part (a) as a function of $R$.
        \item Does the approximation in part (a) produce a stable, bound ground state for \ce{H2+}?
        \item In a second sketch, compare the potential energy surface in part (a) with the bonding and antibonding potential energy surfaces of \ce{H2+} that were derived in class from the trial wave function above.
        \item Using the trial wave function for comparison, explain briefly in terms of bonding --- the sharing of electrons --- the limitation of the wave function in part (a).
        \item Furnish the trial wave function that within the variational approximation would give the same potential energy surface as the application of first-order perturbation theory in (a).
    \end{enumerate}
    \item The Coulomb integral for \ce{H2+} is given by
    \begin{equation*}
        J(R) = \e[-2R]\left( 1+\frac{1}{R} \right)
    \end{equation*}
    \begin{enumerate}
        \item By making a sketch of $J(R)$ as a function of the internuclear distance $R$, show that $J(R)$ is nonnegative for all $R$.
        \item The $J(R)$ results from two competing forces: (i) the attraction of an electron on A to the proton at B and (ii) the repulsion of the proton at A from the proton at B. What does the nonnegativity of $J(R)$ in part (a) say about the relative strengths of these competing forces?
    \end{enumerate}
    \item 
    \begin{enumerate}
        \item Give the molecular orbital diagrams for the molecules \ce{N2}, \ce{N2+}, \ce{O2}, and \ce{O2+}.
        \item What is the bond order for each molecule?
        \item Explain why \ce{N2} has a larger dissociation energy than \ce{N2+}, but \ce{O2+} has a larger dissociation energy than \ce{O2}.
        \item Using Grassmann notation, give the ground-state wave function for \ce{N2} in molecular orbital theory.
    \end{enumerate}
    \item Using the worksheet "Huckel Theory" with the Quantum Chemistry Toolbox for Maple, answer the lettered questions.
    \item In photoelectron spectroscopy, radiation interacts with gaseous molecules to eject electrons whose kinetic energies are measured (recall the photoelectric effect).
    \begin{enumerate}
        \item Explain why measuring the kinetic energy of the ejected electrons tells us something about the molecular orbital energies.
        \item If the incident radiation has a frequency of $\SI{57.8}{\nano\meter}$, what is the largest electron binding energy that can be measured?
    \end{enumerate}
    \item 
    \begin{enumerate}
        \item Sketch the hybrid orbitals of \ce{Be} in the molecule \ce{BeH2} and illustrate their role in bonding.
        \item Using the hybrid orbitals, assemble a molecular orbital diagram for \ce{BeH2}.
        \item With Grassmann notation, express the MO ground-state wave function for \ce{BeH2}.
    \end{enumerate}
    \item Using the \emph{hybridized} $sp^3$ wave functions of \ce{CH4} on \textcite[376]{bib:McQuarrieSimon}, give the probability that an electron in one of the $sp^3$ wave functions is\dots
    \begin{enumerate}
        \item In carbon's $2p_x$ orbital;
        \item In carbon's $2p_z$ orbital.
        \item By symmetry, should the results in parts (a) and (b) be the same or different?
    \end{enumerate}
    \item The $\pi$-electrons may be approximated in conjugated and aromatic hydrocarbons through Huckel theory. Use Huckel theory to approximate the energies and the wave functions for\dots
    \begin{enumerate}
        \item The allyl radical;
        \item Cyclobutadiene;
        \item The cyclopentadienyl radical.
        \item For each wave function computed in parts (a), (b), and (c), give a schematic sketch of the molecular orbital contributions. Refer to Figure 10.23 on \textcite[396]{bib:McQuarrieSimon} for an example of such a sketch.
    \end{enumerate}
\end{enumerate}




\end{document}