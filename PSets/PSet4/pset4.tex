\documentclass[../psets.tex]{subfiles}

\pagestyle{main}
\renewcommand{\leftmark}{Problem Set \thesection}
\setcounter{section}{3}

\begin{document}




\section{Harmonic Oscillators II and the Hydrogen Atom}
\begin{enumerate}
    \item \marginnote{10/27:}The $J=0$ to $J=1$ transition for carbon monoxide (\ce{{}^{12}C{}^{16}O}) occurs at $\SI{1.153e5}{\mega\hertz}$.
    \begin{enumerate}
        \item Calculate the value of the bond length in carbon monoxide.
        \item Predict the $J=1$ to $J=2$ transition for carbon monoxide.
    \end{enumerate}
    \item The harmonic oscillator has a finite zero-point energy because of the uncertainty relation. In contrast, the lowest possible energy for the 2D rigid rotor is zero.
    \begin{enumerate}
        \item For the ground state of the 2D rigid rotor, what is the expectation value of the angular momentum, and what is the uncertainty $\Delta L_z$ in the expectation value? Recall that
        \begin{equation*}
            (\Delta L_z)^2 = \prb{\hat{L}_z^2}-\prb{\hat{L}_z}^2
        \end{equation*}
        \item In words, describe the uncertainty in position.
        \item Using your answers to (a) and (b), explain briefly why the 2D rigid rotor can have a vanishing zero-point energy and yet still remain consistent with the uncertainty relation.
    \end{enumerate}
    \item For the ground state of the hydrogen atom, compute
    \begin{enumerate}
        \item The \emph{average} distance from the nucleus for finding the electron.
        \item The \emph{most probable} distance from the nucleus for finding the electron.
        \item Repeat the calculation for the second excited state ($n=3$ and $l=0$) and compare your results with the ground state.
    \end{enumerate}
    \item Using non-relativistic quantum mechanics, compute the ratio of the ground-state energy of hydrogen to that of atomic tritium.
    \item The Hamiltonian operator for a hydrogen atom in a magnetic field where the field is in the $z$-direction is given by
    \begin{equation*}
        \hat{H} = \hat{H}_0+\frac{\beta_BB_z}{\hbar}\hat{L}_z
    \end{equation*}
    where $\hat{H}_0$ is the Hamiltonian operator in the absence of the magnetic field, $B_z$ is the $z$-component of the magnetic field, and $\beta_B$ is a constant called the Bohr magneton.
    \begin{enumerate}
        \item Show that the wave functions of the Schr\"{o}dinger equation for a hydrogen atom in a magnetic field are the same as those for the hydrogen atom in the absence of the field.
        \item Show that the energy associated with the wave function $\psi_{n,l,m}$ is
        \begin{equation*}
            E = E_n^{(0)}+\beta_BB_zm
        \end{equation*}
        where $E_n^{(0)}$ is the energy in the absence of the field and $m$ is the magnetic quantum number.
    \end{enumerate}
\end{enumerate}




\end{document}