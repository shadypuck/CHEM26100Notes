\documentclass[../psets.tex]{subfiles}

\pagestyle{main}
\renewcommand{\leftmark}{Problem Set \thesection}
\setcounter{section}{5}

\begin{document}




\section{Perturbation Corrections and Atomic and Molecular Orbitals}
\begin{enumerate}
    \item \marginnote{11/10:}Consider an electron in a potential with a Hamiltonian
    \begin{equation*}
        H = H_0+\lambda V
    \end{equation*}
    where
    \begin{equation*}
        H_0 = -\frac{\hbar^2}{2m}\dv[2]{x}+\frac{1}{2}x^2
    \end{equation*}
    and the perturbative term $V=x^2$.
    \begin{enumerate}
        \item Compute the first-order correction to the energy for the ground- and the first-excited states for $\lambda=1/5$.
        \item Optimize the $\alpha$ in the trial wave function $\e[-\alpha x^2/2]$ to estimate the ground-state energy of this system for $\lambda=1/5$.
        \item Which approximation yields a better estimate for the ground-state energy?
    \end{enumerate}
    \item 
    \begin{enumerate}
        \item Using the $1s$, $2s$, and $2p$ orbitals, give all electronic configurations for the carbon atom where both the $1s$ and the $2s$ orbitals are filled.
        \item Which configuration has the lowest energy according to Hund's rule?
        \item For this configuration, express the molecular wave function in Grassmann product notation (i.e., with $\wedge$'s).
    \end{enumerate}
    \item Using the worksheet "Molecular Orbitals" with the Quantum Chemistry Toolbox for Maple, answer the following questions.
    \begin{enumerate}
        \item In performing the geometry optimization of hydrogen fluoride, what is the computed bond length in the STO-3G basis set, and how does this bond length compare to the experimental bond length?
        \item What is the computed dipole moment of the molecule in Debyes, and is this result consistent with the experimental dipole moment in Debyes? (Hint: Search the \href{https://cccbdb.nist.gov/exp1x.asp}{NIST database}.)
        \item Draw a molecular orbital (MO) diagram for hydrogen fluoride.
        \item Based on the results from the worksheet, provide a sketch of each MO, and explain which atomic orbitals (AOs) contribute significantly to each MO. Are these results consistent with your MO diagram?
    \end{enumerate}
\end{enumerate}




\end{document}