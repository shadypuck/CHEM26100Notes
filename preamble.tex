\usepackage[margin=1in]{geometry}
\usepackage{csquotes}
\usepackage{fancyhdr}
\usepackage{marginnote}
\usepackage[style=apa]{biblatex}
\usepackage{enumitem}
\usepackage{scrextend}
\usepackage[bottom]{footmisc}
\usepackage{xr}
\usepackage{float}
\usepackage{subcaption}
\usepackage{tikz}
\usepackage{booktabs}
\usepackage{amsmath,amssymb,amsthm}
\usepackage{physics,bm,mathtools,nicematrix}
\usepackage{mhchem,chemfig}
\usepackage{siunitx}
\usepackage{multirow}
\usepackage[hidelinks]{hyperref}

\MakeOuterQuote{"}

\fancypagestyle{main}{
    \fancyhf{}
    \fancyhead[L]{\leftmark}
    \fancyhead[R]{CHEM 26100}
    \fancyfoot[R]{Labalme\ \thepage}
}
\fancypagestyle{plain}{
    \fancyhead{}
    \renewcommand{\headrulewidth}{0pt}
}

\reversemarginpar

\addbibresource{../main.bib}
\DefineBibliographyStrings{english}{bibliography={References}}

\setlist[itemize,3]{label={\scriptsize$\blacksquare$}}

\deffootnotemark{\textsuperscript{\textup{[}\thefootnotemark\textup{]}}}
\deffootnote[2.1em]{0em}{0em}{\textsuperscript{\thefootnote}}

\usetikzlibrary{fpu,decorations.pathmorphing,decorations.pathreplacing,angles,intersections,backgrounds}
\colorlet{blx}{blue!90!green!80}
\colorlet{grx}{green!50!black!90!yellow!80}
\colorlet{orx}{orange!80!black!90!yellow!80}
\colorlet{rex}{red!80!black!90!orange!80}
\colorlet{rey}{red!80!black!90!orange!40}
\colorlet{rez}{red!80!black!90!orange!15}
\colorlet{yex}{yellow!50!orange}

\DeclareMathOperator{\Prob}{Prob}
\DeclareMathOperator{\sinc}{sinc}

\newtheorem{postulate}{Postulate}

\NiceMatrixOptions{cell-space-limits=1pt}

\setchemfig{atom sep=2em,fixed length=true,bond offset=3pt,cram width=3pt}

\sisetup{range-phrase=-,range-units=single,table-alignment-mode=format,table-number-alignment=center}
\DeclareSIUnit{\angstrom}{\textup{\AA}}
\DeclareSIUnit{\atomicunit}{a.u.}
\DeclareSIUnit{\debye}{D}
\DeclareSIUnit{\hartree}{\emph{E}_h}
\DeclareSIUnit{\torr}{torr}
\DeclareSIUnit{\gauss}{G}
\DeclareSIUnit{\ppm}{ppm}

\newcommand{\N}{\mathbb{N}}
\newcommand{\Z}{\mathbb{Z}}
\newcommand{\R}{\mathbb{R}}

\newcommand{\ibf}{\mathbf{i}}
\newcommand{\jbf}{\mathbf{j}}
\newcommand{\ubf}{\mathbf{u}}
\newcommand{\Cbf}{\mathbf{C}}

\newcommand{\e}[1][]{\text{e}^{#1}}
\newcommand{\prb}[1]{\langle{#1}\rangle}

\usepackage{subfiles}