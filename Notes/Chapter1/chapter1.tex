\documentclass[../notes.tex]{subfiles}

\pagestyle{main}
\renewcommand{\chaptermark}[1]{\markboth{\chaptername\ \thechapter\ (#1)}{}}

\begin{document}




\chapter{From Classical to Quantum Mechanics}
\section{Blackbody Radiation}
\begin{itemize}
    \item \marginnote{9/27:}The surface of a hot body emits energy in the form of EM radiation.
    \item Changes that occur with temperature:
    \begin{itemize}
        \item If less than $\SI{500}{\celsius}$, we have IR Radiation (heat).
        \item If $\SIrange{500}{600}{\celsius}$, we have visible radiation (a glowing body).
        \item If $\SI{5000}{\celsius}$, we have a "white hot" body (short wavelength).
    \end{itemize}
    \item As a body gets hotter, it emits shorter wavelength radiation.
    \item \textbf{Stefan-Boltzmann law}: The the total emissive power $R$ (recall that power is en / time) of a blackbody (BB) is given by
    \begin{equation*}
        R(T) = \sigma T^4
    \end{equation*}
    where $\sigma\approx\SI{5.67e-8}{\watt\per\square\meter\per\kelvin\tothe{4}}$ is \textbf{Stefan's constant}.
    \begin{itemize}
        \item Work done by Stefan and Boltzmann (c. 1870 / 1884, respectively).
    \end{itemize}
    \item \textbf{Wien's 1st Law}: The wavelength for maximum emissive power obeys the equation
    \begin{equation*}
        \lambda_\text{max}T = b
    \end{equation*}
    where $b=\SI{2.898e-3}{\meter\kelvin}$ is \textbf{Wein's displacement constant}. \emph{Also known as} \textbf{Wien's displacement law}.
    \begin{figure}[h!]
        \centering
        \begin{tikzpicture}
            \footnotesize
            \draw [<->] (0,4.5) -- node[left]{$R(\lambda,T)$} (0,0) -- node[below=2mm]{$\lambda\ (\si{\micro\meter})$} (7.5,0);
            \foreach \x in {1,...,7} {
                \draw (\x,0.1) -- ++(0,-0.2);
            }
            \foreach \y in {1,...,4} {
                \draw (0.1,\y) -- ++(-0.2,0);
            }
            
            \begin{scope}[xscale=1.8]
                \draw [blx,thick,samples=100,smooth,/pgf/fpu,/pgf/fpu/output format=fixed] plot[domain=0.1:4] (\x,{5*10^2/(\x^5*(e^(9900/(2000*\x))-1))});
                \draw [grx,thick,samples=100,smooth,/pgf/fpu,/pgf/fpu/output format=fixed] plot[domain=0.1:4] (\x,{5*10^2/(\x^5*(e^(9900/(1750*\x))-1))});
                \draw [orx,thick,samples=100,smooth,/pgf/fpu,/pgf/fpu/output format=fixed] plot[domain=0.1:4] (\x,{5*10^2/(\x^5*(e^(9900/(1500*\x))-1))});
                \draw [rex,thick,samples=100,smooth,/pgf/fpu,/pgf/fpu/output format=fixed] plot[domain=0.1:4] (\x,{5*10^2/(\x^5*(e^(9900/(1000*\x))-1))});
        
                \draw [semithick] (0.997,3.667) node[above right]{$\lambda_\text{max}$} -- ++(0,-0.2);
                \draw [semithick] (1.139,1.93) -- ++(0,-0.2);
                \draw [semithick] (1.329,0.946) -- ++(0,-0.2);
                \draw [semithick] (1.994,0.211) -- ++(0,-0.2);
            \end{scope}
    
            \begin{scope}[xshift=6cm,yshift=3cm]
                \node at (0,1.2) {${\color{blx}\blacksquare}=\SI{2000}{\kelvin}$};
                \node at (0,0.8) {${\color{grx}\blacksquare}=\SI{1750}{\kelvin}$};
                \node at (0,0.4) {${\color{orx}\blacksquare}=\SI{1500}{\kelvin}$};
                \node at (0,0) {${\color{rex}\blacksquare}=\SI{1000}{\kelvin}$};
            \end{scope}
        \end{tikzpicture}
        \caption{Wein's 1st Law.}
        \label{fig:WeinLaw1}
    \end{figure}
    \item Area under the curve (found with integration) is the total emissive power.
    \item We now change variables from emissive power $R$ to energy density $\rho$ in the BB cavity.
    \begin{equation*}
        \rho(\lambda,T) = \frac{4}{c}R(\lambda,T)
    \end{equation*}
    \item Wien's 2nd Law (1893): The energy density must have a functional relationship with the following form.
    \begin{equation*}
        \rho(\lambda,T) = \frac{f(\lambda T)}{\lambda^5}
    \end{equation*}
    \begin{itemize}
        \item $f(\lambda T)$ cannot be determined from thermodynamics. Thus, something else is needed!
    \end{itemize}
    \item Lord Rayleigh and his graduate student Jeans (1899) propose a solution.
    \begin{itemize}
        \item EM: The thermal radiation within a cavity must exist in the form of standing waves.
        \item RJ showed that the number $n$ of standing waves per unit volume, per wavelength has the following form.
        \begin{equation*}
            n(\lambda) = \frac{8\pi}{\lambda^4}
        \end{equation*}
        \item If $\bar{\epsilon}$ is the average energy in the mode with wavelength $\lambda$, then
        \begin{equation*}
            \rho(\lambda,T) = \frac{8\pi}{\lambda^4}\bar{\epsilon}
        \end{equation*}
        \item Waves come from atoms in the walls of the BB cavity, which act as linear harmonic oscillators at a frequency $\nu=c/\lambda$.
        \item Assuming thermal equilibrium, we obtain
        \begin{align*}
            \bar{\epsilon} &= \frac{\int_0^\infty\epsilon\e[-\epsilon/kT]}{\int_0^\infty\e[-\epsilon/kT]}\\
            &= -\pdv{\beta}\ln\left( \int_0^\infty\e[-\beta\epsilon]\dd{\epsilon} \right)\\
            &= \frac{1}{\beta}\\
            &= kT
        \end{align*}
        where $k$ is the Boltzmann constant.
        \begin{itemize}
            \item Basically, we sum all energies $\epsilon$, weighted by the probability $\e[-\epsilon/kT]$ of the energy existing, and divided by the total energy.
            \item The first equation is equivalent to the second with $\beta=1/kT$.
        \end{itemize}
        \item Therefore,
        \begin{equation*}
            \rho(\lambda,T) = \frac{8\pi kT}{\lambda^4}
        \end{equation*}
    \end{itemize}
    \item UV catastrophe: Rayleigh's formula diverges from the experimental data for short wavelength.
    \begin{itemize}
        \item The above formula diverges to $+\infty$, driven by the $\lambda^4$ term in the denominator, as $\lambda\to 0$. However, the amount of radiation of shorter wavelengths should decrease past a point, as seen in Figure \ref{fig:WeinLaw1}.
    \end{itemize}
    \item Max Planck comes in, proposes an idea to the German acacdemy that's so radical, they think he's insane, but he's actually right and it lays a key idea for quantum mechanics.
    \item Planck's key insight: The energy levels of the oscillators are not continuous, but are quantized.
    \begin{itemize}
        \item So we can't actually take an integral as Rayleigh did; we have to take an infinite series.
        \item In reality,
        \begin{align*}
            \bar{\epsilon} &= \frac{\sum_{n=0}^\infty n\epsilon_0\e[-\beta n\epsilon_0]}{\sum_{n=0}^\infty\e[-\beta n\epsilon_0]}\\
            &= \frac{\epsilon_0}{\e[\beta\epsilon_0]-1}
        \end{align*}
        \item Thus,
        \begin{equation*}
            \rho(\lambda,T) = \frac{8\pi\epsilon_0}{\lambda^4(\e[\epsilon/kT]-1)}
        \end{equation*}
        \item But to satisfy Wien's 2nd law, we must let $\epsilon_0\propto 1/\lambda$. More specifically, $\epsilon_0=hc/\lambda=h\nu$, where $h$ is Planck's constant.
        \begin{itemize}
            \item This setup allowed us to get an accurate value for Planck's constant for the first time in history.
        \end{itemize}
        \item Planck's theory predicts the data of Figure 1.
    \end{itemize}
    \item A perfect blackbody absorbs and emits radiation at all frequencies.
    \begin{itemize}
        \item A star is pretty close to a blackbody. The graphite in a pencil is 97\% a blackbody. We are all blackbodies.
        \item The entire universe can be viewed as a blackbody.
    \end{itemize}
    \item Princeton and Bell Labs telescopes find \textbf{Cosmic Background Radiation} (A. A. Penzias and R. W. Wilson, 1964).
    \begin{itemize}
        \item Background radiation from the universe itself.
        \item $\lambda_\text{max}=\SI{7.35}{\centi\meter}$.
        \item Isotropic radio signal, that comes form everywhere.
        \item From this, you can workout the temperature of the universe from Wein's first law.
        \item Thus, the whole universe is a blackbody with a temperature of approximately $\SI{3}{\kelvin}$.
    \end{itemize}
\end{itemize}



\section{Chapter 1: The Dawn of the Quantum Theory}
\emph{From \textcite{bib:McQuarrieSimon}.}
\begin{itemize}
    \item \marginnote{9/28:}\textbf{Blackbody}: A body which absorbs and emits all frequencies. \emph{Also known as} \textbf{ideal body}.
    \item "Many theoretical physicists tried to derive expressions consistent with these experimental curves of intensity versus frequency [see Figure \ref{fig:WeinLaw1}], but they were all unsuccessful. In fact, the expression that is derived according to the laws of nineteenth century physics is" as follows \parencite[3]{bib:McQuarrieSimon}.
    \item \textbf{Rayleigh-Jeans law}: The equation
    \begin{equation*}
        \dd{\rho(\nu,T)} = \rho_\nu(T)\dd{\nu} = \frac{8\pi k_BT}{c^3}\nu^2\dd{\nu}
    \end{equation*}
    where $\rho_\nu(T)\dd{\nu}$ is the "radiant energy density between the frequencies $\nu$ and $\nu+\dd{\nu}$" \parencite[3]{bib:McQuarrieSimon}.
    \item The ultraviolet catastrophe is so named because the frequency increases as the radiation enters the ultraviolet region.
    \item Planck's solution:
    \begin{itemize}
        \item Rayleigh and Jeans assumed (as does classical physics) that the energies of the electronic oscillators responsible for the emission of the radiation could have any value whatsoever.
        \item However, Planck assumed discrete oscillator energies proportional to an integral multiple of the frequency: $E=nh\nu$, where $n\in\mathbb{Z}$.
        \item Using this quantization energy and ideas from statistical thermodynamics (see Chapter 17), Planck derived the \textbf{Planck distribution law for blackbody radiation}.
        \item The only undetermined constant in the above equation was $h$, and Planck showed that if we let $h=\SI{6.626e-34}{\joule\second}$, then this equation gives excellent agreement with the experimental data for all frequencies and temperatures.
    \end{itemize}
    \item \textbf{Planck distribution law for blackbody radiation}: The equation
    \begin{equation*}
        \dd{\rho(\nu,T)} = \rho_\nu(T)\dd{\nu} = \frac{8\pi h}{c^3}\frac{\nu^3\dd{\nu}}{\e[h\nu/k_BT]-1}
    \end{equation*}
    \begin{itemize}
        \item Note that for small frequencies, the Planck distribution law and Rayleigh-Jeans law converge, but they diverge for large frequencies, as expected.
        \item Because $\nu$ and $\lambda$ are related by $\lambda\nu=c$ (and subsequently by $\dd{\nu}=-c/\lambda^2\dd{\lambda}$), we can write the Planck distribution law in terms of wavelength, as well.
        \begin{equation*}
            \dd{\rho(\lambda,T)} = \rho_\lambda(T)\dd{\lambda} = \frac{8\pi hc}{\lambda^5}\frac{\dd{\lambda}}{\e[hc/\lambda k_BT]-1}
        \end{equation*}
    \end{itemize}
    \item Differentiating $\rho_\lambda(T)$ with respect to $\lambda$ gives an alternate formulation for $b$:
    \begin{equation*}
        \lambda_\text{max}T = \frac{hc}{4.965k_B}
    \end{equation*}
    \item Astronomers use the theory of blackbody radiation to estimate the surface temperatures of stars.
    \begin{itemize}
        \item We can measure the electromagnetic spectrum of a star (which will follow a curve similar to one of the ones in Figure \ref{fig:WeinLaw1}).
        \item Then we can find $\lambda_\text{max}$. From here, all that's necessary is to plug into Wien's displacement law:
        \begin{equation*}
            T = \frac{b}{\lambda_\text{max}}
        \end{equation*}
    \end{itemize}
\end{itemize}




\end{document}