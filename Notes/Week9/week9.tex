\documentclass[../notes.tex]{subfiles}

\pagestyle{main}
\renewcommand{\chaptermark}[1]{\markboth{\chaptername\ \thechapter\ (#1)}{}}
\setcounter{chapter}{8}

\begin{document}




\chapter{Quantum Dynamics and Control}
\section{Time-Dependent Schr\"{o}dinger Equation and Spectroscopy}
\begin{itemize}
    \item \marginnote{11/29:}From last time:
    \begin{itemize}
        \item The probability of a two-level system being in state 2 at time $\tau$ is
        \begin{equation*}
            |a_2(\tau)|^2 \equiv a_2^*(\tau)a_2(\tau) \propto \frac{\sin^2[(E_2-E_1-\hbar\omega)t/2\hbar]}{(E_2-E_1-\hbar\omega)^2}
        \end{equation*}
        \item When $\hbar\omega=E_2-E_1$, you'll have the maximum resonance/strongest absorption and the peak in the graph of the $\sinc$ function.
        \item Leads to selection rules: Only certain states directly couple with a weak perturbation.
    \end{itemize}
    \item Now consider the transition dipole moment:
    \begin{equation*}
        (\mu_z)_{12} = \int\phi_2^*\mu_z\phi_1\dd{x}
    \end{equation*}
    \item The integral will vanish in the rigid rotor approximation.
    \begin{itemize}
        \item Up to the approximation of the rigid rotor, $l$ can only change by $\pm 1$ and $m$ stays the same.
    \end{itemize}
    \item Taking the harmonic oscillator as the paradigm for vibrational motion, we find that the quantum number must be $\Delta n=\pm 1$.
    \begin{itemize}
        \item The harmonic oscillator wave functions are of the form
        \begin{equation*}
            \psi_n(x) = N_nH_n(\sqrt{\alpha}x)\e[-\alpha x^2/2]
        \end{equation*}
        where $\alpha=\sqrt{k\mu}/\hbar$.
        \item Assume $\alpha=1$. Then $\psi_n=N_nH_n(x)\e[-x^2/2]$.
        \item We want to evaluate
        \begin{equation*}
            (\mu_z)_{12} = \int\psi_{n'}(q)\mu_z(q)\psi_n(q)\dd{q}
        \end{equation*}
        where
        \begin{equation*}
            \mu_z(q) = \mu_0+\left( \dv{\mu}{q} \right)_{q=0}q+\cdots
        \end{equation*}
        \begin{itemize}
            \item This means that the transition dipole moment changes as a function of the bond length.
            \item When $q=0$ (at equilibrium), the transition dipole moment is the equilibrium one ($\mu_0$).
            \item Then we Taylor series expand to correct the transition dipole moment away from equilibrium.
        \end{itemize}
        \item Therefore,
        \begin{equation*}
            (\mu_z)_{nn'} \equiv \mu_0\int_{-\infty}^\infty H_{n'}(q)H_n(q)\e[-q^2]\dd{q}+\left( \dv{\mu}{q} \right)_{q=0}\int_{-\infty}^\infty H_{n'}(q)qH_n(q)\e[-q^2]\dd{q}+\cdots
        \end{equation*}
        \begin{itemize}
            \item The first term goes to zero as long as $n\neq n'$ because the Hermite polynomials are orthonormal.
        \end{itemize}
        \item Recall that we used a recurrence relation to define the Hermite polynomials.
        \item In addition to the one we used previously, we have the recurrence relation
        \begin{equation*}
            qH_n(q) = nH_{n-1}(q)+\frac{1}{2}H_{n+1}(q)
        \end{equation*}
        for all $n$.
        \item Thus,
        \begin{equation*}
            \int_{-\infty}^\infty H_{n'}(q)qH_n(q)\e[-q^2]\dd{q} = \int_{-\infty}^\infty H_{n'}(q)\left[ nH_{n-1}(q)+\frac{1}{2}H_{n+1}(q) \right]\e[-q^2]\dd{q}
        \end{equation*}
        \item Thus, unless $n'=n\pm 1$, the integral vanishes.
        \item This is the selection rule!
    \end{itemize}
    \item Note that we're just tickling the molecule with a bit of radiation to get this to happen --- if we hit it with too hard of a hammer, the selection rule will no longer hold.
    \item Spectroscopy.
    \begin{itemize}
        \item $\Delta E=E_n-E_l=h\nu$.
        \item Regions of EM:
        \begin{table}[h!]
            \centering
            \small
            \renewcommand{\arraystretch}{1.2}
            \begin{tabular}{l|llll}
                \textbf{Region} & Microwave & Far IR & IR & Vis/UV\\
                \hline
                \textbf{Wavenumber (cm${}^{\bm{-1}}$)} & $\numrange{0.033}{3.3}$ & $\numrange{3.3}{330}$ & $\numrange{330}{14500}$ & $\numrange{14500}{500000}$\\
                \hline
                \textbf{Molecular property} & Rotation of polyatomics & Rotation of small molecules & Vibration & Electron transitions\\
            \end{tabular}
            \caption{Spectroscopy in various regions of the electromagnetic spectrum.}
            \label{fig:spectroscopy}
        \end{table}
    \end{itemize}
    \item The Schr\"{o}dinger equation within the Born-Oppenheimer approximation:
    \begin{equation*}
        \left[ \sum_A\left( -\frac{\nabla_A^2}{2} \right)+\hat{H}_\text{elec} \right]\psi_\text{elec}\psi_\text{nucl} = (E_\text{elec}+E_\text{nucl})\psi_\text{elec}\psi_\text{nucl}
    \end{equation*}
    \begin{itemize}
        \item We invoke the Born-Oppenheimer approximation to split the wavefunction into electronic and nuclear components.
    \end{itemize}
    \item We then split this equation into two parts. The first of which is the electronic Schr\"{o}dinger equation
    \begin{equation*}
        \hat{H}_\text{elec}\psi_\text{elec} = E_\text{elec}\psi_\text{elec}
    \end{equation*}
    \begin{itemize}
        \item The BO approximation is what allows us to split the original Schr\"{o}dinger equation in two.
    \end{itemize}
    \item We now multiply by $\psi_\text{elec}$ and integrate over all of the electrons.
    \begin{equation*}
        \left[ \sum_A\left( -\frac{\nabla_A^2}{2} \right)+V(R) \right]\psi_\text{nucl} = (E_\text{elec}+E_\text{nucl})\psi_\text{nucl}
    \end{equation*}
    \begin{itemize}
        \item $V(R)=E_\text{elec}(R)=\int\psi_\text{elec}^*\hat{H}_\text{elec}\psi_\text{elec}\dd{x}$ is the potential energy surface (PES).
    \end{itemize}
\end{itemize}



\section{Spectroscopy (cont.)}
\begin{itemize}
    \item \marginnote{12/1:}From last time, we have that
    \begin{equation*}
        \left[ \sum_A\left( -\frac{\nabla_A^2}{2\mu_A} \right)+V(R) \right]\psi_\text{nucl} = (E_\text{elec}+E_\text{nucl})\psi_\text{nucl}
    \end{equation*}
    \begin{itemize}
        \item The PES is $V(R)=\int\psi_\text{elec}^*\hat{H}_\text{elec}\psi_\text{elec}\dd{x}$.
        \item This begins to break down where we have \textbf{conical intersections}.
    \end{itemize}
    \item \textbf{Conical intersection}: An intersection between potential energy surfaces, where electrons can jump from one to the other.
    \begin{itemize}
        \item Allows for radiation-less transitions from one surface to another.
    \end{itemize}
    \item Potential energy surface\footnote{Note that we technically have a potential energy curve at this point; we will only have a surface in higher-dimensional systems.} for the ground electronic wave function and its surface.
    \begin{itemize}
        \item Think Figure \ref{fig:parabolicPotentialb} with its zero-point energy in the HO approximation.
        \item The energies of the vibrational levels get closer and closer together until they're continuous forming a continuum at the level of the asymptote, reaching the classical limit.
    \end{itemize}
    \item We have that
    \begin{align*}
        E_\text{nucl} &\approx E_\text{HO}+E_\text{RR}\\
        &\approx \hbar\omega(n+\tfrac{1}{2})+\frac{\hbar^2}{2I}l(l+1)
    \end{align*}
    \begin{itemize}
        \item Since harmonic oscillator transitions are in the IR region and rigid rotor transitions are in the microwave region, vibrational transitions are of higher energy ($\hbar\omega>\hbar^2/2I$).
    \end{itemize}
    \item Rotational levels.
    \begin{figure}[h!]
        \centering
        \begin{tikzpicture}
            \small
            \draw [-stealth] (-3,0) -- node[left]{$E$} ++(0,4.5);
    
            \draw [blx,thick] (-2,4) parabola bend (0,0) (2,4);
    
            \draw [grx,semithick] (-1,1) -- (1,1);
            \draw [grx,semithick] (-1.5,2.25) -- (1.5,2.25);
    
            \draw [grx,yshift=1cm]
                (-0.5,0.05) -- ++(1,0)
                (-0.5,0.15) -- ++(1,0)
                (-0.5,0.3) -- ++(1,0)
                (-0.5,0.5) -- ++(1,0)
                (-0.5,0.75) -- ++(1,0)
            ;
            \draw [grx,yshift=2.25cm]
                (-0.5,0.05) -- ++(1,0)
                (-0.5,0.15) -- ++(1,0)
                (-0.5,0.3) -- ++(1,0)
                (-0.5,0.5) -- ++(1,0)
                (-0.5,0.75) -- ++(1,0)
            ;
        \end{tikzpicture}
        \caption{Rotational and vibrational energy levels.}
        \label{fig:rovibrationalELevels}
    \end{figure}
    \begin{itemize}
        \item Each vibrational energy level has a number of smaller rotational states that are part of it.
    \end{itemize}
    \item Changes in rotational levels often accompany changes in vibrational levels.
    \begin{itemize}
        \item If we change the vibrational level by $\pm 1$, we have to change the rotational level (by $\pm 1$) as well.
    \end{itemize}
    \item IR/vibrational spectroscopy.
    \begin{figure}[H]
        \centering
        \begin{tikzpicture}[
            every node/.style={black}
        ]
            \small
            \draw [stealth-stealth] (0,2) -- node[rotate=90,above]{Absorption} (0,0) -- node[below]{$\si{\per\centi\meter}$} (5.5,0);
    
            \footnotesize
            \draw [rex,thick] (0,0.5)
                -- ++(1,0)
                -- ++(0.05,0.5) -- ++(0.05,-0.5) -- ++(0.05,0) -- ++(0.05,0.7) -- ++(0.05,-0.7) -- ++(0.05,0) -- ++(0.05,0.9) -- ++(0.05,-0.9) -- ++(0.05,0) -- ++(0.05,1) node[below=1cm]{$\Delta J=-1$} node[above]{$P$ branch} -- ++(0.05,-1) -- ++(0.05,0) -- ++(0.05,0.8) -- ++(0.05,-0.8) -- ++(0.05,0) -- ++(0.05,0.6) -- ++(0.05,-0.6) -- ++(0.05,0) -- ++(0.05,0.2) -- ++(0.05,-0.2)
                -- ++(0.5,0)
                -- ++(0.05,0.2) -- ++(0.05,-0.2) -- ++(0.05,0) -- ++(0.05,0.6) -- ++(0.05,-0.6) -- ++(0.05,0) -- ++(0.05,0.8) -- ++(0.05,-0.8) -- ++(0.05,0) -- ++(0.05,1) node[below=1cm]{$\Delta J=+1$} node[above]{$R$ branch} -- ++(0.05,-1) -- ++(0.05,0) -- ++(0.05,0.9) -- ++(0.05,-0.9) -- ++(0.05,0) -- ++(0.05,0.7) -- ++(0.05,-0.7) -- ++(0.05,0) -- ++(0.05,0.5) -- ++(0.05,-0.5)
                -- (5,0.5)
            ;
        \end{tikzpicture}
        \caption{IR/vibrational spectrum.}
        \label{fig:IRspectrum}
    \end{figure}
    \begin{itemize}
        \item The whole spectrum represents a rovibrational transition.
        \item The $P$ branch corresponds to when you go from a lower vibrational state to a higher vibrational state, \emph{but} you go from a higher rotational state to a lower rotational state.
        \item The $R$ branch corresponds to when you go from a lower vibrational state to a higher vibrational state, \emph{and} you go from a lower rotational state to a higher rotational state.
        \item The number of peaks in the $R$ and $P$ branches reveals which rotational states are occupied (although we could theoretically go on forever, realistically, only lower energy rotational states are occupied).
    \end{itemize}
    \item Electronic states.
    \begin{figure}[h!]
        \centering
        \begin{subfigure}[b]{0.49\linewidth}
            \centering
            \begin{tikzpicture}
                \small
                \draw [stealth-stealth] (0,4.5) -- node[left]{$E$} (0,0) -- node[below]{$R$} (6.5,0);
    
                \draw [blx,thick,name path=PES1] (0.5,2) to[out=-85,in=180,in looseness=0.7] (1.3,0.2) to[out=0,in=180,out looseness=0.5,in looseness=2] (4,1);
                \begin{scope}[
                    on background layer
                ]
                    \path [name path=ve1] (0,0.40) -- ++(6,0);
                    \path [name path=ve2] (0,0.57) -- ++(6,0);
                    \path [name path=ve3] (0,0.71) -- ++(6,0);
                    \path [name path=ve4] (0,0.82) -- ++(6,0);
                    \path [name path=ve5] (0,0.90) -- ++(6,0);
                    \path [name path=ve6] (0,0.95) -- ++(6,0);
                    \draw [grx,semithick,name intersections={of=PES1 and ve1}] (intersection-1) -- (intersection-2);
                    \draw [grx,semithick,name intersections={of=PES1 and ve2}] (intersection-1) -- (intersection-2);
                    \draw [grx,semithick,name intersections={of=PES1 and ve3}] (intersection-1) -- (intersection-2);
                    \draw [grx,semithick,name intersections={of=PES1 and ve4}] (intersection-1) -- (intersection-2);
                    \draw [grx,semithick,name intersections={of=PES1 and ve5}] (intersection-1) -- (intersection-2);
                    \draw [grx,semithick,name intersections={of=PES1 and ve6}] (intersection-1) -- (intersection-2);
                \end{scope}
    
                \draw [blx,thick,xshift=0.6cm,yshift=1cm,name path=PES2] (0.6,1.5) to[out=-85,in=180,in looseness=0.7] (1.3,0.2) to[out=0,in=180,out looseness=0.5,in looseness=2] (4,0.75);
                \begin{scope}[
                    on background layer,
                    yshift=1cm
                ]
                    \path [name path=ve1] (0,0.40) -- ++(6,0);
                    \path [name path=ve2] (0,0.57) -- ++(6,0);
                    \path [name path=ve3] (0,0.71) -- ++(6,0);
                    \draw [grx,semithick,name intersections={of=PES2 and ve1}] (intersection-1) -- (intersection-2);
                    \draw [grx,semithick,name intersections={of=PES2 and ve2}] (intersection-1) -- (intersection-2);
                    \draw [grx,semithick,name intersections={of=PES2 and ve3}] (intersection-1) -- (intersection-2);
                \end{scope}
    
                \draw [blx,thick,xshift=1cm,yshift=2cm,name path=PES3] (0.7,1.2) to[out=-85,in=180,in looseness=0.7] (1.3,0.2) to[out=0,in=180,out looseness=0.5,in looseness=2] (4,0.6);
                \begin{scope}[
                    on background layer,
                    yshift=2cm
                ]
                    \path [name path=ve1] (0,0.40) -- ++(6,0);
                    \path [name path=ve2] (0,0.57) -- ++(6,0);
                    \draw [grx,semithick,name intersections={of=PES3 and ve1}] (intersection-1) -- (intersection-2);
                    \draw [grx,semithick,name intersections={of=PES3 and ve2}] (intersection-1) -- (intersection-2);
                \end{scope}
    
                \draw [blx,thick,xshift=1.5cm,yshift=2.5cm] (0.7,1.8) to[out=-70,in=180] (4,0.6);
    
                \path [name path=trans] (1.3,0) -- ++(0,4);
                \draw [orx,semithick,-latex,shorten >=4pt,name intersections={of=trans and PES1,by=start},name intersections={of=trans and PES2,by=end}] (start) -- (end);
            \end{tikzpicture}
            \caption{Electronic states.}
            \label{fig:electronicStatesa}
        \end{subfigure}
        \begin{subfigure}[b]{0.49\linewidth}
            \centering
            \begin{tikzpicture}
                \small
                \draw [stealth-stealth] (0,4.5) -- node[left]{$E$} (0,0) -- node[below]{$R$} (6.5,0);
    
                \draw [blx,thick,name path=PES1] (0.5,2) to[out=-85,in=180,in looseness=0.7] (1.3,0.2) to[out=0,in=180,out looseness=0.5,in looseness=2] (4,1);
                \begin{scope}[
                    on background layer
                ]
                    \draw [grx,semithick] (1.05,0.40) -- ++(0.5,0);
                    \draw [grx,semithick] (1.05,0.57) -- ++(0.5,0);
                    \draw [grx,semithick] (1.05,0.71) -- ++(0.5,0);
                    \draw [grx,semithick] (1.05,0.82) -- ++(0.5,0);
                    \draw [grx,semithick] (1.05,0.90) -- ++(0.5,0);
                    \draw [grx,semithick] (1.05,0.95) -- ++(0.5,0);
                \end{scope}
    
                \draw [blx,thick,yshift=1cm,name path=PES2] (0.6,1.5) to[out=-85,in=180,in looseness=0.7] (1.3,0.2) to[out=0,in=180,out looseness=0.5,in looseness=2] (4,0.75);
                \begin{scope}[
                    on background layer,
                    yshift=1cm
                ]
                    \draw [grx,semithick] (1.05,0.40) -- ++(0.5,0);
                    \draw [grx,semithick] (1.05,0.57) -- ++(0.5,0);
                    \draw [grx,semithick] (1.05,0.71) -- ++(0.5,0);
                \end{scope}
    
                \draw [blx,thick,xshift=0.6cm,yshift=2cm,name path=PES3] (0.7,1.2) to[out=-85,in=180,in looseness=0.7] (1.3,0.2) to[out=0,in=180,out looseness=0.5,in looseness=2] (4,0.6);
                \begin{scope}[
                    on background layer,
                    xshift=0.6cm,yshift=2cm
                ]
                    \draw [grx,semithick] (1.05,0.40) -- ++(0.5,0);
                    \draw [grx,semithick] (1.05,0.57) -- ++(0.5,0);
                \end{scope}
    
                \draw [densely dashed,xshift=1.3cm,yshift=0.2cm] plot [domain=-1:1] (\x,{0.8*e^(-3*\x*\x)});
                \draw [densely dashed,xshift=1.3cm,yshift=1.3cm] plot [domain=-1:1] (\x,{0.6*e^(-3*\x*\x)});
                \draw [densely dashed,xshift=1.9cm,yshift=2.2cm] plot [domain=-1:1] (\x,{0.4*e^(-3*\x*\x)});
    
                \path [name path=trans] (1.3,0) -- ++(0,4);
                \path [name path=trans2] (1.4,0) -- ++(0,4);
                \draw [orx,semithick,-latex,name intersections={of=trans and PES1,by=start},name intersections={of=trans and PES2,by=end}] (start) -- (end);
                \draw [orx,semithick,-latex,shorten >=2pt,name intersections={of=trans2 and PES1,by=start},name intersections={of=trans2 and PES3,by=end}] (start) -- (end);
            \end{tikzpicture}
            \caption{Overlapping Gaussian wave packets.}
            \label{fig:electronicStatesb}
        \end{subfigure}
        \caption{Exciting between electronic states.}
        \label{fig:electronicStates}
    \end{figure}
    \begin{itemize}
        \item You can also do excitations from one potential energy surface to another.
        \item Recall that we have a Gaussian wave packet around the minimum energy on the PES.
        \item It's important that our Gaussian wave packets have some overlap for us to jump from one PES to the other.
        \item In Figure \ref{fig:electronicStatesb}, we'd expect a much higher probability of electronic transitions to the PES's with significant orbital overlap than from either to the third, where there is much less overlap.
    \end{itemize}
    \item \textbf{Franck-Condon Principle}\footnote{Developed at UChicago!}: The intensity of the transitions is proportional to the product of two harmonic oscillator wave functions --- $\psi_1$ from the one vibrational state and $\psi_2$ from the other vibrational state.
    \item In a polyatomic molecule, any bond can be approximated as a spring, and any spring can be approximated as a harmonic oscillator.
    \begin{itemize}
        \item Thus, any polyatomic molecule can be thought of as a bunch of harmonic oscillators linked together.
    \end{itemize}
    \item Normal modes and normal coordinates:
    \begin{itemize}
        \item Degrees of freedom: $3N$ ($x,y,z$ coordinates).
        \begin{itemize}
            \item There are translational, vibrational, and rotational DOFs.
            \item We always have 3 translational DOFs, 2 or 3 rotational DOFs (depending on whether or not the molecule is linear), and $3N-5$ or $3N-6$ vibrational DOFs (depending on whether or not the molecule is linear once again).
        \end{itemize}
    \end{itemize}
    \item Taking the right linear combinations of coupled oscillator stretches gives normal modes that are orthogonal to each other.
    \begin{itemize}
        \item Diagonalizing gives normal modes that are decoupled from each other.
    \end{itemize}
    \item Example (\ce{H2O}):
    \begin{figure}[h!]
        \centering
        \begin{subfigure}[b]{0.3\linewidth}
            \centering
            \begin{tikzpicture}[
                every node/.style={black}
            ]
                \draw [semithick,orx,-latex] (0,1) node[circle,fill,inner sep=1.5pt]{} -- ++(90:0.5);
                \draw [semithick,orx,-latex] (-1,0) node[circle,fill,inner sep=1.5pt]{} -- ++(-135:0.5);
                \draw [semithick,orx,-latex] (1,0) node[circle,fill,inner sep=1.5pt]{} -- ++(-45:0.5);
            \end{tikzpicture}
            \caption{Symmetric stretch.}
            \label{fig:H2ONormalModesa}
        \end{subfigure}
        \begin{subfigure}[b]{0.3\linewidth}
            \centering
            \begin{tikzpicture}[
                every node/.style={black}
            ]
                \draw [semithick,orx,-latex] (0,1) node[circle,fill,inner sep=1.5pt]{} -- ++(0:0.5);
                \draw [semithick,orx,-latex] (-1,0) node[circle,fill,inner sep=1.5pt]{} -- ++(-135:0.5);
                \draw [semithick,orx,-latex] (1,0) node[circle,fill,inner sep=1.5pt]{} -- ++(135:0.5);
            \end{tikzpicture}
            \caption{Asymmetric stretch.}
            \label{fig:H2ONormalModesb}
        \end{subfigure}
        \begin{subfigure}[b]{0.3\linewidth}
            \centering
            \begin{tikzpicture}[
                every node/.style={black}
            ]
                \draw [semithick,orx,-latex] (0,1) node[circle,fill,inner sep=1.5pt]{} -- ++(-90:0.5);
                \draw [semithick,orx,-latex] (-1,0) node[circle,fill,inner sep=1.5pt]{} to[out=180,in=-70] ++(135:0.5);
                \draw [semithick,orx,-latex] (1,0) node[circle,fill,inner sep=1.5pt]{} to[out=0,in=-110] ++(45:0.5);
            \end{tikzpicture}
            \caption{Bending mode.}
            \label{fig:H2ONormalModesc}
        \end{subfigure}
        \caption{Normal modes of \ce{H2O}.}
        \label{fig:H2ONormalModes}
    \end{figure}
    \begin{itemize}
        \item 3 vibrational modes.
        \item Symmetric stretch, asymmetric stretch, and bending mode.
        \item Their respective frequencies are $\nu_1=\SI{3650}{\per\centi\meter}$, $\nu_2=\SI{3760}{\per\centi\meter}$, and $\nu_3=\SI{1600}{\per\centi\meter}$.
    \end{itemize}
\end{itemize}



\section{Chapter 13: Molecular Spectroscopy}
\emph{From \textcite{bib:McQuarrieSimon}.}
\begin{itemize}
    \item \marginnote{11/30:}\textbf{Spectroscopy}: The study of the interaction of electromagnetic radiation with atoms and molecules.
    \item "Electromagnetic radiation is customarily divided into different energy regions reflecting the different types of molecular processes that can be caused by such radiation" \parencite[495-96]{bib:McQuarrieSimon}.
    \item Vibrational selection rule: Transitions among vibrational levels resulting from the absorption of radiation have $\Delta v=\pm 1$ and have a dipole moment that varies during the vibration.
    \begin{itemize}
        \item For a harmonic oscillator, the spectrum consists of one line in the infrared region at the frequency $\nu_\text{obs}=\sqrt{k/\mu}/2\pi$.
    \end{itemize}
    \item \textbf{Vibrational term}: The vibrational energy of a molecule. \emph{Denoted by} $\bm{G(v)}$. \emph{Units} $\si{\per\centi\meter}$. \emph{Given by}
    \begin{equation*}
        G(v) = \frac{E_v}{hc}
    \end{equation*}
    where $E_v=(v+1/2)h\nu$ and $\nu=\sqrt{k/\mu}/2\pi$.
    \item Each energy $E_J=\hbar^2/(2I)\cdot J(J+1)$ of the rigid rotator is associated with degeneracy $g_J=2J+1$.
\end{itemize}




\end{document}