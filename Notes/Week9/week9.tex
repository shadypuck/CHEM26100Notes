\documentclass[../notes.tex]{subfiles}

\pagestyle{main}
\renewcommand{\chaptermark}[1]{\markboth{\chaptername\ \thechapter\ (#1)}{}}
\setcounter{chapter}{8}

\begin{document}




\chapter{Quantum Dynamics and Control}
\section{Time-Dependent Schr\"{o}dinger Equation and Spectroscopy}
\begin{itemize}
    \item \marginnote{11/29:}From last time:
    \begin{itemize}
        \item The probability of a two-level system being in state 2 at time $\tau$ is
        \begin{equation*}
            |a_2(\tau)|^2 \equiv a_2^*(\tau)a_2(\tau) \propto \frac{\sin^2[(E_2-E_1-\hbar\omega)t/2\hbar]}{(E_2-E_1-\hbar\omega)^2}
        \end{equation*}
        \item When $\hbar\omega=E_2-E_1$, you'll have the maximum resonance/strongest absorption and the peak in the graph of the $\sinc$ function.
        \item Leads to selection rules: Only certain states directly couple with a weak perturbation.
    \end{itemize}
    \item Now consider the transition dipole moment:
    \begin{equation*}
        (\mu_z)_{12} = \int\phi_2^*\mu_z\phi_1\dd{x}
    \end{equation*}
    \item The integral will vanish in the rigid rotor approximation.
    \begin{itemize}
        \item Up to the approximation of the rigid rotor, $l$ can only change by $\pm 1$ and $m$ stays the same.
    \end{itemize}
    \item Taking the harmonic oscillator as the paradigm for vibrational motion, we find that the quantum number must be $\Delta n=\pm 1$.
    \begin{itemize}
        \item The harmonic oscillator wave functions are of the form
        \begin{equation*}
            \psi_n(x) = N_nH_n(\sqrt{\alpha}x)\e[-\alpha x^2/2]
        \end{equation*}
        where $\alpha=\sqrt{k\mu}/\hbar$.
        \item Assume $\alpha=1$. Then $\psi_n=N_nH_n(x)\e[-x^2/2]$.
        \item We want to evaluate
        \begin{equation*}
            (\mu_z)_{12} = \int\psi_{n'}(q)\mu_z(q)\psi_n(q)\dd{q}
        \end{equation*}
        where
        \begin{equation*}
            \mu_z(q) = \mu_0+\left( \dv{\mu}{q} \right)_{q=0}q+\cdots
        \end{equation*}
        \begin{itemize}
            \item This means that the transition dipole moment changes as a function of the bond length.
            \item When $q=0$ (at equilibrium), the transition dipole moment is the equilibrium one ($\mu_0$).
            \item Then we Taylor series expand to correct the transition dipole moment away from equilibrium.
        \end{itemize}
        \item Therefore,
        \begin{equation*}
            (\mu_z)_{nn'} \equiv \mu_0\int_{-\infty}^\infty H_{n'}(q)H_n(q)\e[-q^2]\dd{q}+\left( \dv{\mu}{q} \right)_{q=0}\int_{-\infty}^\infty H_{n'}(q)qH_n(q)\e[-q^2]\dd{q}+\cdots
        \end{equation*}
        \begin{itemize}
            \item The first term goes to zero as long as $n\neq n'$ because the Hermite polynomials are orthonormal.
        \end{itemize}
        \item Recall that we used a recurrence relation to define the Hermite polynomials.
        \item In addition to the one we used previously, we have the recurrence relation
        \begin{equation*}
            qH_n(q) = nH_{n-1}(q)+\frac{1}{2}H_{n+1}(q)
        \end{equation*}
        for all $n$.
        \item Thus,
        \begin{equation*}
            \int_{-\infty}^\infty H_{n'}(q)qH_n(q)\e[-q^2]\dd{q} = \int_{-\infty}^\infty H_{n'}(q)\left[ nH_{n-1}(q)+\frac{1}{2}H_{n+1}(q) \right]\e[-q^2]\dd{q}
        \end{equation*}
        \item Thus, unless $n'=n\pm 1$, the integral vanishes.
        \item This is the selection rule!
    \end{itemize}
    \item Note that we're just tickling the molecule with a bit of radiation to get this to happen --- if we hit it with too hard of a hammer, the selection rule will no longer hold.
    \item Spectroscopy.
    \begin{itemize}
        \item $\Delta E=E_n-E_l=h\nu$.
        \item Regions of EM:
        \begin{table}[h!]
            \centering
            \small
            \renewcommand{\arraystretch}{1.2}
            \begin{tabular}{l|llll}
                \textbf{Region} & Microwave & Far IR & IR & Vis/UV\\
                \hline
                \textbf{Wavenumber (cm${}^{\bm{-1}}$)} & $\numrange{0.033}{3.3}$ & $\numrange{3.3}{330}$ & $\numrange{330}{14500}$ & $\numrange{14500}{500000}$\\
                \hline
                \textbf{Molecular property} & Rotation of polyatomics & Rotation of small molecules & Vibration & Electron transitions\\
            \end{tabular}
            \caption{Spectroscopy in various regions of the electromagnetic spectrum.}
            \label{fig:spectroscopy}
        \end{table}
    \end{itemize}
    \item The Schr\"{o}dinger equation within the Born-Oppenheimer approximation:
    \begin{equation*}
        \left[ \sum_A\left( -\frac{\nabla_A^2}{2} \right)+\hat{H}_\text{elec} \right]\psi_\text{elec}\psi_\text{nucl} = (E_\text{elec}+E_\text{nucl})\psi_\text{elec}\psi_\text{nucl}
    \end{equation*}
    \begin{itemize}
        \item We invoke the Born-Oppenheimer approximation to split the wavefunction into electronic and nuclear components.
    \end{itemize}
    \item We then split this equation into two parts. The first of which is the electronic Schr\"{o}dinger equation
    \begin{equation*}
        \hat{H}_\text{elec}\psi_\text{elec} = E_\text{elec}\psi_\text{elec}
    \end{equation*}
    \begin{itemize}
        \item The BO approximation is what allows us to split the original Schr\"{o}dinger equation in two.
    \end{itemize}
    \item We now multiply by $\psi_\text{elec}$ and integrate over all of the electrons.
    \begin{equation*}
        \left[ \sum_A\left( -\frac{\nabla_A^2}{2} \right)+V(R) \right]\psi_\text{nucl} = (E_\text{elec}+E_\text{nucl})\psi_\text{nucl}
    \end{equation*}
    \begin{itemize}
        \item $V(R)=E_\text{elec}(R)=\int\psi_\text{elec}^*\hat{H}_\text{elec}\psi_\text{elec}\dd{x}$ is the potential energy surface (PES).
    \end{itemize}
\end{itemize}



\section{Spectroscopy (cont.)}
\begin{itemize}
    \item \marginnote{12/1:}From last time, we have that
    \begin{equation*}
        \left[ \sum_A\left( -\frac{\nabla_A^2}{2\mu_A} \right)+V(R) \right]\psi_\text{nucl} = (E_\text{elec}+E_\text{nucl})\psi_\text{nucl}
    \end{equation*}
    \begin{itemize}
        \item The PES is $V(R)=\int\psi_\text{elec}^*\hat{H}_\text{elec}\psi_\text{elec}\dd{x}$.
        \item This begins to break down where we have \textbf{conical intersections}.
    \end{itemize}
    \item \textbf{Conical intersection}: An intersection between potential energy surfaces, where electrons can jump from one to the other.
    \begin{itemize}
        \item Allows for radiation-less transitions from one surface to another.
    \end{itemize}
    \item Potential energy surface\footnote{Note that we technically have a potential energy curve at this point; we will only have a surface in higher-dimensional systems.} for the ground electronic wave function and its surface.
    \begin{itemize}
        \item Think Figure \ref{fig:parabolicPotentialb} with its zero-point energy in the HO approximation.
        \item The energies of the vibrational levels get closer and closer together until they're continuous forming a continuum at the level of the asymptote, reaching the classical limit.
    \end{itemize}
    \item We have that
    \begin{align*}
        E_\text{nucl} &\approx E_\text{HO}+E_\text{RR}\\
        &\approx \hbar\omega(n+\tfrac{1}{2})+\frac{\hbar^2}{2I}l(l+1)
    \end{align*}
    \begin{itemize}
        \item Since harmonic oscillator transitions are in the IR region and rigid rotor transitions are in the microwave region, vibrational transitions are of higher energy ($\hbar\omega>\hbar^2/2I$).
    \end{itemize}
    \item Rotational levels.
    \begin{figure}[h!]
        \centering
        \begin{tikzpicture}
            \small
            \draw [-stealth] (-3,0) -- node[left]{$E$} ++(0,4.5);
    
            \draw [blx,thick] (-2,4) parabola bend (0,0) (2,4);
    
            \draw [grx,semithick] (-1,1) -- (1,1);
            \draw [grx,semithick] (-1.5,2.25) -- (1.5,2.25);
    
            \draw [grx,yshift=1cm]
                (-0.5,0.05) -- ++(1,0)
                (-0.5,0.15) -- ++(1,0)
                (-0.5,0.3) -- ++(1,0)
                (-0.5,0.5) -- ++(1,0)
                (-0.5,0.75) -- ++(1,0)
            ;
            \draw [grx,yshift=2.25cm]
                (-0.5,0.05) -- ++(1,0)
                (-0.5,0.15) -- ++(1,0)
                (-0.5,0.3) -- ++(1,0)
                (-0.5,0.5) -- ++(1,0)
                (-0.5,0.75) -- ++(1,0)
            ;
        \end{tikzpicture}
        \caption{Rotational and vibrational energy levels.}
        \label{fig:rovibrationalELevels}
    \end{figure}
    \begin{itemize}
        \item Each vibrational energy level has a number of smaller rotational states that are part of it.
    \end{itemize}
    \item Changes in rotational levels often accompany changes in vibrational levels.
    \begin{itemize}
        \item If we change the vibrational level by $\pm 1$, we have to change the rotational level (by $\pm 1$) as well.
    \end{itemize}
    \item IR/vibrational spectroscopy.
    \begin{figure}[H]
        \centering
        \begin{tikzpicture}[
            every node/.style={black}
        ]
            \small
            \draw [stealth-stealth] (0,2) -- node[rotate=90,above]{Absorption} (0,0) -- node[below]{$\si{\per\centi\meter}$} (5.5,0);
    
            \footnotesize
            \draw [rex,thick] (0,0.5)
                -- ++(1,0)
                -- ++(0.05,0.5) -- ++(0.05,-0.5) -- ++(0.05,0) -- ++(0.05,0.7) -- ++(0.05,-0.7) -- ++(0.05,0) -- ++(0.05,0.9) -- ++(0.05,-0.9) -- ++(0.05,0) -- ++(0.05,1) node[below=1cm]{$\Delta J=-1$} node[above]{$P$ branch} -- ++(0.05,-1) -- ++(0.05,0) -- ++(0.05,0.8) -- ++(0.05,-0.8) -- ++(0.05,0) -- ++(0.05,0.6) -- ++(0.05,-0.6) -- ++(0.05,0) -- ++(0.05,0.2) -- ++(0.05,-0.2)
                -- ++(0.5,0)
                -- ++(0.05,0.2) -- ++(0.05,-0.2) -- ++(0.05,0) -- ++(0.05,0.6) -- ++(0.05,-0.6) -- ++(0.05,0) -- ++(0.05,0.8) -- ++(0.05,-0.8) -- ++(0.05,0) -- ++(0.05,1) node[below=1cm]{$\Delta J=+1$} node[above]{$R$ branch} -- ++(0.05,-1) -- ++(0.05,0) -- ++(0.05,0.9) -- ++(0.05,-0.9) -- ++(0.05,0) -- ++(0.05,0.7) -- ++(0.05,-0.7) -- ++(0.05,0) -- ++(0.05,0.5) -- ++(0.05,-0.5)
                -- (5,0.5)
            ;
        \end{tikzpicture}
        \caption{IR/vibrational spectrum.}
        \label{fig:IRspectrum}
    \end{figure}
    \begin{itemize}
        \item The whole spectrum represents a rovibrational transition.
        \item The $P$ branch corresponds to when you go from a lower vibrational state to a higher vibrational state, \emph{but} you go from a higher rotational state to a lower rotational state.
        \item The $R$ branch corresponds to when you go from a lower vibrational state to a higher vibrational state, \emph{and} you go from a lower rotational state to a higher rotational state.
        \item The number of peaks in the $R$ and $P$ branches reveals which rotational states are occupied (although we could theoretically go on forever, realistically, only lower energy rotational states are occupied).
    \end{itemize}
    \item Electronic states.
    \begin{figure}[h!]
        \centering
        \begin{subfigure}[b]{0.49\linewidth}
            \centering
            \begin{tikzpicture}
                \small
                \draw [stealth-stealth] (0,4.5) -- node[left]{$E$} (0,0) -- node[below]{$R$} (6.5,0);
    
                \draw [blx,thick,name path=PES1] (0.5,2) to[out=-85,in=180,in looseness=0.7] (1.3,0.2) to[out=0,in=180,out looseness=0.5,in looseness=2] (4,1);
                \begin{scope}[
                    on background layer
                ]
                    \path [name path=ve1] (0,0.40) -- ++(6,0);
                    \path [name path=ve2] (0,0.57) -- ++(6,0);
                    \path [name path=ve3] (0,0.71) -- ++(6,0);
                    \path [name path=ve4] (0,0.82) -- ++(6,0);
                    \path [name path=ve5] (0,0.90) -- ++(6,0);
                    \path [name path=ve6] (0,0.95) -- ++(6,0);
                    \draw [grx,semithick,name intersections={of=PES1 and ve1}] (intersection-1) -- (intersection-2);
                    \draw [grx,semithick,name intersections={of=PES1 and ve2}] (intersection-1) -- (intersection-2);
                    \draw [grx,semithick,name intersections={of=PES1 and ve3}] (intersection-1) -- (intersection-2);
                    \draw [grx,semithick,name intersections={of=PES1 and ve4}] (intersection-1) -- (intersection-2);
                    \draw [grx,semithick,name intersections={of=PES1 and ve5}] (intersection-1) -- (intersection-2);
                    \draw [grx,semithick,name intersections={of=PES1 and ve6}] (intersection-1) -- (intersection-2);
                \end{scope}
    
                \draw [blx,thick,xshift=0.6cm,yshift=1cm,name path=PES2] (0.6,1.5) to[out=-85,in=180,in looseness=0.7] (1.3,0.2) to[out=0,in=180,out looseness=0.5,in looseness=2] (4,0.75);
                \begin{scope}[
                    on background layer,
                    yshift=1cm
                ]
                    \path [name path=ve1] (0,0.40) -- ++(6,0);
                    \path [name path=ve2] (0,0.57) -- ++(6,0);
                    \path [name path=ve3] (0,0.71) -- ++(6,0);
                    \draw [grx,semithick,name intersections={of=PES2 and ve1}] (intersection-1) -- (intersection-2);
                    \draw [grx,semithick,name intersections={of=PES2 and ve2}] (intersection-1) -- (intersection-2);
                    \draw [grx,semithick,name intersections={of=PES2 and ve3}] (intersection-1) -- (intersection-2);
                \end{scope}
    
                \draw [blx,thick,xshift=1cm,yshift=2cm,name path=PES3] (0.7,1.2) to[out=-85,in=180,in looseness=0.7] (1.3,0.2) to[out=0,in=180,out looseness=0.5,in looseness=2] (4,0.6);
                \begin{scope}[
                    on background layer,
                    yshift=2cm
                ]
                    \path [name path=ve1] (0,0.40) -- ++(6,0);
                    \path [name path=ve2] (0,0.57) -- ++(6,0);
                    \draw [grx,semithick,name intersections={of=PES3 and ve1}] (intersection-1) -- (intersection-2);
                    \draw [grx,semithick,name intersections={of=PES3 and ve2}] (intersection-1) -- (intersection-2);
                \end{scope}
    
                \draw [blx,thick,xshift=1.5cm,yshift=2.5cm] (0.7,1.8) to[out=-70,in=180] (4,0.6);
    
                \path [name path=trans] (1.3,0) -- ++(0,4);
                \draw [orx,semithick,-latex,shorten >=4pt,name intersections={of=trans and PES1,by=start},name intersections={of=trans and PES2,by=end}] (start) -- (end);
            \end{tikzpicture}
            \caption{Electronic states.}
            \label{fig:electronicStatesa}
        \end{subfigure}
        \begin{subfigure}[b]{0.49\linewidth}
            \centering
            \begin{tikzpicture}
                \small
                \draw [stealth-stealth] (0,4.5) -- node[left]{$E$} (0,0) -- node[below]{$R$} (6.5,0);
    
                \draw [blx,thick,name path=PES1] (0.5,2) to[out=-85,in=180,in looseness=0.7] (1.3,0.2) to[out=0,in=180,out looseness=0.5,in looseness=2] (4,1);
                \begin{scope}[
                    on background layer
                ]
                    \draw [grx,semithick] (1.05,0.40) -- ++(0.5,0);
                    \draw [grx,semithick] (1.05,0.57) -- ++(0.5,0);
                    \draw [grx,semithick] (1.05,0.71) -- ++(0.5,0);
                    \draw [grx,semithick] (1.05,0.82) -- ++(0.5,0);
                    \draw [grx,semithick] (1.05,0.90) -- ++(0.5,0);
                    \draw [grx,semithick] (1.05,0.95) -- ++(0.5,0);
                \end{scope}
    
                \draw [blx,thick,yshift=1cm,name path=PES2] (0.6,1.5) to[out=-85,in=180,in looseness=0.7] (1.3,0.2) to[out=0,in=180,out looseness=0.5,in looseness=2] (4,0.75);
                \begin{scope}[
                    on background layer,
                    yshift=1cm
                ]
                    \draw [grx,semithick] (1.05,0.40) -- ++(0.5,0);
                    \draw [grx,semithick] (1.05,0.57) -- ++(0.5,0);
                    \draw [grx,semithick] (1.05,0.71) -- ++(0.5,0);
                \end{scope}
    
                \draw [blx,thick,xshift=0.6cm,yshift=2cm,name path=PES3] (0.7,1.2) to[out=-85,in=180,in looseness=0.7] (1.3,0.2) to[out=0,in=180,out looseness=0.5,in looseness=2] (4,0.6);
                \begin{scope}[
                    on background layer,
                    xshift=0.6cm,yshift=2cm
                ]
                    \draw [grx,semithick] (1.05,0.40) -- ++(0.5,0);
                    \draw [grx,semithick] (1.05,0.57) -- ++(0.5,0);
                \end{scope}
    
                \draw [densely dashed,xshift=1.3cm,yshift=0.2cm] plot [domain=-1:1] (\x,{0.8*e^(-3*\x*\x)});
                \draw [densely dashed,xshift=1.3cm,yshift=1.3cm] plot [domain=-1:1] (\x,{0.6*e^(-3*\x*\x)});
                \draw [densely dashed,xshift=1.9cm,yshift=2.2cm] plot [domain=-1:1] (\x,{0.4*e^(-3*\x*\x)});
    
                \path [name path=trans] (1.3,0) -- ++(0,4);
                \path [name path=trans2] (1.4,0) -- ++(0,4);
                \draw [orx,semithick,-latex,name intersections={of=trans and PES1,by=start},name intersections={of=trans and PES2,by=end}] (start) -- (end);
                \draw [orx,semithick,-latex,shorten >=2pt,name intersections={of=trans2 and PES1,by=start},name intersections={of=trans2 and PES3,by=end}] (start) -- (end);
            \end{tikzpicture}
            \caption{Overlapping Gaussian wave packets.}
            \label{fig:electronicStatesb}
        \end{subfigure}
        \caption{Exciting between electronic states.}
        \label{fig:electronicStates}
    \end{figure}
    \begin{itemize}
        \item You can also do excitations from one potential energy surface to another.
        \item Recall that we have a Gaussian wave packet around the minimum energy on the PES.
        \item It's important that our Gaussian wave packets have some overlap for us to jump from one PES to the other.
        \item In Figure \ref{fig:electronicStatesb}, we'd expect a much higher probability of electronic transitions to the PES's with significant orbital overlap than from either to the third, where there is much less overlap.
    \end{itemize}
    \item \textbf{Franck-Condon Principle}\footnote{Developed at UChicago!}: The intensity of the transitions is proportional to the product of two harmonic oscillator wave functions --- $\psi_1$ from the one vibrational state and $\psi_2$ from the other vibrational state.
    \item In a polyatomic molecule, any bond can be approximated as a spring, and any spring can be approximated as a harmonic oscillator.
    \begin{itemize}
        \item Thus, any polyatomic molecule can be thought of as a bunch of harmonic oscillators linked together.
    \end{itemize}
    \item Normal modes and normal coordinates:
    \begin{itemize}
        \item Degrees of freedom: $3N$ ($x,y,z$ coordinates).
        \begin{itemize}
            \item There are translational, vibrational, and rotational DOFs.
            \item We always have 3 translational DOFs, 2 or 3 rotational DOFs (depending on whether or not the molecule is linear), and $3N-5$ or $3N-6$ vibrational DOFs (depending on whether or not the molecule is linear once again).
        \end{itemize}
    \end{itemize}
    \item Taking the right linear combinations of coupled oscillator stretches gives normal modes that are orthogonal to each other.
    \begin{itemize}
        \item Diagonalizing gives normal modes that are decoupled from each other.
    \end{itemize}
    \item Example (\ce{H2O}):
    \begin{figure}[h!]
        \centering
        \begin{subfigure}[b]{0.3\linewidth}
            \centering
            \begin{tikzpicture}[
                every node/.style={black}
            ]
                \draw [semithick,orx,-latex] (0,1) node[circle,fill,inner sep=1.5pt]{} -- ++(90:0.5);
                \draw [semithick,orx,-latex] (-1,0) node[circle,fill,inner sep=1.5pt]{} -- ++(-135:0.5);
                \draw [semithick,orx,-latex] (1,0) node[circle,fill,inner sep=1.5pt]{} -- ++(-45:0.5);
            \end{tikzpicture}
            \caption{Symmetric stretch.}
            \label{fig:H2ONormalModesa}
        \end{subfigure}
        \begin{subfigure}[b]{0.3\linewidth}
            \centering
            \begin{tikzpicture}[
                every node/.style={black}
            ]
                \draw [semithick,orx,-latex] (0,1) node[circle,fill,inner sep=1.5pt]{} -- ++(0:0.5);
                \draw [semithick,orx,-latex] (-1,0) node[circle,fill,inner sep=1.5pt]{} -- ++(-135:0.5);
                \draw [semithick,orx,-latex] (1,0) node[circle,fill,inner sep=1.5pt]{} -- ++(135:0.5);
            \end{tikzpicture}
            \caption{Asymmetric stretch.}
            \label{fig:H2ONormalModesb}
        \end{subfigure}
        \begin{subfigure}[b]{0.3\linewidth}
            \centering
            \begin{tikzpicture}[
                every node/.style={black}
            ]
                \draw [semithick,orx,-latex] (0,1) node[circle,fill,inner sep=1.5pt]{} -- ++(-90:0.5);
                \draw [semithick,orx,-latex] (-1,0) node[circle,fill,inner sep=1.5pt]{} to[out=180,in=-70] ++(135:0.5);
                \draw [semithick,orx,-latex] (1,0) node[circle,fill,inner sep=1.5pt]{} to[out=0,in=-110] ++(45:0.5);
            \end{tikzpicture}
            \caption{Bending mode.}
            \label{fig:H2ONormalModesc}
        \end{subfigure}
        \caption{Normal modes of \ce{H2O}.}
        \label{fig:H2ONormalModes}
    \end{figure}
    \begin{itemize}
        \item 3 vibrational modes.
        \item Symmetric stretch, asymmetric stretch, and bending mode.
        \item Their respective frequencies are $\nu_1=\SI{3650}{\per\centi\meter}$, $\nu_2=\SI{3760}{\per\centi\meter}$, and $\nu_3=\SI{1600}{\per\centi\meter}$.
    \end{itemize}
\end{itemize}



\section{Lasers}
\begin{itemize}
    \item \marginnote{12/3:}Consider a 2-level system.
    \begin{figure}[h!]
        \centering
        \begin{tikzpicture}[
            every node/.style=black
        ]
            \footnotesize
            \draw (0,1.5) --
                node[pos=0.4,above=1pt,circle,draw,inner sep=1.5pt]{}
                node[pos=0.5,above=1pt,circle,draw,inner sep=1.5pt]{}
                node[pos=0.6,above=1pt,circle,draw,inner sep=1.5pt]{}
            ++(2,0) node[right]{$E_2$};
            \draw (0,0) --
                node[pos=0.3,above=1pt,circle,draw,inner sep=1.5pt]{}
                node[pos=0.4,above=1pt,circle,draw,inner sep=1.5pt]{}
                node[pos=0.5,above=1pt,circle,draw,inner sep=1.5pt]{}
                node[pos=0.6,above=1pt,circle,draw,inner sep=1.5pt]{}
                node[pos=0.7,above=1pt,circle,draw,inner sep=1.5pt]{}
            ++(2,0) node[right]{$E_1$};
    
            \draw [grx,semithick,->,decorate,decoration={snake,post length=1,amplitude=1pt,segment length=6pt}] (-0.5,1) -- node[above right=-2pt]{$h\nu_{12}$} (0.3,0.1);
            \draw [grx,semithick,->] (1.8,0.1) to[bend right=20] ++(0,1.3);
        \end{tikzpicture}
        \caption{Two-level system.}
        \label{fig:2LevelSystem}
    \end{figure}
    \begin{itemize}
        \item Hit it with a photon of frequency $\nu_{12}=(E_2-E_1)/h$.
        \item Population conditions:
        \begin{align*}
            \dv{N_1}{t}+\dv{N_2}{t} &= 0&
            N_1+N_2 &= N_\text{tot}
        \end{align*}
        \item Radiation:
        \begin{itemize}
            \item Energy density $\rho$ in units of $\si{\joule\per\cubic\meter}$.
            \item Spectral energy density $\rho_\nu(\nu_{12})$ in units of energy density per unit frequency (i.e., $\si{\joule\second\per\cubic\meter}$).
        \end{itemize}
        \item Three types of changes tha can occur to our population.
        \begin{enumerate}
            \item Absorption: A photon comes in, gets absorbed by a molecule in the ground state, and gets excited to the excited state.
            \begin{equation*}
                -\dv{N_1}{t} = B_{12}\rho_\nu(\nu_{12})N_1(t)
            \end{equation*}
            \begin{itemize}
                \item This is very similar to chemical kinetics!
                \item We have a rate dependent on quantity. Our rate constant is $B_{12}$ and $\rho_\nu(\nu_{12})$ is like a photonic reagent interacting with the number of molecules in the state of interest.
                \item The whole thing is essentially a binary reaction.
                \item The $B$ is Einstein's coefficient.
            \end{itemize}
            \item Stimulated emission:
            \begin{equation*}
                -\dv{N_2}{t} = B_{21}\rho_\nu(\nu_{12})N_2(t)
            \end{equation*}
            \begin{itemize}
                \item The photon field can cause the molecule to emit a photon as well.
            \end{itemize}
            \item Spontaneous emission:
            \begin{equation*}
                -\dv{N_2}{t} = A_{21}N_2(t)
            \end{equation*}
            \begin{itemize}
                \item This looks like a unimolecular reaction.
            \end{itemize}
        \end{enumerate}
        \item This yields the overall rate equation
        \begin{equation*}
            -\dv{N_1}{t} = \dv{N_2}{t} = B_{12}\rho_\nu(\nu_{12})N_1(t)-A_{12}N_2(t)-B_{21}\rho_\nu(\nu_{12})N_2(t)
        \end{equation*}
    \end{itemize}
    \item We now get to lasers.
    \item Lasers work by \textbf{population inversion}.
    \item Consider a 3-level system.
    \begin{figure}[h!]
        \centering
        \begin{tikzpicture}[
            every node/.style=black
        ]
            \footnotesize
            \draw (0,2) --
                node[pos=0.25,above=1pt,circle,draw,inner sep=1.5pt]{}
                node[pos=0.30,above=1pt,circle,draw,inner sep=1.5pt]{}
                node[pos=0.35,above=1pt,circle,draw,inner sep=1.5pt]{}
            ++(4,0) node[right]{$E_3$};
            \draw (2,1.5) --
                node[pos=0.2,above=1pt,circle,draw,inner sep=1.5pt]{}
                node[pos=0.3,above=1pt,circle,draw,inner sep=1.5pt]{}
                node[pos=0.4,above=1pt,circle,draw,inner sep=1.5pt]{}
                node[pos=0.5,above=1pt,circle,draw,inner sep=1.5pt]{}
                node[pos=0.6,above=1pt,circle,draw,inner sep=1.5pt]{}
                node[pos=0.7,above=1pt,circle,draw,inner sep=1.5pt]{}
                node[pos=0.8,above=1pt,circle,draw,inner sep=1.5pt]{}
            ++(2,0) node[right]{$E_2$};
            \draw (0,0) --
                node[pos=0.20,above=1pt,circle,draw,inner sep=1.5pt]{}
                node[pos=0.25,above=1pt,circle,draw,inner sep=1.5pt]{}
                node[pos=0.30,above=1pt,circle,draw,inner sep=1.5pt]{}
                node[pos=0.35,above=1pt,circle,draw,inner sep=1.5pt]{}
                node[pos=0.40,above=1pt,circle,draw,inner sep=1.5pt]{}
            ++(4,0) node[right]{$E_1$};
    
            \draw [grx,semithick,->,shorten <=1pt,shorten >=1pt] (0.3,0) -- node[right]{pump} ++(0,2);
            \draw [gray,semithick,decorate,decoration={coil,segment length=5pt,aspect=0.2,post length=4pt,amplitude=2pt},->,shorten >=1pt] (2.1,2) -- node[left]{fast decay} ++(0,-0.5);
            \draw [rex,semithick,->,shorten <=1pt,shorten >=1pt] (3,1.5) -- node[right,text width=1cm]{laser action} ++(0,-1.5);
        \end{tikzpicture}
        \caption{Three-level system.}
        \label{fig:3LevelSystem}
    \end{figure}
    \begin{itemize}
        \item We first pump molecules from $E_1$ into $E_3$ (this is called the \textbf{pump phase}).
        \item Then there's a \textbf{fast decay} to the second excited state, followed by a delay before moving into the first state. Thus, you can get a lot of molecules in $E_2$.
        \item The \textbf{laser action} between states 2 and 1 occurs in the subsequent relaxation from the second to the first excited state.
        \item Pump phase:
        \begin{itemize}
            \item Sending in a photon with frequency $\nu_{13}=(E_3-E_1)/h$.
        \end{itemize}
    \end{itemize}
    \item Note that we can't get a laser built in only two levels.
    \begin{itemize}
        \item The same photons that build our population inversion eat away at it through stimulated emission.
        \item From the differential equation: Once the levels are equally occupied, the $B$ terms cancel; only spontaneous emission is left (which will also keep us below an inversion).
        \begin{itemize}
            \item $\dv*{N_2}{t}=0$ when $N_1(t)=N_2(t)$.
        \end{itemize}
    \end{itemize}
    \item As an aside, note that
    \begin{equation*}
        B_{12} = B_{21} = \frac{\pi}{3\hbar^3\epsilon_0}|\mu_{12}|^2
    \end{equation*}
    \begin{itemize}
        \item I.e., the rates of absorption depend on the transition dipole moment!
    \end{itemize}
    \item Example: Ruby laser.
    \begin{itemize}
        \item We have \ce{Al2O3} with \ce{Cr^2+} impurities absorbing radiation at $\lambda=\SI{5500}{\angstrom}$ (green is absorbed for red to be emitted; opposite colors on the color wheel).
        \item We pump with $h\nu_{13}$ (green photons) and emit with $h\nu_{12}$ (ruby photons).
        \begin{itemize}
            \item $\nu_{12}$ leads to $\lambda_{12}=\SI{6943}{\angstrom}$.
        \end{itemize}
        \item The ruby photons, as they're emitted can stimulate emission from more atoms in $E_2$ in an exponentially increasing (amplified) pattern.
        \begin{itemize}
            \item This is why the population inversion is key, i.e., to create an avalanche.
        \end{itemize}
        \item Lasers are important to create coherent radiation, which allow us to really carefully interrogate our molecules as chemists.
    \end{itemize}
    \item For a laser pointer, you can set up the photons to all come out in the same direction, but that's more of an engineering question.
    \item Relating $A$ and $B$.
    \begin{itemize}
        \item Solve our differential equation for
        \begin{equation*}
            \rho_\nu(\nu_{12}) = \frac{A_{21}}{(N_2/N_1)B_{12}-B_{21}}
        \end{equation*}
        \item Invoking the Boltzmann distribution and setting this equal to the Planck blackbody distribution yields
        \begin{equation*}
            \frac{8\pi h}{c^3}\frac{\nu_{12}^3}{\e[h\nu_{12}/k_\text{B}T]-1} = \frac{A_{21}}{B_{12}\e[h\nu_{12}/k_\text{B}T]-B_{12}}
        \end{equation*}
        \begin{itemize}
            \item So our first notions of quantum mechanics connect back to our most modern advances. The first line of the story is also the last.
        \end{itemize}
    \end{itemize}
\end{itemize}



\section{Chapter 13: Molecular Spectroscopy}
\emph{From \textcite{bib:McQuarrieSimon}.}
\begin{itemize}
    \item \marginnote{11/30:}\textbf{Spectroscopy}: The study of the interaction of electromagnetic radiation with atoms and molecules.
    \item "Electromagnetic radiation is customarily divided into different energy regions reflecting the different types of molecular processes that can be caused by such radiation" \parencite[495-96]{bib:McQuarrieSimon}.
    \item Vibrational selection rule: Transitions among vibrational levels resulting from the absorption of radiation have $\Delta v=\pm 1$ and have a dipole moment that varies during the vibration.
    \begin{itemize}
        \item For a harmonic oscillator, the spectrum consists of one line in the infrared region at the frequency $\nu_\text{obs}=\sqrt{k/\mu}/2\pi$.
    \end{itemize}
    \item \textbf{Vibrational term}: The vibrational energy of a molecule. \emph{Denoted by} $\bm{G(v)}$. \emph{Units} $\si{\per\centi\meter}$. \emph{Given by}
    \begin{equation*}
        G(v) = \frac{E_v}{hc}
    \end{equation*}
    where $E_v=(v+1/2)h\nu$ and $\nu=\sqrt{k/\mu}/2\pi$.
    \item Each energy $E_J=\hbar^2/(2I)\cdot J(J+1)$ of the rigid rotator is associated with degeneracy $g_J=2J+1$.
    \item \marginnote{12/1:}\textbf{Rotational term}: The rotational energy of a molecule. \emph{Denoted by} $\bm{F(J)}$. \emph{Units} $\si{\per\centi\meter}$. \emph{Given by}
    \begin{equation*}
        F(J) = \frac{E_J}{hc}
    \end{equation*}
    \item Rotational selection rule: Transitions among rotational levels resulting from the absorption of radiation have $\Delta J=\pm 1$ and have a permanent dipole moment.
    \item The rotational and vibrational energy of a diatomic molecule within the rigid rotator-harmonic oscillator approximation is
    \begin{equation*}
        \tilde{E}_{v,J} = G(v)+F(J) = (v+\tfrac{1}{2})\tilde{\nu}+\tilde{B}J(J+1)
    \end{equation*}
    where $\tilde{\nu}=\sqrt{k/\mu}/2\pi c$, $\tilde{B}=h/8\pi ^2cI$, and $v,J=0,1,2,\dots$.
    \begin{itemize}
        \item Since $\tilde{\nu}$ and $\tilde{B}$ are on the order of $\SI{e3}{\per\centi\meter}$ and $\SI{1}{\per\centi\meter}$, respectively, vibrational energy levels are usually spaced about 100 to 1000 times farther apart than rotational energy levels.
    \end{itemize}
    \item Rovibrational selection rule: $\Delta v=\pm 1$ and $\Delta J=\pm 1$.
    \item Observed frequencies for rovibrational transitions.
    \begin{itemize}
        \item $\Delta J=+1$.
        \begin{equation*}
            \tilde{\nu}_\text{obs} = \tilde{E}_{v+1,J+1}-\tilde{E}_{v,J} = \tilde{\nu}+2\tilde{B}(J+1)
        \end{equation*}
        \item $\Delta J=-1$.
        \begin{equation*}
            \tilde{\nu}_\text{obs} = \tilde{E}_{v+1,J-1}-\tilde{E}_{v,J} = \tilde{\nu}-2\tilde{B}J
        \end{equation*}
        \item Note that $J$ is the initial quantum number in both of the above equations.
    \end{itemize}
    \item \textbf{$\bm{R}$ branch}: The series toward the high-frequency side of the rotational-vibrational spectrum, due to rotational transitions with $\Delta J=+1$.
    \item \textbf{$\bm{P}$ branch}: The series toward the low frequency side of the rotational-vibrational spectrum, due to rotational transitions with $\Delta J=-1$.
    \item \textbf{Vibration-rotation interaction}: The decrease in $\tilde{B}$ as $v$ increases.
    \begin{itemize}
        \item Cause: As $v$ increases, $R_e$ (the equilibrium bond length) increases (Figure \ref{fig:parabolicPotentiala}). Thus $\tilde{B}\propto 1/R_e$ decreases.
    \end{itemize}
    \item Thus, for the $v=0\to 1$ transition, the frequencies of the $R$ and $P$ branches are truly given by
    \begin{align*}
        \tilde{\nu}_R &= E_{1,J+1}-E_{0,J} = \tilde{\nu}+2\tilde{B}_1+(3\tilde{B}_1-\tilde{B}_0)J+(\tilde{B}_1-\tilde{B}_0)J^2\\
        \tilde{\nu}_P &= E_{1,J-1}-E_{0,J} = \tilde{\nu}-(\tilde{B}_1+\tilde{B}_0)J+(\tilde{B}_1-\tilde{B}_0)J^2
    \end{align*}
    for $J=0,1,2,\dots$ and $J=1,2,3,\dots$, respectively.
    \begin{itemize}
        \item $\tilde{B}_v$ denotes the value of $\tilde{B}$ at vibrational energy level $v$.
        \item The above equations reduce to the original ones for $\tilde{B}_1=\tilde{B}_0$.
        \item Since $\tilde{B}_1<\tilde{B}_0$, "the spacing between the lines in the $R$ branch decreases and the spacing between the lines in the $P$ branch increases with increasing $J$" \parencite[502]{bib:McQuarrieSimon}.
    \end{itemize}
    \item The dependence of $\tilde{B}$ on $v$ is usually expressed as
    \begin{equation*}
        \tilde{B}_v = \tilde{B}_e-\tilde{\alpha}_e(v+\tfrac{1}{2})
    \end{equation*}
    \item Rotational lines are not exactly equally spaced since a chemical bond stretches due to the centrifugal force as a molecule rotates more and more energetically.
    \begin{itemize}
        \item This small deviation from the rigid rotator approximation can be treated by perturbation theory, resulting in
        \begin{equation*}
            F(J) = \tilde{B}J(J+1)-\tilde{D}J^2(J+1)^2
        \end{equation*}
        where $\tilde{D}$ is the \textbf{centrifugal distortion constant}.
        \item Modified frequencies of the absorption:
        \begin{equation*}
            \tilde{\nu} = F(J+1)-F(J) = 2\tilde{B}(J+1)-4\tilde{D}(J+1)^2
        \end{equation*}
        \item Rotational energy levels of a nonrigid rotator are spaced closer together than those of a rigid rotator.
    \end{itemize}
    \item The harmonic-oscillator approximation of a vibrating diatomic predicts only one line in its vibrational spectrum.
    \begin{itemize}
        \item However, while there is an experimental \textbf{fundamental}, there are also \textbf{overtones}.
    \end{itemize}
    \item \textbf{Fundamental}: The dominant line in the vibrational spectrum of a diatomic molecule.
    \item \textbf{Overtone}: A line in the vibrational spectrum of a diatomic molecules of weaker intensity than the fundamental, appearing at an almost integral multiple of the fundamental.
    \item Terms past the quadratic in the Taylor series expansion of the potential energy well $V(R)$ about the equilibrium bond length $R_e$ can be accounted for by applying perturbation theory to the harmonic oscillator approximation.
    \begin{itemize}
        \item Doing so gives rise to the vibrational term
        \begin{equation*}
            G(v) = \tilde{\nu}_e(v+\tfrac{1}{2})-\tilde{x}_e\tilde{\nu}_e(v+\tfrac{1}{2})^2+\cdots
        \end{equation*}
        for $v=0,1,2,\dots$, where $\tilde{x}_e$ is the \textbf{anharmonicity constant}.
        \item The separation between the energy levels of an anharmonic oscillator decreases with increasing $v$.
        \item "The selection rule for an anharmonic oscillator is that $\Delta v$ can have any integral value, although the intensities of the $\Delta v=\pm 2,\pm 3,\dots$ transitions are much less than for the $\Delta v=\pm 1$ transitions" \parencite[506]{bib:McQuarrieSimon}.
        \item Since most diatomics are in the ground vibrational state at room temperature, the frequencies of the observed $0\to v$ transitions will be
        \begin{equation*}
            \tilde{\nu}_\text{obs} = G(v)-G(0) = \tilde{\nu}_ev-\tilde{x}_e\tilde{\nu}_ev(v+1)
        \end{equation*}
        for $v=1,2,\dots$.
    \end{itemize}
    \item "Just as rotational transitions accompany vibrational transitions, both rotational and vibrational transitions accompany electronic transitions" \parencite[507]{bib:McQuarrieSimon}.
    \item The BO approximation allows us to separate the electronic energy from the vibrational-rotational energy since vibration and rotation, overall, are nuclear-motion phenomena.
    \item \textbf{Vibronic transition}: A vibrational transition in an electronic spectrum.
    \item Vibronic selection rule: $\Delta v$ may take on any integral value.
    \begin{itemize}
        \item Rotational transitions in electronic spectra are not considered because they are so much smaller.
    \end{itemize}
    \item Vibronic transitions usually originate from the $v=0$ vibrational state, yielding predicted frequencies
    \begin{equation*}
        \tilde{\nu}_\text{obs} = \tilde{T}_e+(\tfrac{1}{2}\tilde{\nu}_e'-\tfrac{1}{4}\tilde{x}_e'\tilde{\nu}_e')-(\tfrac{1}{2}\tilde{\nu}_e''-\tfrac{1}{4}\tilde{x}_e''\tilde{\nu}_e'')+\tilde{\nu}_e'v'-\tilde{x}_e'\tilde{\nu}_e'v'(v'+1)
    \end{equation*}
    where $\tilde{T}_e$ is the difference in energies of the minima of the two electronic potential energy curves in wave numbers, and the single and double primes indicate the upper and lower energy states, respectively.
    \begin{figure}[h!]
        \centering
        \begin{tikzpicture}[
            every node/.style=black
        ]
            \small
            \draw [stealth-stealth] (0,12.5) -- node[left]{$E$} (0,0) -- node[below]{$R$} (8.5,0);
    
            \footnotesize
            \draw [blx,thick,name path=PES1] (0.7,6) to[out=-85,in=180,in looseness=0.5] (3,0.2) to[out=0,in=180,out looseness=0.7,in looseness=1.5] (8,4);
            \begin{scope}[
                on background layer
            ]
                \path [name path=ve01] (0,0.50) -- ++(8,0);
                \path [name path=ve02] (0,0.92) -- ++(8,0);
                \path [name path=ve03] (0,1.32) -- ++(8,0);
                \path [name path=ve04] (0,1.70) -- ++(8,0);
                \path [name path=ve05] (0,2.06) -- ++(8,0);
                \path [name path=ve06] (0,2.40) -- ++(8,0);
                \path [name path=ve07] (0,2.72) -- ++(8,0);
                \path [name path=ve08] (0,3.02) -- ++(8,0);
                \path [name path=ve09] (0,3.30) -- ++(8,0);
                \path [name path=ve10] (0,3.56) -- ++(8,0);
                \path [name path=ve11] (0,3.80) -- ++(8,0);
                \draw [grx,semithick,name intersections={of=PES1 and ve01}] (intersection-1) node[left=1pt]{0} -- (intersection-2) coordinate (v00b);
                \draw [grx,semithick,name intersections={of=PES1 and ve02}] (intersection-1) node[left]{1} -- (intersection-2);
                \draw [grx,semithick,name intersections={of=PES1 and ve03}] (intersection-1) node[left]{2} -- (intersection-2);
                \draw [grx,semithick,name intersections={of=PES1 and ve04}] (intersection-1) node[left]{3} -- (intersection-2);
                \draw [grx,semithick,name intersections={of=PES1 and ve05}] (intersection-1) node[left]{$v''$} -- (intersection-2);
                \draw [grx,semithick,name intersections={of=PES1 and ve06}] (intersection-1) -- (intersection-2);
                \draw [grx,semithick,name intersections={of=PES1 and ve07}] (intersection-1) -- (intersection-2);
                \draw [grx,semithick,name intersections={of=PES1 and ve08}] (intersection-1) -- (intersection-2);
                \draw [grx,semithick,name intersections={of=PES1 and ve09}] (intersection-1) -- (intersection-2);
                \draw [grx,semithick,name intersections={of=PES1 and ve10}] (intersection-1) -- (intersection-2);
                \draw [grx,semithick,name intersections={of=PES1 and ve11}] (intersection-1) -- (intersection-2);
            \end{scope}
    
            \draw [blx,thick,yshift=6cm,name path=PES2] (0.7,6) to[out=-85,in=180,in looseness=0.5] (3,0.2) to[out=0,in=180,out looseness=0.7,in looseness=1.5] (8,4);
            \begin{scope}[
                on background layer,
                yshift=6cm
            ]
                \path [name path=ve01] (0,0.50) -- ++(8,0);
                \path [name path=ve02] (0,0.92) -- ++(8,0);
                \path [name path=ve03] (0,1.31) -- ++(8,0);
                \path [name path=ve04] (0,1.67) -- ++(8,0);
                \path [name path=ve05] (0,2.00) -- ++(8,0);
                \path [name path=ve06] (0,2.30) -- ++(8,0);
                \path [name path=ve07] (0,2.57) -- ++(8,0);
                \path [name path=ve08] (0,2.81) -- ++(8,0);
                \path [name path=ve09] (0,3.02) -- ++(8,0);
                \path [name path=ve10] (0,3.20) -- ++(8,0);
                \path [name path=ve11] (0,3.35) -- ++(8,0);
                \draw [grx,semithick,name intersections={of=PES2 and ve01}] (intersection-1) node[left=1pt]{0} -- (intersection-2) coordinate (v00t);
                \draw [grx,semithick,name intersections={of=PES2 and ve02}] (intersection-1) node[left]{1} -- (intersection-2);
                \draw [grx,semithick,name intersections={of=PES2 and ve03}] (intersection-1) node[left]{2} -- (intersection-2);
                \draw [grx,semithick,name intersections={of=PES2 and ve04}] (intersection-1) node[left]{3} -- (intersection-2);
                \draw [grx,semithick,name intersections={of=PES2 and ve05}] (intersection-1) node[left]{$v'$} -- (intersection-2);
                \draw [grx,semithick,name intersections={of=PES2 and ve06}] (intersection-1) -- (intersection-2);
                \draw [grx,semithick,name intersections={of=PES2 and ve07}] (intersection-1) -- (intersection-2);
                \draw [grx,semithick,name intersections={of=PES2 and ve08}] (intersection-1) -- (intersection-2);
                \draw [grx,semithick,name intersections={of=PES2 and ve09}] (intersection-1) -- (intersection-2);
                \draw [grx,semithick,name intersections={of=PES2 and ve10}] (intersection-1) -- (intersection-2);
                \draw [grx,semithick,name intersections={of=PES2 and ve11}] (intersection-1) -- (intersection-2);
            \end{scope}
    
            \small
            \draw [very thin] (v00t) -- (v00t -| 4.7,0);
            \draw [very thin] (v00b) -- (v00b -| 4.7,0);
            \draw [<->,shorten <=1pt,shorten >=1pt] (v00b -| 4.5,0) -- node[pos=0.7,fill=white,inner sep=1.5pt]{$\tilde{\nu}_{0\,0}$} (v00t -| 4.5,0);
            \draw [very thin] (3,6.2) -- (6.7,6.2);
            \draw [very thin] (3,0.2) -- (6.7,0.2);
            \draw [<->,shorten <=1pt,shorten >=1pt] (6.5,0.2) -- node[fill=white,inner sep=1.5pt]{$\tilde{T}_e$} (6.5,6.2);
            \path [name path=D0] (7,0) -- ++(0,6);
            \draw [very thin] (v00b -| 6.8,0) -- ++(0.4,0);
            \draw [<->,shorten <=1pt,shorten >=1pt,name intersections={of=PES1 and D0}] (v00b -| 7,0) -- node[fill=white,inner sep=1.5pt]{$D_0$} (intersection-1);
            \path [name path=De] (7.5,0) -- ++(0,6);
            \draw [very thin] (7.3,0.2) -- ++(0.4,0);
            \draw [<->,shorten <=1pt,shorten >=1pt,name intersections={of=PES1 and De}] (7.5,0.2) -- node[fill=white,inner sep=1.5pt]{$D_e$} (intersection-1);
        \end{tikzpicture}
        \caption{The quantities describing an electronic transition.}
        \label{fig:electronicQuantities}
    \end{figure}
    \item $\bm{D_e}$: The difference in energy between the minimum of the potential energy curve and the dissociated atoms.
    \item $\bm{D_0}$: The corresponding dissociation energy from the ground-vibrational level.
    \item We thus have that $D_e=D_0+\frac{1}{2}h\nu$ in the harmonic oscillator approximation and $D_e=D_0+\frac{1}{2}h(\nu_e-\frac{1}{2}x_e\nu_e)$ in the anharmonic oscillator approximation.
    \item Since the nuclei do not move appreciably during an electronic transition, such transitions can be depicted as vertical lines in the energy diagram, as we have done thus far.
    \item \textcite{bib:McQuarrieSimon} covers the Franck-Condon principle.
    \item Consider a rigid body, a polyatomic molecule modeled as a rigid network of $N$ atoms.
    \item \textbf{Moments of inertia} (of a rigid body): Given a Cartesian coordinate system, the following values. \emph{Denoted by} $\bm{I_{xx}},\bm{I_{yy}},\bm{I_{zz}}$. \emph{Given by}
    \begingroup
    \allowdisplaybreaks
    \begin{align*}
        I_{xx} &= \sum_{j=1}^Nm_j[(y_j-y_{cm})^2+(z_j-z_{cm})^2]\\
        I_{yy} &= \sum_{j=1}^Nm_j[(x_j-x_{cm})^2+(z_j-z_{cm})^2]\\
        I_{zz} &= \sum_{j=1}^Nm_j[(x_j-x_{cm})^2+(y_j-y_{cm})^2]
    \end{align*}
    \endgroup
    where $m_j$ is the mass of the $j^\text{th}$ atom situated at the point $(x_j,y_j,z_j)$ and $(x_{cm},y_{cm},z_{cm})$ are the coordinates of the center of mass of the rigid body.
    \item \textbf{Products of inertia} (of a rigid body): The quantities of the form
    \begin{equation*}
        I_{xy} = -\sum_{j=1}^Nm_j(x_j-x_{cm})(y_j-y_{cm})
    \end{equation*}
    \item \textbf{Principal axes} (of a rigid body): The particular set of Cartesian coordinates $X,Y,Z$ passing through the center of mass of a rigid body such that all of the products of inertia vanish.
    \item \textbf{Principal moments of inertia} (of a rigid body): The moments of inertia of a rigid body about the principal axes. \emph{Denoted by} $\bm{I_A},\bm{I_B},\bm{I_C}$ where $I_A\leq I_B\leq I_C$.
    \begin{itemize}
        \item Usually given in terms of rotational constants in units of reciprocal centimeters, e.g., $\tilde{A}=h/8\pi^2cI_A$.
    \end{itemize}
    \item \textbf{Spherical top}: A rigid body with all three principal moments of inertial equal.
    \begin{itemize}
        \item For example, \ce{CH4} and \ce{SF6}.
    \end{itemize}
    \item \textbf{Symmetric top}: A rigid body with two principal moments of inertial equal.
    \begin{itemize}
        \item For example, \ce{NH3} and \ce{C6H6}.
    \end{itemize}
    \item \textbf{Asymmetric top}: A rigid body with all three principal moments of inertial unequal.
    \begin{itemize}
        \item For example, \ce{H2O}.
    \end{itemize}
    \item The quantum-mechanical problem of a spherical top can be solved exactly, yielding energy levels $F(J)=\tilde{B}J(J+1)$ for $J=0,1,2,\dots$ (the same as for a linear molecule) with respective degeneracies $g_J=(2J+1)^2$.
    \item Since spherical tops cannot have a permanent dipole moment, they do not have pure rotational spectra.
    \item The quantum-mechanical problem of a symmetric top can also be solved exactly.
    \item \textbf{Oblate symmetric top}: A symmetric top with unique moment of inertia larger than the two equal ones.
    \begin{itemize}
        \item For example, \ce{C6H6} or an O-ring.
        \item Energy levels $F(J,K)=\tilde{B}J(J+1)+(\tilde{C}-\tilde{B})K^2$ for $J=0,1,2,\dots$, $K=0,\pm 1,\pm 2,\dots,\pm J$, and degeneracy $g_{JK}=2J+1$.
        \begin{itemize}
            \item $J$ is a measure of the total rotational angular momentum of the molecule.
            \item $K$ is a measure of the component of the rotational angular momentum along the unique axis of the symmetric top.
        \end{itemize}
    \end{itemize}
    \item \textbf{Prolate symmetric top}: A symmetric top with unique moment of inertia smaller than the two equal ones.
    \begin{itemize}
        \item For example, \ce{CH3Cl} or a cigar.
        \item Energy levels $F(J,K)=\tilde{B}J(J+1)+(\tilde{A}-\tilde{B})K^2$ with the same quantum numbers and degeneracy as in the oblate case.
    \end{itemize}
    \item Symmetric top molecule selection rule: The dipole moment must be directed along the axis of symmetry, and then we may have $\Delta J=0,\pm 1$, $\Delta K=0$ for $K\neq 0$ and $\Delta J=\pm 1$, $\Delta J=0$ for $K=0$.
    \item Centrifugal distortion effects are larger for larger molecules.
    \item To understand the vibrational spectra of polyatomic molecules in terms of the harmonic-oscillator approximation, we introduce normal coordinates.
    \item \textcite{bib:McQuarrieSimon} reviews degrees of freedom.
    \begin{itemize}
        \item A complete specification of a molecule containing $N$ nuclei in space requires $3N$ coordinates.
        \item However, using some to specify the overall molecules position and orientation in space leaves the rest of the degrees of freedom available to describe vibration.
    \end{itemize}
    \item "In the absence of external fields, the energy of a molecule does not depend upon the position of its center of mass or its orientation" \parencite[519]{bib:McQuarrieSimon}.
    \item $\bm{N_\textbf{vib}}$: The number of vibrational degrees of freedom.
    \item Thus, the potential energy is solely a function of the $N_\text{vib}$ vibrational coordinates. Letting the displacements about the equilibrium values of these coordinates be $q_1,\dots,q_{N_\text{vib}}$, we have that
    \begin{equation*}
        \Delta V = V(q_1,\dots,q_{N_\text{vib}})-V(0,\dots,0) = \frac{1}{2}\sum_{i=1}^{N_\text{vib}}\sum_{j=1}^{N_\text{vib}}\left( \pdv{V}{q_i}{q_j} \right)q_iq_j+\cdots
    \end{equation*}
    where the $\cdots$ terms are anharmonic.
    \item The cross terms in the above expression make the solution to the corresponding Schr\"{o}dinger equation quite tedious, but a theorem of classical mechanics and a straightforward procedure using matrix algebra allows us to find a new set of coordinates $\{Q_j\}$ called \textbf{normal coordinates} or \textbf{normal modes} such that
    \begin{equation*}
        \Delta V = \frac{1}{2}\sum_{j=1}^{N_\text{vib}}F_jQ_j^2
    \end{equation*}
    \item It follows that the vibrational Schr\"{o}dinger equation $\hat{H}_\text{vib}\psi_\text{vib}=E_\text{vib}\psi_\text{vib}$ is separable and defined by
    \begin{gather*}
        \hat{H}_\text{vib} = \sum_{j=1}^{N_\text{vib}}\left( -\frac{\hbar^2}{2\mu_j}\dv[2]{Q_j}+\frac{1}{2}F_jQ_j^2 \right)\\
        \psi_\text{vib}(Q_1,\dots,Q_{N_\text{vib}}) = \psi_\text{vib,1}(Q_1)\cdots\psi_{\text{vib},N_\text{vib}}(Q_{N_\text{vib}})\\
        E_\text{vib} = \sum_{j=1}^{N_\text{vib}}h\nu_j(v_j+\tfrac{1}{2})
    \end{gather*}
    \item Thus, the vibrational motion of a polyatomic molecule appears as $N_\text{vib}$ independent harmonic oscillators, each with their own characteristic fundamental frequency $\nu_j$.
    \item Vibrational absorption spectroscopy selection rule: The dipole moment of the molecule must vary during the normal mode.
    \item \textbf{Infrared active} (normal mode): A normal mode such that the dipole moment of the molecule does vary during the prescribed motion.
    \item \textbf{Infrared inactive} (normal mode): A normal mode such that the dipole moment of the molecule does not vary during the prescribed motion.
    \item \textcite{bib:McQuarrieSimon} analyzes the normal modes of \ce{CO2}, \ce{H2CO}, and \ce{CH3Cl}.
    \item \textbf{Parallel band}: An absorption band corresponding to a normal mode with dipole moment oscillating parallel to the molecular axis.
    \begin{itemize}
        \item Governed by the selection rule $\Delta v=+1$, $\Delta J=\pm 1$, just like a diatomic molecules.
        \item Generates a vibration-rotation spectrum consisting of a $P$ branch and an $R$ branch.
    \end{itemize}
    \item The case of a dipole moment oscillating perpendicular to the molecular axis.
    \begin{itemize}
        \item Governed by the selection rule $\Delta v=+1$, $\Delta J=0,\pm 1$.
        \item The band due to $\Delta J=0$ is called the \textbf{$\bm{Q}$ branch}, and is centered between the $P$ and $R$ branches.
    \end{itemize}
    \item Identifying normal coordinates with irreducible representations.
    \begin{figure}[h!]
        \centering
        \small
        \begin{align*}
            \hat{E}\left( 
                \tikz[every node/.style={black},baseline={(0,0.1)}]{
                    \draw [semithick,orx,-latex] (0,0.5) node[circle,fill,inner sep=1.5pt]{} -- ++(0:0.4);
                    \draw [semithick,orx,-latex] (-0.5,0) node[circle,fill,inner sep=1.5pt]{} -- ++(-135:0.4);
                    \draw [semithick,orx,-latex] (0.5,0) node[circle,fill,inner sep=1.5pt]{} -- ++(135:0.4);
                }
            \right)\quad &=\quad \tikz[every node/.style={black},baseline={(0,0.1)}]{
                \draw [semithick,orx,-latex] (0,0.5) node[circle,fill,inner sep=1.5pt]{} -- ++(0:0.4);
                \draw [semithick,orx,-latex] (-0.5,0) node[circle,fill,inner sep=1.5pt]{} -- ++(-135:0.4);
                \draw [semithick,orx,-latex] (0.5,0) node[circle,fill,inner sep=1.5pt]{} -- ++(135:0.4);
            }&
            \hat{\sigma}_v\left( 
                \tikz[every node/.style={black},baseline={(0,0.1)}]{
                    \draw [semithick,orx,-latex] (0,0.5) node[circle,fill,inner sep=1.5pt]{} -- ++(0:0.4);
                    \draw [semithick,orx,-latex] (-0.5,0) node[circle,fill,inner sep=1.5pt]{} -- ++(-135:0.4);
                    \draw [semithick,orx,-latex] (0.5,0) node[circle,fill,inner sep=1.5pt]{} -- ++(135:0.4);
                }
            \right)\quad &=\quad \tikz[every node/.style={black},baseline={(0,0.1)}]{
                \draw [semithick,orx,-latex] (0,0.5) node[circle,fill,inner sep=1.5pt]{} -- ++(180:0.4);
                \draw [semithick,orx,-latex] (-0.5,0) node[circle,fill,inner sep=1.5pt]{} -- ++(45:0.4);
                \draw [semithick,orx,-latex] (0.5,0) node[circle,fill,inner sep=1.5pt]{} -- ++(-45:0.4);
            }\\[5mm]
            \hat{C}_2\left( 
                \tikz[every node/.style={black},baseline={(0,0.1)}]{
                    \draw [semithick,orx,-latex] (0,0.5) node[circle,fill,inner sep=1.5pt]{} -- ++(0:0.4);
                    \draw [semithick,orx,-latex] (-0.5,0) node[circle,fill,inner sep=1.5pt]{} -- ++(-135:0.4);
                    \draw [semithick,orx,-latex] (0.5,0) node[circle,fill,inner sep=1.5pt]{} -- ++(135:0.4);
                }
            \right)\quad &=\quad \tikz[every node/.style={black},baseline={(0,0.1)}]{
                \draw [semithick,orx,-latex] (0,0.5) node[circle,fill,inner sep=1.5pt]{} -- ++(180:0.4);
                \draw [semithick,orx,-latex] (-0.5,0) node[circle,fill,inner sep=1.5pt]{} -- ++(45:0.4);
                \draw [semithick,orx,-latex] (0.5,0) node[circle,fill,inner sep=1.5pt]{} -- ++(-45:0.4);
            }&
            \hat{\sigma}_v'\left( 
                \tikz[every node/.style={black},baseline={(0,0.1)}]{
                    \draw [semithick,orx,-latex] (0,0.5) node[circle,fill,inner sep=1.5pt]{} -- ++(0:0.4);
                    \draw [semithick,orx,-latex] (-0.5,0) node[circle,fill,inner sep=1.5pt]{} -- ++(-135:0.4);
                    \draw [semithick,orx,-latex] (0.5,0) node[circle,fill,inner sep=1.5pt]{} -- ++(135:0.4);
                }
            \right)\quad &=\quad \tikz[every node/.style={black},baseline={(0,0.1)}]{
                \draw [semithick,orx,-latex] (0,0.5) node[circle,fill,inner sep=1.5pt]{} -- ++(0:0.4);
                \draw [semithick,orx,-latex] (-0.5,0) node[circle,fill,inner sep=1.5pt]{} -- ++(-135:0.4);
                \draw [semithick,orx,-latex] (0.5,0) node[circle,fill,inner sep=1.5pt]{} -- ++(135:0.4);
            }
        \end{align*}
        \caption{The asymmetric stretch normal mode under the operations of the $\Cbf_{2v}$ point group.}
        \label{fig:asymmetricStretchB2}
    \end{figure}
    \begin{itemize}
        \item Consider, for example, \ce{H2O}, belonging to the $\Cbf_{2v}$ point group.
        \item Let the asymmetric stretch normal coordinate be $Q_{as}$.
        \item Then by Figure \ref{fig:asymmetricStretchB2}, we have that
        \begin{align*}
            \hat{E}Q_{as} &= Q_{as}&
            \hat{C}_2Q_{as} &= -Q_{as}&
            \hat{\sigma}_vQ_{as} &= -Q_{as}&
            \hat{\sigma}_v'Q_{as} &= Q_{as}
        \end{align*}
        so $Q_{as}$ belongs to $B_2$.
    \end{itemize}
    \item \textcite{bib:McQuarrieSimon} also derives the relevant irreducible representations as in \textcite[40]{bib:IChemNotes}.
    \item We now begin the derivation of selection rules.
    \item Since we are considering \emph{transitions}, we will need the time-dependent Schr\"{o}dinger equation.
    \begin{itemize}
        \item Stationary states, such as those wave functions we have considered thus far, only pertain to isolated systems with Hamiltonians that do not depend on time.
        \item As such, to consider transitions between states, we will be working not just with the time-dependent Schr\"{o}dinger equation but also with a time-dependent Hamiltonian.
    \end{itemize}
    \item In particular, let the molecule interact with electromagnetic field
    \begin{equation*}
        \mathbf{E} = \mathbf{E}_0\cos 2\pi\nu t
    \end{equation*}
    where $\nu$ is the frequency of the radiation incident on the molecule and $\mathbf{E}_0$ is the electric field vector.
    \item It follows by Problem \ref{prb:13-49} that the Hamiltonian operator for the interaction of the electric field with the molecule is
    \begin{equation*}
        \hat{H}^{(1)} = -\bm{\mu}\cdot\mathbf{E} = -\bm{\mu}\cdot\mathbf{E}_0\cos 2\pi\nu t
    \end{equation*}
    \item Thus, the overall problem is to solve
    \begin{equation*}
        \hat{H}\Psi = i\hbar\dv{\Psi}{t}
    \end{equation*}
    where $\hat{H}=\hat{H}^{(0)}+\hat{H}^{(1)}=\hat{H}^{(0)}-\bm{\mu}\cdot\mathbf{E}_0\cos 2\pi\nu t$ and $\hat{H}^{(0)}$ is the Hamiltonian of the isolated molecule.
    \item Treating $\hat{H}^{(1)}$ as a small perturbation, we can solve the above with \textbf{time-dependent perturbation theory}, an extension of time-independent perturbation theory worked through as follows.
    \begin{itemize}
        \item For simplicity, we consider only a two-state system (in spite of the fact that an isolated molecule generally has an infinite number of stationary states).
    \end{itemize}
    \item In such a system, the solutions to
    \begin{equation*}
        \hat{H}^{(0)}\Psi = i\hbar\dv{\Psi}{t}
    \end{equation*}
    are
    \begin{align*}
        \Psi_1(t) &= \psi_1\e[-iE_1t/\hbar]&
        \Psi_2(t) &= \psi_2\e[-iE_2t/\hbar]
    \end{align*}
    where $\psi_1,\psi_2$ are the two stationary states.
    \item Let the system initially be in state 1, and assume that the overall solution $\Psi(t)$ is a linear combination of $\Psi_1(t),\Psi_2(t)$ that evolves over time. In particular, assume
    \begin{equation*}
        \Psi(t) = a_1(t)\Psi_1(t)+a_2(t)\Psi_2(t)
    \end{equation*}
    and let our initial conditions be $a_1(t)=1$ and $a_2(t)=0$.
    \begin{itemize}
        \item Recall that $a_i^*a_i$ gives the probability that the molecule is in state $i$.
    \end{itemize}
    \item Substituting our wave function into the full time-dependent Schr\"{o}dinger equation gives
    \begin{align*}
        (\hat{H}^{(0)}+\hat{H}^{(1)})\Psi(t) &= i\hbar\dv{\Psi}{t}\\
        a_1(t)\hat{H}^{(0)}\Psi_1+a_2(t)\hat{H}^{(0)}\Psi_2+a_1(t)\hat{H}^{(1)}\Psi_1+a_2(t)\hat{H}^{(1)}\Psi_2 &= a_1(t)i\hbar\dv{\Psi_1}{t}+i\hbar\Psi_1\dv{a_1}{t}+a_2(t)i\hbar\dv{\Psi_2}{t}+i\hbar\Psi_2\dv{a_2}{t}\\
        a_1(t)\hat{H}^{(1)}\Psi_1+a_2(t)\hat{H}^{(1)}\Psi_2 &= i\hbar\Psi_1\dv{a_1}{t}+i\hbar\Psi_2\dv{a_2}{t}
    \end{align*}
    where we cancel terms that are equal by the TD SE in going from the second to the third equality.
    \item We now multiply through by $\psi_2^*$, integrate over all space, and simplify.
    \begin{align*}
        \psi_2^*a_1(t)\hat{H}^{(1)}\Psi_1+\psi_2^*a_2(t)\hat{H}^{(1)}\Psi_2 &= \psi_2^*i\hbar\Psi_1\dv{a_1}{t}+\psi_2^*i\hbar\Psi_2\dv{a_2}{t}\\
        a_1(t)\int\psi_2^*\hat{H}^{(1)}\Psi_1\dd{\tau}+a_2(t)\int\psi_2^*\hat{H}^{(1)}\Psi_2\dd{\tau} &= i\hbar\dv{a_1}{t}\int\psi_2^*\psi_1\e[-iE_1t/\hbar]\dd{\tau}+i\hbar\dv{a_2}{t}\int\psi_2^*\psi_2\e[-iE_2t/\hbar]\dd{\tau}\\
        a_1(t)\int\psi_2^*\hat{H}^{(1)}\Psi_1\dd{\tau}+a_2(t)\int\psi_2^*\hat{H}^{(1)}\Psi_2\dd{\tau} &= i\hbar\e[-iE_1t/\hbar]\dv{a_1}{t}\int\psi_2^*\psi_1\dd{\tau}+i\hbar\e[-iE_2t/\hbar]\dv{a_2}{t}\int\psi_2^*\psi_2\dd{\tau}\\
        a_1(t)\int\psi_2^*\hat{H}^{(1)}\Psi_1\dd{\tau}+a_2(t)\int\psi_2^*\hat{H}^{(1)}\Psi_2\dd{\tau} &= i\hbar\e[-iE_2t/\hbar]\dv{a_2}{t}
    \end{align*}
    \item Solve for $i\hbar\dv*{a_2}{t}$ and simplify.
    \begin{align*}
        i\hbar\dv{a_2}{t} &= a_1(t)\e[iE_2t/\hbar]\int\psi_2^*\hat{H}^{(1)}\psi_1\e[-iE_1t/\hbar]\dd{\tau}+a_2(t)\e[iE_2t/\hbar]\int\psi_2^*\hat{H}^{(1)}\psi_2\e[-iE_2t/\hbar]\dd{\tau}\\
        &= a_1(t)\e[-i(E_1-E_2)t/\hbar]\int\psi_2^*\hat{H}^{(1)}\psi_1\dd{\tau}+a_2(t)\int\psi_2^*\hat{H}^{(1)}\psi_2\dd{\tau}
    \end{align*}
    \item Since $\hat{H}^{(1)}$ is a small perturbation, the initial values corresponding to the TD SE may be approximated as identical to those corresponding to the TI SE. As such, substitute $a_1(0)=1$ and $a_2(0)=0$ into the above to obtain
    \begin{equation*}
        i\hbar\dv{a_2}{t} = \e[-i(E_1-E_2)t/\hbar]\int\psi_2^*\hat{H}^{(1)}\psi_1\dd{\tau}
    \end{equation*}
    \item Take the electric field to be in the $z$-direction. Then
    \begin{align*}
        \hat{H}^{(1)} &= -\mu_zE_{0z}\cos 2\pi\nu t\\
        &= -\frac{\mu_zE_{0z}}{2}(\e[i2\pi\nu t]+\e[-i2\pi\nu t])
    \end{align*}
    \item Substituting this into the above then yields
    \begin{align*}
        i\hbar\dv{a_2}{t} &= \e[-i(E_1-E_2)t/\hbar]\int\psi_2^*\cdot -\frac{\mu_zE_{0z}}{2}(\e[i2\pi\nu t]+\e[-i2\pi\nu t])\psi_1\dd{\tau}\\
        &= -\frac{E_{0z}}{2}\e[i(E_2-E_1)t/\hbar](\e[ih\nu t2\pi/h]+\e[-ih\nu t2\pi/h])\int\psi_2^*\mu_z\psi_1\dd{\tau}\\
        \dv{a_2}{t} &= -\frac{1}{2i\hbar}\int\psi_2^*\mu_z\psi_1\dd{\tau}E_{0z}(\e[i(E_2-E_1+h\nu)t/\hbar]+\e[i(E_2-E_1-h\nu)t/\hbar])\\
        &\propto (\mu_z)_{12}E_{0z}(\e[i(E_2-E_1+h\nu)t/\hbar]+\e[i(E_2-E_1-h\nu)t/\hbar])
    \end{align*}
    where we define
    \begin{equation*}
        (\mu_z)_{12} = \int\psi_2^*\mu_z\psi_1\dd{\tau}
    \end{equation*}
    to be the $z$-component of the \textbf{transition dipole moment} between states 1 and 2.
    \begin{itemize}
        \item Note that if $(\mu_z)_{12}=0$, then $\dv*{a_2}{0}=0$ and there will be no transition out of state 1 and into state 2.
        \item In other words, the dipole must change during the transition, and we have derived the first part of the selection rule!
    \end{itemize}
    \item Before finishing our derivation of explicit selection rules, integrate the above.
    \begin{align*}
        a_2(t) &\propto \int_0^t(\mu_z)_{12}E_{0z}(\e[i(E_2-E_1+h\nu)t/\hbar]+\e[i(E_2-E_1-h\nu)t/\hbar])\dd{t}\\
        &\propto (\mu_z)_{12}E_{0z}\left( \frac{1-\e[i(E_2-E_1+h\nu)t/\hbar]}{E_2-E_1+h\nu}+\frac{1-\e[i(E_2-E_1-h\nu)t/\hbar]}{E_2-E_1-h\nu} \right)
    \end{align*}
    \begin{itemize}
        \item Note that since $E_2>E_1$, when $E_2-E_1\approx h\nu$, the \textbf{resonance denominators} cause the second term above to be of major importance in determining $a_2(t)$.
        \item Thus, the Bohr frequency condition arises naturally from the quantum mechanics!
    \end{itemize}
    \item \textbf{Resonance denominators}: The two denominators in the above equation for $a_2(t)$.
    \item Given the above function, we can now determine the probability of observing the molecule in state 2.
    \begin{itemize}
        \item This quantity is proportional to both the probability of absorption and the intensity of absorption.
        \item It is given by $a_2^*a_2$. In calculating this quantity, though, we need only take into account the second term in the definition of $a_2$ as per the above discussion of resonance denominators.
        \item This gives us (see Problem \ref{prb:13-40})
        \begin{align*}
            a_2^*(t)a_2(t) &\propto \frac{\sin^2[(E_2-E_1-\hbar\omega)t/2\hbar]}{(E_2-E_1-\hbar\omega)^2}
        \end{align*}
        which has its largest peak (largest probability of a transition) at $\hbar\omega=E_2-E_1$ (see Figure \ref{fig:sincFunction}).
    \end{itemize}
    \item We now derive the selection rule in the rigid-rotator approximation.
    \item We require that
    \begin{equation*}
        0 \neq (\mu_z)_{J,M;J',M'} = \int_0^{2\pi}\int_0^\pi Y_{J'}^{M'}(\theta,\phi)^*\mu_zY_J^M(\theta,\phi)\sin\theta\dd{\theta}\dd{\phi}
    \end{equation*}
    \item Invoke $\mu_z=\mu\cos\theta$.
    \begin{equation*}
        (\mu_z)_{J,M;J',M'} = \mu\int_0^{2\pi}\int_0^\pi Y_{J'}^{M'}(\theta,\phi)^*Y_J^M(\theta,\phi)\cos\theta\sin\theta\dd{\theta}\dd{\phi}
    \end{equation*}
    \begin{itemize}
        \item Thus, $\mu\neq 0$ for a nonzero transition dipole moment.
        \item In other words, the molecule must have a permanent dipole moment for it to have a pure rotational spectrum, as asserted earlier!
    \end{itemize}
    \item As to the rest of the selection rule, substitute the spherical harmonics into the above equation and simplify. Also substitute $x=\cos\theta$.
    \begin{align*}
        (\mu_z)_{J,M;J',M'} &= \mu\int_0^{2\pi}\int_0^\pi(N_{J'M'}P_{J'}^{|M'|}(\cos\theta)\e[-iM'\phi])(N_{JM}P_J^{|M|}(\cos\theta)\e[iM\phi])\cos\theta\sin\theta\dd{\theta}\dd{\phi}\\
        &= \mu N_{JM}N_{J'M'}\int_0^{2\pi}\dd{\phi}\e[i(M-M')\phi]\int_0^\pi\dd{\theta}P_{J'}^{|M'|}(\cos\theta)P_J^{|M|}(\cos\theta)\cos\theta\sin\theta\\
        &= \mu N_{JM}N_{J'M'}\int_0^{2\pi}\dd{\phi}\e[i(M-M')\phi]\int_{-1}^1\dd{x}xP_{J'}^{|M'|}(x)P_J^{|M|}(x)
    \end{align*}
    \begin{itemize}
        \item For the leftmost integral above to be nonzero, we must have $M=M'$.
        \item This is the $\Delta M=0$ selection rule!
    \end{itemize}
    \item Evaluating the leftmost integral yields 
    \begin{equation*}
        (\mu_z)_{J,M;J',M'} = 2\pi\mu N_{JM}N_{J'M'}\int_{-1}^1\dd{x}xP_{J'}^{|M'|}(x)P_J^{|M|}(x)
    \end{equation*}
    \item Given the recursion rule
    \begin{equation*}
        (2J+1)xP_J^{|M|}(x) = (J-|M|+1)P_{J+1}^{|M|}(x)+(J+|M|)P_{J-1}^{|M|}(x)
    \end{equation*}
    pertaining to the associated Legendre functions, we have
    \begin{equation*}
        (\mu_z)_{J,M;J',M} = 2\pi\mu N_{JM}N_{J'M}\int_{-1}^1\dd{x}P_{J'}^{|M|}(x)\left[ \frac{J-|M|+1}{2J+1}P_{J+1}^{|M|}(x)+\frac{J+|M|}{2J+1}P_{J-1}^{|M|}(x) \right]
    \end{equation*}
    \begin{itemize}
        \item Thus, by the orthogonality of the associated Legendre functions, we must have $J'=J+1$ or $J'=J-1$ for the above integral not to vanish.
        \item This is the $\Delta J=\pm 1$ selection rule!
    \end{itemize}
    \item We now derive the selection rule in the harmonic-oscillator approximation.
    \item As before, let
    \begin{equation*}
        (\mu_z)_{v,v'} = \int_{-\infty}^\infty[N_{v'}H_{v'}(\sqrt{\alpha}q)\e[-\alpha q^2/2]]\mu_z(q)[N_vH_v(\sqrt{\alpha}q)\e[-\alpha q^2/2]]\dd{q}
    \end{equation*}
    \item Expand $\mu_z(q)$ about the equilibrium nuclear separation to two terms
    \begin{equation*}
        \mu_z(q) = \mu_0+\left( \dv{\mu}{q} \right)_0q
    \end{equation*}
    where $\mu_0$ is the dipole moment at the equilibrium bond length and $q$ is the displacement from that equilibrium value.
    \item Substituting this expansion into the original equation yields
    \begin{align*}
        (\mu_z)_{v,v'} &= \int_{-\infty}^\infty[N_{v'}H_{v'}(\sqrt{\alpha}q)\e[-\alpha q^2/2]]\left[ \mu_0+\left( \dv{\mu}{q} \right)_0q \right][N_vH_v(\sqrt{\alpha}q)\e[-\alpha q^2/2]]\dd{q}\\
        &= N_vN_{v'}\mu_0\int_{-\infty}^\infty H_{v'}(\sqrt{\alpha}q)H_v(\sqrt{\alpha}q)\e[-\alpha q^2]\dd{q}+N_vN_{v'}\left( \dv{\mu}{q} \right)_0\int_{-\infty}^\infty H_{v'}(\sqrt{\alpha}q)qH_v(\sqrt{\alpha}q)\e[-\alpha q^2]\dd{q}
    \end{align*}
    \item The first integral above vanishes if $v\neq v'$ by the orthogonality of the Hermite polynomials with respect to a Gaussian weighting function.
    \item As to the second integral, invoke the Hermite polynomial recursion formula
    \begin{equation*}
        \xi H_v(\xi) = vH_{v-1}(\xi)+\tfrac{1}{2}H_{v+1}(\xi)
    \end{equation*}
    from Problem \ref{prb:5-24} to get
    \begin{equation*}
        (\mu_z)_{v,v'} = \frac{N_vN_{v'}}{\alpha}\left( \dv{\mu}{q} \right)_0\int_{-\infty}^\infty H_{v'}(\xi)[vH_{v-1}(\xi)+\tfrac{1}{2}H_{v+1}(\xi)]\e[-\xi^2]\dd{\xi}
    \end{equation*}
    \begin{itemize}
        \item Note that we substitute $\xi=\sqrt{\alpha}q$.
        \item Thus, by the orthogonality of the Hermite polynomials, we must have $v'=v\pm 1$ for the above integral not to vanish.
        \item This is the $\Delta v=\pm 1$ selection rule!
        \item Additionally, the $(\dv*{\mu}{q})_0$ term indicates that the dipole moment must vary during the vibration.
    \end{itemize}
    \item A normal mode will have a nonzero transition only if it belongs to the same irreducible representation as one of the $\mu_{x,y,z}$\footnote{Related to Module 12 on \textcite[49-50]{bib:IChemNotes}.}.
    \begin{itemize}
        % \item Consider the more general transition dipole moment in terms of normal coordinates as follows.
        % \begin{equation*}
        %     I_{0\to 1} = \int\psi_0(Q_1,\dots,Q_{N_\text{vib}})
        %     \begin{Bmatrix}
        %         \mu_x\\
        %         \mu_y\\
        %         \mu_z\\
        %     \end{Bmatrix}
        %     \psi_1(Q_1,\dots,Q_{N_\text{vib}})\dd{Q_1}\cdots\dd{Q_{N_\text{vib}}}
        % \end{equation*}
        \item We have
        \begin{equation*}
            \psi_0(Q_1,\dots,Q_{N_\text{vib}}) = c\e[-\alpha_1Q_1^2-\cdots-\alpha_{N_\text{vib}}Q_{N_\text{vib}}^2]
        \end{equation*}
        where $c$ is a normalization constant and $\alpha_j=\sqrt{\mu_jk_j}/2\hbar$.
        \begin{itemize}
            \item Since the normal modes belong to the irreducible representations of the relevant molecular point group, the effect of an arbitrary symmetry operation $\hat{R}$ on $Q_j$ gives $\pm Q_j$.
            \item Thus, $\psi_0$ as a function of quadratic terms of $Q_j$ is wholly invariant under any $\hat{R}$.
        \end{itemize}
        \item We also have
        \begin{equation*}
            \psi_1(Q_1,\dots,Q_{N_\text{vib}}) = c'Q_j\e[-\alpha_1Q_1^2-\cdots-\alpha_{N_\text{vib}}Q_{N_\text{vib}}^2]
        \end{equation*}
        since an excited state necessitates exciting one and only one normal mode.
        \begin{itemize}
            \item Thus, $\psi_1$ transforms as $Q_j$ under the symmetry operations of the group.
        \end{itemize}
        \item Since $I_{0\to 1}$ is invariant under all operations of the group,
        \begin{equation*}
            I_{0\to 1} = \hat{R}I_{0\to 1} = \int(\hat{R}\psi_0)(\hat{R}\mu_x)(\hat{R}\psi_1)\dd{Q_1}\cdots\dd{Q_{N_\text{vib}}} = \chi_{A_1}(\hat{R})\chi_{\mu_x}(\hat{R})\chi_{Q_j}(\hat{R})I_{0,1}
        \end{equation*}
        \item Thus, the product of the characters equals 1. More specifically, since $\chi_{A_1}(\hat{R})$ always equals 1, the excited state and dipole must transform together, i.e., must belong to the same irreducible representation for the transition to be nonzero.
    \end{itemize}
\end{itemize}


\subsection*{Problems}
\begin{enumerate}[label={\textbf{13-\arabic*.}},ref={13-\arabic*}]
    \setcounter{enumi}{39}
    \item \label{prb:13-40}Derive Equation 13.57 from Equation 13.55 in \textcite{bib:McQuarrieSimon}.
    \begin{proof}[Answer]
        Since we are considering proportionality and ignoring the first term, we will work with the following version of Equation 13.55.
        \begin{equation*}
            a_2(t) \propto \frac{1-\e[i(E_2-E_1-h\nu)t/\hbar]}{E_2-E_1-h\nu}
        \end{equation*}
        Let
        \begin{equation*}
            x = E_2-E_1-h\nu
            = E_2-E_1-\frac{h}{2\pi}\cdot 2\pi\nu
            = E_2-E_1-\hbar\omega
        \end{equation*}
        Then
        \begin{align*}
            a_2^*(t)a_2(t) &= \frac{1-\e[-ixt/\hbar]}{x}\cdot\frac{1-\e[ixt/\hbar]}{x}\\
            &= \frac{1-\e[ixt/\hbar]-\e[-ixt/\hbar]+\e[0]}{x^2}\\
            &= \frac{2-(\e[ixt/\hbar]+\e[-ixt/\hbar])}{x^2}\\
            &= \frac{2-2\cos(xt/\hbar)}{x^2}\\
            &= \frac{4}{x^2}\cdot\frac{1-\cos(2(xt/2\hbar))}{2}\\
            &= \frac{4}{x^2}\cdot\sin^2(xt/2\hbar)\\
            &\propto \frac{\sin^2(xt/2\hbar)}{x^2}
        \end{align*}
        Returning the substitution yields the desired result.
    \end{proof}
    \setcounter{enumi}{48}
    \item \label{prb:13-49}Consider a molecule with a dipole moment $\bm{\mu}$ in an electric field $\bm{E}$. We picture the dipole moment as a positive charge and a negative charge of magnitude $q$ separated by a vector $\mathbf{l}$.
    \begin{center}
        \begin{tikzpicture}
            \footnotesize
            \draw (-2,0) -- (0,0) coordinate (O) -- (2,0) coordinate (A);
    
            \draw [thick,latex-latex] (30:2) ++(-60:0.7) -- ++(-60:-0.7) -- ++(0:0.7) node[right]{$f=qE_z$};
            \draw [thick,latex-latex] (-150:2) ++(120:0.7) -- ++(120:-0.7) -- ++(180:0.7) node[left]{$f=qE_z$};
            \draw [thick] (-150:2) node[circle,draw,fill=white,inner sep=0pt]{$-$} -- (30:2) coordinate (B) node[circle,draw,fill=white,inner sep=0pt]{$+$};
    
            \pic [draw,angle eccentricity=1.2,angle radius=8mm,pic text={$\theta$}] {angle=A--O--B};
            \draw [shorten <=1pt,shorten >=1pt,<->] (-150:2) ++(120:0.3) -- node[pos=0.6,fill=white,inner sep=1.5pt]{$l$} ++(30:4);
    
            \draw [thick,-latex] (1.7,-1) -- ++(0.6,0) node[right]{$E_z$};
        \end{tikzpicture}
    \end{center}
    The field $\bm{E}$ causes the dipole to rotate into a direction parallel to $\bm{E}$. Therefore, work is required to rotate the dipole to an angle $\theta$ to $\bm{E}$. The force causing the molecule to rotate is actually a torque (torque is the angular analog of force) and is given by $l/2$ times the force perpendicular to $\mathbf{l}$ at each end of the vector $\mathbf{l}$. Show that this torque is equal to $\mu E\sin\theta$ and that the energy required to rotate the dipole from some initial angle $\theta_0$ to some arbitrary angle $\theta$ is
    \begin{equation*}
        V = \int_{\theta_0}^\theta\mu E\sin\theta'\dd{\theta'}
    \end{equation*}
    Given that $\theta_0$ is customarily taken to be $\pi/2$, show that
    \begin{equation*}
        V = -\mu E\cos\theta = -\bm{\mu}\cdot\bm{E}
    \end{equation*}
    The magnetic analog of this result will be given in Chapter 14.
\end{enumerate}



\section{Chapter 15: Lasers, Laser Spectroscopy, and Photochemistry}
\emph{From \textcite{bib:McQuarrieSimon}.}
\begin{itemize}
    \item \marginnote{12/2:}"Laser is an acronym for \underline{l}ight \underline{a}mplification by \underline{s}timulated \underline{e}mission of \underline{r}adiation" \parencite[591]{bib:McQuarrieSimon}.
    \item \textbf{Photochemistry}: The study of light-initiated reactions.
    \item \textbf{Photodissociation}: Light-induced dissociation.
    \item \textbf{Radiative transition}: A transition between energy levels that involves either the absorption or the emission of radiation.
    \item \textbf{Nonradiative transition}: A transition between energy levels that occurs without the absorption or the emission of radiation.
    \item \textbf{Vibrational relaxation}: A collision between an excited molecule and another molecule in the sample resulting in an exchange of energy that removes some of the excess vibrational energy.
    \begin{itemize}
        \item Causes an excited molecule to quickly relax to the lowest vibrational state of the electronic excited state.
    \end{itemize}
    \item \textbf{Fluorescence}: The radiative decay process involving a transition between states of the same spin multiplicity.
    \item \textbf{Internal conversion}: The nonradiative decay process involving a transition between states of the same spin multiplicity.
    \item \textbf{Intersystem crossing}: A nonradiative transition between states of the same spin multiplicity enabled by overlap between vibrational and rotational states.
    \begin{itemize}
        \item Requires a change in the spin of the electron.
    \end{itemize}
    \item \textbf{Phosphorescence}: The radiative decay process involving a transition between states of differing spin multiplicities.
    \item Einstein's approach to the dynamics of atomic spectroscopic transitions.
    \begin{itemize}
        \item Einstein makes several assumptions that can be justified via time-dependent quantum mechanics.
        \item No quantum mechanics is required for Einstein's approach except that the energy levels of the atom are assumed to be quantized.
    \end{itemize}
    \item The approach (absorption only).
    \begin{itemize}
        \item Consider a sample of $N_\text{total}$ identical atoms having only two nondegenerate electronic energy levels $E_1<E_2$.
        \item Let number of atoms in stated $i$ be denoted by $N_i$.
        \item Let the sample of atoms have insufficient thermal energy to occupy state 2, i.e., $k_\text{B}T<<E_2-E_1$.
        \item Expose the sample to EM radiation of frequency $\nu_{12}$ where $h\nu_{12}=E_2-E_1$.
        \item The energy density of said EM radiation is described by the \textbf{radiant energy density} and the \textbf{spectral radiant energy density}.
        \item Einstein proposes that the rate $-\dv*{N_1}{t}=\dv*{N_2}{t}$ of excitation from the ground electronic state to the excited electronic state is proportional to $\rho_\nu(\nu_{12})$ and $N_1(t)$. In particular,
        \begin{equation*}
            -\dv{N_1}{t} = B_{12}\rho_\nu(\nu_{12})N_1(t)
        \end{equation*}
        where $B_{12}$ is a proportionality constant called an \textbf{Einstein coefficient}.
    \end{itemize}
    \item \textbf{Radiant energy density}: The radiant energy per unit volume. \emph{Denoted by} $\bm{\rho}$. \emph{Units} $\si{\joule\per\cubic\meter}$.
    \begin{itemize}
        \item The intensity $I$ of light is of the form $I=\rho c$.
    \end{itemize}
    \item \textbf{Spectral radiant energy density}: The radiant energy density per unit frequency. \emph{Denoted by} $\bm{\rho_\nu}$. \emph{Units} $\si{\joule\per\cubic\meter\second}$. \emph{Given by}
    \begin{equation*}
        \rho_\nu = \dv{\rho}{v}
    \end{equation*}
    \item The approach (relaxation).
    \begin{itemize}
        \item Einstein proposes \textbf{spontaneous emission} and \textbf{stimulated emission} as two pathways by which atoms atoms relax back into the ground state.
    \end{itemize}
    \item \textbf{Spontaneous emission}: The process by which atoms spontaneously emit a photon of energy $h\nu_{12}=E_2-E_1$ at some time after the excitation.
    \begin{itemize}
        \item The rate can be described by $-\dv*{N_2}{t}$, which we assume is simply proportional to the number of atoms $N_2(t)$ in the excited state at time $t$. In particular,
        \begin{equation*}
            -\dv{N_2}{t} = A_{21}N_2(t)
        \end{equation*}
        where $A_{21}$ is another Einstein coefficient.
    \end{itemize}
    \item \textbf{Stimulated emission}: The process by which exposure of an atom in an excited electronic state to electromagnetic radiation of energy $h\nu_{12}=E_2-E_1$ could stimulate the emission of a photon and thereby regenerate the ground state.
    \begin{itemize}
        \item The rate, we assume, is proportional to $N_2(t)$ and $\rho_\nu(\nu_{12})$. In particular,
        \begin{equation*}
            -\dv{N_2}{t} = B_{21}\rho_\nu(\nu_{12})N_2(t)
        \end{equation*}
        where $B_{21}$ is a third Einstein coefficient.
        \item In this process, one photon at frequency $\nu_{12}$ stimulates an atom to emit another. These two photons can then go on to cause more excited atoms to release photons, and on and on resulting in a substantial amplification of an incident light beam at frequency $\nu_{12}$.
        \begin{itemize}
            \item Lasers (light amplification by stimulated emission of radiation devices) exploit this phenomenon.
        \end{itemize}
    \end{itemize}
    \item Since a sample of atoms exposed to light undergoes absorption, spontaneous emission, and stimulated emission, the true rate of change in the population is
    \begin{equation*}
        -\dv{N_1}{t} = \dv{N_2}{t} = B_{12}\rho_\nu(\nu_{12})N_1(t)-A_{21}N_2(t)-B_{21}\rho_\nu(\nu_{12})N_2(t)
    \end{equation*}
    \item Deriving a relation between the three Einstein coefficients.
    \begin{itemize}
        \item Consider the limit at which the two energy states are in thermal equilibrium, i.e., $-\dv*{N_1}{t}=\dv*{N_2}{t}=0$. Also let $\rho_\nu(\nu_{12})$ be the equilibrium spectral radiant energy density, which we assume comes from a thermal blackbody radiation source.
        \item Solving the above overall population rate of change equation for $\rho_\nu(\nu_{12})$ yields
        \begin{equation*}
            \rho_\nu(\nu_{12}) = \frac{A_{21}}{(N_1/N_2)B_{12}-B_{21}}
        \end{equation*}
        \item Applying the fact (see Chapter 17) that for a system in equilibrium at temperature $T$, the number of atoms or molecules in the state $j$ with energy $E_j$ is $N_j=c\e[-E_j/k_\text{B}T]$, we have that
        \begin{equation*}
            \frac{N_2}{N_1} = \e[-(E_2-E_1)/k_\text{B}T] = \e[-h\nu_{12}/k_\text{B}T]
        \end{equation*}
        \item Plug this into the above equation and set it equal to Planck's blackbody distribution law.
        \begin{equation*}
            \frac{8\pi h}{c^3}\frac{\nu_{12}^3}{\e[h\nu_{12}/k_\text{B}T]-1} = \frac{A_{21}}{B_{12}\e[h\nu_{12}/k_\text{B}T]-B_{21}}
        \end{equation*}
        \item From the above equation it is clear that
        \begin{equation*}
            B_{12} = B_{21}
        \end{equation*}
        and
        \begin{equation*}
            A_{21} = \frac{8h\pi\nu_{12}^3}{c^3}B_{21}
        \end{equation*}
        \item This allows us to write the rate equation for a nondegenerate two-level system as
        \begin{equation*}
            -\dv{N_1}{t} = \dv{N_2}{t} = B\rho_\nu(\nu_{12})[N_1(t)-N_2(t)]-AN_2(t)
        \end{equation*}
    \end{itemize}
    \item For a laser to amplify light, a photon that passes through the sample must have a greater probability of stimulating emission from an excited atom then of being absorbed by an atom in its ground state.
    \begin{itemize}
        \item Symbolically, we must have
        \begin{align*}
            \text{rate stimulated emission} &> \text{rate absorption}\\
            B_{21}\rho_\nu(\nu_{12})N_2 &> B_{12}\rho_\nu(\nu_{12})N_1\\
            N_2 &> N_1
        \end{align*}
    \end{itemize}
    \item \textbf{Population inversion}: A situation where the population of the excited state is greater than that of the lower state.
    \begin{itemize}
        \item By the above equation for $N_2/N_1$, we must have $N_2/N_1<1$ when the system is at equilibrium since $h\nu_{12}/k_\text{B}T$ is a strictly positive quantity.
        \item Thus, population inversions constitute nonequilibrium situations.
    \end{itemize}
    \item A population inversion cannot occur in a nondegenerate two-level system.
    \begin{itemize}
        \item Integrating the rate equation for a such a system yields
        \begin{equation*}
            N_2(t) = \frac{B\rho_\nu(\nu_{12})N_\text{total}}{A+2B\rho_\nu(\nu_{12})}\{1-\e[-[{A+2B\rho_\nu(\nu_{12})]t}]\}
        \end{equation*}
        \item This equation's behavior as $t\to\infty$ converges to
        \begin{equation*}
            \frac{N_2}{N_\text{total}} = \frac{B\rho_\nu(\nu_{12})}{A+2B\rho_\nu(\nu_{12})}
        \end{equation*}
        which, since $A>0$, will always be strictly less than $1/2$.
        \item Therefore, no such population inversion can be achieved.
    \end{itemize}
    \item Relaxing back to equilibrium after an incident light source is turned off.
    \begin{itemize}
        \item The rate equation simplifies to just spontaneous emission, i.e.,
        \begin{align*}
            \dv{N_2}{t} &= -AN_2\\
            N_2(t) &= N_2(0)\e[-At]
        \end{align*}
    \end{itemize}
    \item \textbf{Fluorescence lifetime}: The reciprocal of $A$. \emph{Also known as} \textbf{radiative lifetime}. \emph{Denoted by} $\tau_R$.
    \item \marginnote{12/8:}We now consider a three-level system, which can achieve a population inversion.
    \item \textbf{Pump source}: A light beam that is used to create excited state populations.
    \item Although we can derive rate equations for every energy level of the three-level system, only the rate equation of the second, intermediate level is needed to give a useful result.
    \begin{equation*}
        \dv{N_2}{t} = A_{32}N_3-A_{21}N_2+\rho_\nu(\nu_{32})B_{32}N_3-\rho_\nu(\nu_{32})B_{32}N_2
    \end{equation*}
    \begin{itemize}
        \item At equilibrium (e.g., $\dv*{N_2}{t}=0$), the above equation can be rearranged to the form
        \begin{equation*}
            \frac{N_3}{N_2} = \frac{A_{21}+B_{32}\rho_\nu(\nu_{32})}{A_{32}+B_{32}\rho_\nu(\nu_{32})}
        \end{equation*}
        \item This implies that $N_3>N_2$ if $A_{21}>A_{32}$, i.e., if it takes a while for atoms in state 3 to decay to state 2, but atoms in state 2 decay quickly to state 1.
    \end{itemize}
    \item \textbf{Gain medium}: A system of atoms that has a population inversion between states 2 and 3.
    \item The insides of a laser.
    \begin{figure}[h!]
        \centering
        \begin{tikzpicture}[
            scale=0.8,
            every node/.style=black
        ]
            \footnotesize
            \path (-8,0) -- (8,0);
            \fill [rex] (-6,0.7) -- (6,1.3) -- node[above]{laser beam} ++(2,0) -- ++(0,-0.6) -- ++(-2,0) -- (6,0.7) -- (-6,1.3) -- cycle;
    
            \filldraw [fill=gray!20] (-6,0) -- ++(0,2) -- ++(0.5,0) arc[start angle=150,end angle=210,radius=2cm] -- node[below,text width=1.5cm,align=center]{$100\%$ reflective mirror} cycle;
            \filldraw [fill=gray!20] (6,0) -- ++(0,2) -- ++(-0.5,0) arc[start angle=30,end angle=-30,radius=2cm] -- node[below,text width=1.5cm,align=center]{$<100\%$ reflective mirror} cycle;
    
            \small
            \node [draw,fill=white,semithick,minimum width=4.8cm,minimum height=1.6cm] at (0,1) {Gain medium};
            \node [draw,fill=white,semithick,minimum width=4.8cm,minimum height=0.8cm] at (0,-3) {Pump source};
    
            \draw [fill=grx!30] (-2.5,-2.3) -- ++(0,1.4) -- ++(-0.2,0) -- ++(0.3,0.7) -- ++(0.3,-0.7) -- ++(-0.2,0) -- ++(0,-1.4);
            \draw [fill=grx!30] (-1.5,-2.3) -- ++(0,1.4) -- ++(-0.2,0) -- ++(0.3,0.7) -- ++(0.3,-0.7) -- ++(-0.2,0) -- ++(0,-1.4);
            \draw [fill=grx!30] (-0.5,-2.3) -- ++(0,1.4) -- ++(-0.2,0) -- ++(0.3,0.7) -- ++(0.3,-0.7) -- ++(-0.2,0) -- ++(0,-1.4);
            \draw [fill=grx!30] (0.5,-2.3) -- ++(0,1.4) -- ++(-0.2,0) -- ++(0.3,0.7) -- ++(0.3,-0.7) -- ++(-0.2,0) -- ++(0,-1.4);
            \draw [fill=grx!30] (1.5,-2.3) -- ++(0,1.4) -- ++(-0.2,0) -- ++(0.3,0.7) -- ++(0.3,-0.7) -- ++(-0.2,0) -- ++(0,-1.4);
            \draw [fill=grx!30] (2.5,-2.3) -- ++(0,1.4) -- ++(-0.2,0) -- ++(0.3,0.7) -- ++(0.3,-0.7) -- ++(-0.2,0) -- ++(0,-1.4);
        \end{tikzpicture}
        \caption{The insides of a laser.}
        \label{fig:laserInsides}
    \end{figure}
    \item \textbf{Laser cavity}: The area containing the components in Figure \ref{fig:laserInsides}.
    \item Gain medium.
    \begin{itemize}
        \item A solid-state material, liquid solution, or a gas mixture.
        \item The first laser used a solid syntheticallly grown ruby rod.
        \begin{itemize}
            \item Ruby is a crystal of corundum (\ce{Al2O3}) where sum \ce{Al^3+} ions are replaced by \ce{Cr^3+} ions.
            \item The ruby must be synthetic since naturally occurring crystals contain natural strains and defects. We use \ce{Al2O3} doped at about 0.05 mass percent with \ce{Cr^3+}.
            \item The photophysical properties of the \textbf{active ions} (\ce{Cr^3+}) in the \textbf{host material} (\ce{Al203}) are suitable for achieving a population inversion.
        \end{itemize}
        \item Depending on the gain media, laser output can be a continuous light beam or a short burst of light.
        \item \textcite{bib:McQuarrieSimon} lists data on various solid-state and gas phase gain media.
        \item Gas phase lasers produce light in the ultraviolet, visible, and infrared spectra. Some are even capable of producing multiple wavelengths.
        \item Since the stimulated-emission process requries that the phases of the incident light wave and stimulated light wave have the same phase, lasers emit \textbf{coherent} light.
    \end{itemize}
    \item \textbf{Coherent} (light): Light the constituent waves of which are all in phase.
    \item \textbf{Radiant power}: Radiant energy per unit time. \emph{Units} $\si{\watt}$.
    \item \textbf{Average radiant power}: Total energy emitted per second. \emph{Units} $\si{\watt}$.
    \item Pump source.
    \begin{itemize}
        \item Gain media are pumped either via \textbf{optical excitation} or \textbf{electrical excitation}.
    \end{itemize}
    \item \textbf{Optical excitation}: Using a high-intensity light source such as a continuous lamp, flash lamp, or laser to excite the gain medium.
    \begin{itemize}
        \item Since only light with the right frequency is absorbed, lasers tend to be inefficient devices.
        \item Gas phase lasers convert only $\SIrange{0.001}{0.1}{\percent}$ of input energy to laser light.
        \item Solid state lasers convert a few percent.
        \item The \ce{CO2} laser and some semiconductor lasers convert $\SIrange{50}{70}{\percent}$.
        \item This is why some lasers use other lasers as pump sources.
    \end{itemize}
    \item \textbf{Electrical excitation}: Using intense electrical discharges to excite the gain medium.
    \begin{itemize}
        \item Commonly used in gas lasers.
    \end{itemize}
    \item Laser cavity design.
    \begin{itemize}
        \item To sufficiently amplify the light, the light must be directed back and forth through the gain medium a number of times using a \textbf{resonator}.
        \item The less than $100\%$ reflectivity of one of the mirrors allows some light to escape.
    \end{itemize}
    \item \textbf{Resonator}: An optical cavity including a pair of mirrors that direct light back and forth through the gain medium.
    \item \ce{He}-\ce{Ne} laser.
    \begin{itemize}
        \item The first continuous-wave laser.
        \item Can be made to produce light at a number of wavelengths, but most are red or infrared.
        \item Very common (supermarket scanners, range finders, etc.).
        \item Uses $\SI{1.0}{\torr}$ of helium and $\SI{0.1}{\torr}$ neon in a glass gas chamber as a gain medium with a high current DC pump source as electrical excitation.
        \item The constant current delivered leads to continuous lasing.
        \item How it works:
        \begin{itemize}
            \item Electrons collide with helium atoms, generating a number of excited states but notably the long lived $2s\prescript{3}{}{S}_1$ and $2s\prescript{1}{}{S}_0$ excited states, with respective lifetimes $\SI{e-4}{\second}$ and $\SI{5e-6}{\second}$.
            \item On average, within $\SI{e-7}{\second}$, the excited helium atom will collide with a neon atom, initiating a nonradiative energy transfer to neon's fortuitously near equivalent energy levels.
            \begin{gather*}
                \ce{He^{$*$}(2s{}^3S1) + Ne_{(g)} -> He_{(g)} + Ne^{$*$}(2p^5 4s)}\\
                \ce{He^{$*$}(2s{}^1S0) + Ne_{(g)} -> He_{(g)} + Ne^{$*$}(2p^5 5s)}
            \end{gather*}
            \item The lifetime of the excited neon states are such that a population inversion can be set up and maintained.
        \end{itemize}
    \end{itemize}
    \item \textbf{Spectral resolution}: The range over which a spectrophotometer cannot distinguish a difference in wavenumbers.
    \item \textbf{Hyperfine interaction}: Small changes in the energies of the rotational state that are caused by the interaction of the electron spins with the nuclear spins.
    \begin{itemize}
        \item It is possible to include such interactions in the Hamiltonian operator and predict the spacings.
    \end{itemize}
    \item Time-resolved laser spectroscopy can be used to study the dynamics of chemical reactions.
    \item \textbf{Photochemical reaction}: A chemical reaction that results from the absorption of light.
    \item Examples:
    \begin{gather*}
        \ce{O3 ->[\SI{300}{\nano\meter}] O2 + O}\\
        \ce{\emph{trans}-butadiene ->[\SI{250}{\nano\meter}] \emph{cis}-butadiene}
    \end{gather*}
    \begin{itemize}
        \item The top reaction above is \textbf{photodissociation}, while the bottom one is \textbf{photoisomerization}.
    \end{itemize}
    \item \textbf{Quantum yield} (of a photochemical reaction): The number of molecules that undergo a reaction per photon absorbed. \emph{Denoted by} $\bm{\Phi}$.
    \item Time-resolved laser apparatuses split short pulses into two time-separated pulses.
    \item \textbf{Pump pulse}: The laser pulse that initiates the photochemistry.
    \item \textbf{Probe pulse}: The laser pulse that is used to record changes in the sample since the pump pulse arrived.
    \item \textcite{bib:McQuarrieSimon} does a worked example with \ce{ICN}.
    \item \textbf{Laser-induced fluorescence}: A form of detection that relies on using a laser to cause the product molecules to fluoresce.
\end{itemize}




\end{document}