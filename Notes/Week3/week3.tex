\documentclass[../notes.tex]{subfiles}

\pagestyle{main}
\renewcommand{\chaptermark}[1]{\markboth{\chaptername\ \thechapter\ (#1)}{}}
\setcounter{chapter}{2}

\begin{document}




\chapter{Mathematical Formulation of Quantum Mechanics}
\section{Vibrational Motion and the Harmonic Oscillator}
\begin{itemize}
    \item \marginnote{10/11:}Suppose we have an attractive force $F$ proportional to the displacement $x$ from the center of a system
    \begin{equation*}
        F = -kx
    \end{equation*}
    \begin{itemize}
        \item Then we also have an associated potential energy
        \begin{equation*}
            V(x) = \frac{1}{2}kx^2
        \end{equation*}
        \item Recall that $F=-\pdv*{V}{x}$.
    \end{itemize}
    \item Thus, we have a harmonic (or parabolic) potential well.
    \begin{figure}[h!]
        \centering
        \begin{subfigure}[b]{0.4\linewidth}
            \centering
            \begin{tikzpicture}[
                every node/.append style={black}
            ]
                \footnotesize
                \draw (0,1.5) -- node[left]{\small$E$} (0,0) -- node[below right]{\small$x$} (4,0);
    
                \draw [grx,semithick]
                    ({2-0.4},0.2)  -- ({2+0.4},0.2)
                    ({2-0.55},0.4) -- ({2+0.55},0.4)
                    ({2-0.67},0.6) -- ({2+0.67},0.6)
                    ({2-0.8},0.8)  -- ({2+0.8},0.8)
                ;
    
                \draw [blx,thick,<->] (1,1.3) parabola bend (2,0) (3,1.3) node[below right]{$V(x)$};
            \end{tikzpicture}
            \caption{Harmonic (parabolic) well.}
            \label{fig:parabolicPotentiala}
        \end{subfigure}
        \begin{subfigure}[b]{0.4\linewidth}
            \centering
            \begin{tikzpicture}[
                every node/.append style={black}
            ]
                \footnotesize
                \draw [stealth-stealth] (0,1.5) -- node[left]{\small$W(x)$} (0,0) -- node[below right]{\small$R$} (4,0);
                \draw (1.5,0.1) -- ++(0,-0.2) node[below]{$a$};
    
                \draw [grx,semithick]
                    ({1.5-0.13},0.5)  -- ({1.5+0.13},0.5)
                    ({1.5-0.19},0.6)  -- ({1.5+0.19},0.6)
                    ({1.5-0.23},0.7)  -- ({1.5+0.23},0.7)
                ;
    
                \draw [blx,thick] (1,1.3)
                    to[out=-80,in=180,in looseness=0.7] (1.5,0.4)
                    to[out=0,in=-140,in looseness=0.5,out looseness=0.7] (1.8,0.55)
                    to[out=40,in=-180,out looseness=0.5] (3.8,0.8)
                ;
                \draw [blx,semithick,dashed] (1.1,1.3) parabola bend (1.5,0.4) (1.9,1.3);
            \end{tikzpicture}
            \caption{Approximating a potential well.}
            \label{fig:parabolicPotentialb}
        \end{subfigure}
        \caption{Parabolic potential wells.}
        \label{fig:parabolicPotential}
    \end{figure}
    \begin{itemize}
        \item However, because this is a quantum system, the attainable energy levels will be quantized (see Figure \ref{fig:parabolicPotentiala}).
        \item We can use a parabolic well to approximate the minimum of the potential well (see Figure \ref{fig:parabolicPotentialb}).
    \end{itemize}
    \item \textbf{Reduced mass}: For two objects of mass $m_A,m_B$, the quantity
    \begin{equation*}
        \mu = \frac{m_Am_B}{m_A+m_B}
    \end{equation*}
    \begin{itemize}
        \item We can map the two body problem of two atoms being drawn together and pulled apart onto the one body problem of a single harmonic oscillator of reduced mass $\mu$ at the center of mass of the diatomic system.
    \end{itemize}
    \item The Taylor series expansion of the Moir\'{e} Potential about $x=a$ where $a$ is the minimum potential:
    \begin{align*}
        W(x) &= W(a)+(x-a)W'(a)+\frac{1}{2!}(x-a)^2W''(a)+\cdots\\
        &= W(a)+\frac{1}{2}(x-a)^2W''(a)\\
        &= \frac{1}{2}kx^2
    \end{align*}
    \begin{itemize}
        \item We reduce by noting that $W'(a)=0$ at the minimum of the potential well, we can let $W(a)=0$, and we can set $a=0$ to be the origin of our coordinate system.
    \end{itemize}
    \item The Schr\"{o}dinger equation describing this system is
    \begin{equation*}
        -\frac{\hbar}{2m}\dv[2]{x}\psi(x)+\frac{1}{2}m\omega^2x^2\psi(x) = E\psi(x)
    \end{equation*}
    \begin{itemize}
        \item Note that since $\omega=\sqrt{k/m}$, we substituted $k=m\omega^2$.
    \end{itemize}
    \item If we let $x=y\sqrt{\hbar/m\omega}$ and $E=\omega\hbar\epsilon/2$, then we can simplify the above to the form
    \begin{equation*}
        \dv[2]{y}\psi(y)+(\epsilon-y^2)\psi(y) = 0
    \end{equation*}
    \item Asymptotic solution: In the limit of large $y$, the finite value of $\epsilon$ becomes negligible, that is
    \begin{equation*}
        \dv[2]{y}\psi(y)-y^2\psi(y) = 0
    \end{equation*}
    \item General form solution:
    \begin{equation*}
        \psi(y) = y^p\e[-y^2/2]
    \end{equation*}
    \begin{itemize}
        \item This is the Gaussian exponential.
        \item $p$ is any integer.
    \end{itemize}
    \item Because the sign of the exponential must be negative for the wave function to be bounded, we have the form
    \begin{equation*}
        \psi(y) = \e[-y^2/2]H(y)
    \end{equation*}
    where $H(y)$ are polynomials.
    \item Hermite equation:
    \begin{equation*}
        \dv[2]{y}H(y)-2y\dv{y}H(y)-(\epsilon-1)H(y) = 0
    \end{equation*}
    \begin{itemize}
        \item Whenever $H(y)$ solves this equation, it yields a full solution.
    \end{itemize}
    \item What are the correct polynomials?
    \begin{itemize}
        \item The polynomials that are even about the origin will give us the even solutions, and vice versa for the odd ones.
        \item Even solution: Because the potential has a definite parity (even-ness or odd-ness), we know that the solution to the polynomials must be even or odd.
        \item Expanding in an infinite power series:
        \begin{equation*}
            H(y) = \sum_{j=0}^\infty c_jy^{2j}
        \end{equation*}
    \end{itemize}
    \item This is the power series solution to differential equations. We have to plug into the differential equation and get a recursion relation.
    \begin{itemize}
        \item Upon substitution,
        \begin{equation*}
            \sum_{j=0}^\infty\left( 2j(2j-1)c_jy^{2j-2}+(\epsilon-1-4j)c_jy^{2j} \right) = 0
        \end{equation*}
        \item The indices are arbitrary, so
        \begin{equation*}
            \sum_{j=0}^\infty(2(j+1)(2j+1)c_{j+1}+(\epsilon-1-4j))y^{2j} = 0
        \end{equation*}
        \item Recursion relation: The whole coefficient above must equal 0 for all $j$, but that gives us a relationship between $c_j$ and $c_{j+1}$! Explicitly,
        \begin{equation*}
            c_{j+1} = \frac{4j+1-\epsilon}{2(j+1)(2j+1)}c_j
        \end{equation*}
    \end{itemize}
    \item How do we know when to stop?
    \begin{itemize}
        \item If the recursion never stops, then the ratio is approximately equal to
        \begin{equation*}
            \frac{c_{j+1}}{c_j} = \frac{1}{j}
        \end{equation*}
        \item But this means that asymptotically, the boundary conditions will be violated because it will keep expanding. The probability of finding the particle will actually diverge (infinite probability at infinite distances). Thus, the expansion procedure \emph{must} terminate.
        \item The truncation of this expansion requires us to pick a particular energy $\epsilon$ (in particular, one such that $\epsilon=4j+1$).
    \end{itemize}
    \item Test on Friday:
    \begin{itemize}
        \item 5 questions. Each question is approximately 20 points.
        \item There will be a formula page in the back with formulas and constants.
        \item There will be a periodic table provided.
        \item Topics: Everything in problem sets 1-2. BB radiation, Photoelectric effect, SG experiment, particle-wave duality, Heisenberg uncertainty relations, Gaussian wave packets and their role in the Heisenberg uncertainty, de Broglie formula, free particles, particle in a box, potential step, and a bit of the harmonic oscillator.
        \item Study for it by going back to the problem sets and seeing which ones might be doable in a 50 minute test.
        \item Go back to your notes and review some of the key highlights of each of the topics in the topic list.
        \item You're allowed to use a calculator. The test won't be too calculator-heavy though.
        \item Deriving vs. understanding and applying: Emphasis on applying and getting answers.
    \end{itemize}
\end{itemize}



\section{Harmonic Oscillator (cont.)}
\begin{itemize}
    \item \marginnote{10/13:}Assume that the $(n+1)^\text{th}$ coefficient vanishes by the recurrence relation; this causes the energy to be quantized.
    \begin{itemize}
        \item Therefore, $\epsilon=4n+1$.
    \end{itemize}
    \item For each value of $n$, there is an even Hermite polynomial\footnote{So named because they were studied by the mathematician Charles Hermite before they were utilized in Quantum Mechanics.}.
    \begin{itemize}
        \item Thus, our even solutions include $H_0(y)=1$, $H_2(y)=4y^2-2$, $H_4(y)=16y^4-48y^2-2$, for example.
    \end{itemize}
    \item Odd solutions:
    \begin{itemize}
        \item Let
        \begin{equation*}
            H(y) = \sum_{j=0}^\infty d_jy^{2j+1}
        \end{equation*}
        \item Our recurrence relation works out to be
        \begin{equation*}
            d_{j+1} = \frac{4j+3-\epsilon}{2(j+1)(2j+3)}d_j
        \end{equation*}
        \item Again, if it does not terminate, $d_{j+1}/d_j\approx 1/j$, so the solutions will blow up at the edges due to the high powers of $y$. Therefore, the series must truncate.
        \item If the coefficient at $n$ exists but the next one will vanish, it must be true that $4n+3=\epsilon$.
        \item Example odd solutions: $H_1(y)=2y$, $H_3(y)=8y^3-12y$, $H_5(y)=32y^5-160y^3+120y$.
        \begin{itemize}
            \item Note that you can make the coefficients pretty much of any scale because they will be normalized later as part of the wave function.
        \end{itemize}
    \end{itemize}
    \item Energies and wave functions:
    \begin{itemize}
        \item The energy levels are $\epsilon_n=2n+1$ for $n=0,1,2,\dots$ or $E_n=(n+\frac{1}{2})\hbar\omega$ for $n=0,1,2,\dots$.
        \item It follows that if $\psi(y)=N_y\e[-y^2/2]H_n(y)$, where
        \begin{equation*}
            N_y = \frac{1}{\sqrt{2^nn!\sqrt{\pi}}}
        \end{equation*}
        \item In terms of $x$, we get
        \begin{equation*}
            N_x = \sqrt[4]{\frac{\omega m}{\hbar}}N_y
        \end{equation*}
        \item Observations:
        \begin{enumerate}
            \item The energy levels are quantized (discrete).
            \item The energy levels are equally spaced apart where $\Delta E=\hbar\nu$.
            \item The energy levels are non-degenerate.
            \item The zero-point energy is equal to $\hbar\omega/2$. (Recall that the finite energy is there due to the Uncertainty Relation.)
            \item Similar to the levels discovered by Max Planck.
        \end{enumerate}
    \end{itemize}
    \item Classical harmonic oscillator:
    \begin{itemize}
        \item Position: $x=x_0\sin(\omega t)$.
        \item Velocity: $v=\cos x_0\sin(\omega t)$.
        \item Energy: $E=m\omega^2x_0^2/2$.
        \item Turning point: $x_0=\sqrt{2E/m\omega^2}$.
        \item Probability of the oscillator being at $x$:
        \begin{equation*}
            P(x)\dd{x} = \frac{\frac{2\dd{x}}{v}}{T} = \frac{\dd{x}}{\pi\sqrt{x_0^2-x^2}}
        \end{equation*}
        \begin{itemize}
            \item Thus, classically, the oscillator spends most of its time at the turning points (this makes intuitive sense because a pendulum slows down at the turning points, spending more time there).
            \item In quantum mechanics, we don't have a hard turning point the way we do classically.
            \item Classical limit of quantum theory: In higher and higher order Hermite polynomials, the probability gets pushed to the edges.
        \end{itemize}
    \end{itemize}
\end{itemize}




\end{document}