\documentclass[../notes.tex]{subfiles}

\pagestyle{main}
\renewcommand{\chaptermark}[1]{\markboth{\chaptername\ \thechapter\ (#1)}{}}
\setcounter{chapter}{6}

\begin{document}




\chapter{Molecular Spectroscopy and Group Theory}
\section{Hydrogen Molecule}
\begin{itemize}
    \item \marginnote{11/8:}Reviews what would happen if electrons were bozons.
    \begin{itemize}
        \item The wave function $\psi$ of the lithium atom would be
        \begin{equation*}
            \psi(123) = 1s\alpha(1)\vee 1s\alpha(2)\vee 1s\alpha(3)
        \end{equation*}
        \item Reviews wedge products.
        \item Conclusion: All electrons could occupy the same orbital.
    \end{itemize}
    \item A sodium atom (the nucleus and the electrons jointly) acts like a bozon.
    \begin{itemize}
        \item At temperatures on the order of microkelvin, $10^{11}$ atoms have been placed in the same ground-state orbital.
        \item These substances are known as \textbf{Bose-Einstein condensates}.
        \item We use a magnetic field to confine the atoms to a harmonic potential. Since the atoms form a Gaussian curve at the bottom of said potential, they are all in the ground state (see Figure \ref{fig:harmonicPrbDensity}).
        \item Evaporative cooling and laser cooling allow you to reach such temperatures.
        \item Fermionic atoms cannot condense in such a way (because of the Pauli Exclusion Principle).
        \item Superconductivity is a condensation phenomena.
        \item Bose-Einstein condensates were predicted by Einstein in the 1930s but not experimentally verified until the 1990s.
        \item A very dilute gas was used here. In such a condition, the atoms feel the statistics force of the wedge product which forces them into such a state.
    \end{itemize}
    \item Consider the Boron atom:
    \begin{itemize}
        \item It has 5 electrons.
        \item It's electron configuration is $1s^22s^2sp^1$.
        \item It's wave function is
        \begin{equation*}
            \psi(12345) = 1s\alpha(1)\wedge 1s\beta(2)\wedge 2s\alpha(3)\wedge 2s\beta(4)\wedge 2p\alpha(5)
        \end{equation*}
        \begin{itemize}
            \item Recall that this is not the exact wave function; this is still a product of hydrogenlike orbitals at the Hartree-Fock level.
            \item More general wave functions can be used to obtain more accurate results.
        \end{itemize}
    \end{itemize}
    \item Consider the diatomic molecule \ce{H2}.
    \begin{figure}[H]
        \centering
        \begin{tikzpicture}[scale=1.5]
            \footnotesize
            \coordinate (HA) at (0,0);
            \coordinate (HB) at (2,0);
            \coordinate (e1) at (0.3,1);
            \coordinate (e2) at (1.7,1);
    
            \draw
                (HA)
                    -- node[below]{$R$} (HB)
                    -- node[right]{$r_{2\text{B}}$} (e2)
                    -- node[above]{$r_{12}$} (e1)
                    -- node[left]{$r_{1\text{A}}$} cycle
                (HA) -- node[pos=0.3,above=2pt]{$r_{2\text{A}}$} (e2)
                (e1) -- node[pos=0.7,above=1pt]{$r_{1\text{B}}$} (HB)
            ;
    
            \node [circle,fill=rex,inner sep=2pt,label={below:\ce{H_A}}] at (HA) {};
            \node [circle,fill=rex,inner sep=2pt,label={below:\ce{H_B}}] at (HB) {};
            \node [circle,fill=grx,inner sep=1.5pt,label={above:$\e_1$}] at (e1) {};
            \node [circle,fill=grx,inner sep=1.5pt,label={above:$\e_2$}] at (e2) {};
        \end{tikzpicture}
        \caption{\ce{H2} distances.}
        \label{fig:H2Distances}
    \end{figure}
    \begin{itemize}
        \item We have hydrogen atoms \ce{H_A} and \ce{H_B}, separated by a distance $R$.
        \item We have electrons $\e_1$ and $\e_2$.
        \item The distance from object $i$ to $j$ where $i,j=\text{A},\text{B},1,2$ and $i\neq j$ is $r_{ij}$.
        \item Hamiltonian:
        \begin{equation*}
            \hat{H} = -\frac{1}{2}(\nabla_1^2+\nabla_2^2)-\frac{1}{r_{1\text{A}}}-\frac{1}{r_{1\text{B}}}-\frac{1}{r_{2\text{A}}}-\frac{1}{r_{2\text{B}}}+\frac{1}{r_{12}}+\frac{1}{R}
        \end{equation*}
    \end{itemize}
    \item This is pretty complicated.
    \item Thus, let's start with the hydrogen molecular ion (\ce{H2+}).
    \begin{figure}[h!]
        \centering
        \begin{tikzpicture}[scale=1.5]
            \footnotesize
            \coordinate (HA) at (0,0);
            \coordinate (HB) at (1,0);
            \coordinate (e1) at (0.5,1);
    
            \draw
                (HA)
                -- node[below]{$R$} (HB)
                -- node[right]{$r_\text{B}$} (e1)
                -- node[left]{$r_\text{A}$} cycle
            ;
    
            \node [circle,fill=rex,inner sep=2pt,label={below:\ce{H_A}}] at (HA) {};
            \node [circle,fill=rex,inner sep=2pt,label={below:\ce{H_B}}] at (HB) {};
            \node [circle,fill=grx,inner sep=1.5pt,label={above:$\e_1$}] at (e1) {};
        \end{tikzpicture}
        \caption{\ce{H2+} distances.}
        \label{fig:H2plusDistances}
    \end{figure}
    \begin{itemize}
        \item Hamiltonian:
        \begin{equation*}
            \hat{H} = -\frac{1}{2}\nabla^2-\frac{1}{r_\text{A}}-\frac{1}{r_\text{B}}+\frac{1}{R}
        \end{equation*}
        \item This \emph{can} be solved exactly in cylindrical coordinates, but it's nasty.
        \item Thus, let's approximate with the following variational wave function (originally by Heitler and London in the 1960s).
        \begin{equation*}
            \psi(1) = c_11s_\text{A}+c_21s_\text{B}
        \end{equation*}
        \begin{itemize}
            \item Albeit simple, this wave function gives pretty good results.
        \end{itemize}
        \item By the variational principle, $\mathbb{H}\vec{c}=E\mathbb{S}\vec{c}$, or
        \begin{equation*}
            \begin{vmatrix}
                H_{\text{A}\text{A}}-ES_{\text{A}\text{A}} & H_{\text{A}\text{B}}-ES_{\text{A}\text{B}}\\
                H_{\text{B}\text{A}}-ES_{\text{B}\text{A}} & H_{\text{B}\text{B}}-ES_{\text{B}\text{B}}\\
            \end{vmatrix}
            = 0
        \end{equation*}
        solves for $E$.
        \item We have that
        \begin{align*}
            H_{\text{A}\text{A}} = H_{\text{B}\text{B}} &= \int\dd{\vec{r}}1s_\text{A}^*\hat{H}1s_\text{A} = \int\dd{\vec{r}}1s_\text{B}^*\hat{H}1s_\text{B}\\
            H_{\text{A}\text{B}} = H_{\text{B}\text{A}} &= \int\dd{\vec{r}}1s_\text{A}^*\hat{H}1s_\text{B} = \int\dd{\vec{r}}1s_\text{B}^*\hat{H}1s_\text{A}\\
            S_{\text{A}\text{A}} = S_{\text{B}\text{B}} &= \int\dd{\vec{r}}1s_\text{A}^*1s_\text{A} = \int\dd{\vec{r}}1s_\text{B}^*1s_\text{B}\\
            S_{\text{A}\text{B}} = S_{\text{B}\text{A}} &= \int\dd{\vec{r}}1s_\text{A}^*1s_\text{B} = \int\dd{\vec{r}}1s_\text{B}^*1s_\text{A}
        \end{align*}
        \item We can show that
        \begin{align*}
            H_{\text{A}\text{A}} = H_{\text{B}\text{B}} &= E_{1s}+J\\
            H_{\text{A}\text{B}} = H_{\text{B}\text{A}} &= E_{1s}S+K
        \end{align*}
        where $E_{1s}$ is the energy of the $1s$ orbital of the hydrogen atom, $J$ is the \textbf{Coulomb integral}
        \begin{equation*}
            J = \int\dd{\vec{r}}1s_\text{A}^*\left( -\frac{1}{r_\text{B}}+\frac{1}{R} \right)1s_\text{A}
        \end{equation*}
        and $K$ is the \textbf{exchange integral}
        \begin{equation*}
            K = \int\dd{\vec{r}}1s_\text{B}^*\left( -\frac{1}{r_\text{B}}+\frac{1}{R} \right)1s_\text{A}
        \end{equation*}
    \end{itemize}
\end{itemize}



\section{The Hydrogen Molecular Ion}
\begin{itemize}
    \item \marginnote{11/10:}Continuing from last time, the determinant for the equation $\mathbb{H}\vec{c}_n=E_n\mathbb{S}\vec{c}_n$ is
    \begin{equation*}
        \begin{vmatrix}
            E_{1s}+J-E & E_{1s}S+K\\
            E_{1s}S+K & E_{1s}+J-E\\
        \end{vmatrix}
    \end{equation*}
    \item Therefore, the characteristic polynomial is a quadratic equation in $E$.
    \item Solving said quadratic gives us
    \begin{equation*}
        E_\pm = E_{1s}+\frac{J\pm K}{1\pm S}
    \end{equation*}
    where
    \begin{align*}
        J(R) &= \e[-2R]\left( 1+\frac{1}{R} \right)&
        S(R) &= \e[-R]\left( 1+R+\frac{R^2}{3} \right)&
        K(R) &= \frac{S(R)}{R}-\e[-R](1+R)
    \end{align*}
    \item We then determine $\vec{c}_n$ in the two different cases. But this yields
    \begin{equation*}
        |c_1| = |c_2| = k
    \end{equation*}
    \item Therefore,
    \begin{equation*}
        \psi_\pm = k(1s_\text{A}\pm 1s_\text{B})
    \end{equation*}
    \begin{figure}[H]
        \centering
        \footnotesize
        \begin{subfigure}[b]{0.4\linewidth}
            \centering
            \begin{tikzpicture}
                \path (-1,0) -- (5,0);
                \draw [stealth-stealth] (0,2) -- node[left]{$\psi_+$} (0,0) -- (4,0);
    
                \node (A) [circle,fill=rex,inner sep=2pt,label={below:A}] at (1,0) {};
                \node (B) [circle,fill=rex,inner sep=2pt,label={below:B}] at (3,0) {}
                    edge [semithick,decorate,decoration={brace,raise=2pt}] node[below=4pt]{$R$} (A)
                ;
    
                \draw [blx,thick] plot[domain=0.1:3.9,samples=500,smooth] (\x,{e^(-1*abs(\x-1))+e^(-1*abs(\x-3))});
            \end{tikzpicture}
            \caption{Sum wave function.}
            \label{fig:H2plusBondinga}
        \end{subfigure}
        \begin{subfigure}[b]{0.4\linewidth}
            \centering
            \begin{tikzpicture}
                \path (-1,0) -- (5,0);
                \draw [stealth-stealth] (0,2) -- node[left]{$\psi_-$} (0,-2);
                \draw [-stealth] (0,0) -- (4,0);
    
                \node (A) [circle,fill=rex,inner sep=2pt,label={below:A}] at (1,0) {};
                \node (B) [circle,fill=rex,inner sep=2pt,label={below:B}] at (3,0) {}
                    edge [semithick,decorate,decoration={brace,raise=1cm}] node[below=1.1cm]{$R$} (A)
                ;
    
                \draw [blx,thick] plot[domain=0.1:3.9,samples=500,smooth] (\x,{e^(-1*abs(\x-1))-e^(-1*abs(\x-3))});
            \end{tikzpicture}
            \caption{Difference wave function.}
            \label{fig:H2plusBondingb}
        \end{subfigure}\\[2em]
        \begin{subfigure}[b]{0.4\linewidth}
            \centering
            \begin{tikzpicture}
                \path (-1,0) -- (5,0);
                \draw [stealth-stealth] (0,2) -- node[left]{$|\psi_+|^2$} (0,0) -- (4,0);
    
                \node (A) [circle,fill=rex,inner sep=2pt,label={below:A}] at (1,0) {};
                \node (B) [circle,fill=rex,inner sep=2pt,label={below:B}] at (3,0) {}
                    edge [semithick,decorate,decoration={brace,raise=2pt}] node[below=4pt]{$R$} (A)
                ;
    
                \draw [blx,thick] plot[domain=0.1:3.9,samples=500,smooth] (\x,{(e^(-1*abs(\x-1))+e^(-1*abs(\x-3)))^2});
            \end{tikzpicture}
            \caption{Sum probability.}
            \label{fig:H2plusBondingc}
        \end{subfigure}
        \begin{subfigure}[b]{0.4\linewidth}
            \centering
            \begin{tikzpicture}
                \path (-1,0) -- (5,0);
                \draw [stealth-stealth] (0,2) -- node[left]{$|\psi_-|^2$} (0,0) -- (4,0);
    
                \node (A) [circle,fill=rex,inner sep=2pt,label={below:A}] at (1,0) {};
                \node (B) [circle,fill=rex,inner sep=2pt,label={below:B}] at (3,0) {}
                    edge [semithick,decorate,decoration={brace,raise=2pt}] node[below=4pt]{$R$} (A)
                ;
    
                \draw [blx,thick] plot[domain=0.1:3.9,samples=500,smooth] (\x,{(e^(-1*abs(\x-1))-e^(-1*abs(\x-3)))^2});
            \end{tikzpicture}
            \caption{Difference probability.}
            \label{fig:H2plusBondingd}
        \end{subfigure}
        \caption{Hydrogen ion bonding.}
        \label{fig:H2plusBonding}
    \end{figure}
    \begin{itemize}
        \item Note that in $\psi_-$, a node arises naturally from the quantum mechanics.
        \item Thus, $\psi_+$ is a bonding orbital and $\psi_-$ is an antibonding orbital.
    \end{itemize}
    \item We now consider the \textbf{potential energy surface} or \textbf{PES} of the molecule.
    \begin{figure}[h!]
        \centering
        \begin{tikzpicture}[xscale=0.5,yscale=2]
            \path (-4,0) -- (13,0);
            \small
            \draw [stealth-stealth] (0,1) -- node[rotate=90,above=9mm]{$E$ (\si{\hartree})} (0,0) -- node[below=7mm]{$R$} (9,0);
            \footnotesize
            \draw
                (0.2,0.5) -- ++(-0.4,0) node[left]{$-0.5$}
                (2.493,0.05) -- ++(0,-0.1) node[below=3mm]{$R_e$}
                foreach \x in {2,4,6,8} {(\x,0.05) -- ++(0,-0.1) node[below]{$\x$}}
            ;
    
            \draw [dashed] (0.2,0.5) -- (8.9,0.5);
    
            \draw [grx,thick] plot[domain=0.735:8.9,samples=500,smooth] (\x,{0.5+((e^(-2*\x)*(1+1/\x))+((e^(-\x)*(1+\x+\x*\x/3))/\x-e^(-\x)*(1+\x)))/(1+(e^(-\x)*(1+\x+\x*\x/3)))});
            \draw [grx,thick] plot[domain=1.625:8.9,samples=500,smooth] (\x,{0.5+((e^(-2*\x)*(1+1/\x))-((e^(-\x)*(1+\x+\x*\x/3))/\x-e^(-\x)*(1+\x)))/(1-(e^(-\x)*(1+\x+\x*\x/3)))});
    
            \node at (3.5,0.8) {$E_-(R)$};
            \node at (3.5,0.3) {$E_+(R)$};
        \end{tikzpicture}
        \caption{Potential energy surface of the hydrogen molecular ion.}
        \label{fig:H2plusPES}
    \end{figure}
    \begin{itemize}
        \item The $x,y$-axis units are Bohrs and Hartrees, respectively.
        \item The bound state only occurs in the bonding orbital; if the electron is excited to the antibonding orbital, the atoms will drift apart to $\infty$ to minimize energy.
    \end{itemize}
    \item \textbf{Born-Oppenheimer approximation}.
    \begin{itemize}
        \item Throughout this derivation, we neglected the kinetic energy of the nuclei.
        \item Thus, technically the total Hamiltonian is
        \begin{equation*}
            \hat{H}_\text{tot} = -\frac{\hbar^2}{2M}(\hat{\nabla}_\text{A}^2+\hat{\nabla}_\text{B}^2)+\hat{H}_\text{electr}
        \end{equation*}
        where $\hat{H}_\text{electr}$ is the Hamiltonian associated with Figure \ref{fig:H2plusDistances}.
        \item We have assumed that the nuclei are fixed relative to the motion of the electrons. We can do this because $m_e/M\approx 10^{-3}$, i.e., the electrons travel much faster than the nuclei. Therefore, the kinetic energy of the electrons is more important.
        \item The wave functions of the nuclei (which do exist) are very sharp peaks, so the nuclei don't move much, so we may regard them as fixed.
    \end{itemize}
    \item \textbf{Molecular orbital}: A linear combination of atomic orbitals. \emph{Also known as} \textbf{MO}.
    \item Example (\ce{H2}):
    \begin{itemize}
        \item $\psi_\pm=1s_\text{A}\pm 1s_\text{B}$ (bonding and antibonding).
        \item $\phi_\text{MO}=\phi_{1s_\text{A}}+\phi_{1s_\text{B}}$.
        \item $\phi(12)=\phi_\text{MO}\alpha(1)\wedge\phi_\text{MO}\beta(2)$; thus, the MO diagram is connected back to the rigorous mathematics of Schr\"{o}dinger.
    \end{itemize}
    \item Filling rules: Fill the MOs that are lower in energy first.
    \item Example (\ce{C2}):
    \begin{itemize}
        \item The MO diagram is identical to Figure III.17 from \textcite{bib:IChemNotes} except that the $\sigma_g$ corresponding to the $2p$ orbitals has higher energy than the $\pi_u$'s due to mixing.
    \end{itemize}
\end{itemize}



\section{Huckel Theory}
\begin{itemize}
    \item \marginnote{11/12:}Reviews that diatomics with more electrons will have less mixing.
    \item \textbf{Bond order}: A measure of stability. \emph{Given by}
    \begin{equation*}
        BO = \frac{1}{2}[(\text{\# of electrons in }BO_1)-(\text{\# of electrons in }ABO_1)]
    \end{equation*}
    \begin{itemize}
        \item For example, for \ce{C2}, we will have $BO=\frac{1}{2}(8-4)=2$.
    \end{itemize}
    \item Huckel was a physicist/chemist from the 1930s who worked out the MO theory for conjugated molecules.
    \item Huckel's strategy.
    \begin{enumerate}
        \item $\mathbb{H}\vec{c}=E\mathbb{S}\vec{c}$ becomes $\mathbb{H}\vec{c}=E\vec{c}$, where the only orbitals considered are the $p_z$ orbitals.
        \item We let $\mathbb{S}=\mathbb{I}$, i.e., we assume all orbitals are orthonormal. $S_{ij}=\delta_{ij}$.
        \item $H_{ii}=\alpha$ (the energy of the electrons in $p_z$).
        \begin{itemize}
            \item $H_{i,i+1}=H_{i+1,i}=\beta$ (interaction between adjacent $p_z$ orbitals).
            \item Everything else is zero!
        \end{itemize}
    \end{enumerate}
    \item Example: Ethene.
    \begin{figure}[h!]
        \centering
        \footnotesize
        \chemfig{@{C1}C(-[:120]H)(-[:-120]H)=@{C2}C(-[:60]H)(-[:-60]H)}
        \chemmove{
            \draw [-,rey,line width=2mm] ([yshift=5mm]C1.90) to[bend left=30] ([yshift=5mm]C2.90);
            \draw [-,rez,line width=2mm] ([yshift=-5mm]C1.-90) to[bend right=30] ([yshift=-5mm]C2.-90);
            %
            \filldraw [-,shorten <=2pt,shorten >=2pt,semithick,draw=rex,fill=rey] (C1.110) to[bend left=110,looseness=30] (C1.70);
            \filldraw [-,shorten <=2pt,shorten >=2pt,semithick,draw=rex,fill=rez] (C1.-70) to[bend left=110,looseness=30] (C1.-110);
            %
            \filldraw [-,shorten <=2pt,shorten >=2pt,semithick,draw=rex,fill=rey] (C2.110) to[bend left=110,looseness=30] (C2.70);
            \filldraw [-,shorten <=2pt,shorten >=2pt,semithick,draw=rex,fill=rez] (C2.-70) to[bend left=110,looseness=30] (C2.-110);
        }
        \caption{Ethene orbital diagram.}
        \label{fig:ethene}
    \end{figure}
    \begin{itemize}
        \item The Hamiltonian matrix is
        \begin{equation*}
            \mathbb{H} =
            \begin{pmatrix}
                \alpha & \beta\\
                \beta & \alpha\\
            \end{pmatrix}
        \end{equation*}
        \item If we let $(\alpha-E)/\beta=x$, then
        \begin{align*}
            0 &=
            \begin{vmatrix}
                \alpha-E & \beta\\
                \beta & \alpha-E\\
            \end{vmatrix}\\
            &= \frac{1}{\beta^2}
            \begin{vmatrix}
                \alpha-E & \beta\\
                \beta & \alpha-E\\
            \end{vmatrix}\\
            &=
            \begin{vmatrix}
                x & 1\\
                1 & x\\
            \end{vmatrix}
        \end{align*}
        \item Thus, our characteristic polynomial is
        \begin{align*}
            0 &= x^2-1\\
            &= (x+1)(x-1)
        \end{align*}
        so $x=\pm 1$.
        \item It follows from returning the substitution that $\alpha\pm\beta=E_\mp$.
        \item This yields the following energy diagram.
        \begin{figure}[h!]
            \centering
            \begin{tikzpicture}[
                yscale=1.2,
                every node/.style={black}
            ]
                \footnotesize
                \draw [stealth-stealth] (0,2) -- (0,0) -- (4,0);
        
                \draw [dashed] (0,1) -- ++(4,0) node[right]{$\alpha$};
        
                \draw [grx,ultra thick] (1.25,1.4) -- ++(0.5,0) node[right]{$\alpha-\beta$};
                \draw [grx,ultra thick] (1.25,0.6) -- node{\large$\upharpoonleft$\hspace{-1mm}$\downharpoonright$} ++(0.5,0) node[right]{$\alpha+\beta$};
            \end{tikzpicture}
            \caption{Ethene energy diagram.}
            \label{fig:etheneEN}
        \end{figure}
        \item Note that the ground state MO is more stable than the 1 electron in each $p_z$ orbital scenario; thus bonding occurs.
        \item It follows that
        \begin{align*}
            \psi_{\alpha+\beta} &= 2p_{z_\text{A}}+2p_{z_\text{B}}&
            \psi_{\alpha-\beta} &= 2p_{z_\text{A}}-2p_{z_\text{B}}
        \end{align*}
        \begin{itemize}
            \item Thus, $\psi_{\alpha+\beta}$ has $\pi_u$ symmetry and $\psi_{\alpha-\beta}$ has $\pi_g$ symmetry.
        \end{itemize}
    \end{itemize}
    \item Example: 1,3-butadiene.
    \begin{itemize}
        \item Our determinant here will be
        \begin{align*}
            0 &=
            \begin{vmatrix}
                \alpha-E & \beta & 0 & 0\\
                \beta & \alpha-E & \beta & 0\\
                0 & \beta & \alpha-E & \beta\\
                0 & 0 & \beta & \alpha-E\\
            \end{vmatrix}\\
            &=
            \begin{vmatrix}
                x & 1 & 0 & 0\\
                1 & x & 1 & 0\\
                0 & 1 & x & 1\\
                0 & 0 & 1 & x\\
            \end{vmatrix}\\
            &= x^4-3x^2+1\\
            &= y^2-3y+1\\
            y &= \frac{3\pm\sqrt{5}}{2}\\
            x &= \pm 1.618,\pm 0.618
        \end{align*}
        \item Thus, our energy diagram will be
        \begin{figure}[h!]
            \centering
            \begin{tikzpicture}[
                yscale=1.2,
                every node/.style={black}
            ]
                \footnotesize
                \draw [stealth-stealth] (0,2) -- (0,0) -- (4,0);
        
                \draw [dashed] (0,1) -- ++(4,0) node[right]{$\alpha$};
        
                \draw [grx,ultra thick] (1.25,{1+0.4*1.618}) -- ++(0.5,0) node[right]{$\alpha-1.618\beta$};
                \draw [grx,ultra thick] (1.25,{1+0.4*0.618}) -- ++(0.5,0) node[right]{$\alpha-0.618\beta$};
                \draw [grx,ultra thick] (1.25,{1-0.4*0.618}) -- node{\large$\upharpoonleft$\hspace{-1mm}$\downharpoonright$} ++(0.5,0) node[right]{$\alpha+0.618\beta$};
                \draw [grx,ultra thick] (1.25,{1-0.4*1.618}) -- node{\large$\upharpoonleft$\hspace{-1mm}$\downharpoonright$} ++(0.5,0) node[right]{$\alpha+1.618\beta$};
            \end{tikzpicture}
            \caption{1,3-butadiene energy diagram.}
            \label{fig:butadieneEN}
        \end{figure}
        \item This yields the bond order $BO=\frac{1}{2}(4-0)=2$.
        \item Solving for the possible $\vec{c}$'s yields four wave function LCAOs analogous to the particle in a box wave functions (even function, odd function with one node, even function with two nodes, odd function with three nodes).
        \begin{itemize}
            \item Thus, the Huckel approximation agrees with the approximation of four particles in a box.
        \end{itemize}
    \end{itemize}
    \item Example 3: 1,3,5-hexatriene.
    \begin{itemize}
        \item Working out the determinant yields
        \begin{align*}
            0 &= x^6-5x^4+6x^2-1\\
            &= y^3-5y^2+6y-1
        \end{align*}
        \item Our energy diagram is as one might expect with 3 bonding orbitals and 3 antibonding orbitals.
    \end{itemize}
\end{itemize}



\section{Office Hours (Mazziotti)}
\begin{itemize}
    \item How does the asymptotic analysis guarantee us the $n=1$ solution, from Problem Set 5.1e?
    \begin{itemize}
        \item $n=1$ dominates in the asymptotic analysis, especially among the ground states.
    \end{itemize}
    \item Math that we're using?
    \begin{itemize}
        \item We're doing a lot of linear algebra, ordinary differential equations, and partial differential equations (boundary condition differential equations). Typically elliptic boundary conditions.
        \item Eigenvalue equations, determinants, etc.
        \item We're in a Hilbert space. Hilbert spaces are subsets of the Banach spaces, so all of that linear algebra applies.
        \item Differential equations in a subspace can be rewritten as eigenvalue equations?
        \item Ties into real analysis and complex analysis with $\e[i\phi]$.
        \item The Grassmann algebra is a piece of classical mathematics that plays in.
        \item Point groups and spin groups fall under abstract algebra; more specifically, Lie algebra.
    \end{itemize}
\end{itemize}



\section{Chapter 9: The Chemical Bond --- Diatomic Molecules}
\emph{From \textcite{bib:McQuarrieSimon}.}
\begin{itemize}
    \item \marginnote{11/8:}Quantum mechanics was the first theory to explain why atoms combined to form a stable bond.
    \item Since \ce{H2+} has the simplest chemical bond, we will discuss it in detail.
    \begin{itemize}
        \item The ideas developed will be applicable to more complex molecules, motivating molecular orbitals.
    \end{itemize}
    \item Describes the Hamiltonian for \ce{H2}, as in the discussion associated with Figure \ref{fig:H2Distances}.
    \item \textbf{Born-Oppenheimer approximation}: The approximation of neglecting the nuclear motion, allowing us to ignore $\nabla_{A,B}$ terms.
    \begin{itemize}
        \item We can correct for the BO approximation using perturbation theory, but realistically we don't really need to (corrections are on the order of the mass ratio $10^{-3}$).
    \end{itemize}
    \item \textbf{Molecular-orbital theory}: The method we will use to describe the bonding properties of molecules.
    \item \textbf{Molecular orbital}: A single-electron wave function corresponding to a molecule.
    \item Like we constructed atomic wave functions in terms of determinants involving atomic orbitals, we will construct molecular wave functions in terms of determinants involving molecular orbitals.
    \item Note that \ce{H2+} is a stable, well-studied species in real life.
    \item Although the Schr\"{o}dinger equation for \ce{H2+} can be solved exactly within the BO approximation, the solutions are not easy to use and their mathematical form gives little physical insight into how and why bonding occurs.
    \begin{itemize}
        \item Thus, we use approximate solutions that provide good physical insight and are in good agreement with experimental observations.
    \end{itemize}
    \item As a first trial wave function $\psi(r_A,r_B;R)$, use
    \begin{equation*}
        \psi_\pm = c_11s_\text{A}\pm c_21s_\text{B}
    \end{equation*}
    where $1s_\text{A,B}$ are the hydrogen atomic orbitals centered on nuclei A and B, respectively.
    \begin{itemize}
        \item By symmetry, $c_1=c_2$ for \ce{H2+}.
    \end{itemize}
    \item \textbf{LCAO molecular orbital}: A molecular orbital that is a linear combination of atomic orbitals.
    \item We seek to find $E_\pm$ by taking the quotient of the integrals $\int\dd{\mathbf{r}}\psi_\pm^*\hat{H}\psi_\pm$ and $\int\dd{\mathbf{r}}\psi_\pm^*\psi_\pm$.
    \item Considering $\int\dd{\mathbf{r}}\psi_\pm^*\psi_\pm$ first, we are led to define the \textbf{overlap integral}.
    \item \textbf{Overlap integral}: The following integral, which is only significant where there is a large overlap between the two hydrogenlike atomic orbitals. \emph{Denoted by} $\bm{S}$. \emph{Given by}
    \begin{equation*}
        S = \int\dd{\mathbf{r}}nl_\text{A}^*nl_\text{B}
        = \int\dd{\mathbf{r}}nl_\text{B}^*nl_\text{A}
        = \int\dd{\mathbf{r}}nl_\text{A}nl_\text{B}
    \end{equation*}
    \begin{figure}[h!]
        \centering
        \begin{subfigure}[b]{0.4\linewidth}
            \centering
            \begin{tikzpicture}
                \footnotesize
                \node (A) [circle,fill=rex,inner sep=2pt,label={above:A}] at (0,-0.5) {};
                \node (B) [circle,fill=rex,inner sep=2pt,label={above:B}] at (2,-0.5) {}
                    edge [<->] node[below]{$R$} (A)
                ;
    
                \draw [blx,thick] plot [domain=-1.5:1.5,smooth,samples=100] (\x,{e^(-2*abs(\x))});
                \draw [blx,thick] plot [domain=-1.5:1.5,smooth,samples=100] ({\x+2},{e^(-2*abs(\x))});
            \end{tikzpicture}
            \caption{Small overlap.}
            \label{fig:overlapIntegrala}
        \end{subfigure}
        \begin{subfigure}[b]{0.4\linewidth}
            \centering
            \begin{tikzpicture}
                \footnotesize
                \node (A) [circle,fill=rex,inner sep=2pt,label={above:A}] at (0,-0.5) {};
                \node (B) [circle,fill=rex,inner sep=2pt,label={above:B}] at (1,-0.5) {}
                    edge [<->] node[below]{$R$} (A)
                ;
    
                \draw [blx,thick] plot [domain=-1.5:1.5,smooth,samples=100] (\x,{e^(-2*abs(\x))});
                \draw [blx,thick] plot [domain=-1.5:1.5,smooth,samples=100] ({\x+1},{e^(-2*abs(\x))});
            \end{tikzpicture}
            \caption{Large overlap.}
            \label{fig:overlapIntegralb}
        \end{subfigure}
        \caption{Overlap integral vs. internuclear distance.}
        \label{fig:overlapIntegral}
    \end{figure}
    \begin{itemize}
        \item As $R\to 0$, $S\to 1$. As $R\to\infty$, $S\to 0$.
    \end{itemize}
    \item The overlap integral when $nl=1s$ can be evaluated analytically, giving
    \begin{equation*}
        S(R) = \e[-R]\left( 1+R+\frac{R^2}{3} \right)
    \end{equation*}
    \item The overlap integral allows us to express $\int\dd{\mathbf{r}}\psi_\pm^*\psi_\pm$ in the following convenient form.
    \begin{align*}
        \int\dd{\mathbf{r}}\psi_+^*\psi_+ &= \int\dd{\mathbf{r}}(1s_\text{A}^*+1s_\text{B}^*)(1s_\text{A}+1s_\text{B})\\
        &= \int\dd{\mathbf{r}}1s_\text{A}^*1s_\text{A}+\int\dd{\mathbf{r}}1s_\text{A}^*1s_\text{B}+\int\dd{\mathbf{r}}1s_\text{B}^*1s_\text{A}+\int\dd{\mathbf{r}}1s_\text{B}^*1s_\text{B}\\
        &= 1+S+S+1\\
        &= 2(1+S)
    \end{align*}
    \item It follows that the normalized $\psi_\pm$ are
    \begin{equation*}
        \psi_\pm = \frac{1}{\sqrt{2(1\pm S)}}(1s_\text{A}\pm 1s_\text{B})
    \end{equation*}
    \item \marginnote{11/14:}We now look to evaluate the integral $\int\dd{\mathbf{r}}\psi_+^*\hat{H}\psi_+$.
    \item To begin, we have that
    \begin{align*}
        \int\dd{\mathbf{r}}\psi_+^*\hat{H}\psi_+ ={}& \int\dd{\mathbf{r}}(1s_\text{A}^*+1s_\text{B}^*)\left( -\frac{1}{2}\nabla^2-\frac{1}{r_\text{A}}-\frac{1}{r_\text{B}}+\frac{1}{R} \right)(1s_\text{A}+1s_\text{B})\\
        \begin{split}
            ={}& \int\dd{\mathbf{r}}(1s_\text{A}^*+1s_\text{B}^*)\left( -\frac{1}{2}\nabla^2-\frac{1}{r_\text{A}}-\frac{1}{r_\text{B}}+\frac{1}{R} \right)1s_\text{A}\\
            & +\int\dd{\mathbf{r}}(1s_\text{A}^*+1s_\text{B}^*)\left( -\frac{1}{2}\nabla^2-\frac{1}{r_\text{A}}-\frac{1}{r_\text{B}}+\frac{1}{R} \right)1s_\text{B}
        \end{split}\\
        \begin{split}
            ={}& \int\dd{\mathbf{r}}(1s_\text{A}^*+1s_\text{B}^*)\left( E_{1s}-\frac{1}{r_\text{B}}+\frac{1}{R} \right)1s_\text{A}\\
            & +\int\dd{\mathbf{r}}(1s_\text{A}^*+1s_\text{B}^*)\left( E_{1s}-\frac{1}{r_\text{A}}+\frac{1}{R} \right)1s_\text{B}
        \end{split}\\
        \begin{split}
            ={}& \int\dd{\mathbf{r}}\left( E_{1s}-\frac{1}{r_\text{B}}+\frac{1}{R} \right)1s_\text{A}^*1s_\text{A}+\int\dd{\mathbf{r}}\left( E_{1s}-\frac{1}{r_\text{B}}+\frac{1}{R} \right)1s_\text{B}^*1s_\text{A}\\
            & +\int\dd{\mathbf{r}}\left( E_{1s}-\frac{1}{r_\text{B}}+\frac{1}{R} \right)1s_\text{A}^*1s_\text{B}+\int\dd{\mathbf{r}}\left( E_{1s}-\frac{1}{r_\text{B}}+\frac{1}{R} \right)1s_\text{B}^*1s_\text{B}
        \end{split}\\
        \begin{split}
            ={}& E_{1s}\cdot 1+\int\dd{\mathbf{r}}\left( -\frac{1}{r_\text{B}}+\frac{1}{R} \right)1s_\text{A}^*1s_\text{A}+E_{1s}\cdot S+\int\dd{\mathbf{r}}\left( -\frac{1}{r_\text{B}}+\frac{1}{R} \right)1s_\text{B}^*1s_\text{A}\\
            & +E_{1s}\cdot 1+\int\dd{\mathbf{r}}\left( -\frac{1}{r_\text{B}}+\frac{1}{R} \right)1s_\text{A}^*1s_\text{B}+E_{1s}\cdot S+\int\dd{\mathbf{r}}\left( -\frac{1}{r_\text{B}}+\frac{1}{R} \right)1s_\text{B}^*1s_\text{B}
        \end{split}\\
        \begin{split}
            ={}& 2E_{1s}(1+S)\\
            & +\int\dd{\mathbf{r}}\left( -\frac{1}{r_\text{B}}+\frac{1}{R} \right)1s_\text{A}^*1s_\text{A}+\int\dd{\mathbf{r}}\left( -\frac{1}{r_\text{B}}+\frac{1}{R} \right)1s_\text{B}^*1s_\text{A}\\
            & +\int\dd{\mathbf{r}}\left( -\frac{1}{r_\text{B}}+\frac{1}{R} \right)1s_\text{A}^*1s_\text{B}+\int\dd{\mathbf{r}}\left( -\frac{1}{r_\text{B}}+\frac{1}{R} \right)1s_\text{B}^*1s_\text{B}
        \end{split}
    \end{align*}
    where $E_{1s}=-E_\text{h}/2$ is the ground-state energy of the hydrogen atom (the energy corresponding to the $1s_\text{A}$ wave function).
    \item \textbf{Coulomb integral}: The following integral, which reflects both the charge density of the electron about nucleus A interacting with nucleus B via the Coulumb potential and the internuclear repulsion. \emph{Denoted by} $\bm{J}$. \emph{Given by}
    \begin{equation*}
        J = \int\dd{\mathbf{r}}1s_\text{A}^*\left( -\frac{1}{r_\text{B}}+\frac{1}{R} \right)1s_\text{A}
    \end{equation*}
    \item \textbf{Exchange integral}: The following integral. \emph{Also known as} \textbf{resonance integral}. \emph{Denoted by} $\bm{K}$. \emph{Given by}
    \begin{equation*}
        K = \int\dd{\mathbf{r}}1s_\text{B}^*\left( -\frac{1}{r_\text{B}}+\frac{1}{R} \right)1s_\text{A}
    \end{equation*}
    \begin{itemize}
        \item The exchange integral describes a purely quantum-mechanical effect; it has no analogy in classical mechanics.
    \end{itemize}
    \item Using the definitions of the Coluomb and exchange integrals, we have that
    \begin{equation*}
        \int\dd{\mathbf{r}}\psi_+^*\hat{H}\psi_+ = 2E_{1s}(1+S)+2J+2K
    \end{equation*}
    \item We can now determine a final expression for the energy $E_+$ corresponding to $\psi_+$.
    \begin{align*}
        E_+ &= \frac{\int\dd{\mathbf{r}}\psi_+^*\hat{H}\psi_+}{\int\dd{\mathbf{r}}\psi_+^*\psi_+}\\
        &= \frac{2E_{1s}(1+S)+2J+2K}{2(1+S)}\\
        &= E_{1s}+\frac{J+K}{1+S}
    \end{align*}
    \item The Coulomb and exchange integrals can be evaluated analytically, giving
    \begin{align*}
        J &= \e[-2R]\left( 1+\frac{1}{R} \right)&
        K &= \frac{S}{R}-\e[-R](1+R)
    \end{align*}
    \begin{itemize}
        \item Note that since the Coulomb integral is always positive, the exchange integral is entirely responsible for the existence of the chemical bond in \ce{H2+}.
        \item This highlights the importance of the quantum-mechanical nature of the chemical bond.
    \end{itemize}
    \item \textbf{Bonding orbital}: A state that exhibits a stable chemical bond.
    \begin{itemize}
        \item Example: $\psi_+$.
    \end{itemize}
    \item \textbf{Antibonding orbital}: A state that leads to a repulsive interaction between the two nuclei for all internuclear distances.
    \begin{itemize}
        \item Example: $\psi_-$.
    \end{itemize}
    \item The bonding orbital $\psi_\text{b}=\psi_+$ describes the ground state of \ce{H2+}, while the antibonding orbital $\psi_\text{a}=\psi_-$ describes an excited state.
    \item Note that we have only found two molecular orbitals (as opposed to the infinitely many hydrogenlike atomic orbitals) because we used a trial wave function that was the linear combination of only two atomic orbitals. We could use longer linear combinations to find more molecular orbitals, but for pedagogical reasons, we will limit ourselves to two for now.
    \item "Because $\psi_\text{b}$ is the molecular orbital corresponding to the ground-state energy of \ce{H2+}, we can describe the ground state of \ce{H2} by placing two electrons with opposite spins in $\psi_\text{b}$, just as we placed two electrons in a $1s$ atomic orbital to describe the helium atom" \parencite[336]{bib:McQuarrieSimon}.
    \item This leads to the Slater determinantal wave function
    \begin{equation*}
        \psi = \frac{1}{\sqrt{2!}}
        \begin{vmatrix}
            \psi_\text{b}\alpha(1) & \psi_\text{b}\beta(1)\\
            \psi_\text{b}\alpha(2) & \psi_\text{b}\beta(2)\\
        \end{vmatrix}
    \end{equation*}
    for \ce{H2}.
    \begin{itemize}
        \item As before, the spatial and spin parts of this two-electron wave function separate.
    \end{itemize}
    \item It follows that the spatial molecular wave function is
    \begin{align*}
        \psi_\text{MO} &= \psi_\text{b}(1)\psi_\text{b}(2)\\
        &= \frac{1}{2(1+S)}[1s_\text{A}(1)+1s_\text{B}(1)][1s_\text{A}(2)+1s_\text{B}(2)]
    \end{align*}
    \item \textbf{Linear combination of atomic orbitals-molecular orbitals method}: The method of constructing molecular wave functions by taking the product of molecular orbitals, which are in turn linear combinations of atomic orbitals. \emph{Also known as} \textbf{LCAO-MO method}.
    \begin{itemize}
        \item This is how we constructed $\psi_\text{MO}$, above.
    \end{itemize}
    \item Knowing that $\psi_\text{MO}$ is normalized, we have that
    \begin{equation*}
        E_\text{MO} = \int\dd{\mathbf{r}_1}\dd{\mathbf{r}_2}\psi_\text{MO}^*(1,2)\hat{H}\psi_\text{MO}(1,2)
    \end{equation*}
    \item \textbf{$\bm{\sigma}$ orbital}: An orbital that is symmetric about the internuclear axis.
    \begin{itemize}
        \item Both $\psi_\pm$ are $\sigma$ orbitals.
        \item Molecular orbitals constructed from $1s$ orbitals are denoted by $\sigma 1s$.
    \end{itemize}
    \item Distinguishing between $\sigma 1s$ bonding and antibonding orbitals.
    \begin{itemize}
        \item We may let $\sigma 1s$ refer to $\psi_+$ and $\sigma^*1s$ refer to $\psi_-$.
        \item We may let $\sigma_g1s$ refer to $\psi_+$ ($g$ being short for "gerade" [the German word for even] since $\psi_+$ does not change sign under inversion) and $\sigma_u1s$ refer to $\psi_-$ ($u$ being short for "ungerade" [the German word for odd] since $\psi_-$ does change sign under inversion).
        \item \textcite{bib:McQuarrieSimon} will favor the latter notation.
    \end{itemize}
    \item $\sigma 2s$ molecular orbitals.
    \begin{itemize}
        \item Of the form $2s_\text{A}\pm 2s_\text{B}$.
        \item Have radial nodes about each nuclei (owing to the structure of the $2s$ hydrogenlike atomic orbital) in addition to the possible planar node in the middle of the molecule.
        \item Have higher energies than $\sigma 1s$ MOs since they're associated with higher energy atomic orbitals. This can be demonstrated rigorously the same way we calculated the energies for the $\sigma 1s$ MOs.
        \item In terms of energy, $\sigma_g1s<\sigma_u1s<\sigma_g2s<\sigma_u2s$.
    \end{itemize}
    \item $2p$ molecular orbitals.
    \begin{itemize}
        \item Choose the internuclear axis to be the $z$-axis.
        \item $2p_{z_\text{A}}\pm 2p_{z_\text{B}}$ are cylindrically symmetric about the internuclear axis and are therefore $\sigma_u2p_z$ and $\sigma_g2p_z$ orbitals, respectively.
        \item $2p_{x,y_\text{A}}\pm 2p_{x,y_\text{B}}$ are \textbf{$\bm{\pi}$ orbitals}, denoted $\pi_u2p_{x,y}$ and $\pi_g2p_{x,y}$, respectively.
        \item $\pi_u2p_x,\pi_u2p_y$ and $\pi_g2p_x,\pi_g2p_y$ are degenerate pairs of orbitals since they are identical except for their spatial orientation.
    \end{itemize}
    \item \textbf{$\bm{\pi}$ orbital}: A molecular orbital with one nodal plane that contains the internuclear axis.
    \item From \ce{Li2} through \ce{N2}, the energy order is
    \begin{equation*}
        \sigma_g2s < \sigma_u2s
        < \pi_u2p_{x,y}
        < \sigma_g2p_z
        < \pi_g2p_{x,y}
        < \sigma_u2p_z
    \end{equation*}
    \begin{itemize}
        \item From \ce{O2} through \ce{F2}, the energy order is
        \begin{equation*}
            \sigma_g2s < \sigma_u2s
            < \sigma_g2p_z
            < \pi_u2p_{x,y}
            < \pi_g2p_{x,y}
            < \sigma_u2p_z
        \end{equation*}
    \end{itemize}
    \item Diatomic helium does not exist since the bonding and antibonding electrons cancel each other out, leaving only repulsive nuclei.
    \begin{itemize}
        \item Bond order formalizes this notion.
        \item Note that in 1993, Gentry and co. reported spectroscopic observations of \ce{He2} in a gas-phase sample of helium of temperature $\SI{0.001}{\kelvin}$, possessing by far the weakest chemical bond known.
    \end{itemize}
    \item Note that since there is little difference between the electron densities of core electron-derived molecular orbitals, we need only consider electrons in the valence shell when discussing chemical bonding.
    \item We denote the filled $n=1$ shell by $K$, e.g., the ground state electron configuration of \ce{Li2} is $KK(\sigma_g2s)^2$.
    \item MO theory accurately predicts that \ce{O2} is paramagnetic.
    \begin{itemize}
        \item MO theory also predicts bond orders of $2.5,2,1.5,1$ for \ce{O2+}, \ce{O2}, \ce{O2-}, \ce{O2^2-}, respectively. This would suggest that bond length increases and bond energy decreases down this list, as we do indeed observe experimentally.
    \end{itemize}
    \item \textcite{bib:McQuarrieSimon} provides data on the ground-state electron configuration, bond order, bond length, and bond energy for the homonuclear diatomics of lithium through neon.
    \item \textbf{Ionization energy}: The energy required to eject an electron from a molecule.
    \item \textbf{Photoelectron spectroscopy}: The measurement of the energies of the electrons ejected by radiation incident on gaseous molecules.
    \begin{itemize}
        \item Goes through how PES supports MO theory, as on \textcite[56]{bib:IChemNotes}.
    \end{itemize}
    \item \textbf{Heteronuclear diatomic molecule}: A diatomic molecule in which the two nuclei are different.
    \item For heteronuclear diatomics whose constituent atoms' atomic numbers differ by 1 or 2, the scheme we used for homonuclear diatomics is good enough.
    \item For heteronuclear diatomics whose constituent atoms' atomic numbers differ greatly (e.g., \ce{HF}), we need a new scheme.
    \begin{itemize}
        \item Noting that the energy of the $1s$ atomic orbital of \ce{H} is similar to the energy of the $2p$ atomic orbitals of \ce{F}, we look for constructive and destructive interference and find that $1s_{\ce{H}}$ and $2p_{z_{\ce{F}}}$ interact nicely.
        \item Thus, a first approximation the molecular orbital would be $\psi=c_11s_{\ce{H}}+c_22p_{z_{\ce{F}}}$.
        \item Note that both MOs have $\sigma$ symmetry.
    \end{itemize}
    \item The scheme presented thus far is the simplest possible.
    \item More generally, we may create trial wave functions that include more and orbitals, minimizing the coefficients with the variational method along the way, until we reach the Hartree-Fock limit.
    \item \textbf{SCF-LCAO-MO wave function}: A wave function obtained from molecular orbitals that are linear combinations of atomic orbitals whose coefficients are determined by a self-consistent field method.
    \begin{itemize}
        \item Such orbitals, obtained from many atomic orbitals, cannot be meaningfully classified as $\sigma 2s$ or $\pi 2p$, for example, so we classify them as the first $\sigma_g$ orbital ($1\sigma_g$), the first $\sigma_u$ orbital ($1\sigma_u$), and so on and so forth.
        \item An SCF-LCAO-MO is only the same as a Hartree-Fock orbital if the SCF-LCAO-MO contains enough terms to reach the Hartree-Fock limit.
    \end{itemize}
    \item \textbf{Hartree-Fock-Roothaan method}: The method for determining a SCF-LCAO-MO wave function.
    \item \textcite{bib:McQuarrieSimon} lists data on the Hartree-Fock limit vs. experimental calculations.
    \item Molecular term symbols.
    \item Excited states of molecules.
    \begin{itemize}
        \item Consider \ce{H2} with one electron in the $\sigma_g1s$ orbital and one electron in the $\sigma_g2s$ orbital. This species has a slightly longer bond length than ground state \ce{H2}.
    \end{itemize}
\end{itemize}



\section{Chapter 10: Bonding in Polyatomic Molecules}
\emph{From \textcite{bib:McQuarrieSimon}.}
\begin{itemize}
    \item \marginnote{11/15:}\textbf{$\bm{\pi}$-electron approximation}: The approximation of the delocalized $\pi$ electrons of a conjugated polyene or benzene as moving in some fixed, effective, electrostatic potential due to the electrons in the $\sigma$ framework.
    \begin{itemize}
        \item Note that this can be rigorously developed from the Schr\"{o}dinger equation.
        \item We may write the wave function of the $\pi$ orbital of ethene as
        \begin{equation*}
            \psi_\pi = c_12p_{z_{\ce{A}}}+c_22p_{z_{\ce{B}}}
        \end{equation*}
        \item The diagonal entries in the corresponding secular determinant are \textbf{Coulomb integrals} and the off-diagonal entries are \textbf{resonance integrals}.
        \item To evaluate the determinant, we either need to specify the Hamiltonian or come up with another approximation for the integrals.
    \end{itemize}
    \item \textbf{H\"{u}ckel molecular-orbital theory}: An approximation composed of the following three assertions.
    \begin{enumerate}
        \item The overlap integrals $S_{ij}=\delta_{ij}$.
        \item All of the Coulomb integrals are assumed to be the same for all carbon atoms and are commonly denoted by $\alpha$.
        \item The resonance integrals involving nearest-neighbor carbon atoms are assumed to be the same and are denoted by $\beta$.
    \end{enumerate}
    \item Once we calculate $E$ from the reworked secular determinant, we plug in experimental values for $\alpha$ and $\beta$ and go from there (no effective Hamiltonian needed).
    \item \textcite{bib:McQuarrieSimon} covers the same butadiene derivation from class.
    \begin{itemize}
        \item Addendum: The total $\pi$-electronic energy of butadiene is
        \begin{align*}
            E_\pi &= 2(\alpha+1.618\beta)+2(\alpha+0.618\beta)\\
            &= 4\alpha+4.472\beta
        \end{align*}
        \item Comparing this energy to the energy of two ethene molecules gives us the \textbf{delocalization energy}.
        \item The molecular orbitals are given and their schematic diagrams drawn.
    \end{itemize}
    \item \textcite{bib:McQuarrieSimon} treats benzene the same way. See Nocera Lectures 6-7 from \textcite{bib:IChemNotes}.
\end{itemize}




\end{document}