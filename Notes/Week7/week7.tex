\documentclass[../notes.tex]{subfiles}

\pagestyle{main}
\renewcommand{\chaptermark}[1]{\markboth{\chaptername\ \thechapter\ (#1)}{}}
\setcounter{chapter}{6}

\begin{document}




\chapter{Molecular Spectroscopy and Group Theory}
\section{Hydrogen Molecule}
\begin{itemize}
    \item \marginnote{11/8:}Reviews what would happen if electrons were bozons.
    \begin{itemize}
        \item The wave function $\psi$ of the lithium atom would be
        \begin{equation*}
            \psi(123) = 1s\alpha(1)\vee 1s\alpha(2)\vee 1s\alpha(3)
        \end{equation*}
        \item Reviews wedge products.
        \item Conclusion: All electrons could occupy the same orbital.
    \end{itemize}
    \item A sodium atom (the nucleus and the electrons jointly) acts like a bozon.
    \begin{itemize}
        \item At temperatures on the order of microkelvin, $10^{11}$ atoms have been placed in the same ground-state orbital.
        \item These substances are known as \textbf{Bose-Einstein condensates}.
        \item We use a magnetic field to confine the atoms to a harmonic potential. Since the atoms form a Gaussian curve at the bottom of said potential, they are all in the ground state (see Figure \ref{fig:harmonicPrbDensity}).
        \item Evaporative cooling and laser cooling allow you to reach such temperatures.
        \item Fermionic atoms cannot condense in such a way (because of the Pauli Exclusion Principle).
        \item Superconductivity is a condensation phenomena.
        \item Bose-Einstein condensates were predicted by Einstein in the 1930s but not experimentally verified until the 1990s.
        \item A very dilute gas was used here. In such a condition, the atoms feel the statistics force of the wedge product which forces them into such a state.
    \end{itemize}
    \item Consider the Boron atom:
    \begin{itemize}
        \item It has 5 electrons.
        \item It's electron configuration is $1s^22s^2sp^1$.
        \item It's wave function is
        \begin{equation*}
            \psi(12345) = 1s\alpha(1)\wedge 1s\beta(2)\wedge 2s\alpha(3)\wedge 2s\beta(4)\wedge 2p\alpha(5)
        \end{equation*}
        \begin{itemize}
            \item Recall that this is not the exact wave function; this is still a product of hydrogenlike orbitals at the Hartree-Fock level.
            \item More general wave functions can be used to obtain more accurate results.
        \end{itemize}
    \end{itemize}
    \item Consider the diatomic molecule \ce{H2}.
    \begin{figure}[H]
        \centering
        \begin{tikzpicture}[scale=1.5]
            \footnotesize
            \coordinate (HA) at (0,0);
            \coordinate (HB) at (2,0);
            \coordinate (e1) at (0.3,1);
            \coordinate (e2) at (1.7,1);
    
            \draw
                (HA)
                    -- node[below]{$R$} (HB)
                    -- node[right]{$r_{2\text{B}}$} (e2)
                    -- node[above]{$r_{12}$} (e1)
                    -- node[left]{$r_{1\text{A}}$} cycle
                (HA) -- node[pos=0.3,above=2pt]{$r_{2\text{A}}$} (e2)
                (e1) -- node[pos=0.7,above=1pt]{$r_{1\text{B}}$} (HB)
            ;
    
            \node [circle,fill=rex,inner sep=2pt,label={below:\ce{H_A}}] at (HA) {};
            \node [circle,fill=rex,inner sep=2pt,label={below:\ce{H_B}}] at (HB) {};
            \node [circle,fill=grx,inner sep=1.5pt,label={above:$\e_1$}] at (e1) {};
            \node [circle,fill=grx,inner sep=1.5pt,label={above:$\e_2$}] at (e2) {};
        \end{tikzpicture}
        \caption{\ce{H2} distances.}
        \label{fig:H2Distances}
    \end{figure}
    \begin{itemize}
        \item We have hydrogen atoms \ce{H_A} and \ce{H_B}, separated by a distance $R$.
        \item We have electrons $\e_1$ and $\e_2$.
        \item The distance from object $i$ to $j$ where $i,j=\text{A},\text{B},1,2$ and $i\neq j$ is $r_{ij}$.
        \item Hamiltonian:
        \begin{equation*}
            \hat{H} = -\frac{1}{2}(\nabla_1^2+\nabla_2^2)-\frac{1}{r_{1\text{A}}}-\frac{1}{r_{1\text{B}}}-\frac{1}{r_{2\text{A}}}-\frac{1}{r_{2\text{B}}}+\frac{1}{r_{12}}+\frac{1}{R}
        \end{equation*}
    \end{itemize}
    \item This is pretty complicated.
    \item Thus, let's start with the hydrogen molecular ion (\ce{H2+}).
    \begin{figure}[h!]
        \centering
        \begin{tikzpicture}[scale=1.5]
            \footnotesize
            \coordinate (HA) at (0,0);
            \coordinate (HB) at (1,0);
            \coordinate (e1) at (0.5,1);
    
            \draw
                (HA)
                -- node[below]{$R$} (HB)
                -- node[right]{$r_\text{B}$} (e1)
                -- node[left]{$r_\text{A}$} cycle
            ;
    
            \node [circle,fill=rex,inner sep=2pt,label={below:\ce{H_A}}] at (HA) {};
            \node [circle,fill=rex,inner sep=2pt,label={below:\ce{H_B}}] at (HB) {};
            \node [circle,fill=grx,inner sep=1.5pt,label={above:$\e_1$}] at (e1) {};
        \end{tikzpicture}
        \caption{\ce{H2+} distances.}
        \label{fig:H2plusDistances}
    \end{figure}
    \begin{itemize}
        \item Hamiltonian:
        \begin{equation*}
            \hat{H} = -\frac{1}{2}\nabla^2-\frac{1}{r_\text{A}}-\frac{1}{r_\text{B}}+\frac{1}{R}
        \end{equation*}
        \item This \emph{can} be solved exactly in cylindrical coordinates, but it's nasty.
        \item Thus, let's approximate with the following variational wave function (originally by Heitler and London in the 1960s).
        \begin{equation*}
            \psi(1) = c_11s_\text{A}+c_21s_\text{B}
        \end{equation*}
        \begin{itemize}
            \item Albeit simple, this wave function gives pretty good results.
        \end{itemize}
        \item By the variational principle, $\mathbb{H}\vec{c}=E\mathbb{S}\vec{c}$, or
        \begin{equation*}
            \begin{vmatrix}
                H_{\text{A}\text{A}}-ES_{\text{A}\text{A}} & H_{\text{A}\text{B}}-ES_{\text{A}\text{B}}\\
                H_{\text{B}\text{A}}-ES_{\text{B}\text{A}} & H_{\text{B}\text{B}}-ES_{\text{B}\text{B}}\\
            \end{vmatrix}
            = 0
        \end{equation*}
        solves for $E$.
        \item We have that
        \begin{align*}
            H_{\text{A}\text{A}} = H_{\text{B}\text{B}} &= \int\dd{\vec{r}}1s_\text{A}^*\hat{H}1s_\text{A} = \int\dd{\vec{r}}1s_\text{B}^*\hat{H}1s_\text{B}\\
            H_{\text{A}\text{B}} = H_{\text{B}\text{A}} &= \int\dd{\vec{r}}1s_\text{A}^*\hat{H}1s_\text{B} = \int\dd{\vec{r}}1s_\text{B}^*\hat{H}1s_\text{A}\\
            S_{\text{A}\text{A}} = S_{\text{B}\text{B}} &= \int\dd{\vec{r}}1s_\text{A}^*1s_\text{A} = \int\dd{\vec{r}}1s_\text{B}^*1s_\text{B}\\
            S_{\text{A}\text{B}} = S_{\text{B}\text{A}} &= \int\dd{\vec{r}}1s_\text{A}^*1s_\text{B} = \int\dd{\vec{r}}1s_\text{B}^*1s_\text{A}
        \end{align*}
        \item We can show that
        \begin{align*}
            H_{\text{A}\text{A}} = H_{\text{B}\text{B}} &= E_{1s}+J\\
            H_{\text{A}\text{B}} = H_{\text{B}\text{A}} &= E_{1s}\mathbb{S}+K
        \end{align*}
        where $E_{1s}$ is the energy of the $1s$ orbital of the hydrogen atom, $J$ is the \textbf{Coulomb integral}
        \begin{equation*}
            J = \int\dd{\vec{r}}1s_\text{A}^*\left( -\frac{1}{r_\text{B}}+\frac{1}{R} \right)1s_\text{A}
        \end{equation*}
        and $K$ is the \textbf{exchange integral}
        \begin{equation*}
            K = \int\dd{\vec{r}}1s_\text{B}^*\left( -\frac{1}{r_\text{B}}+\frac{1}{R} \right)1s_\text{A}
        \end{equation*}
    \end{itemize}
\end{itemize}



\section{The Hydrogen Molecular Ion}
\begin{itemize}
    \item \marginnote{11/10:}Continuing from last time, the determinant for the equation $\mathbb{H}\vec{c}_n=E_n\mathbb{S}\vec{c}_n$ is
    \begin{equation*}
        \begin{vmatrix}
            E_{1s}+J-E & E_{1s}S+K\\
            E_{1s}S+K & E_{1s}+J-E\\
        \end{vmatrix}
    \end{equation*}
    \item Therefore, the characteristic polynomial is a quadratic equation in $E$.
    \item Solving said quadratic gives us
    \begin{equation*}
        E_\pm = E_{1s}+\frac{J\pm K}{1\pm S}
    \end{equation*}
    where
    \begin{align*}
        J(R) &= \e[-2R]\left( 1+\frac{1}{R} \right)&
        S(R) &= \e[-R]\left( 1+R+\frac{R^2}{3} \right)&
        K(R) &= \frac{S(R)}{R}-\e[-R](1+R)
    \end{align*}
    \item We then determine $\vec{c}_n$ in the two different cases. But this yields
    \begin{equation*}
        |c_1| = |c_2| = k
    \end{equation*}
    \item Therefore,
    \begin{equation*}
        \psi_\pm = k(1s_\text{A}\pm 1s_\text{B})
    \end{equation*}
    \begin{figure}[H]
        \centering
        \footnotesize
        \begin{subfigure}[b]{0.4\linewidth}
            \centering
            \begin{tikzpicture}
                \path (-1,0) -- (5,0);
                \draw [stealth-stealth] (0,2) -- node[left]{$\psi_+$} (0,0) -- (4,0);
    
                \node (A) [circle,fill=rex,inner sep=2pt,label={below:A}] at (1,0) {};
                \node (B) [circle,fill=rex,inner sep=2pt,label={below:B}] at (3,0) {}
                    edge [semithick,decorate,decoration={brace,raise=2pt}] node[below=4pt]{$R$} (A)
                ;
    
                \draw [blx,thick] plot[domain=0.1:3.9,samples=500,smooth] (\x,{e^(-1*abs(\x-1))+e^(-1*abs(\x-3))});
            \end{tikzpicture}
            \caption{Sum wave function.}
            \label{fig:H2plusBondinga}
        \end{subfigure}
        \begin{subfigure}[b]{0.4\linewidth}
            \centering
            \begin{tikzpicture}
                \path (-1,0) -- (5,0);
                \draw [stealth-stealth] (0,2) -- node[left]{$\psi_-$} (0,-2);
                \draw [-stealth] (0,0) -- (4,0);
    
                \node (A) [circle,fill=rex,inner sep=2pt,label={below:A}] at (1,0) {};
                \node (B) [circle,fill=rex,inner sep=2pt,label={below:B}] at (3,0) {}
                    edge [semithick,decorate,decoration={brace,raise=1cm}] node[below=1.1cm]{$R$} (A)
                ;
    
                \draw [blx,thick] plot[domain=0.1:3.9,samples=500,smooth] (\x,{e^(-1*abs(\x-1))-e^(-1*abs(\x-3))});
            \end{tikzpicture}
            \caption{Difference wave function.}
            \label{fig:H2plusBondingb}
        \end{subfigure}\\[2em]
        \begin{subfigure}[b]{0.4\linewidth}
            \centering
            \begin{tikzpicture}
                \path (-1,0) -- (5,0);
                \draw [stealth-stealth] (0,2) -- node[left]{$|\psi_+|^2$} (0,0) -- (4,0);
    
                \node (A) [circle,fill=rex,inner sep=2pt,label={below:A}] at (1,0) {};
                \node (B) [circle,fill=rex,inner sep=2pt,label={below:B}] at (3,0) {}
                    edge [semithick,decorate,decoration={brace,raise=2pt}] node[below=4pt]{$R$} (A)
                ;
    
                \draw [blx,thick] plot[domain=0.1:3.9,samples=500,smooth] (\x,{(e^(-1*abs(\x-1))+e^(-1*abs(\x-3)))^2});
            \end{tikzpicture}
            \caption{Sum probability.}
            \label{fig:H2plusBondingc}
        \end{subfigure}
        \begin{subfigure}[b]{0.4\linewidth}
            \centering
            \begin{tikzpicture}
                \path (-1,0) -- (5,0);
                \draw [stealth-stealth] (0,2) -- node[left]{$|\psi_-|^2$} (0,0) -- (4,0);
    
                \node (A) [circle,fill=rex,inner sep=2pt,label={below:A}] at (1,0) {};
                \node (B) [circle,fill=rex,inner sep=2pt,label={below:B}] at (3,0) {}
                    edge [semithick,decorate,decoration={brace,raise=2pt}] node[below=4pt]{$R$} (A)
                ;
    
                \draw [blx,thick] plot[domain=0.1:3.9,samples=500,smooth] (\x,{(e^(-1*abs(\x-1))-e^(-1*abs(\x-3)))^2});
            \end{tikzpicture}
            \caption{Difference probability.}
            \label{fig:H2plusBondingd}
        \end{subfigure}
        \caption{Hydrogen ion bonding.}
        \label{fig:H2plusBonding}
    \end{figure}
    \begin{itemize}
        \item Note that in $\psi_-$, a node arises naturally from the quantum mechanics.
        \item Thus, $\psi_+$ is a bonding orbital and $\psi_-$ is an antibonding orbital.
    \end{itemize}
    \item We now consider the \textbf{potential energy surface} or \textbf{PES} of the molecule.
    \begin{figure}[h!]
        \centering
        \begin{tikzpicture}[xscale=0.5,yscale=2]
            \path (-4,0) -- (13,0);
            \small
            \draw [stealth-stealth] (0,1) -- node[rotate=90,above=9mm]{$E$ (\si{\hartree})} (0,0) -- node[below=7mm]{$R$} (9,0);
            \footnotesize
            \draw
                (0.2,0.5) -- ++(-0.4,0) node[left]{$-0.5$}
                (2.493,0.05) -- ++(0,-0.1) node[below=3mm]{$R_e$}
                foreach \x in {2,4,6,8} {(\x,0.05) -- ++(0,-0.1) node[below]{$\x$}}
            ;
    
            \draw [dashed] (0.2,0.5) -- (8.9,0.5);
    
            \draw [grx,thick] plot[domain=0.735:8.9,samples=500,smooth] (\x,{0.5+((e^(-2*\x)*(1+1/\x))+((e^(-\x)*(1+\x+\x*\x/3))/\x-e^(-\x)*(1+\x)))/(1+(e^(-\x)*(1+\x+\x*\x/3)))});
            \draw [grx,thick] plot[domain=1.625:8.9,samples=500,smooth] (\x,{0.5+((e^(-2*\x)*(1+1/\x))-((e^(-\x)*(1+\x+\x*\x/3))/\x-e^(-\x)*(1+\x)))/(1-(e^(-\x)*(1+\x+\x*\x/3)))});
    
            \node at (3.5,0.8) {$E_-(R)$};
            \node at (3.5,0.3) {$E_+(R)$};
        \end{tikzpicture}
        \caption{Potential energy surface of the hydrogen molecular ion.}
        \label{fig:H2plusPES}
    \end{figure}
    \begin{itemize}
        \item The $x,y$-axis units are Bohrs and Hartrees, respectively.
        \item The bound state only occurs in the bonding orbital; if the electron is excited to the antibonding orbital, the atoms will drift apart to $\infty$ to minimize energy.
    \end{itemize}
    \item \textbf{Born-Oppenheimer approximation}.
    \begin{itemize}
        \item Throughout this derivation, we neglected the kinetic energy of the nuclei.
        \item Thus, technically the total Hamiltonian is
        \begin{equation*}
            \hat{H}_\text{tot} = -\frac{\hbar^2}{2M}(\hat{\nabla}_\text{A}^2+\hat{\nabla}_\text{B}^2)+\hat{H}_\text{electr}
        \end{equation*}
        where $\hat{H}_\text{electr}$ is the Hamiltonian associated with Figure \ref{fig:H2plusDistances}.
        \item We have assumed that the nuclei are fixed relative to the motion of the electrons. We can do this because $m_e/M\approx 10^{-3}$, i.e., the electrons travel much faster than the nuclei. Therefore, the kinetic energy of the electrons is more important.
        \item The wave functions of the nuclei (which do exist) are very sharp peaks, so the nuclei don't move much, so we may regard them as fixed.
    \end{itemize}
    \item \textbf{Molecular orbital}: A linear combination of atomic orbitals. \emph{Also known as} \textbf{MO}.
    \item Example (\ce{H2}):
    \begin{itemize}
        \item $\psi_\pm=1s_\text{A}\pm 1s_\text{B}$ (bonding and antibonding).
        \item $\phi_\text{MO}=\phi_{1s_\text{A}}+\phi_{1s_\text{B}}$.
        \item $\phi(12)=\phi_\text{MO}\alpha(1)\wedge\phi_\text{MO}\beta(2)$; thus, the MO diagram is connected back to the rigorous mathematics of Schr\"{o}dinger.
    \end{itemize}
    \item Filling rules: Fill the MOs that are lower in energy first.
    \item Example (\ce{C2}):
    \begin{itemize}
        \item The MO diagram is identical to Figure III.17 from \textcite{bib:IChemNotes} except that the $\sigma_g$ corresponding to the $2p$ orbitals has higher energy than the $\pi_u$'s due to mixing.
    \end{itemize}
\end{itemize}



\section{Chapter 9: The Chemical Bond --- Diatomic Molecules}
\emph{From \textcite{bib:McQuarrieSimon}.}
\begin{itemize}
    \item \marginnote{11/8:}Quantum mechanics was the first theory to explain why atoms combined to form a stable bond.
    \item Since \ce{H2+} has the simplest chemical bond, we will discuss it in detail.
    \begin{itemize}
        \item The ideas developed will be applicable to more complex molecules, motivating molecular orbitals.
    \end{itemize}
    \item Describes the Hamiltonian for \ce{H2}, as in the discussion associated with Figure \ref{fig:H2Distances}.
    \item \textbf{Born-Oppenheimer approximation}: The approximation of neglecting the nuclear motion, allowing us to ignore $\nabla_{A,B}$ terms.
    \begin{itemize}
        \item We can correct for the BO approximation using perturbation theory, but realistically we don't really need to (corrections are on the order of the mass ratio $10^{-3}$).
    \end{itemize}
    \item \textbf{Molecular-orbital theory}: The method we will use to describe the bonding properties of molecules.
    \item \textbf{Molecular orbital}: A single-electron wave function corresponding to a molecule.
    \item Like we constructed atomic wave functions in terms of determinants involving atomic orbitals, we will construct molecular wave functions in terms of determinants involving molecular orbitals.
    \item Note that \ce{H2+} is a stable, well-studied species in real life.
    \item Although the Schr\"{o}dinger equation for \ce{H2+} can be solved exactly within the BO approximation, the solutions are not easy to use and their mathematical form gives little physical insight into how and why bonding occurs.
    \begin{itemize}
        \item Thus, we use approximate solutions that provide good physical insight and are in good agreement with experimental observations.
    \end{itemize}
    \item As a first trial wave function $\psi(r_A,r_B;R)$, use
    \begin{equation*}
        \psi_\pm = c_11s_\text{A}\pm c_21s_\text{B}
    \end{equation*}
    where $1s_\text{A,B}$ are the hydrogen atomic orbitals centered on nuclei A and B, respectively.
    \begin{itemize}
        \item By symmetry, $c_1=c_2$ for \ce{H2+}.
    \end{itemize}
    \item \textbf{LCAO molecular orbital}: A molecular orbital that is a linear combination of atomic orbitals.
    \item \textbf{Overlap integral}: The following integral. \emph{Denoted by} $\bm{S}$. \emph{Given by}
    \begin{equation*}
        S = \int\dd{\mathbf{r}}nl_\text{A}^*nl_\text{B}
        = \int\dd{\mathbf{r}}nl_\text{B}^*nl_\text{A}
        = \int\dd{\mathbf{r}}nl_\text{A}nl_\text{B}
    \end{equation*}
    \begin{figure}[h!]
        \centering
        \begin{subfigure}[b]{0.4\linewidth}
            \centering
            \begin{tikzpicture}
                \footnotesize
                \node (A) [circle,fill=rex,inner sep=2pt,label={above:A}] at (0,-0.5) {};
                \node (B) [circle,fill=rex,inner sep=2pt,label={above:B}] at (2,-0.5) {}
                    edge [<->] node[below]{$R$} (A)
                ;
    
                \draw [blx,thick] plot [domain=-1.5:1.5,smooth,samples=100] (\x,{e^(-2*abs(\x))});
                \draw [blx,thick] plot [domain=-1.5:1.5,smooth,samples=100] ({\x+2},{e^(-2*abs(\x))});
            \end{tikzpicture}
            \caption{Small overlap.}
            \label{fig:overlapIntegrala}
        \end{subfigure}
        \begin{subfigure}[b]{0.4\linewidth}
            \centering
            \begin{tikzpicture}
                \footnotesize
                \node (A) [circle,fill=rex,inner sep=2pt,label={above:A}] at (0,-0.5) {};
                \node (B) [circle,fill=rex,inner sep=2pt,label={above:B}] at (1,-0.5) {}
                    edge [<->] node[below]{$R$} (A)
                ;
    
                \draw [blx,thick] plot [domain=-1.5:1.5,smooth,samples=100] (\x,{e^(-2*abs(\x))});
                \draw [blx,thick] plot [domain=-1.5:1.5,smooth,samples=100] ({\x+1},{e^(-2*abs(\x))});
            \end{tikzpicture}
            \caption{Large overlap.}
            \label{fig:overlapIntegralb}
        \end{subfigure}
        \caption{Overlap integral vs. internuclear distance.}
        \label{fig:overlapIntegral}
    \end{figure}
    \begin{itemize}
        \item So named because it is only significant where there is a large overlap between the two hydrogenlike atomic orbitals.
        \item As $R\to 0$, $S\to 1$. As $R\to\infty$, $S\to 0$.
    \end{itemize}
    \item The overlap integral when $nl=1s$ can be evaluated analytically, giving
    \begin{equation*}
        S(R) = \e[-R]\left( 1+R+\frac{R^2}{3} \right)
    \end{equation*}
    \item It follows that the normalized $\psi_\pm$ are
    \begin{equation*}
        \psi_\pm = \frac{1}{\sqrt{2(1\pm S)}}(1s_\text{A}\pm 1s_\text{B})
    \end{equation*}
\end{itemize}




\end{document}