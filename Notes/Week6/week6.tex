\documentclass[../notes.tex]{subfiles}

\pagestyle{main}
\renewcommand{\chaptermark}[1]{\markboth{\chaptername\ \thechapter\ (#1)}{}}
\setcounter{chapter}{5}

\begin{document}




\chapter{Multi-electron Atoms and Molecules}
\section{Many-electron Atoms and Molecules}
\begin{itemize}
    \item \marginnote{11/1:}Picking up from last time, since $Z=2$ for helium,
    \begin{equation*}
        \dv{E}{\lambda}\bigg|_{\lambda=0} = \frac{5}{8}Z = \frac{5}{4}
    \end{equation*}
    \item Additionally,
    \begin{equation*}
        E_{\ce{He}} = -Z^2+\frac{5}{8}Z = -4+\frac{5}{4} = \SI{-2.75}{\atomicunit}
    \end{equation*}
    \begin{itemize}
        \item The first equality is the ground state energy of two-electron atoms or ions.
    \end{itemize}
    \item Summary:
    \begin{itemize}
        \item Zeroeth-order Perturbation Theory approximation: $E_0=\SI{-4}{\atomicunit}$
        \item First-order approximation: $E_0+E'=\SI{-2.75}{\atomicunit}$
        \item Second-order: $E_0+E'+\frac{1}{2}E''=\SI{-2.9077}{\atomicunit}$
        \item Thirteenth-order: $\SI{-2.90372433}{\atomicunit}$
    \end{itemize}
    \item Variational calculation of Helium:
    \begin{itemize}
        \item Simplest: $\psi_0(12)=\text{1s}(1)\text{1s}(2)$. Gives $E=\SI{-2.75}{\atomicunit}$
        \item Trial $\psi$ with one parameter: $\psi_0(1,2)=\tilde{\text{1s}}(1)\tilde{\text{1s}}(2)$, where $\tilde{\text{1s}}(1)=\sqrt{Z^3/\pi}\e[-Zr_1]$. Energy: $E=\SI{-2.8477}{\atomicunit}$
    \end{itemize}
    \item Ionization energy.
    \begin{itemize}
        \item Simplest approximation:
        \begin{align*}
            \text{IE} &= E_{\ce{He+}}-E_{\ce{He}}\\
            &= -2-(-2.75)\\
            &= \SI{0.75}{\atomicunit}\\
            &= \SI[per-mode=symbol]{1969}{\kilo\joule\per\mole}
        \end{align*}
        \item Exact number:
        \begin{align*}
            \text{IE} &= E_{\ce{He+}}-E_{\ce{He}}\\
            &= -2-(\num{-2.903724})\\
            &= \SI{0.9033}{\atomicunit}\\
            &= \SI[per-mode=symbol]{2372}{\kilo\joule\per\mole}
        \end{align*}
    \end{itemize}
    \item Optimal orbitals (Hartree-Fock): Optimizing the orbitals to lower the energy as much as possible.
    \begin{itemize}
        \item $\psi(12)=\phi(r_1)\phi(r_2)$.
        \item The orbital energies converge to
        \begin{equation*}
            E_\text{HF} = \SI{-2.8617}{\atomicunit}
        \end{equation*}
    \end{itemize}
    \item Allow the $\psi(12)$ to move beyond a simple product of orbitals (Hylleras (1930)).
    \begin{itemize}
        \item $\psi(r_1,r_2,r_{12})=\e[-Zr_1]\e[-Zr_2](1+cr_{12})$.
        \item Accounting for the electron-electron repulsion (the \textbf{electron correlation energy}) gives us an energy much better than the Hartree-Fock calculation:
        \begin{equation*}
            E = \SI{-2.8913}{\atomicunit}
        \end{equation*}
        \item Pekeris (1959): Did a variational calculation with 1078 parameters. Was working at IBM, who told him to do something with their newest computer that would be impressive to the world. Pekeris tackled this, and got
        \begin{equation*}
            E = \SI{-2.903724375}{\atomicunit}
        \end{equation*}
        which is even more accurate than 13th order perturbation theory.
        \begin{itemize}
            \item This value cannot be accurately measured to this precision in the laboaratory. Additionally, relativistic quantum mechanics (using the Dirac equation of which the Schr\"{o}dinger equation is only a part) predicts a value that diverges from this one around the fifth decimal point, and this is the experimentally verifiable value.
            \item This is important because scientists want to figure out how accurately can we account for the electron cusp.
        \end{itemize}
    \end{itemize}
    \item To summarize, the improvement tiers are
    \begin{enumerate}
        \item \ce{H} orbitals.
        \item MO picture (Hartree-Fock).
        \item Solution of the Schr\"{o}dinger equation (electron correlation).
    \end{enumerate}
    \item Recall that the electron has spin (from the Stern-Gerlach and Uhlenbeck-Goudsmit experiments).
    \begin{itemize}
        \item This is analogous to orbital angular momentum:
        \begin{align*}
            \hat{L}^2Y_{lm} &= l(l+1)\hbar^2Y_{lm}&
            \hat{L_z}Y_{lm} &= m\hbar Y_{lm}
        \end{align*}
        \item These two operators give rise to
        \begin{align*}
            \hat{S}^2\sigma &= s(s+1)\hbar^2\sigma&
            \hat{S}_z\sigma &= m_s\hbar\sigma
        \end{align*}
        where $\sigma$ is the spin eigenfunction, $s$ is the total spin angular momentum quantum number, and $m_s$ is the spin quantum number.
        \item We have $\sigma=\alpha$ or $\sigma=\beta$ where $\alpha$ represents "up" and $\beta$ represents "down."
        \item $m_s=\pm 1/2$ where $s=1/2$, so $m_s=-s,s$.
        \item Spin eigenfunctions are orthonormal, i.e.,
        \begin{align*}
            \int\alpha^*\alpha &= \int\beta^*\beta = 1&
            \int\alpha^*\beta &= \int\beta^*\alpha = 0
        \end{align*}
        \item Hence, each electron has four degrees of freedom (three spatial and one spin).
        \item This gives rise to a \textbf{spin orbital} $\psi(x,y,z,\sigma)=\phi(x,y,z)\sigma$, which is the product of the spin and spatial orbitals.
    \end{itemize}
    \item In fact, our ability to calculate the energy of the helium atom was complete luck --- we cannot calculate the energy of any other element on the periodic table without accounting for spin.
\end{itemize}



\section{Many-electron Atoms and Molecules / Spin}
\begin{itemize}
    \item \marginnote{11/3:}\textbf{Spin orbital}: A spatial orbital augmented with a spin eigenfunction. \emph{Given by}
    \begin{equation*}
        \psi(x,y,z,\sigma) = \phi(x,y,z)\sigma
    \end{equation*}
    \item Key idea: Electrons are indistinguishable, i.e., if $\psi(1,2,\dots,N)$ where each number represents the three spatial coordinates of an electron, the $N$ electrons in $\psi$ must be indistinguishable.
    \item There are two possible ways to achieve \emph{mathematical} indistinguishability.
    \begin{enumerate}
        \item We can require that
        \begin{equation*}
            \psi(1,2,3,\dots,N) = \psi(2,1,3,\dots,N)
        \end{equation*}
        \begin{itemize}
            \item Essentially, this is a symmetric permutation of all pairs of particles.
        \end{itemize}
        \item We can alternatively require that
        \begin{equation*}
            \psi(1,2,3,\dots,N) = -\psi(2,1,3,\dots,N)
        \end{equation*}
        \begin{itemize}
            \item Because the probability is invariant to the sign (think about taking the modulus squared and how that eliminates sign considerations).
            \item Essentially, this is an antisymmetric permutation of all pairs of particles.
        \end{itemize}
    \end{enumerate}
    \item Experimental observation shows two types of particles in nature.
    \begin{enumerate}
        \item Bozons (symmetric permutations).
        \item Fermions (antisymmetric permutations).
    \end{enumerate}
    \item Electrons are Fermions.
    \item $\Prob=\psi^*(1,2,\dots,N)\psi(1,2,\dots,N)$ tells us the probability of finding the $N$ electrons at the positions specified by 1 through $N$.
    \item Example (helium):
    \begin{itemize}
        \item We had $\psi(1,2)=1s\alpha(1)1s\beta(2)$.
        \item But since the electrons are distinguishable here, we have to make them indistinguishable by letting
        \begin{align*}
            \psi(1,2) &= 1s\alpha(1)12\beta(2)-1s\alpha(2)1s\beta(1)\\
            &= 1s(1)1s(2)[\alpha(1)\beta(2)-\alpha(2)\beta(1)]
        \end{align*}
        \begin{itemize}
            \item Here, we've separated out the spatial and spin parts.
        \end{itemize}
    \end{itemize}
    \item How to represent antisymmetry.
    \begin{itemize}
        \item Grassmann Wedge Product (1850s): Herman de Grassmann was studying linear algebra and would learn about higher dimensional spaces in his dreams.
        \item de Grassman says
        \begin{align*}
            \psi(1,2) &= 1s\alpha(1)\wedge 1s\beta(2)\\
            \frac{1}{2!}(1s\alpha(1)1s\beta(2)-1s\alpha(2)1s\beta(1))
        \end{align*}
        where $\wedge$ is the wedge product, a tensor product.
        \begin{itemize}
            \item We can normalize the above by substituting $\sqrt{2!}$ for $2!$.
        \end{itemize}
        \item Slater determinant (1930): Took
        \begin{equation*}
            \psi(1,2) = \frac{1}{\sqrt{2!}}
            \begin{vmatrix}
                1s\alpha(1) & 1s\beta(1)\\
                1s\alpha(2) & 1s\beta(2)\\
            \end{vmatrix}
        \end{equation*}
    \end{itemize}
    \item We now consider lithium.
    \begin{itemize}
        \item Let's put 3 electrons in a single orbital:
        \begin{equation*}
            \psi(1,2,3) = 1s\alpha(1)\wedge 1s\beta(2)\wedge 1s\beta(3)
        \end{equation*}
        \begin{itemize}
            \item $1s\beta(2)\wedge 1s\beta(3)$ will vanish as per Grassmann algebra since the electrons have the same spin state. In general, $\phi\wedge\phi=0$.
        \end{itemize}
        \item Thus, in general,
        \begin{equation*}
            \psi(1,2,3) = \sqrt{3!}(1s\alpha(1)\wedge 1s\beta(2)\wedge 2s\alpha(3))
        \end{equation*}
        or, using the Slater determinant,
        \begin{equation*}
            \psi(1,2,3) = \frac{1}{\sqrt{3!}}
            \begin{vmatrix}
                1s\alpha(1) & 1s\beta(1) & 2s\alpha(1)\\
                1s\alpha(2) & 1s\beta(2) & 2s\alpha(2)\\
                1s\alpha(3) & 1s\beta(3) & 2s\alpha(3)\\
            \end{vmatrix}
        \end{equation*}
    \end{itemize}
    \item \textbf{Pauli exclusion principle}: Each electron must occupy a distinct spin orbital, that is, it must have a distinct set of 4 quantum numbers $n,l,m,m_s$.
    \begin{itemize}
        \item Emerges naturally from the quantum mechanics as per the above.
    \end{itemize}
    \item What if electrons were bozons?
    \begin{itemize}
        \item Then
        \begin{equation*}
            \psi(1,2,3) = 1s\alpha(1)\vee 1s\beta(2)\vee 2s\alpha(3)
        \end{equation*}
        where $\vee$ is the bozonic operator, i.e., the symmetric wedge product, which denotes the positive sum of all permutations.
        \item We have
        \begin{align*}
            \phi(1)\vee\phi(2) &= \phi(1)\phi(2)+\phi(2)\phi(1)\\
            &= 2\phi(1)\phi(2)
        \end{align*}
        \item Thus, we could put an infinite number of electrons in the same orbital (all electrons could occupy the same orbital) if electrons were bozons, and our shell structure would disintegrate.
        \begin{itemize}
            \item Everything would also merge; we could not form matter as we know it.
        \end{itemize}
    \end{itemize}
    \item Sodium atoms are bozons.
\end{itemize}




\end{document}