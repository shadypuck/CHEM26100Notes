\documentclass[../notes.tex]{subfiles}

\pagestyle{main}
\renewcommand{\chaptermark}[1]{\markboth{\chaptername\ \thechapter\ (#1)}{}}
\setcounter{chapter}{5}
\setcounter{postulate}{5}

\begin{document}




\chapter{Multi-electron Atoms and Molecules}
\section{Many-electron Atoms and Molecules}
\begin{itemize}
    \item \marginnote{11/1:}Picking up from last time, since $Z=2$ for helium,
    \begin{equation*}
        \dv{E}{\lambda}\bigg|_{\lambda=0} = \frac{5}{8}Z = \frac{5}{4}
    \end{equation*}
    \item Additionally,
    \begin{equation*}
        E_{\ce{He}} = -Z^2+\frac{5}{8}Z = -4+\frac{5}{4} = \SI{-2.75}{\atomicunit}
    \end{equation*}
    \begin{itemize}
        \item The first equality is the ground state energy of two-electron atoms or ions.
    \end{itemize}
    \item Summary:
    \begin{itemize}
        \item Zeroeth-order Perturbation Theory approximation: $E_0=\SI{-4}{\atomicunit}$
        \item First-order approximation: $E_0+E'=\SI{-2.75}{\atomicunit}$
        \item Second-order: $E_0+E'+\frac{1}{2}E''=\SI{-2.9077}{\atomicunit}$
        \item Thirteenth-order: $\SI{-2.90372433}{\atomicunit}$
    \end{itemize}
    \item Variational calculation of Helium:
    \begin{itemize}
        \item Simplest: $\psi_0(12)=\text{1s}(1)\text{1s}(2)$. Gives $E=\SI{-2.75}{\atomicunit}$
        \item Trial $\psi$ with one parameter: $\psi_0(1,2)=\tilde{\text{1s}}(1)\tilde{\text{1s}}(2)$, where $\tilde{\text{1s}}(1)=\sqrt{Z^3/\pi}\e[-Zr_1]$. Energy: $E=\SI{-2.8477}{\atomicunit}$
    \end{itemize}
    \item Ionization energy.
    \begin{itemize}
        \item Simplest approximation:
        \begin{align*}
            \text{IE} &= E_{\ce{He+}}-E_{\ce{He}}\\
            &= -2-(-2.75)\\
            &= \SI{0.75}{\atomicunit}\\
            &= \SI[per-mode=symbol]{1969}{\kilo\joule\per\mole}
        \end{align*}
        \item Exact number:
        \begin{align*}
            \text{IE} &= E_{\ce{He+}}-E_{\ce{He}}\\
            &= -2-(\num{-2.903724})\\
            &= \SI{0.9033}{\atomicunit}\\
            &= \SI[per-mode=symbol]{2372}{\kilo\joule\per\mole}
        \end{align*}
    \end{itemize}
    \item Optimal orbitals (Hartree-Fock): Optimizing the orbitals to lower the energy as much as possible.
    \begin{itemize}
        \item $\psi(12)=\phi(r_1)\phi(r_2)$.
        \item The orbital energies converge to
        \begin{equation*}
            E_\text{HF} = \SI{-2.8617}{\atomicunit}
        \end{equation*}
    \end{itemize}
    \item Allow the $\psi(12)$ to move beyond a simple product of orbitals (Hylleras (1930)).
    \begin{itemize}
        \item $\psi(r_1,r_2,r_{12})=\e[-Zr_1]\e[-Zr_2](1+cr_{12})$.
        \item Accounting for the electron-electron repulsion (the \textbf{electron correlation energy}) gives us an energy much better than the Hartree-Fock calculation:
        \begin{equation*}
            E = \SI{-2.8913}{\atomicunit}
        \end{equation*}
        \item Pekeris (1959): Did a variational calculation with 1078 parameters. Was working at IBM, who told him to do something with their newest computer that would be impressive to the world. Pekeris tackled this, and got
        \begin{equation*}
            E = \SI{-2.903724375}{\atomicunit}
        \end{equation*}
        which is even more accurate than 13th order perturbation theory.
        \begin{itemize}
            \item This value cannot be accurately measured to this precision in the laboaratory. Additionally, relativistic quantum mechanics (using the Dirac equation of which the Schr\"{o}dinger equation is only a part) predicts a value that diverges from this one around the fifth decimal point, and this is the experimentally verifiable value.
            \item This is important because scientists want to figure out how accurately can we account for the electron cusp.
        \end{itemize}
    \end{itemize}
    \item To summarize, the improvement tiers are
    \begin{enumerate}
        \item \ce{H} orbitals.
        \item MO picture (Hartree-Fock).
        \item Solution of the Schr\"{o}dinger equation (electron correlation).
    \end{enumerate}
    \item Recall that the electron has spin (from the Stern-Gerlach and Uhlenbeck-Goudsmit experiments).
    \begin{itemize}
        \item This is analogous to orbital angular momentum:
        \begin{align*}
            \hat{L}^2Y_{lm} &= l(l+1)\hbar^2Y_{lm}&
            \hat{L_z}Y_{lm} &= m\hbar Y_{lm}
        \end{align*}
        \item These two operators give rise to
        \begin{align*}
            \hat{S}^2\sigma &= s(s+1)\hbar^2\sigma&
            \hat{S}_z\sigma &= m_s\hbar\sigma
        \end{align*}
        where $\sigma$ is the spin eigenfunction, $s$ is the total spin angular momentum quantum number, and $m_s$ is the spin quantum number.
        \item We have $\sigma=\alpha$ or $\sigma=\beta$ where $\alpha$ represents "up" and $\beta$ represents "down."
        \item $m_s=\pm 1/2$ where $s=1/2$, so $m_s=-s,s$.
        \item Spin eigenfunctions are orthonormal, i.e.,
        \begin{align*}
            \int\alpha^*\alpha &= \int\beta^*\beta = 1&
            \int\alpha^*\beta &= \int\beta^*\alpha = 0
        \end{align*}
        \item Hence, each electron has four degrees of freedom (three spatial and one spin).
        \item This gives rise to a \textbf{spin orbital} $\psi(x,y,z,\sigma)=\phi(x,y,z)\sigma$, which is the product of the spin and spatial orbitals.
    \end{itemize}
    \item In fact, our ability to calculate the energy of the helium atom was complete luck --- we cannot calculate the energy of any other element on the periodic table without accounting for spin.
\end{itemize}



\section{Many-electron Atoms and Molecules / Spin}
\begin{itemize}
    \item \marginnote{11/3:}\textbf{Spin orbital}: A spatial orbital augmented with a spin eigenfunction. \emph{Given by}
    \begin{equation*}
        \psi(x,y,z,\sigma) = \phi(x,y,z)\sigma
    \end{equation*}
    \item Key idea: Electrons are indistinguishable, i.e., if $\psi(1,2,\dots,N)$ where each number represents the three spatial coordinates of an electron, the $N$ electrons in $\psi$ must be indistinguishable.
    \item There are two possible ways to achieve \emph{mathematical} indistinguishability.
    \begin{enumerate}
        \item We can require that
        \begin{equation*}
            \psi(1,2,3,\dots,N) = \psi(2,1,3,\dots,N)
        \end{equation*}
        \begin{itemize}
            \item Essentially, this is a symmetric permutation of all pairs of particles.
        \end{itemize}
        \item We can alternatively require that
        \begin{equation*}
            \psi(1,2,3,\dots,N) = -\psi(2,1,3,\dots,N)
        \end{equation*}
        \begin{itemize}
            \item Because the probability is invariant to the sign (think about taking the modulus squared and how that eliminates sign considerations).
            \item Essentially, this is an antisymmetric permutation of all pairs of particles.
        \end{itemize}
    \end{enumerate}
    \item Experimental observation shows two types of particles in nature.
    \begin{enumerate}
        \item Bozons (symmetric permutations).
        \item Fermions (antisymmetric permutations).
    \end{enumerate}
    \item Electrons are Fermions.
    \item $\Prob=\psi^*(1,2,\dots,N)\psi(1,2,\dots,N)$ tells us the probability of finding the $N$ electrons at the positions specified by 1 through $N$.
    \item Example (helium):
    \begin{itemize}
        \item We had $\psi(1,2)=1s\alpha(1)1s\beta(2)$.
        \item But since the electrons are distinguishable here, we have to make them indistinguishable by letting
        \begin{align*}
            \psi(1,2) &= 1s\alpha(1)12\beta(2)-1s\alpha(2)1s\beta(1)\\
            &= 1s(1)1s(2)[\alpha(1)\beta(2)-\alpha(2)\beta(1)]
        \end{align*}
        \begin{itemize}
            \item Here, we've separated out the spatial and spin parts.
        \end{itemize}
    \end{itemize}
    \item How to represent antisymmetry.
    \begin{itemize}
        \item Grassmann Wedge Product (1850s): Herman de Grassmann was studying linear algebra and would learn about higher dimensional spaces in his dreams.
        \item de Grassman says
        \begin{align*}
            \psi(1,2) &= 1s\alpha(1)\wedge 1s\beta(2)\\
            \frac{1}{2!}(1s\alpha(1)1s\beta(2)-1s\alpha(2)1s\beta(1))
        \end{align*}
        where $\wedge$ is the wedge product, a tensor product.
        \begin{itemize}
            \item We can normalize the above by substituting $\sqrt{2!}$ for $2!$.
        \end{itemize}
        \item Slater determinant (1930): Took
        \begin{equation*}
            \psi(1,2) = \frac{1}{\sqrt{2!}}
            \begin{vmatrix}
                1s\alpha(1) & 1s\beta(1)\\
                1s\alpha(2) & 1s\beta(2)\\
            \end{vmatrix}
        \end{equation*}
    \end{itemize}
    \item We now consider lithium.
    \begin{itemize}
        \item Let's put 3 electrons in a single orbital:
        \begin{equation*}
            \psi(1,2,3) = 1s\alpha(1)\wedge 1s\beta(2)\wedge 1s\beta(3)
        \end{equation*}
        \begin{itemize}
            \item $1s\beta(2)\wedge 1s\beta(3)$ will vanish as per Grassmann algebra since the electrons have the same spin state. In general, $\phi\wedge\phi=0$.
        \end{itemize}
        \item Thus, in general,
        \begin{equation*}
            \psi(1,2,3) = \sqrt{3!}(1s\alpha(1)\wedge 1s\beta(2)\wedge 2s\alpha(3))
        \end{equation*}
        or, using the Slater determinant,
        \begin{equation*}
            \psi(1,2,3) = \frac{1}{\sqrt{3!}}
            \begin{vmatrix}
                1s\alpha(1) & 1s\beta(1) & 2s\alpha(1)\\
                1s\alpha(2) & 1s\beta(2) & 2s\alpha(2)\\
                1s\alpha(3) & 1s\beta(3) & 2s\alpha(3)\\
            \end{vmatrix}
        \end{equation*}
    \end{itemize}
    \item \textbf{Pauli exclusion principle}: Each electron must occupy a distinct spin orbital, that is, it must have a distinct set of 4 quantum numbers $n,l,m,m_s$.
    \begin{itemize}
        \item Emerges naturally from the quantum mechanics as per the above.
    \end{itemize}
    \item What if electrons were bozons?
    \begin{itemize}
        \item Then
        \begin{equation*}
            \psi(1,2,3) = 1s\alpha(1)\vee 1s\beta(2)\vee 2s\alpha(3)
        \end{equation*}
        where $\vee$ is the bozonic operator, i.e., the symmetric wedge product, which denotes the positive sum of all permutations.
        \item We have
        \begin{align*}
            \phi(1)\vee\phi(2) &= \phi(1)\phi(2)+\phi(2)\phi(1)\\
            &= 2\phi(1)\phi(2)
        \end{align*}
        \item Thus, we could put an infinite number of electrons in the same orbital (all electrons could occupy the same orbital) if electrons were bozons, and our shell structure would disintegrate.
        \begin{itemize}
            \item Everything would also merge; we could not form matter as we know it.
        \end{itemize}
    \end{itemize}
    \item Sodium atoms are bozons.
\end{itemize}



\section{Chapter 8: Multielectron Atoms}
\emph{From \textcite{bib:McQuarrieSimon}.}
\begin{itemize}
    \item \marginnote{11/7:}\textbf{Atomic units}: A system of units widely adopted for atomic and molecular calculations to simplify the equations.
    \begin{table}[H]
        \centering
        \small
        \renewcommand{\arraystretch}{1.4}
        \begin{tabular}{lll}
            \toprule
            Property & Atomic unit & SI Equivalent\\
            \midrule
            mass & mass of an electron, $m_e$ & $\SI{9.1094e-31}{\kilo\gram}$\\
            charge & charge on a proton, $e$ & $\SI{1.6022e-19}{\coulomb}$\\
            angular momentum & reduced Planck constant, $\hbar$ & $\SI{1.0546e-34}{\joule\second}$\\
            distance & Bohr radius, $a_0=4\pi\epsilon_0\hbar^2/m_ee^2$ & $\SI{5.2918e-11}{\meter}$\\
            energy & $e^2/4\pi\epsilon_0a_0=E_\text{h}$ & $\SI{4.3597e-18}{\joule}$\\
            permittivity & $4\pi\epsilon_0$ & $\SI{1.1127e-10}{\square\coulomb\per\joule\per\meter}$\\
            \bottomrule
        \end{tabular}
        \caption{Atomic units and their SI equivalents.}
        \label{tab:atomicUnits}
    \end{table}
    \item \textbf{Hartree}: The atomic unit of energy. \emph{Denoted by} $\bm{E}_\textbf{h}$.
    \begin{itemize}
        \item Note that in atomic units, the ground-state energy of a hydrogen atom (in the fixed nucleus approximation) is $-E_\text{h}/2$.
    \end{itemize}
    \item Another good reason to use atomic units is that we are still refining the values of $m_e$, $e$, $\hbar$, etc., so results computed in atomic units will hold even as these constants evolve.
    \item Both perturbation theory and the variational method can yield excellent results for helium.
    \item \textbf{Slater orbital}\footnote{First introduced by American physicist John Slater in the 1930s.}: An orbital of the following form. \emph{Denoted by} $\bm{S_{nlm}(r,\theta,\phi)}$. \emph{Given by}
    \begin{equation*}
        S_{nlm}(r,\theta,\phi) = N_{nl}r^{n-1}\e[-\zeta r]Y_l^m(\theta,\phi)\footnotemark
    \end{equation*}
    \footnotetext{$\zeta$ is the Greek "zeta."}
    where $N_{nl}=(2\zeta)^{n+1/2}/\sqrt{(2n)!}$ and the $Y_l^m(\theta,\phi)$ are the spherical harmonics.
    \begin{itemize}
        \item The parameter $\zeta$ is arbitrary, i.e., not necessarily $Z/n$ as in hydrogenlike orbitals.
        \item The radial components of the Slater orbitals do not have nodes like the hydrogenlike orbitals.
    \end{itemize}
    \item \textbf{Orbital}: A one-electron wave function.
    \item \textbf{Hartree-Fock limit}: The best value of the energy that can be obtained using a trial function of the form of a product of orbitals.
    \item As such, it is advantageous to generalize our trial wave functions past simple products of orbitals.
    \begin{itemize}
        \item Better wave functions typically include the interelectronic distance.
    \end{itemize}
    \item This realization led scientists to abandon the orbital concept altogether in favor of finding Hartree-Fock orbitals (because they are still a useful model) and correcting them using an approach such as perturbation theory.
    \item Self-consistent field method for finding the Hartree-Fock orbitals (of helium).
    \begin{itemize}
        \item Write $\psi(\mathbf{r}_1,\mathbf{r}_2)$ as a product of orbitals $\phi(\mathbf{r}_1)\phi(\mathbf{r}_2)$.
        \item It follows that the potential energy that electron 1 experiences at point $\mathbf{r}_1$ due to electron 2 is
        \begin{equation*}
            V_1^\text{eff}(\mathbf{r}_1) = \int\phi^*(\mathbf{r}_2)\frac{1}{r_{12}}\phi(\mathbf{r}_2)\dd{\mathbf{r}_2}
        \end{equation*}
        where the superscript "eff" emphasizes that $V_1^\text{eff}(\mathbf{r}_1)$ is an effective, or average, potential.
        \item This allows us to define the effective one-electron Hamiltonian for electron 1 by
        \begin{equation*}
            \hat{H}_1^\text{eff}(\mathbf{r}_1) = -\frac{1}{2}\nabla_1^2-\frac{2}{r_1}+V_1^\text{eff}(\mathbf{r}_1)
        \end{equation*}
        \item Using this Hamiltonian and the corresponding Schr\"{o}dinger equation, we can determine $\phi(\mathbf{r}_1)$, given an estimate for $\phi(\mathbf{r}_2)$. Since there is analogous Schr\"{o}dinger equation for $\phi(\mathbf{r}_2)$, the self-consistent field method consists of starting with a guess for $\phi(\mathbf{r}_2)$, calculating from that $\phi(\mathbf{r}_1)$, calculating from that a better guess for $\phi(\mathbf{r}_2)$, and so on and so forth until the two wave equations are reasonably close, or \textbf{self-consistent}.
        \item In practice, linear combinations of Slater orbitals are used for each $\phi(\mathbb{r})$.
    \end{itemize}
    \item \textbf{Uncorrelated} (electrons): A set of electrons that are taken to be independent of each other, or at least to interact only through some average (or effective) potential.
    \begin{itemize}
        \item Uncorrelated electrons occupy mathematically separable orbitals, as in $\psi(\mathbf{r}_1,\mathbf{r}_2)=\phi(\mathbf{r}_1)\phi(\mathbf{r}_2)$.
    \end{itemize}
    \item \textbf{Correlation energy}: The difference between the exact energy and the Hartree-Fock approximation energy. \emph{Denoted by} $\textbf{CE}$.
    \item \textbf{Spin quantum number}: The fourth quantum number, which represents the $z$-component of the electron spin angular momentum. \emph{Denoted by} $\bm{m_s}$. \emph{Given by}
    \begin{equation*}
        m_s = \pm\frac{1}{2}\,\si{\atomicunit}
    \end{equation*}
    \begin{itemize}
        \item Uhlenbeck and Goudsmit first suggested in 1925 that electrons behave like spinning tops having $z$-components of spin angular momentum $\pm\hbar/2$ to explain the phenomena of sodium having a doublet in its atomic spectrum where quantum theory predicts it should have a singlet.
    \end{itemize}
    \item We graft spin onto quantum theory in an ad hoc manner here. This will suit our purposes.
    \begin{itemize}
        \item Note, however, that Dirac developed in the 1930s a relativistic extension of quantum mechanics in which spin arises in a perfectly natural way.
    \end{itemize}
    \item As part of this grafting, we \emph{define} the \textbf{spin operators} $\hat{S}^2$ and $\hat{S}_z$ in an analogous manner to how we defined the angular momentum operators $\hat{L}^2$ and $\hat{L}_z$, except that we introduce half-integral angular momentum for electron spin: Indeed, let
    \begin{align*}
        \hat{S}^2\alpha &= \frac{1}{2}\left( \frac{1}{2}+1 \right)\hbar^2\alpha&
        \hat{S}^2\beta  &= \frac{1}{2}\left( \frac{1}{2}+1 \right)\hbar^2\beta
    \end{align*}
    and let
    \begin{align*}
        \hat{S}_z\alpha &= m_s\alpha = \frac{1}{2}\hbar\alpha&
        \hat{S}_z\beta  &= m_2\beta  = -\frac{1}{2}\hbar\beta
    \end{align*}
    \begin{itemize}
        \item We can formally associate $\alpha=Y_{1/2}^{1/2}$ and $\beta=Y_{1/2}^{-1/2}$ in analogy with $\hat{L}^2$ and $\hat{L}_z$, but we do not have to.
        \item To continue the analogy, we can actually say that the square of the \emph{spin} angular momentum is $S^2=\hbar^2s(s+1)$, where $s=\pm 1/2$.
        \item Since $s$ cannot assume large values like $l$ and thus cannot approach a classical limit, spin is strictly nonclassical.
    \end{itemize}
    \item \textbf{Spin eigenfunctions}: The functions $\alpha$ and $\beta$.
    \item The spin operators are Hermitian, so $\alpha,\beta$ are orthonormal.
    \begin{itemize}
        \item Formally,
        \begin{align*}
            \int\alpha^*(\sigma)\alpha(\sigma)\dd{\sigma} &= \int\beta^*(\sigma)\beta(\sigma)\dd{\sigma} = 1\\
            \int\alpha^*(\sigma)\beta(\sigma)\dd{\sigma} &= \int\beta^*(\sigma)\alpha(\sigma)\dd{\sigma} = 0
        \end{align*}
        where $\sigma$ is the \textbf{spin variable}, an object with no classical analog.
    \end{itemize}
    \item We now include the spin function with the spatial wave function and postulate that the two are independent so that one of the following two cases holds.
    \begin{align*}
        \psi(x,y,z,\sigma) &= \psi(x,y,z)\alpha(\sigma)&
        \psi(x,y,z,\sigma) &= \psi(x,y,z)\beta(\sigma)
    \end{align*}
    \item \textbf{Spin orbital}: The complete one-electron wave function $\psi$ dependent on $x,y,z,\sigma$.
    \item For example, the first two spin orbitals of a hydrogenlike atom are
    \begin{align*}
        \psi_{100\frac{1}{2}} &= \sqrt{\frac{Z^3}{\pi}}\e[-Zr]\alpha&
        \psi_{100-\frac{1}{2}} &= \sqrt{\frac{Z^3}{\pi}}\e[-Zr]\beta
    \end{align*}
    \begin{itemize}
        \item Note that these orbitals are orthonormal since
        \begin{equation*}
            \int\psi_{100\frac{1}{2}}^*(\mathbf{r},\sigma)\psi_{100\frac{1}{2}}(\mathbf{r},\sigma)4\pi r^2\dd{r}\dd{\sigma} = \int_0^\infty\frac{Z^3}{\pi}\e[-2Zr]4\pi r^2\dd{r}\int\alpha^*\alpha\dd{\sigma} = 1
        \end{equation*}
        and similarly for $\psi_{100-\frac{1}{2}}$, and since
        \begin{equation*}
            \int\psi_{100\frac{1}{2}}^*(\mathbf{r},\sigma)\psi_{100-\frac{1}{2}}(\mathbf{r},\sigma)4\pi r^2\dd{r}\dd{\sigma} = \int_0^\infty\frac{Z^3}{\pi}\e[-2Zr]4\pi r^2\dd{r}\int\alpha^*\beta\dd{\sigma} = 0
        \end{equation*}
    \end{itemize}
    \item \textbf{Pauli Exclusion Principle} (elementary): No two electrons in an atom can have the same values of all four quantum numbers.
    \item Consider the helium atom.
    \begin{itemize}
        \item Let $\psi(1,2)=1s\alpha(1)1s\beta(2)$ where $1s\alpha$ and $1s\beta$ represent $\psi_{100\frac{1}{2}}$ and $\psi_{100-\frac{1}{2}}$, respectively, and 1 and 2 represent all four coordinates (3 spatial plus 1 spin) of electrons 1 and 2, respectively.
        \item Similarly, $\psi(2,1)=1s\alpha(2)1s\beta(1)$.
        \item Since no known experiment can distinguish one electron from another, $\psi(1,2)=\psi(2,1)$.
        \item Formally, we must consider the two possible states
        \begin{align*}
            \psi_1 &= \psi(1,2)+\psi(2,1)&
            \psi_2 &= \psi(1,2)-\psi(2,1)
        \end{align*}
        of two indistinguishable electrons.
        \item Experimentally, $\psi_2$ describes the ground state of helium since $\psi_2(1,2)=-\psi_2(2,1)$, so $\psi_2$ is \textbf{antisymmetric}.
        \item Note that the normalization constant of $\psi_2$ is $1/\sqrt{2}$.
    \end{itemize}
    \item \textbf{Antisymmetric wave function}: A wave function that changes sign when two electrons are interchanged.
    \item The importance of antisymmetry is captured by the following more fundamental statement of the Pauli Exclusion Principle.
    \begin{postulate}
        All electronic wave functions must be antisymmetric under the interchange of any two electrons.
    \end{postulate}
    \item We were able to ignore spin in previous treatments of helium since $\psi_2$ can be factored into a spatial and a spin part, and any contributions from spin cancel under a Hamiltonian that does not contain any spin operators.
    \begin{itemize}
        \item Note that factorization of $\psi^*\hat{H}\psi$ and $\psi^*\psi$ does not occur in general, but does occur for two-electron systems.
    \end{itemize}
    \item \textbf{Determinantal wave function}: A wave function given by the determinant of a matrix whose entries are individual spin orbitals.
    \begin{itemize}
        \item Since the determinant representation changes signs when two rows are interchanged (representing interchanging two electrons) and is equal to zero under repeated rows/columns (representing multiple electrons in the same spin orbital), the determinant representation is antisymmetric and satisfies the Pauli Exclusion Principle, respectively.
    \end{itemize}
    \item \textbf{Normalized $\bm{N}$-electron determinantal wave function}: The normalized determinantal wave function
    \begin{equation*}
        \psi(1,\dots,N) = \frac{1}{\sqrt{N!}}
        \begin{vmatrix}
            u_1(1) & \cdots & u_N(1)\\
            \vdots & \ddots & \vdots\\
            u_1(N) & \cdots & u_N(N)\\
        \end{vmatrix}
    \end{equation*}
    where the $u$'s are orthonormal spin orbitals.
    \item \textbf{Fock operator}: The more general effective Hamiltonian operator used for self-consistent field solutions to systems more complex than helium. \emph{Denoted by} $\bm{\hat{F}_i}$.
    \item \textbf{Hartree-Fock orbital}: An eigenfunction of the Fock operator. \emph{Denoted by} $\bm{\phi_i}$.
    \item \textbf{Orbital energy}: An eigenvalue of the Fock operator. \emph{Denoted by} $\bm{\epsilon_i}$.
    \begin{itemize}
        \item Alternatively, the energy of a Hartree-Fock orbital.
        \item $\epsilon_i$ approximates the ionization energy of an electron from the $i^\text{th}$ orbital.
    \end{itemize}
\end{itemize}




\end{document}