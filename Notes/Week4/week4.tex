\documentclass[../notes.tex]{subfiles}

\pagestyle{main}
\renewcommand{\chaptermark}[1]{\markboth{\chaptername\ \thechapter\ (#1)}{}}
\setcounter{chapter}{3}

\begin{document}




\chapter{The Hydrogen Atom and Angular Momentum}
\section{Rotational Motion}
\begin{itemize}
    \item \marginnote{10/18:}Consider the diatomic molecule \ce{AB} at a distance $r_0$ apart rotating about its center of mass.
    \begin{figure}[h!]
        \centering
        \begin{subfigure}[b]{0.3\linewidth}
            \centering
            \begin{tikzpicture}
                \footnotesize
                \draw (0,0) node[circle,fill=rex,inner sep=1.5pt,label={below:$A$}]{}
                    -- node[pos=0.65,fill,inner sep=1.5pt,label={below:CM}]{}
                    (10:2) node[circle,fill=blx,inner sep=1.5pt,label={below:$B$}]{}
                ;
                \draw [decorate,decoration=brace] (100:0.2) -- node[above=2pt]{$r_0$} ++(10:2);
            \end{tikzpicture}
            \caption{Diatomic.}
            \label{fig:partRota}
        \end{subfigure}
        \begin{subfigure}[b]{0.3\linewidth}
            \centering
            \begin{tikzpicture}
                \footnotesize
                \draw [semithick,-latex] (10:2) -- ++(100:0.5);
                \draw (0,0) node[fill,inner sep=1.5pt,label={left:origin}]{}
                    --
                    (10:2) node[circle,fill=blx!65!rex,inner sep=1.5pt,label={right:$\mu$}]{}
                ;
                \draw [decorate,decoration={brace,mirror}] (-80:0.2) -- node[below=2pt]{$r_0$} ++(10:2);
            \end{tikzpicture}
            \caption{An equivalent system.}
            \label{fig:partRotb}
        \end{subfigure}
        \caption{Diatomic rotation.}
        \label{fig:partRot}
    \end{figure}
    \begin{itemize}
        \item To simplify the problem, replace the two particles rotating about the center of mass with one particle of reduced mass $\mu$ rotating about the center of mass with lever arm $r_0$.
    \end{itemize}
    \item Classically, the kinetic energy of the translational motion is
    \begin{equation*}
        T = \frac{L^2}{2I}
    \end{equation*}
    where $I=\mu r_0^2$ and $L=p\times r_0=pr_0$ (for this kind of rotation; see Figure \ref{fig:partRotb}).
    \item To further talk about this problem, we should introduce \textbf{spherical coordinates}.
    \item \textbf{Spherical coordinates}: The coordinate system $(r,\theta,\phi)$ related to the Cartesian coordinates $(x,y,z)$ by
    \begin{align*}
        x &= r\sin\theta\cos\phi&
        y &= r\sin\theta\sin\phi&
        z &= r\cos\theta
    \end{align*}
    \item Classically, we will have
    \begin{equation*}
        H = \frac{1}{2\mu}(p_x^2+p_y^2+p_z^2)+V(x,y,z)
    \end{equation*}
    in Cartesian coordinates.
    \item In spherical coordinates, this becomes
    \begin{equation*}
        H = \frac{1}{2\mu}\left( p_r^2+\frac{L^2}{r^2} \right)+V(r)
    \end{equation*}
    \item Thus, in quantum mechanics, we get
    \begin{align*}
        \hat{H} &= \frac{1}{2\mu}(\hat{p}_x^2+\hat{p}_y^2+\hat{p}_z^2)+V(x,y,z)\\
        &= \frac{1}{2\mu}\left( \hat{p}_r^2+\frac{\hat{L}^2}{r^2} \right)+V(r)
    \end{align*}
    \item Thus, we have in spherical coordinates that
    \begin{equation*}
        \hat{T}\psi = -\frac{\hbar^2}{2\mu}\left[ \frac{1}{r^2}\pdv{r}\left( r^2\pdv{r}\psi \right)+\frac{1}{r^2\sin\theta}\left( \pdv{\theta}\left( \sin\theta\pdv{\theta}\psi \right) \right)+\frac{1}{r^2\sin^2\theta}\pdv[2]{\phi}\psi \right]
    \end{equation*}
    \item 2D rigid rotor:
    \begin{itemize}
        \item Let the system from Figure \ref{fig:partRotb} be confined to rotating in two dimensions.
        \item This simplifies the problem since both the $\pdv*{r}$ and $\pdv*{\theta}$ terms in the kinetic energy operator disappear (since, respectively, the particle is at a fixed distance from the center of mass and it cannot move out of the 2D plane).
        \item Thus, our Schr\"{o}dinger equation for this system is
        \begin{equation*}
            -\frac{\hbar^2}{2\mu r_0^2}\pdv[2]{\phi}\psi(\phi) = E\psi(\phi)
        \end{equation*}
        \item Solution: Let $\psi(\phi)=\e[im\phi]$; then
        \begin{equation*}
            E_m = \frac{\hbar^2m^2}{2\mu r_0^2}
        \end{equation*}
        \begin{itemize}
            \item $m=0,1,2,\dots$ is a new quantum number.
            \item $m$ doesn't go to infinity because $|m|$ is bounded by $\ell$ (the total angular momentum).
        \end{itemize}
        \item Remembering our original restriction, we have that this math describes the system from Figure \ref{fig:partRota} but confined to rotate in the $xy$ plane with angular momentum in the $z$ direction.
        \begin{itemize}
            \item Thus, for example, the energies of the system from Figure \ref{fig:partRota} are dependent on $m$ and $I=\mu r_0^2$.
        \end{itemize}
        \item Such a system occurs in physical reality when we put the diatomic in an external field, or attach to it a big functional group.
        \item Zero point energy: $m=0$ does not violate the UR since we still have $\Delta L\Delta\theta\geq\hbar/2$ (as everything is still rotating in the sense that we have equal probability of the particle being everywhere [as opposed to more localized/normal rotation with higher values of $m$]).
    \end{itemize}
    \item 3D rigid rotor:
    \begin{itemize}
        \item Assume that the potential energy is zero on the surface of the sphere (so we basically have a particle on a sphere).
        \item Then
        \begin{equation*}
            \hat{H} = \frac{\hat{L}^2}{2\mu r_0^2} = \frac{\hat{L}^2}{2I}
        \end{equation*}
        \item Solving $\hat{H}\psi=E\psi$ asserts that the eigenfunctions of the Hamiltonian are the spherical harmonics $Y_{\ell m}(\theta,\phi)$.
        \item Energy:
        \begin{equation*}
            E_\ell = \frac{\hbar^2}{2I}\ell(\ell+1)
        \end{equation*}
        where $\ell=0,1,2,\dots$.
        \item Recall that $m$ corresponds to the projection of angular momentum onto the $z$-axis, so that
        \begin{equation*}
            m = -\ell,\dots,+\ell
        \end{equation*}
    \end{itemize}
\end{itemize}



\section{Hydrogen Atom}
\begin{itemize}
    \item \marginnote{10/20:}Microwaves (for food) excite the rotational motion of water molecules.
    \item Spherical harmonics: The solution of $\psi_{lm}(\theta,\phi)=Y_{lm}(\theta,\phi)$, where $l,m$ are quantum numbers.
    \item $E_l=\hbar^2/2I\cdot l(l+1)$ for $l=0,1,2,\dots$.
    \item Form of the spherical harmonics:
    \begin{equation*}
        Y(\theta,\phi) = P_{lm}(\cos\theta)\e[im\phi]
    \end{equation*}
    where $P_{lm}(cos\theta)$ is a polynomial.
    \item The polynomials $P_{lm}(\cos\theta)$ are the associated Legendre polynomials.
    \item When $m=0$, we have the Legendre polynomials.
    \begin{itemize}
        \item The differential equation describing these is
        \begin{equation*}
            \dv{x}\left( (1-x^2)\dv{x}P_l(x)+l(l+1)P_l(x) \right) = 0
        \end{equation*}
        \item The Legendre polynomials converge very quickly to functions on $[-1,1]$.
        \item People map these polynomials onto other domains, too, to solve a variety of problems.
        \item Legendre polynomials have more of their roots at the boundaries --- since the boundary conditions are the most important part of solving a differential equation, it makes sense that accurate representations would sample near the boundary more.
        \item Examples:
        \begin{align*}
            P_0(x) &= 1&
            P_1(x) &= x\\
            P_2(x) &= \frac{3}{2}x^2-\frac{1}{2}&
            P_3(x) &= \frac{5}{2}x^3-\frac{3}{2}x
        \end{align*}
    \end{itemize}
    \item Consider \ce{HCl}.
    \begin{itemize}
        \item For it,
        \begin{equation*}
            \frac{\hbar^2}{2I} = \SI{1.3e-3}{\electronvolt}
        \end{equation*}
        \item Rotational spectral lines may arise from different values of the quantum number $l$.
        \item The molecule vibrates in harmonic oscillation with spacings $\approx\SI{0.1}{\electronvolt}$ (it's not strictly rigid).
        \item Rovibrational spectra includes both forms of movement.
        \begin{itemize}
            \item Very high precision.
            \item Very big in the 90s.
        \end{itemize}
        \item Electronic spectra: A few electron volts.
    \end{itemize}
    \item The hydrogen atom.
    \begin{itemize}
        \item Two generalizations of the 3D rigid rotor combine to treat the hydrogen atom:
        \begin{enumerate}
            \item An addition of the kinetic energy in the radial direction $\hat{r}$.
            \item An addition of the Coulomb potential.
        \end{enumerate}
        \item Schr\"{o}dinger equation:
        \begin{equation*}
            \hat{H}\psi(r,\theta,\phi) = E\psi(r,\theta,\phi)
        \end{equation*}
        where
        \begin{gather*}
            \hat{H} = \frac{1}{2\mu}\left( \hat{p}_r^2+\frac{\hat{L}^2}{r^2} \right)+V(r)\\
            \hat{p}_r^2 = -\frac{\hbar^2}{r^2}\pdv{r}\left( r^2\pdv{r}\psi(r) \right)\\
            V(r) = -\frac{e(eZ)}{4\pi\epsilon_0r}
        \end{gather*}
        \item Because the particle has spherical symmetry (that it, does not depend on $\theta$ or $\phi$), the wave function is separable, that is, it may be written as a product
        \begin{equation*}
            \psi(r,\theta,\phi) = R_n(r)Y_{lm}(\theta,\phi)
        \end{equation*}
        \item Note that there is no analytic solution to the Schr\"{o}dinger equation in Cartesian coordinates --- we need spherical coordinates to take advantage of the spherical symmetry.
        \item Substitution into the Schr\"{o}dinger equation yields
        \begin{equation*}
            \left( \frac{1}{2\mu}\left( \hat{p}_r^2+\frac{\hat{L}^2}{r^2} \right)+V(r) \right)R(r)Y(\theta,\phi) = ER(r)Y(\theta,\phi)
        \end{equation*}
        where $\mu$ is the reduced mass of the electron and the nucleus (which is approximately the mass of the electron).
        \item But since
        \begin{equation*}
            \frac{1}{2\mu r^2}\hat{L}^2Y_{lm}(\theta,\phi) = \frac{l(l+1)}{2\mu r^2}\hbar^2Y_{lm}(\theta,\phi)
        \end{equation*}
        we have that
        \begin{equation*}
            \frac{1}{2\mu}\left( \hat{p}_r^2+\frac{l(l+1)\hbar^2}{r^2}+V(r) \right)R_n(r) = E_nR_n(r)
        \end{equation*}
        \begin{itemize}
            \item We have reduced the three-dimensional case of the hydrogen atom to a one-dimensional differential equation.
        \end{itemize}
    \end{itemize}
\end{itemize}



\section{Chapter 5: The Harmonic Oscillator and the Rigid Rotator --- Two Spectroscopic Models}
\emph{From \textcite{bib:McQuarrieSimon}.}
\begin{itemize}
    \item \marginnote{10/19:}\textbf{Rigid-rotator model}: Two point masses $m_1$ and $m_2$ at fixed distances $r_1$ and $r_2$ from their center of mass.
    \begin{itemize}
        \item Since the vibrational amplitude of a rotating molecule is small compared to its amplitude, this is a good model.
    \end{itemize}
    \item Kinetic energy of the rigid rotator:
    \begin{align*}
        K &= \frac{1}{2}m_1v_1^2+\frac{1}{2}m_2v_2^2\\
        &= \frac{1}{2}(m_1r_1^2+m_2r_2^2)\omega^2\\
        &= \frac{1}{2}I\omega^2
    \end{align*}
    \begin{itemize}
        \item Note that
        \begin{align*}
            I &= m_1r_1^2+m_2r_2^2\\
            &= \mu r^2
        \end{align*}
        (see Problem \ref{prb:5-29}).
    \end{itemize}
    \item It follows that we the two-body problem of the rigid rotator is equivalent to the one-body problem of a single body of mass $\mu$ rotating at a distance $r$ from a fixed center.
    \item Since there are no external forces on the rigid rotator (we're not applying any electric or magnetic fields), the energy of the molecule is solely kinetic (i.e., there is no potential energy term in the Hamiltonian).
    \begin{itemize}
        \item Thus, for a rigid rotator,
        \begin{equation*}
            \hat{H} = \hat{K} = -\frac{\hbar^2}{2\mu}\nabla^2
        \end{equation*}
    \end{itemize}
    \item Since this particle has a natural center of spherical symmetry, we opt for spherical coordinates. However, this necessitates expressing $\nabla^2$ as the following.
    \begin{equation*}
        \nabla^2 = \frac{1}{r^2}\pdv{r}\left( r^2\pdv{r} \right)_{\theta,\phi}+\frac{1}{r^2\sin\theta}\pdv{\theta}\left( \sin\theta\pdv{\theta} \right)_{r,\phi}+\frac{1}{r^2\sin^2\theta}\left( \pdv[2]{\phi} \right)_{r,\theta}
    \end{equation*}
    \begin{itemize}
        \item See Problem \ref{prb:5-32} for a derivation.
    \end{itemize}
    \item With respect to the rigid rotator, $r$ is constant. Thus,
    \begin{gather*}
        \nabla^2 = \frac{1}{r^2}\frac{1}{\sin\theta}\pdv{\theta}\left( \sin\theta\pdv{\theta} \right)+\frac{1}{r^2}\frac{1}{\sin^2\theta}\pdv[2]{\phi}\\
        \bar{H} = -\frac{\hbar^2}{2I}\left[ \frac{1}{\sin\theta}\pdv{\theta}\left( \sin\theta\pdv{\theta} \right)+\frac{1}{\sin^2\theta}\left( \pdv[2]{\phi} \right) \right]\\
        \bar{L}^2 = -\hbar^2\left[ \frac{1}{\sin\theta}\pdv{\theta}\left( \sin\theta\pdv{\theta} \right)+\frac{1}{\sin^2\theta}\left( \pdv[2]{\phi} \right) \right]
    \end{gather*}
    \begin{itemize}
        \item Note that since both $\theta$ and $\phi$ are unitless, the units of angular momentum for quantum system are $\hbar$.
    \end{itemize}
    \item Rigid-rotator wave functions are customarily denoted by $Y(\theta,\phi)$.
    \item In solving $\hat{H}Y(\theta,\phi)=EY(\theta,\phi)$, it will be useful to multiply the original Schr\"{o}dinger equation by $\sin^2\theta$ and let $\beta=2IE/\hbar^2$ to obtain the partial differential equation
    \begin{equation*}
        \sin\theta\pdv{\theta}\left( \sin\theta\pdv{Y}{\theta} \right)+\pdv[2]{Y}{\phi}+(\beta\sin^2\theta)Y = 0
    \end{equation*}
    \begin{itemize}
        \item The solutions to the above equation are intimately linked to those for the hydrogen atom.
        \item Solving the above equation yields the condition that $\beta=J(J+1)$ for $J=0,1,2,\dots$. Therefore,
        \begin{equation*}
            E_J = \frac{\hbar^2}{2I}J(J+1)
        \end{equation*}
        for $J=0,1,2,\dots$.
        \begin{itemize}
            \item Each energy level has a degeneracy $g_J=2J+1$ as well.
        \end{itemize}
    \end{itemize}
    \item Once again, electromagnetic radiation can cause a rigid rotator to undergo transitions from one state to another subject to the selection rules that only transitions between adjacent states are allowed and the molecule must possess a permanent dipole moment.
    \item As before, we can calculate
    \begin{equation*}
        \Delta E = \frac{h^2}{4\pi^2I}(J+1)
    \end{equation*}
    and the frequencies at which absorption transitions occur are
    \begin{equation*}
        \nu = \frac{h}{4\pi^2I}(J+1)
    \end{equation*}
    for $J=0,1,2,\dots$.
    \item It follows from reduced mass, bond length, and moment of inertia data that the frequencies typically lie in the microwave region.
    \item \textbf{Microwave spectroscopy}: The direct study of rotational transitions.
    \item \textbf{Rotational constant} (of a molecule): The following quantity. \emph{Given by}
    \begin{equation*}
        B = \frac{h}{8\pi^2I}
    \end{equation*}
    \begin{itemize}
        \item We often write the absorption frequencies as $\nu=2B(J+1)$.
    \end{itemize}
    \item The spacing of lines in a microwave spectrum is $2B$.
    \item Like IR spectroscopy can be used to determine the force constants of molecular attractions in diatomics, microwave spectroscopy can be used to determine the bond lengths of diatomics.
\end{itemize}


\subsection*{Problems}
\begin{enumerate}[label={\textbf{5-\arabic*.}},ref={5-\arabic*}]
    \setcounter{enumi}{28}
    \item \label{prb:5-29}Show that the moment of inertia for a rigid rotator can be written as $I=\mu r^2$ where $r=r_1+r_2$ (the fixed separation of the two masses) and $\mu$ is the reduced mass.
    \begin{proof}[Answer]
        First, note that $m_1r_1=m_2r_2$ for such a rotation about the center of mass. Then
        \begingroup
        \allowdisplaybreaks
        \begin{align*}
            I &= \mu r^2\\
            &= \frac{m_1m_2}{m_1+m_2}(r_1+r_2)^2\\
            &= \frac{m_1m_2r_1}{m_1r_1+m_2r_1}(r_1+r_2)^2\\
            &= \frac{m_1m_2r_1}{m_2r_2+m_2r_1}(r_1+r_2)^2\\
            &= \frac{m_1r_1}{r_2+r_1}(r_1+r_2)^2\\
            &= m_1r_1(r_1+r_2)\\
            &= m_1r_1^2+m_1r_1r_2\\
            &= m_1r_1^2+m_2r_2^2
        \end{align*}
        \endgroup
        as desired.
    \end{proof}
\end{enumerate}




\end{document}