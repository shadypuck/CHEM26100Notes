\documentclass[../notes.tex]{subfiles}

\pagestyle{main}
\renewcommand{\chaptermark}[1]{\markboth{\chaptername\ \thechapter\ (#1)}{}}
\setcounter{chapter}{3}

\begin{document}




\chapter{The Hydrogen Atom and Angular Momentum}
\section{Rotational Motion}
\begin{itemize}
    \item \marginnote{10/18:}Consider the diatomic molecule \ce{AB} at a distance $r_0$ apart rotating about its center of mass.
    \begin{figure}[h!]
        \centering
        \begin{subfigure}[b]{0.3\linewidth}
            \centering
            \begin{tikzpicture}
                \footnotesize
                \draw (0,0) node[circle,fill=rex,inner sep=1.5pt,label={below:$A$}]{}
                    -- node[pos=0.65,fill,inner sep=1.5pt,label={below:CM}]{}
                    (10:2) node[circle,fill=blx,inner sep=1.5pt,label={below:$B$}]{}
                ;
                \draw [decorate,decoration=brace] (100:0.2) -- node[above=2pt]{$r_0$} ++(10:2);
            \end{tikzpicture}
            \caption{Diatomic.}
            \label{fig:partRota}
        \end{subfigure}
        \begin{subfigure}[b]{0.3\linewidth}
            \centering
            \begin{tikzpicture}
                \footnotesize
                \draw [semithick,-latex] (10:2) -- ++(100:0.5);
                \draw (0,0) node[fill,inner sep=1.5pt,label={left:origin}]{}
                    --
                    (10:2) node[circle,fill=blx!65!rex,inner sep=1.5pt,label={right:$\mu$}]{}
                ;
                \draw [decorate,decoration={brace,mirror}] (-80:0.2) -- node[below=2pt]{$r_0$} ++(10:2);
            \end{tikzpicture}
            \caption{An equivalent system.}
            \label{fig:partRotb}
        \end{subfigure}
        \caption{Diatomic rotation.}
        \label{fig:partRot}
    \end{figure}
    \begin{itemize}
        \item To simplify the problem, replace the two particles rotating about the center of mass with one particle of reduced mass $\mu$ rotating about the center of mass with lever arm $r_0$.
    \end{itemize}
    \item Classically, the kinetic energy of the translational motion is
    \begin{equation*}
        T = \frac{L^2}{2I}
    \end{equation*}
    where $I=\mu r_0^2$ and $L=p\times r_0=pr_0$ (for this kind of rotation; see Figure \ref{fig:partRotb}).
    \item To further talk about this problem, we should introduce \textbf{spherical coordinates}.
    \item \textbf{Spherical coordinates}: The coordinate system $(r,\theta,\phi)$ related to the Cartesian coordinates $(x,y,z)$ by
    \begin{align*}
        x &= r\sin\theta\cos\phi&
        y &= r\sin\theta\sin\phi&
        z &= r\cos\theta
    \end{align*}
    \item Classically, we will have
    \begin{equation*}
        H = \frac{1}{2\mu}(p_x^2+p_y^2+p_z^2)+V(x,y,z)
    \end{equation*}
    in Cartesian coordinates.
    \item In spherical coordinates, this becomes
    \begin{equation*}
        H = \frac{1}{2\mu}\left( p_r^2+\frac{L^2}{r^2} \right)+V(r)
    \end{equation*}
    \item Thus, in quantum mechanics, we get
    \begin{align*}
        \hat{H} &= \frac{1}{2\mu}(\hat{p}_x^2+\hat{p}_y^2+\hat{p}_z^2)+V(x,y,z)\\
        &= \frac{1}{2\mu}\left( \hat{p}_r^2+\frac{\hat{L}^2}{r^2} \right)+V(r)
    \end{align*}
    \item Thus, we have in spherical coordinates that
    \begin{equation*}
        \hat{T}\psi = -\frac{\hbar^2}{2\mu}\left[ \frac{1}{r^2}\pdv{r}\left( r^2\pdv{r}\psi \right)+\frac{1}{r^2\sin\theta}\left( \pdv{\theta}\left( \sin\theta\pdv{\theta}\psi \right) \right)+\frac{1}{r^2\sin^2\theta}\pdv[2]{\phi}\psi \right]
    \end{equation*}
    \item 2D rigid rotor:
    \begin{itemize}
        \item Let the system from Figure \ref{fig:partRotb} be confined to rotating in two dimensions.
        \item This simplifies the problem since both the $\pdv*{r}$ and $\pdv*{\theta}$ terms in the kinetic energy operator disappear (since, respectively, the particle is at a fixed distance from the center of mass and it cannot move out of the 2D plane).
        \item Thus, our Schr\"{o}dinger equation for this system is
        \begin{equation*}
            -\frac{\hbar^2}{2\mu r_0^2}\pdv[2]{\phi}\psi(\phi) = E\psi(\phi)
        \end{equation*}
        \item Solution: Let $\psi(\phi)=\e[im\phi]$; then
        \begin{equation*}
            E_m = \frac{\hbar^2m^2}{2\mu r_0^2}
        \end{equation*}
        \begin{itemize}
            \item $m=0,1,2,\dots$ is a new quantum number.
            \item $m$ doesn't go to infinity because $|m|$ is bounded by $\ell$ (the total angular momentum).
        \end{itemize}
        \item Remembering our original restriction, we have that this math describes the system from Figure \ref{fig:partRota} but confined to rotate in the $xy$ plane with angular momentum in the $z$ direction.
        \begin{itemize}
            \item Thus, for example, the energies of the system from Figure \ref{fig:partRota} are dependent on $m$ and $I=\mu r_0^2$.
        \end{itemize}
        \item Such a system occurs in physical reality when we put the diatomic in an external field, or attach to it a big functional group.
        \item Zero point energy: $m=0$ does not violate the UR since we still have $\Delta L\Delta\theta\geq\hbar/2$ (as everything is still rotating in the sense that we have equal probability of the particle being everywhere [as opposed to more localized/normal rotation with higher values of $m$]).
    \end{itemize}
    \item 3D rigid rotor:
    \begin{itemize}
        \item Assume that the potential energy is zero on the surface of the sphere (so we basically have a particle on a sphere).
        \item Then
        \begin{equation*}
            \hat{H} = \frac{\hat{L}^2}{2\mu r_0^2} = \frac{\hat{L}^2}{2I}
        \end{equation*}
        \item Solving $\hat{H}\psi=E\psi$ asserts that the eigenfunctions of the Hamiltonian are the spherical harmonics $Y_{\ell m}(\theta,\phi)$.
        \item Energy:
        \begin{equation*}
            E_\ell = \frac{\hbar^2}{2I}\ell(\ell+1)
        \end{equation*}
        where $\ell=0,1,2,\dots$.
        \item Recall that $m$ corresponds to the projection of angular momentum onto the $z$-axis, so that
        \begin{equation*}
            m = -\ell,\dots,+\ell
        \end{equation*}
    \end{itemize}
\end{itemize}



\section{Hydrogen Atom}
\begin{itemize}
    \item \marginnote{10/20:}Microwaves (for food) excite the rotational motion of water molecules.
    \item Spherical harmonics: The solution of $\psi_{lm}(\theta,\phi)=Y_{lm}(\theta,\phi)$, where $l,m$ are quantum numbers.
    \item $E_l=\hbar^2/2I\cdot l(l+1)$ for $l=0,1,2,\dots$.
    \item Form of the spherical harmonics:
    \begin{equation*}
        Y(\theta,\phi) = P_{lm}(\cos\theta)\e[im\phi]
    \end{equation*}
    where $P_{lm}(cos\theta)$ is a polynomial.
    \item The polynomials $P_{lm}(\cos\theta)$ are the associated Legendre polynomials.
    \item When $m=0$, we have the Legendre polynomials.
    \begin{itemize}
        \item The differential equation describing these is
        \begin{equation*}
            \dv{x}\left( (1-x^2)\dv{x}P_l(x)+l(l+1)P_l(x) \right) = 0
        \end{equation*}
        \item The Legendre polynomials converge very quickly to functions on $[-1,1]$.
        \item People map these polynomials onto other domains, too, to solve a variety of problems.
        \item Legendre polynomials have more of their roots at the boundaries --- since the boundary conditions are the most important part of solving a differential equation, it makes sense that accurate representations would sample near the boundary more.
        \item Examples:
        \begin{align*}
            P_0(x) &= 1&
            P_1(x) &= x\\
            P_2(x) &= \frac{3}{2}x^2-\frac{1}{2}&
            P_3(x) &= \frac{5}{2}x^3-\frac{3}{2}x
        \end{align*}
    \end{itemize}
    \item Consider \ce{HCl}.
    \begin{itemize}
        \item For it,
        \begin{equation*}
            \frac{\hbar^2}{2I} = \SI{1.3e-3}{\electronvolt}
        \end{equation*}
        \item Rotational spectral lines may arise from different values of the quantum number $l$.
        \item The molecule vibrates in harmonic oscillation with spacings $\approx\SI{0.1}{\electronvolt}$ (it's not strictly rigid).
        \item Rovibrational spectra includes both forms of movement.
        \begin{itemize}
            \item Very high precision.
            \item Very big in the 90s.
        \end{itemize}
        \item Electronic spectra: A few electron volts.
    \end{itemize}
    \item The hydrogen atom.
    \begin{itemize}
        \item Two generalizations of the 3D rigid rotor combine to treat the hydrogen atom:
        \begin{enumerate}
            \item An addition of the kinetic energy in the radial direction $\hat{r}$.
            \item An addition of the Coulomb potential.
        \end{enumerate}
        \item Schr\"{o}dinger equation:
        \begin{equation*}
            \hat{H}\psi(r,\theta,\phi) = E\psi(r,\theta,\phi)
        \end{equation*}
        where
        \begin{gather*}
            \hat{H} = \frac{1}{2\mu}\left( \hat{p}_r^2+\frac{\hat{L}^2}{r^2} \right)+V(r)\\
            \hat{p}_r^2 = -\frac{\hbar^2}{r^2}\pdv{r}\left( r^2\pdv{r}\psi(r) \right)\\
            V(r) = -\frac{e(eZ)}{4\pi\epsilon_0r}
        \end{gather*}
        \item Because the particle has spherical symmetry (that it, does not depend on $\theta$ or $\phi$), the wave function is separable, that is, it may be written as a product
        \begin{equation*}
            \psi(r,\theta,\phi) = R_n(r)Y_{lm}(\theta,\phi)
        \end{equation*}
        \item Note that there is no analytic solution to the Schr\"{o}dinger equation in Cartesian coordinates --- we need spherical coordinates to take advantage of the spherical symmetry.
        \item Substitution into the Schr\"{o}dinger equation yields
        \begin{equation*}
            \left( \frac{1}{2\mu}\left( \hat{p}_r^2+\frac{\hat{L}^2}{r^2} \right)+V(r) \right)R(r)Y(\theta,\phi) = ER(r)Y(\theta,\phi)
        \end{equation*}
        where $\mu$ is the reduced mass of the electron and the nucleus (which is approximately the mass of the electron).
        \item But since
        \begin{equation*}
            \frac{1}{2\mu r^2}\hat{L}^2Y_{lm}(\theta,\phi) = \frac{l(l+1)}{2\mu r^2}\hbar^2Y_{lm}(\theta,\phi)
        \end{equation*}
        we have that
        \begin{equation*}
            \frac{1}{2\mu}\left( \hat{p}_r^2+\frac{l(l+1)\hbar^2}{r^2}+V(r) \right)R_n(r) = E_nR_n(r)
        \end{equation*}
        \begin{itemize}
            \item We have reduced the three-dimensional case of the hydrogen atom to a one-dimensional differential equation.
        \end{itemize}
    \end{itemize}
\end{itemize}



\section{Hydrogen Atom (cont.)}
\begin{itemize}
    \item \marginnote{10/22:}We want to solve the SE $\hat{H}\psi_n=E\psi_n$ for the hydrogen atom.
    \item To exploit the spherical symmetry of the hydrogen atom, we use $(r,\theta,\phi)$.
    \item Thus, our Hamiltonian is equal to
    \begin{equation*}
        \hat{H} = \frac{1}{2\mu}\left( \hat{p}_r^2+\frac{\hat{L}^2}{r^2} \right)+V(r)
    \end{equation*}
    \begin{itemize}
        \item Since $V$ is just a function of $r$, the wave function is separable: $\psi(r,\theta,\phi)=R_n(r)Y_{l,m}(\theta,\phi)$.
        \item "Whenever you have a separation of variables additively in the Hamiltonian, you have a separation of variables multiplicatively in the wave function."
    \end{itemize}
    \item Substitute the wave function into the Schr\"{o}dinger equation:
    \begin{align*}
        \left( \frac{1}{2\mu}\left( \hat{p}_r^2+\frac{\hat{L}^2}{r^2} \right)+V(r) \right)R_n(r)Y_{lm}(\theta,\phi) &= ER_n(r)Y_{lm}(\theta,\phi)\\
        \left( \frac{1}{2\mu}\left( \hat{p}_r^2+\frac{l(l+1)\hbar^2}{r^2}\right)+V(r) \right)R_n(r)Y_{lm}(\theta,\phi) &= ER_n(r)Y_{lm}(\theta,\phi)\\
        \left( \frac{1}{2\mu}\left( \hat{p}_r^2+\frac{l(l+1)\hbar^2}{r^2} \right)+V(r) \right)R_n(r) &= E_{nl}R_n(r)
    \end{align*}
    \item Noting that we only care about the behavior of the differential equation on $[0,\infty)$, specifically at really large distances, we perform an asymptotic analysis.
    \begin{equation*}
        \lim_{r\to\infty}\left( \frac{1}{2\mu}\hat{p}_r^2 \right)R(r) = ER(r)
    \end{equation*}
    \item Plugging in the value of the momentum operator, we have that
    \begin{align*}
        -\frac{\hbar^2}{2\mu}\dv[2]{r}R(r) &= ER(r)\\
        R(r) &= \e[-\alpha r]
    \end{align*}
    where $\alpha=i\hbar/\sqrt{2\mu E}$.
    \item But we want to multiply $\e[-\alpha r]$ by some type of polynomial. Namely, \textbf{Laguerre polynomials}.
    \begin{itemize}
        \item These polynomials may be found by expanding a power series.
        \item Let
        \begin{equation*}
            L\left( \frac{\alpha r}{n} \right) = \sum_{j=0}^\infty c_j\left( \frac{\alpha r}{n} \right)^j
        \end{equation*}
        \item A recursion relation may be found.
        \item The quantization of energy in the hydrogen atom again arises from the truncation of the polynomials.
    \end{itemize}
    \item Thus, the general solution of $R_{nl}(r)$ is
    \begin{equation*}
        R_{nl}(r) = \left( \frac{\alpha r}{n} \right)^lL_{n+1,2l+1}\left( \frac{\alpha r}{n} \right)\e[-\alpha r/2n]
    \end{equation*}
    \item Energy levels:
    \begin{equation*}
        E_n = -\frac{\mu}{2\hbar^2}\left( \frac{(Ze)e}{4\pi\epsilon} \right)^2\frac{1}{n^2}
    \end{equation*}
    \begin{itemize}
        \item These are the discrete energies of Bohr!
    \end{itemize}
    \item But now we also have the electron's probability distributions, i.e., the ground state wave function is
    \begin{equation*}
        \psi(r,\theta,\phi) = \frac{1}{\sqrt{\pi}}\left( \frac{Z}{a_0} \right)^{3/2}\e[-Zr/a_0]
    \end{equation*}
    \item Thus, we can now take a look at the probabilities.
    \emph{picture}
    \begin{itemize}
        \item The probability is
        \begin{equation*}
            \text{Pr} = |\psi(r,\theta,\phi)|^2r^2\sin\theta
        \end{equation*}
        \item The radial probability is
        \begin{equation*}
            \text{Pr}(r) = r^2|R_{nl}(r)|^2
        \end{equation*}
        \begin{itemize}
            \item The radial probability peaks at $a_0$, the \textbf{Bohr radius}.
            \item Thus, the Bohr radius (the radius of the circular orbit of Bohr's hydrogen electron) is just the most probable distance from the nucleus!
        \end{itemize}
    \end{itemize}
    \item The average distance from the nucleus (the expectation value of $r$) is
    \begin{align*}
        \prb{r} &= \int_0^\infty\text{Pr}(r)r\dd{r}\\
        &= \int_0^\infty\psi^*(r)r\psi(r)\dd{r}
    \end{align*}
    \begin{itemize}
        \item Note we also find that $\prb{\hat{H}}=E$, so there is no uncertainty in the energy.
    \end{itemize}
    \item Note that in some ways, quantum mechanics is more certain than classical mechanics since, for instance, in quantum we know the energy exactly.
\end{itemize}



\section{Chapter 5: The Harmonic Oscillator and the Rigid Rotator --- Two Spectroscopic Models}
\emph{From \textcite{bib:McQuarrieSimon}.}
\begin{itemize}
    \item \marginnote{10/19:}\textbf{Rigid-rotator model}: Two point masses $m_1$ and $m_2$ at fixed distances $r_1$ and $r_2$ from their center of mass.
    \begin{itemize}
        \item Since the vibrational amplitude of a rotating molecule is small compared to its amplitude, this is a good model.
    \end{itemize}
    \item Kinetic energy of the rigid rotator:
    \begin{align*}
        K &= \frac{1}{2}m_1v_1^2+\frac{1}{2}m_2v_2^2\\
        &= \frac{1}{2}(m_1r_1^2+m_2r_2^2)\omega^2\\
        &= \frac{1}{2}I\omega^2
    \end{align*}
    \begin{itemize}
        \item Note that
        \begin{align*}
            I &= m_1r_1^2+m_2r_2^2\\
            &= \mu r^2
        \end{align*}
        (see Problem \ref{prb:5-29}).
    \end{itemize}
    \item It follows that we the two-body problem of the rigid rotator is equivalent to the one-body problem of a single body of mass $\mu$ rotating at a distance $r$ from a fixed center.
    \item Since there are no external forces on the rigid rotator (we're not applying any electric or magnetic fields), the energy of the molecule is solely kinetic (i.e., there is no potential energy term in the Hamiltonian).
    \begin{itemize}
        \item Thus, for a rigid rotator,
        \begin{equation*}
            \hat{H} = \hat{K} = -\frac{\hbar^2}{2\mu}\nabla^2
        \end{equation*}
    \end{itemize}
    \item Since this particle has a natural center of spherical symmetry, we opt for spherical coordinates. However, this necessitates expressing $\nabla^2$ as the following.
    \begin{equation*}
        \nabla^2 = \frac{1}{r^2}\pdv{r}\left( r^2\pdv{r} \right)_{\theta,\phi}+\frac{1}{r^2\sin\theta}\pdv{\theta}\left( \sin\theta\pdv{\theta} \right)_{r,\phi}+\frac{1}{r^2\sin^2\theta}\left( \pdv[2]{\phi} \right)_{r,\theta}
    \end{equation*}
    \begin{itemize}
        \item See Problem \ref{prb:5-32} for a derivation.
    \end{itemize}
    \item With respect to the rigid rotator, $r$ is constant. Thus,
    \begin{gather*}
        \nabla^2 = \frac{1}{r^2}\frac{1}{\sin\theta}\pdv{\theta}\left( \sin\theta\pdv{\theta} \right)+\frac{1}{r^2}\frac{1}{\sin^2\theta}\pdv[2]{\phi}\\
        \hat{H} = -\frac{\hbar^2}{2I}\left[ \frac{1}{\sin\theta}\pdv{\theta}\left( \sin\theta\pdv{\theta} \right)+\frac{1}{\sin^2\theta}\left( \pdv[2]{\phi} \right) \right]\\
        \hat{L}^2 = -\hbar^2\left[ \frac{1}{\sin\theta}\pdv{\theta}\left( \sin\theta\pdv{\theta} \right)+\frac{1}{\sin^2\theta}\left( \pdv[2]{\phi} \right) \right]
    \end{gather*}
    \begin{itemize}
        \item Note that since both $\theta$ and $\phi$ are unitless, the units of angular momentum for quantum system are $\hbar$.
    \end{itemize}
    \item Rigid-rotator wave functions are customarily denoted by $Y(\theta,\phi)$.
    \item In solving $\hat{H}Y(\theta,\phi)=EY(\theta,\phi)$, it will be useful to multiply the original Schr\"{o}dinger equation by $\sin^2\theta$ and let $\beta=2IE/\hbar^2$ to obtain the partial differential equation
    \begin{equation*}
        \sin\theta\pdv{\theta}\left( \sin\theta\pdv{Y}{\theta} \right)+\pdv[2]{Y}{\phi}+(\beta\sin^2\theta)Y = 0
    \end{equation*}
    \begin{itemize}
        \item The solutions to the above equation are intimately linked to those for the hydrogen atom.
        \item Solving the above equation yields the condition that $\beta=J(J+1)$ for $J=0,1,2,\dots$. Therefore,
        \begin{equation*}
            E_J = \frac{\hbar^2}{2I}J(J+1)
        \end{equation*}
        for $J=0,1,2,\dots$.
        \begin{itemize}
            \item Each energy level has a degeneracy $g_J=2J+1$ as well.
        \end{itemize}
    \end{itemize}
    \item Once again, electromagnetic radiation can cause a rigid rotator to undergo transitions from one state to another subject to the selection rules that only transitions between adjacent states are allowed and the molecule must possess a permanent dipole moment.
    \item As before, we can calculate
    \begin{equation*}
        \Delta E = \frac{h^2}{4\pi^2I}(J+1)
    \end{equation*}
    and the frequencies at which absorption transitions occur are
    \begin{equation*}
        \nu = \frac{h}{4\pi^2I}(J+1)
    \end{equation*}
    for $J=0,1,2,\dots$.
    \item It follows from reduced mass, bond length, and moment of inertia data that the frequencies typically lie in the microwave region.
    \item \textbf{Microwave spectroscopy}: The direct study of rotational transitions.
    \item \textbf{Rotational constant} (of a molecule): The following quantity. \emph{Given by}
    \begin{equation*}
        B = \frac{h}{8\pi^2I}
    \end{equation*}
    \begin{itemize}
        \item We often write the absorption frequencies as $\nu=2B(J+1)$.
    \end{itemize}
    \item The spacing of lines in a microwave spectrum is $2B$.
    \item Like IR spectroscopy can be used to determine the force constants of molecular attractions in diatomics, microwave spectroscopy can be used to determine the bond lengths of diatomics.
\end{itemize}


\subsection*{Problems}
\begin{enumerate}[label={\textbf{5-\arabic*.}},ref={5-\arabic*}]
    \setcounter{enumi}{28}
    \item \label{prb:5-29}Show that the moment of inertia for a rigid rotator can be written as $I=\mu r^2$ where $r=r_1+r_2$ (the fixed separation of the two masses) and $\mu$ is the reduced mass.
    \begin{proof}[Answer]
        First, note that $m_1r_1=m_2r_2$ for such a rotation about the center of mass. Then
        \begingroup
        \allowdisplaybreaks
        \begin{align*}
            I &= \mu r^2\\
            &= \frac{m_1m_2}{m_1+m_2}(r_1+r_2)^2\\
            &= \frac{m_1m_2r_1}{m_1r_1+m_2r_1}(r_1+r_2)^2\\
            &= \frac{m_1m_2r_1}{m_2r_2+m_2r_1}(r_1+r_2)^2\\
            &= \frac{m_1r_1}{r_2+r_1}(r_1+r_2)^2\\
            &= m_1r_1(r_1+r_2)\\
            &= m_1r_1^2+m_1r_1r_2\\
            &= m_1r_1^2+m_2r_2^2
        \end{align*}
        \endgroup
        as desired.
    \end{proof}
    \item \label{prb:5-30}Consider the transformation from Cartesian coordinates to plane polar coordinates where
    \begin{align*}
        x &= r\cos\theta&
            r &= \sqrt{x^2+y^2}\\
        y &= r\sin\theta&
            \theta &= \tan^{-1}\left( \frac{y}{x} \right)
    \end{align*}
    If a function $f(r,\theta)$ depends upon the polar coordinates $r$ and $\theta$, then the chain rule of partial differentiation says that\footnote{Note that the subscript means that the subscripted variable is held constant.}
    \begin{equation*}
        \left( \pdv{f}{x} \right)_y = \left( \pdv{f}{r} \right)_\theta\left( \pdv{r}{x} \right)_y+\left( \pdv{f}{\theta} \right)_r\left( \pdv{\theta}{x} \right)_y
    \end{equation*}
    and that
    \begin{equation*}
        \left( \pdv{f}{y} \right)_x = \left( \pdv{f}{r} \right)_\theta\left( \pdv{r}{y} \right)_x+\left( \pdv{f}{\theta} \right)_r\left( \pdv{\theta}{y} \right)_x
    \end{equation*}
    For simplicity, we will assume that $r$ is constant so that we can ignore terms involving derivatives with respect to $r$. In other words, we will consider a particle that is constrained to move on the circumference of a circle. This system is sometimes called a \textbf{particle on a ring}. Using the above equations, show that
    \begin{align*}
        \left( \pdv{f}{x} \right)_y &= -\frac{\sin\theta}{r}\left( \pdv{f}{\theta} \right)_r&
        \left( \pdv{f}{y} \right)_x &= \frac{\cos\theta}{r}\left( \pdv{f}{\theta} \right)_r
    \end{align*}
    for $r$ fixed. Now apply the above equations again to show that
    \begin{align*}
        \left( \pdv[2]{f}{x} \right)_y &= \left[ \pdv{x}\left( \pdv{f}{x} \right)_y \right]\\
        &= \left[ \pdv{\theta}\left( \pdv{f}{x} \right)_y \right]_r\left( \pdv{\theta}{x} \right)_y\\
        &= \left\{ \pdv{\theta}\left[ -\frac{\sin\theta}{r}\left( \pdv{f}{\theta} \right)_r \right] \right\}_r\left( -\frac{\sin\theta}{r} \right)\\
        &= \frac{\sin\theta\cos\theta}{r^2}\left( \pdv{f}{\theta} \right)_r+\frac{\sin^2\theta}{r^2}\left( \pdv[2]{f}{\theta} \right)_r
    \end{align*}
    for $r$ fixed. Similarly, show that
    \begin{equation*}
        \left( \pdv[2]{f}{y} \right)_x = -\frac{\sin\theta\cos\theta}{r^2}\left( \pdv{f}{\theta} \right)_r+\frac{\cos^2\theta}{r^2}\left( \pdv[2]{f}{\theta} \right)_r
    \end{equation*}
    and that
    \begin{equation*}
        \nabla^2f = \pdv[2]{f}{x}+\pdv[2]{f}{y} \longrightarrow \frac{1}{r^2}\left( \pdv[2]{f}{\theta} \right)_r
    \end{equation*}
    both for $r$ fixed. Now show that the Schr\"{o}dinger equation for a particle of mass $m$ constrained to move on a circle of radius $r$ (see Problem \ref{prb:3-28}) is
    \begin{equation*}
        -\frac{\hbar^2}{2I}\pdv[2]{\psi(\theta)}{\theta} = E\psi(\theta)
    \end{equation*}
    where $I=mr^2$ is the moment of inertia and $0\leq\theta\leq 2\pi$.
    \begin{proof}[Answer]
        We have that
        \begin{align*}
            \left( \pdv{\theta}{x} \right)_y &= \pdv{x}\left( \tan^{-1}\left( \frac{y}{x} \right) \right)&
                \left( \pdv{\theta}{y} \right)_y &= \pdv{y}\left( \tan^{-1}\left( \frac{y}{x} \right) \right)\\
            &= \frac{1}{\left( \frac{y}{x} \right)^2+1}\cdot y\cdot -\frac{1}{x^2}&
                &= \frac{1}{\left( \frac{y}{x} \right)^2+1}\cdot\frac{1}{x}\\
            &= -\frac{y}{x^2+y^2}&
                &= \frac{x}{x^2+y^2}\\
            &= -\frac{r\sin\theta}{r^2}&
                &= \frac{r\cos\theta}{r^2}\\
            &= -\frac{\sin\theta}{r}&
                &= \frac{\cos\theta}{r}
        \end{align*}
        This combined with the fact that $r$ is constant yields
        \begin{align*}
            \left( \pdv{f}{x} \right)_y &= \left( \pdv{f}{r} \right)_\theta\left( \pdv{r}{x} \right)_y+\left( \pdv{f}{\theta} \right)_r\left( \pdv{\theta}{x} \right)_y&
                \left( \pdv{f}{y} \right)_x &= \left( \pdv{f}{r} \right)_\theta\left( \pdv{r}{y} \right)_x+\left( \pdv{f}{\theta} \right)_r\left( \pdv{\theta}{y} \right)_x\\
            &= 0\cdot\left( \pdv{r}{x} \right)_y+\left( \pdv{f}{\theta} \right)_r\cdot -\frac{\sin\theta}{r}&
                &= 0\cdot\left( \pdv{r}{y} \right)_x+\left( \pdv{f}{\theta} \right)_r\frac{\cos\theta}{r}\\
            &= -\frac{\sin\theta}{r}\left( \pdv{f}{\theta} \right)_r&
                &= \frac{\cos\theta}{r}\left( \pdv{f}{\theta} \right)_r
        \end{align*}
        as desired.\par
        The rest of the differential equation derivations follow fairly easy with simple calculus rules.\par
        As for the Schr\"{o}dinger equation, we know that there is no potential field for this particle. Additionally, although we could choose to express $\psi$ as a function of $x,y$, owing to the radial symmetry, we choose to express it as a function of $r,\theta$. In fact, we need only express $\psi$ as a function of $\theta$ since $r$ is invariant throughout the chosen free space. Thus, expanding from the general form, we get
        \begin{align*}
            \hat{H}\psi(\theta) &= E\psi(\theta)\\
            \left( -\frac{\hbar^2}{2m}\nabla^2 \right)\psi(\theta) &= E\psi(\theta)\\
            -\frac{\hbar^2}{2m}\cdot\frac{1}{r^2}\left( \pdv[2]{\psi(\theta)}{\theta} \right)_r &= E\psi(\theta)\\
            -\frac{\hbar^2}{2I}\pdv[2]{\psi(\theta)}{\theta} &= E\psi(\theta)
        \end{align*}
        as desired.
    \end{proof}
    \item \label{prb:5-31}Generalize Problem \ref{prb:5-30} to the case of a particle moving in a plane under the influence of a central force; in other words, convert
    \begin{equation*}
        \nabla^2 = \pdv[2]{x}+\pdv[2]{y}
    \end{equation*}
    to plane polar coordinates, this time without assuming that $r$ is constant. Use the method of separation of variables to separate the equation for this problem. Solve the angular equation.
    \begin{proof}[Answer]
        We have
        \begin{align*}
            \pdv{r}{x} &= \frac{1}{2\sqrt{x^2+y^2}}\cdot 2x&
                \pdv{r}{y} &= \frac{1}{2\sqrt{x^2+y^2}}\cdot 2y\\
            &= \frac{x}{r}&
                &= \frac{y}{r}\\
            &= \cos\theta&
                &= \sin\theta
        \end{align*}
        Thus
        \begin{align*}
            \pdv{f}{x} &= \cos\theta\pdv{f}{r}-\frac{\sin\theta}{r}\pdv{f}{\theta}&
            \pdv{f}{y} &= \sin\theta\pdv{f}{r}+\frac{\cos\theta}{r}\pdv{f}{\theta}
        \end{align*}
        Also note that
        \begin{align*}
            \pdv[2]{f}{x} &= \pdv{x}\left( \pdv{f}{x} \right)\\
            &= \pdv{\theta}\left( \pdv{f}{x} \right)\pdv{\theta}{x}+\pdv{r}\left( \pdv{f}{x} \right)\pdv{r}{x}
        \end{align*}
        The rest of the expansions are routine, leading to
        \begin{equation*}
            \nabla^2f = \pdv[2]{u}{r}+\frac{1}{r}\pdv{u}{r}+\frac{1}{r^2}\pdv[2]{u}{\theta}
        \end{equation*}
    \end{proof}
    \item \label{prb:5-32}Using Problems \ref{prb:5-30} and \ref{prb:5-31} as a guide, convert $\nabla^2$ from three-dimensional Cartesian coordinates to spherical coordinates.
\end{enumerate}



\section{Chapter 6: The Hydrogen Atom}
\emph{From \textcite{bib:McQuarrieSimon}.}
\begin{itemize}
    \item \marginnote{10/29:}Idealize the hydrogen atom to a proton fixed at the origin and an electron of mass $m_e$ interacting with the proton through a Coulombic potential
    \begin{equation*}
        V(r) = -\frac{e^2}{4\pi\epsilon_0r}
    \end{equation*}
    \item Because of the spherical geometry of the model, we opt for spherical coordinates.
    \item The appropriate Hamiltonian is thus
    \begin{equation*}
        \hat{H} = -\frac{\hbar}{2m_e}\nabla^2-\frac{e^2}{4\pi\epsilon_0r}
    \end{equation*}
    \item Expressing the Laplacian in spherical coordinates gives us
    \begin{equation*}
        \left\{ -\frac{\hbar^2}{2m_e}\left[ \frac{1}{r^2}\pdv{r}\left( r^2\pdv{\psi}{r} \right)+\frac{1}{r^2\sin\theta}\pdv{\theta}\left( \sin\theta\pdv{\psi}{\theta} \right)+\frac{1}{r^2\sin^2\theta}\pdv[2]{\psi}{\phi} \right]-\frac{e^2}{4\pi\epsilon_0r} \right\}\psi(r,\theta,\phi) = E\psi(r,\theta,\phi)
    \end{equation*}
    as the Schr\"{o}dinger equation for the hydrogen atom.
    \item Multiply through by $2m_er^2$ and move all terms over to the lefthand side of the equality to get the above equation into the following separable form.
    \begin{equation*}
        \left\{ -\hbar^2\left( \pdv{r}r^2\pdv{\psi}{r} \right)-\hbar^2\left[ \frac{1}{\sin\theta}\left( \pdv{\theta}\sin\theta\pdv{\psi}{\theta} \right)+\frac{1}{\sin^2\theta}\pdv[2]{\psi}{\phi} \right]-2m_er^2\left[ \frac{e^2}{4\pi\epsilon_0r}+E \right] \right\}\psi(r,\theta,\phi) = 0
    \end{equation*}
    \item Since all the $\theta,\phi$ dependence occurs within the large square brackets, assume that
    \begin{equation*}
        \psi(r,\theta,\phi) = R(r)Y(\theta,\phi)
    \end{equation*}
    \item Substituting into the differential equation and separating terms (dividing by $R(r)Y(\theta,\phi)$) yields
    \begin{equation*}
        -\frac{\hbar^2}{R(r)}\left[ \dv{r}\left( r^2\dv{R}{r} \right)+\frac{2m_er^2}{\hbar^2}\left( \frac{e^2}{4\pi\epsilon_0r}+E \right)R(r) \right]-\frac{\hbar^2}{Y(\theta,\phi)}\left[ \frac{1}{\sin\theta}\pdv{\theta}\left( \sin\theta\pdv{Y}{\theta} \right)+\frac{1}{\sin^2\theta}\pdv[2]{Y}{\phi} \right] = 0
    \end{equation*}
    \item All terms containing $r$ are in the first set of square brackets, and all terms containing $\theta,\phi$ are in the second set of square brackets. Since the above equation must hold for all $r,\theta,\phi$, and $r,\theta,\phi$ are independent variables, we must have that varying $r$ while holding $\theta,\phi$ constant does not change the $r$ term of the above equation (as this would violate the equality). Similarly, changing $\theta,\phi$ must not change the $\theta,\phi$ term. Therefore, we may let
    \begin{gather*}
        -\beta = -\frac{1}{R(r)}\left[ \dv{r}\left( r^2\dv{R}{r} \right)+\frac{2m_er^2}{\hbar^2}\left( \frac{e^2}{4\pi\epsilon_0r}+E \right)R(r) \right]\\
        \beta = -\frac{1}{Y(\theta,\phi)}\left[ \frac{1}{\sin\theta}\pdv{\theta}\left( \sin\theta\pdv{Y}{\theta} \right)+\frac{1}{\sin^2\theta}\pdv[2]{Y}{\phi} \right]
    \end{gather*}
    where $\beta$ is a \textbf{separation constant} into which we have incorporated $\hbar^2$.
    \item \textbf{Radial equation}: The first equation above.
    \item \textbf{Angular equation}: The second equation above after having been multiplied by $\sin^2\theta Y(\theta,\phi)$. \emph{Given by}
    \begin{equation*}
        \sin\theta\pdv{\theta}\left( \sin\theta\pdv{Y}{\theta}+\pdv[2]{Y}{\phi}+\beta\sin^2\theta Y \right) = 0
    \end{equation*}
    \begin{itemize}
        \item Note that this is identical to the partial differential equation derived in Chapter 5 for the rigid rotator.
        \item Thus, the angular parts of hydrogen atomic orbitals are also rigid-rotator wave functions.
    \end{itemize}
    \item \textbf{Spherical harmonic}: A wave function of the rigid rotator.
    \item We now solve the angular equation.
    \item Invoking separation of variables again with $Y(\theta,\phi)=\Theta(\theta)\Phi(\phi)$ yields
    \begin{equation*}
        \frac{\sin\theta}{\Theta(\theta)}\dv{\theta}\left( \sin\theta\dv{\Theta}{\theta} \right)+\beta\sin\theta+\frac{1}{\Phi(\phi)}\dv[2]{\Phi}{\phi} = 0
    \end{equation*}
    \item We now introduce a new separation constant (written in a form that predicts future algebraic manipulations) to get
    \begin{gather*}
        \frac{\sin\theta}{\Theta(\theta)}\dv{\theta}\left( \sin\theta\dv{\Theta}{\theta} \right)+\beta\sin^2\theta = m^2\\
        \frac{1}{\Phi(\phi)}\dv[2]{\Phi}{\phi} = -m^2
    \end{gather*}
    \item Solving the bottom equation above gives solutions
    \begin{align*}
        \Phi(\phi) &= A_m\e[im\phi]&
        \Phi(\phi) &= A_{-m}\e[-im\phi]
    \end{align*}
    where the constants $A_m,A_{-m}$ may depend on the value of $m$.
    \item From the domain of $\phi$, we have the boundary condition $\Phi(\phi+2\pi)=\Phi(\phi)$.
    \item This yields
    \begin{align*}
        A_m\e[im(\phi+2\pi)] &= A_m\e[im\phi]&
            A_{-m}\e[-im(\phi+2\pi)] &= A_{-m}\e[-im\phi]\\
        \e[i2\pi m] &= 1&
            \e[-i2\pi m] &= 1
    \end{align*}
    from which it follows that we must have
    \begin{align*}
        1+0i &= 1\\
        &= \e[\pm i2\pi m]\\
        &= \cos(2\pi m)\pm i\sin(2\pi m)
    \end{align*}
    \item Therefore, $m=0,\pm 1,\pm 2,\dots$.
    \item It follows that we can express the solutions to the bottom angular equation in the more compact form
    \begin{equation*}
        \Phi_m(\phi) = A_m\e[im\phi]
    \end{equation*}
    where $m=0,\pm 1,\pm 2,\dots$.
    \item Normalizing gives us the coefficient $A_m$:
    \begin{align*}
        1 &= \int_0^{2\pi}\Phi_m^*(\phi)\Phi_m(\phi)\dd{\phi}\\
        &= \int_0^{2\pi}(A_m\e[-im\phi])(A_m\e[im\phi])\dd{\phi}\\
        &= A_m^2\int_0^{2\pi}1\dd{\phi}\\
        A_m &= \frac{1}{\sqrt{2\pi}}
    \end{align*}
    \item Thus, the normalized wave functions of $\Phi(\phi)$ are
    \begin{equation*}
        \Phi(\phi) = \frac{1}{\sqrt{2\pi}}\e[im\phi]
    \end{equation*}
    for $m=0,\pm 1,\pm 2,\dots$.
    \item We now direct our attention to solving the differential equation containing $\Theta(\theta)$.
    \item Let $x=\cos\theta$ and $P(x)=\Theta(\theta)$. Substituting, we have (see Problem \ref{prb:6-2})
    \begin{equation*}
        (1-x^2)\dv[2]{P}{x}-2x\dv{P}{x}+\left[ \beta-\frac{m^2}{1-x^2} \right]P(x) = 0
    \end{equation*}
    where $m=0,\pm 1,\pm 2,\dots$.
    \begin{itemize}
        \item Note that since $x=\cos\theta$, $0\leq\theta\leq\pi$, the range of $x$ is $-1\leq x\leq 1$.
    \end{itemize}
    \item Solving the above equation gives \textbf{Legendre's equation}.
    \item \textbf{Legendre's equation}: The following equation. \emph{Given by}
    \begin{equation*}
        (1-x^2)\dv[2]{P}{x}-2x\dv{P}{x}+\left[ l(l+1)-\frac{m^2}{1-x^2} \right]P(x) = 0
    \end{equation*}
    \begin{itemize}
        \item The key difference between Legendre's equation and the previous one is that solving the previous one finds $\beta$ equal to $l(l+1)$ with $l=0,1,2,\dots$ and $|m|\leq l$ if the solutions are to remain finite.
    \end{itemize}
    \item \textbf{Legendre polynomials}: The solutions to Legendre's equation when $m=0$. \emph{Denoted by} $\bm{P_l(x)}$.
    \begin{table}[h!]
        \centering
        \small
        \renewcommand{\arraystretch}{1.4}
        \begin{tabular}{l}
            \toprule
            $P_0(x)=1$\\
            $P_1(x)=x$\\
            $P_2(x)=\frac{1}{2}(3x^2-1)$\\
            $P_3(x)=\frac{1}{2}(5x^3-3x)$\\
            $P_4(x)=\frac{1}{8}(35x^4-30x^2+3)$\\
            \bottomrule
        \end{tabular}
        \caption{The first few Legendre polynomials.}
        \label{tab:legendrePolynomials}
    \end{table}
    \begin{itemize}
        \item Notice that $P_l(x)$ is an even function if $l$ is even and an odd function if $l$ is odd.
        \item The factors in front of $P_l(x)$ are chosen such that $P_l(1)=1$.
        \item The Legendre polynomials are orthogonal: If $l\neq n$, then
        \begin{equation*}
            \int_{-1}^1P_l(x)P_n(x)\dd{x} = 0
        \end{equation*}
        \item We have that
        \begin{equation*}
            \int_{-1}^1[P_l(x)]^2\dd{x} = \frac{2}{2l+1}
        \end{equation*}
        for each $l$, so the normalization constant of $P_l(x)$ is
        \begin{equation*}
            \sqrt{\frac{2l+l}{2}}
        \end{equation*}
    \end{itemize}
    \item \textbf{Associated Legendre functions}: The solutions to Legendre's equation when $m\neq 0$. \emph{Denoted by} $\bm{P_l^{|m|}(x)}$. \emph{Given by}
    \begin{equation*}
        P_l^{|m|}(x) = (1-x^2)^{|m|/2}\dv[|m|]{x}P_l(x)
    \end{equation*}
    \begin{itemize}
        \item Note that only the magnitude of $m$ is present because $m^2$ is the only form of $m$ present in Legendre's equation.
    \end{itemize}
    \begin{table}[h!]\marginnote{10/30:}
        \centering
        \small
        \renewcommand{\arraystretch}{1.4}
        \begin{tabular}{lll}
            \toprule
            \textbf{Function} & \textbf{$\bm{x}$-coordinates} & \textbf{$\bm{\theta}$-coordinates}\\
            \midrule
            $P_0^0(x)$ & $1$ & $1$\\
            $P_1^0(x)$ & $x$ & $\cos\theta$\\
            $P_1^1(x)$ & $\sqrt{1-x^2}$ & $\sin\theta$\\
            $P_2^0(x)$ & $\frac{1}{2}(3x^2-1)$ & $\frac{1}{2}(3\cos^2\theta-1)$\\
            $P_2^1(x)$ & $3x\sqrt{1-x^2}$ & $3\cos\theta\sin\theta$\\
            $P_2^2(x)$ & $3(1-x^2)$ & $3\sin^2\theta$\\
            $P_3^0(x)$ & $\frac{1}{2}(5x^2-3x)$ & $\frac{1}{2}(5\cos^2\theta-3\cos\theta)$\\
            $P_3^1(x)$ & $\frac{3}{2}(5x^2-1)\sqrt{1-x^2}$ & $\frac{3}{2}(5\cos^2\theta-1)\sin\theta$\\
            $P_3^2(x)$ & $15x(1-x^2)$ & $15\cos\theta\sin^2\theta$\\
            $P_3^3(x)$ & $15\sqrt{1-x^2}$ & $15\sin^2\theta$\\
            \bottomrule
        \end{tabular}
        \caption{The first few associated Legendre functions.}
        \label{tab:associatedLegendreFunctions}
    \end{table}
    \item Relating the normalization and orthogonality conditions back to the variable of interest ($\theta$), we have (for the Legendre polynomials) that
    \begin{equation*}
        \int_0^\pi P_l(\cos\theta)P_n(\cos\theta)\sin\theta\dd{\theta} = \frac{2\delta_{ln}}{2l+1}
    \end{equation*}
    \item Similarly, for the associated Legendre functions, we have that
    \begin{equation*}
        \int_0^\pi P_l^{|m|}(\cos\theta)P_n^{|m|}(\cos\theta)\sin\theta\dd{\theta} = \frac{2}{2l+1}\frac{(l+|m|)!}{(l-|m|)!}\delta_{ln}
    \end{equation*}
    \begin{itemize}
        \item Thus, the normalization constant for the associated Legendre functions is
        \begin{equation*}
            N_{lm} = \sqrt{\frac{2l+1}{2}\frac{(l-|m|)!}{(l+|m|)!}}
        \end{equation*}
    \end{itemize}
    \item Thus, the normalized wave functions of $\Theta(\theta)$ are
    \begin{equation*}
        \Theta(\theta) = N_{lm}P_l^{|m|}(\cos\theta)
    \end{equation*}
    for $l=0,1,2,\dots$ and $m=0,\pm 1,\dots,\pm l$.
    \item \textbf{Spherical harmonics}: The orthonormal set of angular wave functions of the hydrogen atom. \emph{Given by}
    \begin{equation*}
        Y_l^m(\theta,\phi) = \Theta(\theta)\Phi(\phi)
        = \sqrt{\frac{2l+1}{4\pi}\frac{(l-|m|)!}{(l+|m|)!}}P_l^{|m|}(\cos\theta)\e[im\phi]
    \end{equation*}
    \begin{itemize}
        \item Note that $l=0,1,2,\dots$ and $m=0,\pm 1,\dots,\pm l$.
        \item The orthonormality condition:
        \begin{equation*}
            \int_0^\pi\dd{\theta}\sin\theta\int_0^{2\pi}\dd{\phi}Y_l^m(\theta,\phi)^*Y_n^k(\theta,\phi) = \delta_{ln}\delta_{mk}
        \end{equation*}
    \end{itemize}
    \begin{table}[h!]
        \centering
        \small
        \renewcommand{\arraystretch}{2.5}
        \begin{tabular}{ll}
            \toprule
            $Y_0^0=\dfrac{1}{\sqrt{4\pi}}$ & $Y_2^1=\sqrt{\dfrac{15}{8\pi}}\sin\theta\cos\theta\e[i\phi]$\\
            $Y_1^0=\sqrt{\dfrac{3}{4\pi}}\cos\theta$ & $Y_2^{-1}=\sqrt{\dfrac{15}{8\pi}}\sin\theta\cos\theta\e[-i\phi]$\\
            $Y_1^1=\sqrt{\dfrac{3}{8\pi}}\sin\theta\e[i\phi]$ & $Y_2^2=\sqrt{\dfrac{15}{32\pi}}\sin^2\theta\e[2i\phi]$\\
            $Y_1^{-1}=\sqrt{\dfrac{3}{8\pi}}\sin\theta\e[-i\phi]$ & $Y_2^{-2}=\sqrt{\dfrac{15}{32\pi}}\sin^2\theta\e[-2i\phi]$\\
            $Y_2^0=\sqrt{\dfrac{5}{16\pi}}(3\cos^2\theta-1)$ & \\
            \bottomrule
        \end{tabular}
        \caption{The first few spherical harmonics.}
        \label{tab:sphericalHarmonics}
    \end{table}
\end{itemize}


\subsection*{Exercises}
\subsection*{Problems}
\begin{enumerate}[label={\textbf{6-\arabic*.}},ref={6-\arabic*}]
    \stepcounter{enumi}
    \item \label{prb:6-2}\marginnote{10/29:}In terms of the variable $\theta$, Legendre's equation is
    \begin{equation*}
        \sin\theta\dv{\theta}\left( \sin\theta\dv{\Theta(\theta)}{\theta} \right)+(\beta\sin^2\theta-m^2)\Theta(\theta) = 0
    \end{equation*}
    Let $x=\cos\theta$ and $P(x)=\Theta(\theta)$. Show that
    \begin{equation*}
        (1-x^2)\dv[2]{P(x)}{x}-2x\dv{P(x)}{x}+\left[ \beta-\frac{m^2}{1-x^2} \right]P(x) = 0
    \end{equation*}
    \begin{proof}[Answer]
        If $\cos\theta=\frac{x}{1}$, then
        \begin{equation*}
            \sin\theta = \sqrt{1-x^2}
        \end{equation*}
        Additionally,
        \begin{equation*}
            \dv{\Theta}{\theta} = \dv{P}{\theta}
            = \dv{P}{x}\dv{x}{\theta}
            = -\sin\theta\dv{P}{x}
            = -\sqrt{1-x^2}\dv{P}{x}
        \end{equation*}
        Therefore, we have by substitution and simplification that
        \begin{align*}
            0 &= \sqrt{1-x^2}\dv{\theta}\left( \sqrt{1-x^2}\cdot -\sqrt{1-x^2}\dv{P}{x} \right)+(\beta^2(1-x^2)-m^2)P(x)\\
            &= \sqrt{1-x^2}\dv{x}\left( -(1-x^2)\dv{P}{x} \right)\dv{x}{\theta}+(\beta^2(1-x^2)-m^2)P(x)\\
            &= (1-x^2)\dv{x}\left( (1-x^2)\dv{P}{x} \right)+(\beta^2(1-x^2)-m^2)P(x)\\
            &= (1-x^2)\left[ (1-x^2)\dv[2]{P}{x}-2x\dv{P}{x} \right]+(\beta^2(1-x^2)-m^2)P(x)\\
            &= (1-x^2)\dv[2]{P}{x}-2x\dv{P}{x}+\left[ \beta^2-\frac{m^2}{1-x^2} \right]P(x)
        \end{align*}
    \end{proof}
\end{enumerate}




\end{document}