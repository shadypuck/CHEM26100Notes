\documentclass[../notes.tex]{subfiles}

\pagestyle{main}
\renewcommand{\chaptermark}[1]{\markboth{\chaptername\ \thechapter\ (#1)}{}}
\setcounter{chapter}{3}

\begin{document}




\chapter{Vibrational Motion and the Harmonic Oscillator}
\section{Rotational Motion}
\begin{itemize}
    \item \marginnote{10/18:}Consider the diatomic molecule \ce{AB} at a distance $r_0$ apart rotating about its center of mass.
    \begin{figure}[h!]
        \centering
        \begin{subfigure}[b]{0.3\linewidth}
            \centering
            \begin{tikzpicture}
                \footnotesize
                \draw (0,0) node[circle,fill=rex,inner sep=1.5pt,label={below:$A$}]{}
                    -- node[pos=0.65,fill,inner sep=1.5pt,label={below:CM}]{}
                    (10:2) node[circle,fill=blx,inner sep=1.5pt,label={below:$B$}]{}
                ;
                \draw [decorate,decoration=brace] (100:0.2) -- node[above=2pt]{$r_0$} ++(10:2);
            \end{tikzpicture}
            \caption{Diatomic.}
            \label{fig:partRota}
        \end{subfigure}
        \begin{subfigure}[b]{0.3\linewidth}
            \centering
            \begin{tikzpicture}
                \footnotesize
                \draw [semithick,-latex] (10:2) -- ++(100:0.5);
                \draw (0,0) node[fill,inner sep=1.5pt,label={left:origin}]{}
                    --
                    (10:2) node[circle,fill=blx!65!rex,inner sep=1.5pt,label={right:$\mu$}]{}
                ;
                \draw [decorate,decoration={brace,mirror}] (-80:0.2) -- node[below=2pt]{$r_0$} ++(10:2);
            \end{tikzpicture}
            \caption{An equivalent system.}
            \label{fig:partRotb}
        \end{subfigure}
        \caption{Diatomic rotation.}
        \label{fig:partRot}
    \end{figure}
    \begin{itemize}
        \item To simplify the problem, replace the two particles rotating about the center of mass with one particle of reduced mass $\mu$ rotating about the center of mass with lever arm $r_0$.
    \end{itemize}
    \item Classically, the kinetic energy of the translational motion is
    \begin{equation*}
        T = \frac{L^2}{2I}
    \end{equation*}
    where $I=\mu r_0^2$ and $L=p\times r_0=pr_0$ (for this kind of rotation; see Figure \ref{fig:partRotb}).
    \item To further talk about this problem, we should introduce \textbf{spherical coordinates}.
    \item \textbf{Spherical coordinates}: The coordinate system $(r,\theta,\phi)$ related to the Cartesian coordinates $(x,y,z)$ by
    \begin{align*}
        x &= r\sin\theta\cos\phi&
        y &= r\sin\theta\sin\phi&
        z &= r\cos\theta
    \end{align*}
    \item Classically, we will have
    \begin{equation*}
        H = \frac{1}{2\mu}(p_x^2+p_y^2+p_z^2)+V(x,y,z)
    \end{equation*}
    in Cartesian coordinates.
    \item In spherical coordinates, this becomes
    \begin{equation*}
        H = \frac{1}{2\mu}\left( p_r^2+\frac{L^2}{r^2} \right)+V(r)
    \end{equation*}
    \item Thus, in quantum mechanics, we get
    \begin{align*}
        \hat{H} &= \frac{1}{2\mu}(\hat{p}_x^2+\hat{p}_y^2+\hat{p}_z^2)+V(x,y,z)\\
        &= \frac{1}{2\mu}\left( \hat{p}_r^2+\frac{\hat{L}^2}{r^2} \right)+V(r)
    \end{align*}
    \item Thus, we have in spherical coordinates that
    \begin{equation*}
        \hat{T}\psi = -\frac{\hbar^2}{2\mu}\left[ \frac{1}{r^2}\pdv{r}\left( r^2\pdv{r}\psi \right)+\frac{1}{r^2\sin\theta}\left( \pdv{\theta}\left( \sin\theta\pdv{\theta}\psi \right) \right)+\frac{1}{r^2\sin^2\theta}\pdv[2]{\phi}\psi \right]
    \end{equation*}
    \item 2D rigid rotor:
    \begin{itemize}
        \item Let the system from Figure \ref{fig:partRotb} be confined to rotating in two dimensions.
        \item This simplifies the problem since both the $\pdv*{r}$ and $\pdv*{\theta}$ terms in the kinetic energy operator disappear (since, respectively, the particle is at a fixed distance from the center of mass and it cannot move out of the 2D plane).
        \item Thus, our Schr\"{o}dinger equation for this system is
        \begin{equation*}
            -\frac{\hbar^2}{2\mu r_0^2}\pdv[2]{\phi}\psi(\phi) = E\psi(\phi)
        \end{equation*}
        \item Solution: Let $\psi(\phi)=\e[im\phi]$; then
        \begin{equation*}
            E_m = \frac{\hbar^2m^2}{2\mu r_0^2}
        \end{equation*}
        \begin{itemize}
            \item $m=0,1,2,\dots$ is a new quantum number.
            \item $m$ doesn't go to infinity because $|m|$ is bounded by $\ell$ (the total angular momentum).
        \end{itemize}
        \item Remembering our original restriction, we have that this math describes the system from Figure \ref{fig:partRota} but confined to rotate in the $xy$ plane with angular momentum in the $z$ direction.
        \begin{itemize}
            \item Thus, for example, the energies of the system from Figure \ref{fig:partRota} are dependent on $m$ and $I=\mu r_0^2$.
        \end{itemize}
        \item Such a system occurs in physical reality when we put the diatomic in an external field, or attach to it a big functional group.
        \item Zero point energy: $m=0$ does not violate the UR since we still have $\Delta L\Delta\theta\geq\hbar/2$ (as everything is still rotating in the sense that we have equal probability of the particle being everywhere [as opposed to more localized/normal rotation with higher values of $m$]).
    \end{itemize}
    \item 3D rigid rotor:
    \begin{itemize}
        \item Assume that the potential energy is zero on the surface of the sphere (so we basically have a particle on a sphere).
        \item Then
        \begin{equation*}
            \hat{H} = \frac{\hat{L}^2}{2\mu r_0^2} = \frac{\hat{L}^2}{2I}
        \end{equation*}
        \item Solving $\hat{H}\psi=E\psi$ asserts that the eigenfunctions of the Hamiltonian are the spherical harmonics $Y_{\ell m}(\theta,\phi)$.
        \item Energy:
        \begin{equation*}
            E_\ell = \frac{\hbar^2}{2I}\ell(\ell+1)
        \end{equation*}
        where $\ell=0,1,2,\dots$.
        \item Recall that $m$ corresponds to the projection of angular momentum onto the $z$-axis, so that
        \begin{equation*}
            m = -\ell,\dots,+\ell
        \end{equation*}
    \end{itemize}
\end{itemize}




\end{document}