\documentclass[../notes.tex]{subfiles}

\pagestyle{main}
\renewcommand{\chaptermark}[1]{\markboth{\chaptername\ \thechapter\ (#1)}{}}
\setcounter{chapter}{4}

\begin{document}




\chapter{Approximate Solutions of the Schr\"{o}dinger Equation}
\section{Approximation Methods}
\begin{itemize}
    \item \marginnote{10/25:}The \textbf{variational method} and \textbf{perturbation theory} are two methods of approximating solutions to Schr\"{o}dinger equations describing systems more complex than the hydrogen atom.
    \item To begin our investigation of the variational method, we will look at the particle in a box.
    \begin{itemize}
        \item Consider a Hamiltonian for an electron in a box of length $L=\SI{2}{\atomicunit}$ centered around $x=0$.
        \begin{itemize}
            \item Note that we take the electron as the fundamental mass, $\hbar$ as the fundamental unit of energy time, and the charge of the electron as the fundamental unit of charge, and the Bohr radius as the fundamental unit of length.
        \end{itemize}
        \item Our Hamiltonian is
        \begin{equation*}
            H\psi(x) = -\frac{\hbar^2}{2m}\dv[2]{x}\psi(x)
        \end{equation*}
        or, in atomic units,
        \begin{equation*}
            H\psi(x) = -\frac{1}{2}\dv[2]{x}\psi(x)
        \end{equation*}
    \end{itemize}
    \item \textbf{Variational theorem}: The expectation value of our Hamiltonian with respect to a trial wave function produces an approximate energy. Moreover\footnote{We will prove that the approximate energy is an upper bound on the ground state energy in the homework.},
    \begin{equation*}
        E_\text{approx} \geq E_\text{gr. st.}
    \end{equation*}
    \item \textbf{Variational method}: Take $\psi_\text{trial}=\sum_na_n\ket{\psi_n}$ where $\psi_n$ is a trial wave function and the $a_j$'s are parameters of the wave function which we want to optimize to lower $E_\text{trial}$.
    \begin{itemize}
        \item Dirac's ket describes an abstract state of the particle (possibly position, possibly its Fourier transform, momentum).
    \end{itemize}
    \item Back to the particle in a box:
    \begin{itemize}
        \item A possible trial wave function (that satisfies the boundary conditions) is
        \begin{equation*}
            \psi_\text{tr} = (1+x)(1-x) = 1-x^2
        \end{equation*}
        \item The energy of $\psi_\text{tr}$ may be evaluated as follows.
        \begin{align*}
            E &= \frac{\int\psi_\text{tr}^*(x)\hat{H}\psi_\text{tr}(x)\dd{x}}{\int\psi_\text{tr}^*(x)\psi_\text{tr}(x)\dd{x}}\\
            &= \frac{\int_{-1}^1(1-x^2)\left( -\frac{1}{2}\dv[2]{x} \right)(1-x^2)\dd{x}}{\int_{-1}^1(1-x^2)(1-x^2)\dd{x}}\\
            &= \frac{\int_{-1}^1(1-x^2)\dd{x}}{\int_{-1}^1(1-x^2)(1-x^2)\dd{x}}\\
            &= \frac{4/3}{16/15}\\
            &= \frac{5}{4}\\
            &= \SI{1.25}{\atomicunit}
        \end{align*}
        \item From the exact solution to the particle in a box
        \begin{equation*}
            E_1 = 1.23370055 < 1.25 = E_\text{trial}
        \end{equation*}
        so the variational theorem is satisfied.
        \item Next step: Trial wave function as a linear combination is $\psi_\text{tr}(x)=c_1\psi_1(x)+c_2\psi_2(x)$.
        \item Plugging this into the SE yields
        \begin{equation*}
            c_1(\hat{H}-E)\psi_1(x)+c_2(\hat{H}-E)\psi_2(x) = 0
        \end{equation*}
        \begin{itemize}
            \item $\psi_1,\psi_2$ span the (Hilbert) space of solutions.
        \end{itemize}
        \item To solve the above equation, multiply by $\psi_1^*(x)$ and integrate to obtain
        \begin{equation*}
            c_1\int_{-1}^1\psi_1^*(x)(\hat{H}-E)\psi_1(x)\dd{x}+c_2\int_{-1}^1\psi_1^*(x)(\hat{H}-E)\psi_2(x)\dd{x} = 0
        \end{equation*}
        and multiply by $\psi_2^*(x)$ an integrate to obtain
        \begin{equation*}
            c_1\int_{-1}^1\psi_2^*(x)(\hat{H}-E)\psi_1(x)\dd{x}+c_2\int_{-1}^1\psi_2^*(x)(\hat{H}-E)\psi_2(x)\dd{x} = 0
        \end{equation*}
        \item Substituting, we have
        \begin{align*}
            c_1(H_{11}-ES_{11})+c_2(H_{12}-ES_{12}) &= 0&
            c_1(H_{21}-ES_{21})+c_2(H_{22}-ES_{22}) &= 0
        \end{align*}
        \item In matrix form, the above two equations become
        \begin{align*}
            \begin{pmatrix}
                H_{11} & H_{12}\\
                H_{21} & H_{22}\\
            \end{pmatrix}
            \begin{pmatrix}
                c_1\\
                c_2\\
            \end{pmatrix}
            -E
            \begin{pmatrix}
                S_{11} & S_{12}\\
                S_{21} & S_{22}\\
            \end{pmatrix}
            \begin{pmatrix}
                c_1\\
                c_2\\
            \end{pmatrix}
            &=
            \begin{pmatrix}
                0\\
                0\\
            \end{pmatrix}\\
            \mathbb{H}\vec{c}-E\mathbb{S}\vec{c} &= 0
        \end{align*}
        \begin{itemize}
            \item We get a matrix that has the same dimension as the size of the expansion (in the first case, we had a $1\times 1$ matrix).
            \item $\mathbb{S}$ is the overlap matrix because the wave functions aren't normalized.
        \end{itemize}
    \end{itemize}
\end{itemize}



\section{Variational Method}
\begin{itemize}
    \item \marginnote{10/27:}Approximating the ground state energy with some trial wave function and applying
    \begin{equation*}
        E_\text{approx} = \frac{\int\psi_\text{tr}^*\hat{H}\psi_\text{tr}\dd{x}}{\int\psi_\text{tr}^*\psi_\text{tr}\dd{x}}
    \end{equation*}
    where
    \begin{equation*}
        \psi_\text{tr} = \sum_nc_n\psi_n(x)
    \end{equation*}
    \item Example 2:
    \item For our second term, we need another even function (since the ground state wavefunction is even). Thus, choose
    \begin{equation*}
        \psi_\text{tr}(x) = c_1(1-x^2)+c_2(1-x^2)x^2
    \end{equation*}
    \begin{itemize}
        \item Think about this in the context of power series --- we have $(1-x^2)$ times an even power series expansion $(c_1+c_2x^2)$.
    \end{itemize}
    \item To find $c_1,c_2$, we could plug into the approximation integral and minimize.
    \item Alternatively, we can use matrices. We essentially project the Schr\"{o}dinger equation onto the space of the two wave functions.
    \item Take $\hat{H}\psi=E\psi$ and expand it to $\hat{H}(c_1\psi_1+c_2\psi_2)=E(c_1\psi_1+c_2\psi_2)$. In matrix form, $\mathbb{H}\vec{c}=E\mathbb{S}\vec{c}$.
    \item We have an overlap matrix $\mathbb{S}$ because our wave functions aren't normalized. If the basis \emph{is} orthonormal, $\mathbb{S}$ collapses to the identity matrix.
    \begin{itemize}
        \item Each $s_{ij}$ equals
        \begin{equation*}
            s_{ij} = \int\psi_i^*\psi_j\dd{x}
        \end{equation*}
        \item If $\psi_1,\psi_2$ is orthonormal, then $s_{ij}=\delta_{ij}$.
    \end{itemize}
    \item The elements of the Hamiltonian matrix:
    \begin{align*}
        H_{11} &= \int\psi_1^*(x)\hat{H}\psi_1(x)\dd{x}&
            H_{12} &= \int\psi_1^*(x)\hat{H}\psi_2(x)\dd{x}\\
        &= \frac{4}{3}&
            &= \frac{8}{15}
    \end{align*}
    \begin{align*}
        H_{21} &= \frac{8}{15}&
        H_{22} &= \frac{44}{105}
    \end{align*}
    \begin{itemize}
        \item Notice that $\mathbb{H}$ is symmetric with $H_{12}=H_{21}$.
    \end{itemize}
    \item Elements of the overlap matrix:
    \begin{align*}
        S_{11} &= \frac{16}{15}&
        S_{12} &= \frac{32}{105}
    \end{align*}
    \begin{align*}
        S_{21} &= \frac{32}{105}&
        S_{22} &= \frac{16}{315}
    \end{align*}
    \begin{itemize}
        \item Notice that $\mathbb{S}$ is symmetric with $S_{12}=S_{21}$.
    \end{itemize}
    \item Note that there are multiple ways to solve $\mathbb{H}\vec{c}=E\mathbb{S}\vec{c}$; \textcite{bib:McQuarrieSimon} only teaches one. Thus, you can get computers to do the math and solve far bigger systems than you could by hand.
    \item Solving $\mathbb{H}\vec{c}=E\mathbb{S}\vec{c}$ with the textbook method:
    \begin{itemize}
        \item Rewrite as $(\mathbb{H}-E\mathbb{S})\vec{c}=0$. Find the null space of $\mathbb{H}-E\mathbb{S}$.
        \item Since the determinant is the product of the eigenvalues, $\det(\mathbb{H}-E\mathbb{S})=(E_1-E)(E_2-E)$.
        \item This determinant is equal to zero only when $E=E_1$ or $E=E_2$.
        \begin{itemize}
            \item The energy is becoming quantized because of the linear algebra!
        \end{itemize}
        \item Now taking $\det(\mathbb{H}-E\mathbb{S})$ gives a characteristic polynomial in $E$.
        \renewcommand{\arraystretch}{1.6}
        \begin{align*}
            0 &=
            \begin{vmatrix}
                \frac{4}{3}-\frac{16}{15}E & \frac{4}{15}-\frac{16}{105}E\\
                \frac{4}{15}-\frac{16}{105}E & \frac{44}{105}-\frac{16}{315}E\\
            \end{vmatrix}\\
            &= \frac{256}{525}-\frac{2048}{4725}E+\frac{1024}{33075}E^2
        \end{align*}
        \item Solving the quadratic gives us
        \begin{equation*}
            E = 7\pm\frac{\sqrt{133}}{2}
        \end{equation*}
        \item Thus,
        \begin{align*}
            E_1 &= \SI{1.233718705}{\atomicunit}&
            E_2 &= \SI{12.766}{\atomicunit}
        \end{align*}
        \begin{itemize}
            \item Notice that the $E_1$ we found is only \emph{marginally} greater than the real value of $E_1$. Our value is accurate to four decimal places!
        \end{itemize}
        \item Solving for $\vec{c}$ with $E_1$ gives us
        \begin{align*}
            \vec{c}_1 &= \num{-0.9764}&
            \vec{c}_2 &= \num{0.2156}
        \end{align*}
    \end{itemize}
\end{itemize}



\section{Perturbation Theory}
\begin{itemize}
    \item \marginnote{10/29:}Consider the Hamiltonian
    \begin{equation*}
        \hat{H} = \hat{H}_0+\lambda\hat{V}
    \end{equation*}
    where $\hat{H}_0$ is the reference hamiltonian, $\hat{V}$ is the perturbation, and $\lambda$ is the perturbation parameter.
    \item The energy may be expressed as a Taylor series expansion in $\lambda$:
    \begin{equation*}
        E(\lambda) = E(0)+\lambda\dv{E}{\lambda}\bigg|_0+\frac{\lambda^2}{2}\dv[2]{E}{x}\bigg|_0+\cdots
    \end{equation*}
    \begin{itemize}
        \item If $\lambda$ is sufficiently small, we can get good approximations without resorting to higher order derivatives.
    \end{itemize}
    \item It follows that our reference energy is
    \begin{equation*}
        E(0) = \int\psi_0^*\hat{H}_0\psi_0\dd{x}
    \end{equation*}
    \item We now have that
    \begin{equation*}
        E(\lambda) = \int\psi^*(\lambda)\hat{H}(\lambda)\psi(\lambda)\dd{x}
    \end{equation*}
    \item We also have from differentiating that
    \begingroup
    \allowdisplaybreaks
    \begin{align*}
        \dv{E}{\lambda} &= \int\dv{\psi}{\lambda}\hat{H}\psi(\lambda)\dd{x}+\int\psi^*(\lambda)\hat{H}^*\dv{\psi}{\lambda}\dd{x}+\int\psi^*\dv{\hat{H}}{\lambda}\psi(\lambda)\dd{x}\\
        &= E\int\dv{\psi^*}{\lambda}\psi(\lambda)\dd{x}+E\int\psi^*(\lambda)\dv{\psi}{\lambda}\dd{x}+\int\psi^*\dv{\hat{H}}{\lambda}\psi(\lambda)\dd{x}\\
        &= E\dv{\lambda}\left( \int\psi^*(\lambda)\psi(\lambda)\dd{x} \right)+\int\psi^*\hat{H}\psi\dd{x}\\
        &= \int\psi^*(\lambda)\dv{\hat{H}}{\lambda}\psi(\lambda)\dd{x}\\
        &= \int\psi^*(\lambda)\hat{V}\psi(\lambda)\dd{x}
    \end{align*}
    \endgroup
    \begin{itemize}
        \item Note that the commutativity of $\hat{H}$ follows from the fact that it's a Hermitian operator.
    \end{itemize}
    \item It follows that
    \begin{equation*}
        \dv{E}{\lambda}\bigg|_{\lambda=0} = \int\psi_0^*V\psi_0\dd{x}
    \end{equation*}
    \begin{itemize}
        \item Richard Feynman worked this out for his undergraduate thesis at MIT. This laid the foundation of quantum electrodynamics, for which he would eventually win the Nobel prize.
        \item This is known as the \textbf{Hellmann-Feynman theorem} (1939).
        \item Note that the second derivative of $E(\lambda)$ unfortunately depends on $\dv*{\psi}{\lambda}$.
    \end{itemize}
    \item Many electron molecules: The Helium atom.
    \begin{itemize}
        \item We have $\hat{H}\psi=E\psi$ where
        \begin{equation*}
            \hat{H} = -\frac{1}{2}{\hat{\nabla}_1}^2-\frac{1}{2}{\hat{\nabla}_2}^2-\frac{Z}{r_1}-\frac{Z}{r_2}+\frac{1}{r_{12}}
        \end{equation*}
        \begin{itemize}
            \item Note that the $\nabla$'s are Laplacians.
            \item This equation takes into account the kinetic and potential energy of two electrons, plus the electron-electron repulsion.
        \end{itemize}
        \item Solve using perturbation theory. Our reference Hamiltonian is
        \begin{equation*}
            \hat{H}_0 = \hat{H}_{\ce{He+}_1}+\hat{H}_{\ce{He+}_2} = \underbrace{-\frac{1}{2}\hat{\nabla}_1^2-\frac{Z}{r_1}}_{\ce{He+}_1}\underbrace{-\frac{1}{2}\hat{\nabla}_2^2-\frac{Z}{r_2}}_{\ce{He+}_2}
        \end{equation*}
        \begin{itemize}
            \item I.e., it's the sum of the Hamiltonians of two helium ions (one-electron systems like hydrogen).
        \end{itemize}
        \item Since $\hat{V}=+1/r_{12}$, we have that
        \begin{equation*}
            \hat{H}(1) = \hat{H}_0+\hat{V}
        \end{equation*}
        is the Hamiltonian of the atom.
        \item Now we look for the solution to $\hat{H}_0\psi_0+E_0\psi_0$.
        \item We know that
        \begin{equation*}
            \psi_0 = \psi_{1s}(r_1\theta_1\phi_1)\psi_{1s}(r_2\theta_2\psi_2)
        \end{equation*}
        \begin{itemize}
            \item The fact that only two electrons fit in an orbital emerges naturally from the quantum mechanics!
        \end{itemize}
        \item We also know that
        \begin{equation*}
            E_0 = -\frac{Z^2}{2n^2}-\frac{Z^2}{2n^2} = -\SI{4}{\atomicunit}
        \end{equation*}
        \item Thus, by perturbation theory,
        \begin{align*}
            \dv{E}{\lambda}\bigg|_{\lambda=0} &= \int\psi_0^*\hat{V}\psi_0\dd{\vec{r}_1}\dd{\vec{r}_2}\\
            &= \int\text{1s}^*(1)\text{1s}^*(2)\hat{V}\text{1s}(1)\text{1s}(2)\dd{1}\dd{2}\\
            &= \frac{5}{8}Z
        \end{align*}
    \end{itemize}
\end{itemize}



\section{Chapter 7: Approximation Methods}
\emph{From \textcite{bib:McQuarrieSimon}.}
\begin{itemize}
    \item \marginnote{11/2:}Although the Schr\"{o}dinger equation cannot be solved exactly for any system more complicated than the hydrogen atom, we can use either the \textbf{variational method} or \textbf{perturbation theory} to approximate solutions to Schr\"{o}dinger equations to almost any desired accuracy.
    \item We begin with the variational method.
    \item If the ground-state wave function and corresponding energy of some arbitrary system are $\psi_0$ and $E_0$, respectively, then
    \begin{align*}
        \hat{H}\psi_0 &= E_0\psi_0\\
        \psi_0^*\hat{H}\psi_0 &= \psi_0^*E_0\psi_0\\
        \int\psi_0^*\hat{H}\psi_0\dd{\tau} &= E_0\int\psi_0^*\psi_0\dd{\tau}\\
        E_0 &= \frac{\int\psi_0^*\hat{H}\psi_0\dd{\tau}}{\int\psi_0^*\psi_0\dd{\tau}}
    \end{align*}
    where $\dd{\tau}$ represents the appropriate volume element for integrating over all space.
    \item \textbf{Variational principle}: If $\phi$ is any function, then
    \begin{equation*}
        \frac{\int\phi^*\hat{H}\phi\dd{\tau}}{\int\phi^*\phi\dd{\tau}} = E_\phi
        \geq E_0
        = \frac{\int\psi_0^*\hat{H}\psi_0\dd{\tau}}{\int\psi_0^*\psi_0\dd{\tau}}
    \end{equation*}
    and $E_\phi=E_0$ iff $\phi=\psi_0$\footnote{See Problem \ref{prb:7-1} for a proof of the variational principle.}.
    \item Essentially, the variational method consists of choosing a trial function $\phi$ that depends on \textbf{variational parameters} (numbers) so that $E_\phi$ depends on the variational parameters, too, and minimizing $E_\phi$ over the parameters.
    \item \textcite{bib:McQuarrieSimon} does a worked example with the ground state of the hydrogen atom, comparing answers with the exact solution.
    \begin{itemize}
        \item Uses the formula for $E_\phi$ to directly express it in terms of the variational parameter and then differentiates to find the maxima.
        \item Applied to a case that can be solved exactly, the variational method with a trial wave function of the appropriate form will give the exact solution.
    \end{itemize}
    \item Estimating the ground state energy of a helium atom, and deriving effective nuclear charge.
    \begin{itemize}
        \item Consider the Hamiltonian for helium from the end of Chapter 6.
        \item It can be written in the form
        \begin{equation*}
            \hat{H} = \hat{H}_{\ce{H}}(1)+\hat{H}_{\ce{H}}(2)+\frac{e^2}{4\pi\epsilon_0}\frac{1}{r_{12}}
        \end{equation*}
        where $\hat{H}_{\ce{H}}(j)$, $j=1,2$, is the Hamiltonian operator for a single electron around a helium nucleus.
        \item Ignoring the interelectron repulsion term gives us a separable wave function of the form
        \begin{equation*}
            \phi_0(\mathbf{r}_1,\mathbf{r}_2) = \psi_{1s}(\mathbf{r}_1)\psi_{1s}(\mathbf{r}_2)
        \end{equation*}
        \item We use the above as our trial wave function with $Z$ (see Table \ref{tab:hydrogenRealWavefunctions}) as our variational parameter.
        \item Minimizing gives us $Z_\text{min}=27/16$ and a fairly good $E_\text{min}$ value.
        \item Note that $Z_\text{min}$ can naturally be interpreted as the \textbf{effective nuclear charge}.
        \item Since $Z_\text{min}<2$, we have that each electron partially \textbf{screens} the nucleus from the other.
    \end{itemize}
    \item \textcite{bib:McQuarrieSimon} arrives at the two linear algebraic equations Mazziotti did by expanding $\int\phi\hat{H}\phi\dd{\tau}$ with $\phi=c_1f_1+c_2f_2$, applying the condition that $H$ is Hermitian from Chapter 4 to show that $H_{ij}=H_{ji}$ and $S_{ij}=S_{ji}$, and differentiating the resulting energy function with respect to $c_1$ and then with respect to $c_2$.
    \item \textbf{Matrix elements}: The quantities $H_{ij}$ and $S_{ij}$.
    \item \textbf{Secular determinant}: A determinant of the form
    \begin{equation*}
        \begin{vmatrix}
            H_{11}-ES_{11} & \cdots & H_{1n}-ES_{1n}\\
            \vdots &  & \vdots\\
            H_{n1}-ES_{n1} & \cdots & H_{nn}-ES_{nn}\\
        \end{vmatrix}
    \end{equation*}
    \begin{itemize}
        \item There exists a solution to the linear algebraic equations iff the secular determinant vanishes (equals 0).
    \end{itemize}
    \item \textbf{Secular equation}: The equation obtained by expanding the secular determinant.
    \begin{itemize}
        \item The smaller root of a second-degree secular equation is an upper bound on the ground state energy.
        \item The larger root of a second-degree secular equation is an upper bound on the energy of the first excited state (albeit usually a crude one; we do not investigate methods of approximating energies other than that of the ground state here).
    \end{itemize}
    \item Normalizing the trial wave function.
    \begin{itemize}
        \item Since the linear system is homogenous, all of the $c_1$ terms (for example) can be expressed as a linear combination of all of the other terms.
        \item From here, we can find the ratios $c_2/c_1,c_3/c_1,\dots,c_N/c_1$.
        \item Thus, we simply integrate the modulus squared of our trial wave function with $c_1$ free and set the result equal to 1. Solving for $c_1$ normalizes the equation.
    \end{itemize}
    \item Note that trial wave functions can be linear combinations of functions that also contain variational parameters, but minimization here must be done numerically as the system is no longer linear.
    \item We now discuss perturbation theory.
    \item Suppose we cannot solve the Schr\"{o}dinger equation $\hat{H}\psi=E\psi$ but we can solve $\hat{H}^{(0)}\psi^{(0)}=E^{(0)}\psi^{(0)}$, which corresponds to a system that is in some sense similar to the unsolvable system. Then let $\hat{H}=\hat{H}^{(0)}+\hat{H}^{(1)}$.
    \item \textbf{Unperturbed Hamiltonian operator}: The Hamiltonian operator corresponding to the system we can solve exactly. \emph{Denoted by} $\bm{\hat{H}^{(0)}}$.
    \item \textbf{Perturbation}: The difference between the Hamiltonian operator we cannot solve and the unperturbed Hamiltonian operator. \emph{Denoted by} $\bm{\hat{H}^{(1)}}$.
    \item In perturbation theory, we let
    \begin{align*}
        \psi &= \psi^{(0)}+\psi^{(1)}+\psi^{(2)}+\cdots&
        E &= E^{(0)}+E^{(1)}+E^{(2)}+\cdots
    \end{align*}
    where $\psi^{(1)},\psi^{(2)},\dots$ and $E^{(1)},E^{(2)},\dots$ are successive corrections to $\psi^{(0)}$ and $E^{(0)}$, respectively.
    \item We can show (see Problem \ref{prb:7-19}) that
    \begin{equation*}
        E^{(1)} = \int\psi^{(0)*}\hat{H}^{(1)}\psi^{(0)}\dd{\tau}
    \end{equation*}
    \begin{itemize}
        \item Note that we can (although we will not here) derive explicit expressions for higher order corrections. These do get messy though.
    \end{itemize}
    \item Uses perturbation theory to derive the energy of a particle in a one-dimensional slanted box, and separately the energy of a Helium electron.
\end{itemize}




\end{document}