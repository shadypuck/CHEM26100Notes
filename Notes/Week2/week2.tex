\documentclass[../notes.tex]{subfiles}

\pagestyle{main}
\renewcommand{\chaptermark}[1]{\markboth{\chaptername\ \thechapter\ (#1)}{}}
\setcounter{chapter}{1}

\begin{document}




\chapter{The Schr\"{o}dinger Equation}
\section{Particle-Wave Duality and Uncertainty Relations}
\begin{itemize}
    \item \marginnote{10/4:}Particle-wave duality (de Brogelie's original formulation):
    \begin{align*}
        \lambda\nu &= c&
        E &= h\nu&
        p &= \frac{h}{\lambda}
    \end{align*}
    \item \textbf{Angular frequency}: The quantity $\omega=2\pi r$.
    \item \textbf{Wavenumber}: The quantity $k=2\pi/\lambda$.
    \item We can create a symmetrical formulation of the de Broglie relation using these new quantities:
    \begin{align*}
        E &= \hbar\omega&
        p &= \hbar k
    \end{align*}
    \item What is the wave that we might associate with a de Broglie particle?
    \begin{equation*}
        \Psi(x) = A\e[ikx-i\omega t]
    \end{equation*}
    \item Probability:
    \begin{itemize}
        \item Classically, such a wave might be associated with EM radiation hitting a surface with intensity $I=|\Psi(x)|^2=\Psi(x)\Psi^*(x)$.
        \item As soon as we associate a particle (photon) with the wave, the intensity may be re-interpreted as the number of particles reaching the surface or the probability of a particle being at the surface.
        \item Thus, the probability of finding a particle at the surface becomes $|\Psi(x)|^2$, as well.
    \end{itemize}
    \item Following de Broglie, we also associate waves with particles such as electrons.
    \begin{itemize}
        \item With the association of light as a particle, the particle wave duality leads to the appearance of probability.
    \end{itemize}
    \item What is the probability of finding the particle at the origin?
    \begin{align*}
        Pr &= \left| A\e[ik\cdot 0] \right|^2\\
        &= |A|^2\\
    \end{align*}
    \begin{itemize}
        \item Since the probability is not dependent on position, it is the same everywhere.
        \item We also run into issues \textbf{normalizing} this unbounded wavefunction.
        \item We know this particle's momentum exactly, but we know nothing about its position.
    \end{itemize}
    \item \textbf{Normalizing} (a wavefunction): Guaranteeing that the integral for the entire wavefunction is equal to 1.
    \item \textbf{Free particle}: A particle that does not have constraints on where it is more likely to be.
    \item Heisenberg's uncertainty relations are formalized in terms of matrix mechanics.
    \begin{itemize}
        \item We can Fourier transform the wave function of particle to convert it from a function of position to a function of momentum.
        \item The Fourier transform will yield one spike at $\hbar k$ and will be 0 everywhere else --- just like the Dirac delta function.
        \item Thus,
        \begin{equation*}
            \Psi(p) = \delta(p-\hbar k)
        \end{equation*}
    \end{itemize}
    \item Consider a Gaussian wave packet at $p=0$. Then
    \begin{equation*}
        \phi(p) = C\e[-\frac{p^2}{2(\Delta p)^2}]
    \end{equation*}
    \begin{itemize}
        \item $\Delta p$ is the standard deviation of the Gaussian/width of the distribution. It is a constant such that the probability drops to $1/\e$ of its maximum at $p=0$.
    \end{itemize}
    \item With the Fourier Transform of $\Psi(p)$, we obtain
    \begin{equation*}
        \Psi(x) = D\e[-\frac{(\Delta p)^2x^2}{2\hbar^2}]
    \end{equation*}
    \begin{itemize}
        \item Thus, a Gaussian quantum function produces a Gaussian position function via an FT as well, i.e.,
        \begin{equation*}
            \Psi(x) = D\e[-\frac{x^2}{2(\Delta x)^2}]
        \end{equation*}
    \end{itemize}
    \item Now if we set the last two equations equal to each other, we get
    \begin{align*}
        \frac{(\Delta p)^2}{2\hbar^2} &= \frac{1}{2(\Delta x)^2}\\
        (\Delta p)^2(\Delta x)^2 &= \hbar^2\\
        \Delta p\Delta x &= \hbar = \frac{h}{2\pi}
    \end{align*}
    \begin{itemize}
        \item This implies that the spread of the Gaussian in momentum times the spread of the Gaussian in position is a constant.
        \item If we make one Gaussian wave packet more specific, the other gets more spread out, and vice versa.
        \item Note that the above equality does \emph{satisfy} the Heisenberg uncertainty principle, but it is not it itself.
    \end{itemize}
\end{itemize}



\section{The Schr\"{o}dinger Equation and Particle in a Box}
\begin{itemize}
    \item \marginnote{10/6:}Review:
    \begin{itemize}
        \item de Broglie describes an electron as a free particle.
        \begin{equation*}
            \Psi(x) = A\e[ikx]
        \end{equation*}
        \item We can only observe the real part, but being able to access the complex part is important in quantum mechanics.
    \end{itemize}
    \item Schr\"{o}dinger was on vacation in the Swiss Alps with his mistress when he derived the wave equation.
    \begin{itemize}
        \item Schr\"{o}dinger realized that
        \begin{align*}
            \dv{\Psi(x)}{x} &= Aik\e[ikx]\\
            -i\hbar\dv{\Psi(x)}{x} &= Ap\e[ikx]\\
            &= p\Psi(x)
        \end{align*}
        \item Let's introduce operators in quantum mechanics and let $\hat{p}$ be an operator that when it acts on $\Psi(x)$, it gives the above. In other words,
        \begin{equation*}
            \hat{p} = -i\hbar\dv{x}
        \end{equation*}
        \item Thus,
        \begin{equation*}
            \hat{p}\Psi(x) = p\Psi(x)
        \end{equation*}
        \item But energy is more important than momentum, so let's introduce an energy operator $\hat{T}$ related to $\hat{p}$ by
        \begin{equation*}
            \hat{T} = \frac{\hat{p}^2}{2m}
            = \frac{-\hbar^2}{2m}\dv[2]{x}
        \end{equation*}
        since $E=mv^2/2=p^2/(2m)$.
        \item Thus, we have
        \begin{equation*}
            \hat{T}\Psi(x) = \frac{p^2}{2m}\Psi(x)
        \end{equation*}
        \item It follows from classical physics that the total energy operator $\hat{H}$ (the Hamiltonian) is the sum of the kinetic and potential energy operators, i.e., $\hat{H}=\hat{T}+\hat{V}$. Therefore, we must have
        \begin{equation*}
            \hat{H}\Psi(x) = E\Psi(x)
        \end{equation*}
        and that is the Schr\"{o}dinger equation.
    \end{itemize}
    \item The particle in a box is like a single electron in a one-dimensional chamber that runs from $-a$ to $a$ with $L=2a$ (Schr\"{o}dinger figured this out a few days later, still in the Swiss Alps).
    \begin{itemize}
        \item The walls are infinite and have infinite potential.
        \item We need the boundary condition, though, to be able to solve a differential equation like the Schr\"{o}dinger equation.
        \begin{itemize}
            \item Fortunately, we know that at $|x|=a$, we have $\Psi(\pm a)=0$.
            \item Another important point is that $\dv*{\Psi(x)}{x}$ at $a$ is discontinuous.
        \end{itemize}
        \item So we have that
        \begin{align*}
            -\frac{\hbar^2}{2m}\dv[2]{x}\Psi_n(x) &= E_n\Psi_n(x)\\
            \dv[2]{x}\Psi(x) &= -k^2\Psi(x)
        \end{align*}
        where
        \begin{equation*}
            k = \sqrt{\frac{2mE}{\hbar^2}}
        \end{equation*}
        \item The solution of the differential equation will be of the form
        \begin{equation*}
            \Psi(x) = A\cos(kx)+B\sin(kx)
        \end{equation*}
        \item Boundary conditions 1 and 2, respectively:
        \begin{align*}
            0 &= \Psi(a)&
                0 &= \Psi(-a)\\
            &= A\cos(ka)+B\sin(ka)&
                &= A\cos(ka)-B\sin(ka)
        \end{align*}
        \item Adding/subtracting the two equations yields
        \begin{align*}
            A\cos(ka) &= 0&
            B\sin(ka) &= 0
        \end{align*}
        \item We satisfy these equations with either of 2 classes of nontrivial solutions (the trivial solution being $a=0$).
        \begin{enumerate}
            \item $B=0$ and $\cos(ka)=0$, i.e., $k_n=\frac{n\pi}{2a}$ for $n\in 2\N+1$.
            \item $A=0$ and $\sin(ka)=0$, i.e., $k_n=\frac{n\pi}{2a}$ for $n\in 2\N$.
        \end{enumerate}
        \item Thus, either
        \begin{equation*}
            \Psi_n(x) = \frac{1}{\sqrt{a}}\cos\left( \frac{n\pi x}{2a} \right)
        \end{equation*}
        for $n\in 2\N+1$ are the \textbf{even solutions} (because cosine is an even function), and
        \begin{equation*}
            \Psi_n(x) = \frac{1}{\sqrt{a}}\sin\left( \frac{n\pi x}{2a} \right)
        \end{equation*}
        for $n\in 2\N$ are the \textbf{odd solutions} (because sine is an odd function).
        \item Note that we derive the $1/\sqrt{a}$ coefficient by normalizing $\Psi(x)$ with
        \begin{equation*}
            \int_{-a}^a|\Psi(x)|^2\dd{x} = \int_{-a}^a\Psi^*(x)\Psi(x) = 1
        \end{equation*}
        \item The energies come out to
        \begin{equation*}
            E_n = \frac{\hbar^2k_n^2}{2m} = \frac{\hbar^2}{8m}\cdot\frac{\pi^2n^2}{a^2}
        \end{equation*}
        with the substitution $k_n=\frac{n\pi}{2a}$.
        \begin{itemize}
            \item Note that this means that the particle becomes more discrete the smaller the box gets (as uncertainty in position goes down, it acts more and more quantum mechanically).
        \end{itemize}
    \end{itemize}
\end{itemize}



\section{Potential Step}
\begin{itemize}
    \item \marginnote{10/8:}Particle in a box:
    \begin{itemize}
        \item For $n=1$, the potential is defined by one hump of a sine wave.
        \item For $n=2$, the potential is defined by two humps.
        \item The number of nodes is equal to the principal quantum number minus 1.
        \item We have
        \begin{equation*}
            E_n = \frac{\hbar^2}{8m}\frac{\pi^2n^2}{a^2}
        \end{equation*}
        for $n\in\N$.
        \item By the Heisenberg uncertainty relationship, we must have $E_1>0$. In other words, the \textbf{zero-point energy} arises from the UR.
        \item Trend wrt. $a$: As $a\to\infty$, all of the energies become degenerate.
        \item Trend wrt. $m$: As $m\to\infty$, $E_n\to 0$ as well.
        \begin{itemize}
            \item In other words, as $m\to\infty$, the particle behaves more classically!
            \item The zero-point energy also disappears as $a\to\infty$.
        \end{itemize}
    \end{itemize}
    \item \textbf{Zero-point energy}: The lowest possible energy a quantum mechanical system may have.
    \item All of that information comes from the Schr\"{o}dinger equation, so we now know much more than we used to.
    \item Free particle vs. particle in a box:
    \begin{itemize}
        \item For a free particle, we have $\Psi(x)=\e[ikx]$. Boundary condition was a circle (as per the Bohr model).
        \item In the particle in a box, we weed out all of the free particle solutions that don't match the boundary conditions. And the only solutions that match the boundary conditions are the ones that have integers for the quantum number $n$.
        \item The only constraint is that you can retain more particles the bigger the box gets; this is why the particle gets more quantum mechanical as you shrink the box.
    \end{itemize}
    \item \textbf{Potential step}: Let the energy $E$ be 0 up until the origin, where it steps up to potential $V_0$.
    \begin{figure}[h!]
        \centering
        \begin{tikzpicture}[
            every node/.append style={black}
        ]
            \footnotesize
            \draw (-3,-0.5) -- node[pos=0.25,left]{$0$} node[left]{\small$E$} ++(0,2);
            \draw [grx,thick] (-2.8,0) -- node[above=1.1cm]{\small I} (0,0) node[below]{$0$} node[below=5mm]{\small$x$} -- node[right]{$V_0$} ++(0,1) -- node[above=1mm]{\small II} ++(2.8,0);
    
            \node [circle,fill,orx,inner sep=1.5pt,label={below:$E_0$},label={[xshift=1mm]above left:$\e[-]$}] at (-2,0.7) {}
                edge [->,dashed] ++(-0.5,0.7)
            ;
        \end{tikzpicture}
        \caption{Potential step.}
        \label{fig:potentialStep}
    \end{figure}
    \begin{itemize}
        \item We shoot a particle at a potential wall with energies varying from below the top to above the top.
    \end{itemize}
    \item In classical mechanics, we have
    \begin{equation*}
        E = \frac{p^2}{2m}+V
    \end{equation*}
    \begin{itemize}
        \item In region I, there's no potential, so the total energy is all kinetic. The particle is moving with momentum $p_\text{I}=\sqrt{2mE}$.
        \item In region II, the particle is moving with momentum $p_\text{II}=\sqrt{2m(E-V)}$.
        \begin{itemize}
            \item If $E_0<V$, the particle \emph{never} passes from region $\text{I}\to\text{II}$.
            \item If $E_0>V$, the particle \emph{always} passes from $\text{I}\to\text{II}$, but has less KE in II than I\footnote{Note that the classical resolution to the case $E=V_0$ is that the particle never has $E_0=V$; it always has energy $\epsilon$ above or $\epsilon$ below $V$. However, in some sense, there \emph{is} another answer: Classical mechanics is not an "accurate" reflection of reality, and this is a place where it shows. Indeed, we \emph{need} quantum mechanics to treat this case.}.
        \end{itemize}
    \end{itemize}
    \item Quantum particle motion:
    \begin{align*}
        -\frac{\hbar^2}{2m}\dv[2]{x}\Psi(x)+V(x)\Psi(x) &= E\Psi(x)\\
        \dv[2]{x}\Psi(x)+k^2\Psi(x) &= 0
    \end{align*}
    where $k=\sqrt{2m(E-V)/\hbar^2}$.
    \begin{itemize}
        \item The total wave function will be the sum of the LCAOs that fit the boundary condition.
        \item Our general solution has two parts:
        \begin{align*}
            \Psi_\text{I}(x)  &= A\e[i\alpha x]+B\e[-i\beta x]&
            \Psi_\text{II}(x) &= C\e[i\alpha x]+D\e[-i\beta x]
        \end{align*}
        \item Two energy cases: $E>V_0$ and $E<V_0$.
        \item $E>V_0$:
        \begin{itemize}
            \item We must maintain the continuity of the $\Psi(x)$ and $\dv*{\Psi(x)}{x}$ at $x=0$. This yields
            \begin{align*}
                A+B &= C+D&
                i\alpha(A-B) &= i\beta(C-D)
            \end{align*}
            \item It follows that
            \begin{align*}
                A &= \frac{C(\alpha+\beta)}{2\alpha}+\frac{D(\alpha-\beta)}{2\alpha}&
                B &= \frac{C(\alpha-\beta)}{2\alpha}+\frac{D(\alpha+\beta)}{2\alpha}
            \end{align*}
            \item Assume that the particles only travel from left to right in II, i.e., $D=0$.
            \item The flux of the particle: The probability of the particle going left to right in region I is $|A|^2$. Thus, since the incident flux factors in the speed $v_I$ of the particle, the incident flux is $v_\text{I}|A|^2$. Similarly, the transmitted flux of the particle is $v_\text{II}|C|^2$.
            \item Consequently, the reflected flux of the particles is
            \begin{equation*}
                R = \frac{c|B|^2}{c|A|^2} = \frac{|B|^2}{|A|^2} = \frac{(\alpha-\beta)^2}{(\alpha+\beta)^2}
            \end{equation*}
            Note that the speed of the particle (the speed of light, $c$) is the same in both regions.
            \item Conclusion: There is a probability of reflection \emph{even when} $E_0>V_0$, disagreeing with classical mechanics.
            \item Fraction of transmitted particles:
            \begin{equation*}
                T = \frac{v_\text{II}}{v_\text{I}}\frac{|C|^2}{|A|^2} = \frac{4\alpha\beta}{(\alpha+\beta)^2}
            \end{equation*}
        \end{itemize}
        \item $E<V_0$:
        \begin{itemize}
            \item The continuity of $\Psi(x)$ and $\Psi'(x)$ at $x=0$ again gives us
            \begin{align*}
                A+B &= C+D&
                i\alpha(A-B) &= i\beta(C-D)
            \end{align*}
            \item But since we have
            \begin{equation*}
                \beta = \frac{\sqrt{2m(E-V_0)}}{\hbar}
            \end{equation*}
            and $E-V_0<0$, $\beta$ will be a complex number.
            \item Thus, to treat the real and complex portions of $\beta$ separately, we define $\beta_2$ to be a real number.
            \item Consequently, we may write
            \begin{equation*}
                R = \frac{|B|^2}{|A|^2} = \frac{|\alpha-\beta|^2}{|\alpha+\beta|^2} = \frac{|\alpha-i\beta_2|^2}{|\alpha+i\beta_2|^2} = \frac{\alpha^2+\beta_2^2}{\alpha^2+\beta_2^2} = 1
            \end{equation*}
            \item Conclusion: When the energy of the particle is less than the energy of the potential, even quantum mechanics predicts total reflection. However, there's still something subtle happening.
            \item Let's look at the wave function in region II:
            \begin{equation*}
                \Psi_\text{II} = C\e[i\beta x] = C\e[i(i\beta_2)x] = C\e[-\beta_2x]
            \end{equation*}
            where $\beta_2>0$ by definition.
            \begin{figure}[H]
                \centering
                \begin{tikzpicture}[
                    every node/.append style={black}
                ]
                    \footnotesize
                    \draw (-3,-0.5) -- node[pos=0.25,left]{$0$} node[left]{\small$E$} ++(0,2);
                    \draw [grx,thick] (-2.8,0) -- (0,0) node[below]{$0$} node[below=5mm]{\small$x$} -- ++(0,1) -- ++(2.8,0);
            
                    \node [circle,fill,orx,inner sep=1.5pt,label={below:$E_0$},label={[xshift=1mm]above left:$\e[-]$}] at (-2,0.7) {}
                        edge [->,dashed] ++(-0.5,0.7)
                    ;
            
                    \draw [blx,semithick,densely dashed] plot[domain=0:2.2,smooth] (\x,{0.7*e^(-3*\x)});
                    \node (a) at (1.4,0.6) {$\e[-\beta_2x]$}
                        (a.south west) edge [->] (0.7,0.2)
                    ;
                \end{tikzpicture}
                \caption{Quantum tunneling.}
                \label{fig:quantumTunneling}
            \end{figure}
            \item Thus, even though the particle is reflected 100\%, it has some probability of going through the step, namely a probability that decays exponentially the farther you go into the wall.
            \item This is \textbf{quantum tunneling}.
            \item The particle can't ever get to $\infty$, so that's why $T=0$, but it can go into the wall for a little bit, just a sec.
        \end{itemize}
    \end{itemize}
\end{itemize}



\section{MathChapter B: Probability and Statistics}
\begin{itemize}
    \item \marginnote{10/10:}"Consider some experiment, such as the tossing of a coin or the rolling of a die, that has $n$ possible outcomes, each with probability $p_j$, where $j=1,2,\dots,n$" \parencite[63]{bib:McQuarrieSimon}.
    \item If the experiment is repeated indefinitely, we intuitively expect that for each $j=1,\dots,n$
    \begin{equation*}
        p_j = \lim_{N\to\infty}\frac{N_j}{N}
    \end{equation*}
    where $N_j$ is the number of times that the event $j$ occurs and $N$ is the total number of repetitions of the experiment.
    \item The fact that $0\leq N_j\leq N$ implies that $0\leq p_j\leq 1$ by the above condition.
    \item \textbf{Certainty}: An event $j$ such that $p_j=1$.
    \item \textbf{Impossibility}: An event $j$ such that $p_j=0$.
    \item \textbf{Normalization condition}: The result that
    \begin{equation*}
        \sum_{j=1}^np_j = 1
    \end{equation*}
    \begin{itemize}
        \item This follows from the fact that $\sum_{j=1}^nN_j=N$ and the above.
        \item The normalization condition expresses the idea that "the probability that some event occurs is a certainty" \parencite[64]{bib:McQuarrieSimon}.
    \end{itemize}
    \item \textbf{Average} (of $x$): The following quantity, where we associate some number $x_j$ with each outcome $j$. \emph{Also known as} \textbf{mean} (of $x$). \emph{Denoted by} $\bm{\prb{x}}$. \emph{Given by}
    \begin{equation*}
        \prb{x} = \sum_{j=1}^nx_jp_j = \sum_{j=1}^nx_jp(x_j)
    \end{equation*}
    \begin{figure}[h!]
        \centering
        \begin{tikzpicture}[
            every node/.append style={black}
        ]
            \footnotesize
            \draw (-3,0) -- (3,0);
            \draw [-stealth] (0,0) -- (0,2.5) node[above]{$p(x)$};
    
            \draw [rex,thick] (-2.5,0) node[below]{$x_1$} -- ++(0,1.5);
            \draw [rex,thick] (-1.6,0) node[below]{$x_2$} -- ++(0,1.2);
            \draw [rex,thick] (-1,0)   node[below]{$x_3$} -- ++(0,1.7);
            \draw [rex,thick] (1.5,0)  node[below]{$x_4$} -- ++(0,1.6);
        \end{tikzpicture}
        \caption{The discrete probability frequency function.}
        \label{fig:discreteProbabilityDensity}
    \end{figure}
    \begin{itemize}
        \item It is helpful to interpret a probability distribution like $p_j$ as a distribution of a unit mass along the $x$-axis in a discrete manner such that $p_j$ is the fraction of mass located at the point $x_j$.
        \item According to this interpretation, the average value of $x$ is the center of mass of this system.
    \end{itemize}
    \pagebreak
    \item \textbf{Second moment} (of the distribution $\{p_j\}$): The following quantity.
    \begin{equation*}
        \prb{x^2} = \sum_{j=1}^nx_j^2p_j
    \end{equation*}
    \begin{itemize}
        \item Note that $\prb{x^2}\neq\prb{x}^2$.
        \item Analogous to the moment of inertia.
    \end{itemize}
    \item The next quantity is physically more interesting than the second moment.
    \item \textbf{Second central moment}: The following quantity. \emph{Also known as} \textbf{variance}. \emph{Denoted by} $\bm{\sigma_x^2}$. \emph{Given by}
    \begin{equation*}
        \sigma_x^2 = \prb{(x-\prb{x})^2} = \sum_{j=1}^n(x_j-\prb{x})^2p_j
    \end{equation*}
    \begin{itemize}
        \item $\sigma_x^2\geq 0$ because it is a sum of positive terms.
        \item An alternate form of $\sigma_x^2$:
        \begin{align*}
            \sigma_x^2 &= \sum_{j=1}^n(x_j-\prb{x})^2p_j\\
            &= \sum_{j=1}^n(x_j^2-2\prb{x}x_j+\prb{x}^2)-p_j\\
            &= \sum_{j=1}^nx_j^2p_j-2\prb{x}\sum_{j=1}^nx_jp_j+\prb{x}^2\sum_{j=1}^np_j\\
            &= \prb{x^2}-2\prb{x}\cdot\prb{x}+\prb{x}^2\cdot 1\\
            &= \prb{x^2}-\prb{x}^2
        \end{align*}
        \item If $\sigma_x^2=0$ or $\prb{x}^2=\prb{x^2}$, then we must have $x_j=\prb{x}$ for all $j$, i.e., the event is not really probabilistic because the event $j$ occurs on every trial.
    \end{itemize}
    \item \textbf{Standard deviation}: The positive square root of the variance. \emph{Denoted by} $\bm{\sigma_x}$.
    \item Both the standard deviation and the variance are measures of the spread of the distribution about its mean.
    \item We now step into continuous probability distributions.
    \item \textbf{Linear mass density}: The quantity $\rho(x)$ defined by
    \begin{equation*}
        \dd{m} = \rho(x)\dd{x}
    \end{equation*}
    where $\dd{m}$ is the fraction of the mass lying between $x$ and $x+\dd{x}$.
    \item It follows that the probability that, for example, a particle lies between positions $x$ and $x+\dd{x}$ in a box is
    \begin{equation*}
        \Prob(x,x+\dd{x}) = p(x)\dd{x}
    \end{equation*}
    \item Therefore,
    \begin{equation*}
        \Prob(a\leq x\leq b) = \int_a^bp(x)\dd{x}
    \end{equation*}
    \item Furthermore, the continuous normalization condition becomes
    \begin{equation*}
        \int_{-\infty}^\infty p(x)\dd{x} = 1
    \end{equation*}
    \item We may also analogously define
    \begin{align*}
        \prb{x} &= \int_{-\infty}^\infty xp(x)\dd{x}&
        \prb{x^2} &= \int_{-\infty}^\infty x^2p(x)\dd{x}&
        \sigma_x^2 &= \int_{-\infty}^\infty(x-\prb{x})^2p(x)\dd{x}
    \end{align*}
    \item \textbf{Gaussian distribution}: The most commonly occuring and the most important continuous probability distribution. \emph{Given by}
    \begin{equation*}
        p(x)\dd{x} = c\e[-x^2/2a^2]\dd{x}
    \end{equation*}
    \begin{itemize}
        \item Note that the normalization condition implies that
        \begin{equation*}
            c = \frac{1}{\sqrt{2\pi a^2}}
        \end{equation*}
        \item We can also prove that
        \begin{equation*}
            \sigma_x = a
        \end{equation*}
        \item Thus, the standard notation for a normalized Gaussian distribution function is
        \begin{equation*}
            p(x)\dd{x} = \frac{1}{\sqrt{2\pi\sigma_x^2}}\e[-x^2/2\sigma_x^2]\dd{x}
        \end{equation*}
        \item Note that as $\sigma_x$ gets smaller, the bell curves become narrower and taller, and vice versa as it gets larger.
        \item A more general form (one that accounts for a center at $x=\prb{x}$ as opposed to just $x=0$) is
        \begin{equation*}
            p(x)\dd{x} = \frac{1}{\sqrt{2\pi\sigma_x^2}}\e[-(x-\prb{x})^2/2\sigma_x^2]\dd{x}
        \end{equation*}
    \end{itemize}
\end{itemize}




\end{document}