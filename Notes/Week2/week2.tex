\documentclass[../notes.tex]{subfiles}

\pagestyle{main}
\renewcommand{\chaptermark}[1]{\markboth{\chaptername\ \thechapter\ (#1)}{}}
\setcounter{chapter}{1}

\begin{document}




\chapter{The Schr\"{o}dinger Equation}
\section{Particle-Wave Duality and Uncertainty Relations}
\begin{itemize}
    \item \marginnote{10/4:}Particle-wave duality (de Brogelie's original formulation):
    \begin{align*}
        \lambda\nu &= c&
        E &= h\nu&
        p &= \frac{h}{\lambda}
    \end{align*}
    \item \textbf{Angular frequency}: The quantity $\omega=2\pi r$.
    \item \textbf{Wavenumber}: The quantity $k=2\pi/\lambda$.
    \item We can create a symmetrical formulation of the de Broglie relation using these new quantities:
    \begin{align*}
        E &= \hbar\omega&
        p &= \hbar k
    \end{align*}
    \item What is the wave that we might associate with a de Broglie particle?
    \begin{equation*}
        \Psi(x) = A\e[ikx-i\omega t]
    \end{equation*}
    \item Probability:
    \begin{itemize}
        \item Classically, such a wave might be associated with EM radiation hitting a surface with intensity $I=|\Psi(x)|^2=\Psi(x)\Psi^*(x)$.
        \item As soon as we associate a particle (photon) with the wave, the intensity may be re-interpreted as the number of particles reaching the surface or the probability of a particle being at the surface.
        \item Thus, the probability of finding a particle at the surface becomes $|\Psi(x)|^2$, as well.
    \end{itemize}
    \item Following de Broglie, we also associate waves with particles such as electrons.
    \begin{itemize}
        \item With the association of light as a particle, the particle wave duality leads to the appearance of probability.
    \end{itemize}
    \item What is the probability of finding the particle at the origin?
    \begin{align*}
        Pr &= \left| A\e[ik\cdot 0] \right|^2\\
        &= |A|^2\\
    \end{align*}
    \begin{itemize}
        \item Since the probability is not dependent on position, it is the same everywhere.
        \item We also run into issues \textbf{normalizing} this unbounded wavefunction.
        \item We know this particle's momentum exactly, but we know nothing about its position.
    \end{itemize}
    \item \textbf{Normalizing} (a wavefunction): Guaranteeing that the integral for the entire wavefunction is equal to 1.
    \item \textbf{Free particle}: A particle that does not have constraints on where it is more likely to be.
    \item Heisenberg's uncertainty relations are formalized in terms of matrix mechanics.
    \begin{itemize}
        \item We can Fourier transform the wave function of particle to convert it from a function of position to a function of momentum.
        \item The Fourier transform will yield one spike at $\hbar k$ and will be 0 everywhere else --- just like the Dirac delta function.
        \item Thus,
        \begin{equation*}
            \Psi(p) = \delta(p-\hbar k)
        \end{equation*}
    \end{itemize}
    \item Consider a Gaussian wave packet at $p=0$. Then
    \begin{equation*}
        \phi(p) = C\e[-\frac{p^2}{2(\Delta p)^2}]
    \end{equation*}
    \begin{itemize}
        \item $\Delta p$ is the standard deviation of the Gaussian/width of the distribution. It is a constant such that the probability drops to $1/\e$ of its maximum at $p=0$.
    \end{itemize}
    \item With the Fourier Transform of $\Psi(p)$, we obtain
    \begin{equation*}
        \Psi(x) = D\e[-\frac{(\Delta p)^2x^2}{2\hbar^2}]
    \end{equation*}
    \begin{itemize}
        \item Thus, a Gaussian quantum function produces a Gaussian position function via an FT as well, i.e.,
        \begin{equation*}
            \Psi(x) = D\e[-\frac{x^2}{2(\Delta x)^2}]
        \end{equation*}
    \end{itemize}
    \item Now if we set the last two equations equal to each other, we get
    \begin{align*}
        \frac{(\Delta p)^2}{2\hbar^2} &= \frac{1}{2(\Delta x)^2}\\
        (\Delta p)^2(\Delta x)^2 &= \hbar^2\\
        \Delta p\Delta x &= \hbar = \frac{h}{2\pi}
    \end{align*}
    \begin{itemize}
        \item This implies that the spread of the Gaussian in momentum times the spread of the Gaussian in position is a constant.
        \item If we make one Gaussian wave packet more specific, the other gets more spread out, and vice versa.
        \item Note that the above equality does \emph{satisfy} the Heisenberg uncertainty principle, but it is not it itself.
    \end{itemize}
\end{itemize}



\section{The Schr\"{o}dinger Equation and Particle in a Box}
\begin{itemize}
    \item \marginnote{10/6:}Review:
    \begin{itemize}
        \item de Broglie describes an electron as a free particle.
        \begin{equation*}
            \Psi(x) = A\e[ikx]
        \end{equation*}
        \item We can only observe the real part, but being able to access the complex part is important in quantum mechanics.
    \end{itemize}
    \item Schr\"{o}dinger was on vacation in the Swiss Alps with his mistress when he derived the wave equation.
    \begin{itemize}
        \item Schr\"{o}dinger realized that
        \begin{align*}
            \dv{\Psi(x)}{x} &= Aik\e[ikx]\\
            -i\hbar\dv{\Psi(x)}{x} &= Ap\e[ikx]\\
            &= p\Psi(x)
        \end{align*}
        \item Let's introduce operators in quantum mechanics and let $\hat{p}$ be an operator that when it acts on $\Psi(x)$, it gives the above. In other words,
        \begin{equation*}
            \hat{p} = -i\hbar\dv{x}
        \end{equation*}
        \item Thus,
        \begin{equation*}
            \hat{p}\Psi(x) = p\Psi(x)
        \end{equation*}
        \item But energy is more important than momentum, so let's introduce an energy operator $\hat{T}$ related to $\hat{p}$ by
        \begin{equation*}
            \hat{T} = \frac{\hat{p}^2}{2m}
            = \frac{-\hbar^2}{2m}\dv[2]{x}
        \end{equation*}
        since $E=mv^2/2=p^2/(2m)$.
        \item Thus, we have
        \begin{equation*}
            \hat{T}\Psi(x) = \frac{p^2}{2m}\Psi(x)
        \end{equation*}
        \item It follows from classical physics that the total energy operator $\hat{H}$ (the Hamiltonian) is the sum of the kinetic and potential energy operators, i.e., $\hat{H}=\hat{T}+\hat{V}$. Therefore, we must have
        \begin{equation*}
            \hat{H}\Psi(x) = E\Psi(x)
        \end{equation*}
        and that is the Schr\"{o}dinger equation.
    \end{itemize}
    \item The particle in a box is like a single electron in a one-dimensional chamber that runs from $-a$ to $a$ with $L=2a$ (Schr\"{o}dinger figured this out a few days later, still in the Swiss Alps).
    \begin{itemize}
        \item The walls are infinite and have infinite potential.
        \item We need the boundary condition, though, to be able to solve a differential equation like the Schr\"{o}dinger equation.
        \begin{itemize}
            \item Fortunately, we know that at $|x|=a$, we have $\Psi(\pm a)=0$.
            \item Another important point is that $\dv*{\Psi(x)}{x}$ at $a$ is discontinuous.
        \end{itemize}
        \item So we have that
        \begin{align*}
            -\frac{\hbar^2}{2m}\dv[2]{x}\Psi_n(x) &= E_n\Psi_n(x)\\
            \dv[2]{x}\Psi(x) &= -k^2\Psi(x)
        \end{align*}
        where
        \begin{equation*}
            k = \sqrt{\frac{2mE}{\hbar^2}}
        \end{equation*}
        \item The solution of the differential equation will be of the form
        \begin{equation*}
            \Psi(x) = A\cos(kx)+B\sin(kx)
        \end{equation*}
        \item Boundary conditions 1 and 2, respectively:
        \begin{align*}
            0 &= \Psi(a)&
                0 &= \Psi(-a)\\
            &= A\cos(ka)+B\sin(ka)&
                &= A\cos(ka)-B\sin(ka)
        \end{align*}
        \item Adding/subtracting the two equations yields
        \begin{align*}
            A\cos(ka) &= 0&
            B\sin(ka) &= 0
        \end{align*}
        \item We satisfy these equations with either of 2 classes of nontrivial solutions (the trivial solution being $a=0$).
        \begin{enumerate}
            \item $B=0$ and $\cos(ka)=0$, i.e., $k_n=\frac{n\pi}{2a}$ for $n\in 2\N+1$.
            \item $A=0$ and $\sin(ka)=0$, i.e., $k_n=\frac{n\pi}{2a}$ for $n\in 2\N$.
        \end{enumerate}
        \item Thus, either
        \begin{equation*}
            \Psi_n(x) = \frac{1}{\sqrt{a}}\cos\left( \frac{n\pi x}{2a} \right)
        \end{equation*}
        for $n\in 2\N+1$ are the \textbf{even solutions} (because cosine is an even function), and
        \begin{equation*}
            \Psi_n(x) = \frac{1}{\sqrt{a}}\sin\left( \frac{n\pi x}{2a} \right)
        \end{equation*}
        for $n\in 2\N$ are the \textbf{odd solutions} (because sine is an odd function).
        \item Note that we derive the $1/\sqrt{a}$ coefficient by normalizing $\Psi(x)$ with
        \begin{equation*}
            \int_{-a}^a|\Psi(x)|^2\dd{x} = \int_{-a}^a\Psi^*(x)\Psi(x) = 1
        \end{equation*}
        \item The energies come out to
        \begin{equation*}
            E_n = \frac{\hbar^2k_n^2}{2m} = \frac{\hbar^2}{8m}\cdot\frac{\pi^2n^2}{a^2}
        \end{equation*}
        with the substitution $k_n=\frac{n\pi}{2a}$.
        \begin{itemize}
            \item Note that this means that the particle becomes more discrete the smaller the box gets (as uncertainty in position goes down, it acts more and more quantum mechanically).
        \end{itemize}
    \end{itemize}
\end{itemize}




\end{document}