\documentclass[../notes.tex]{subfiles}

\pagestyle{main}
\renewcommand{\chaptermark}[1]{\markboth{\chaptername\ \thechapter\ (#1)}{}}
\setcounter{chapter}{1}

\begin{document}




\chapter{The Schr\"{o}dinger Equation}
\section{Particle-Wave Duality and Uncertainty Relations}
\begin{itemize}
    \item \marginnote{10/4:}Particle-wave duality (de Brogelie's original formulation):
    \begin{align*}
        \lambda\nu &= c&
        E &= h\nu&
        p &= \frac{h}{\lambda}
    \end{align*}
    \item \textbf{Angular frequency}: The quantity $\omega=2\pi r$.
    \item \textbf{Wavenumber}: The quantity $k=2\pi/\lambda$.
    \item We can create a symmetrical formulation of the de Broglie relation using these new quantities:
    \begin{align*}
        E &= \hbar\omega&
        p &= \hbar k
    \end{align*}
    \item What is the wave that we might associate with a de Broglie particle?
    \begin{equation*}
        \Psi(x) = A\e[ikx-i\omega t]
    \end{equation*}
    \item Probability:
    \begin{itemize}
        \item Classically, such a wave might be associated with EM radiation hitting a surface with intensity $I=|\Psi(x)|^2=\Psi(x)\Psi^*(x)$.
        \item As soon as we associate a particle (photon) with the wave, the intensity may be re-interpreted as the number of particles reaching the surface or the probability of a particle being at the surface.
        \item Thus, the probability of finding a particle at the surface becomes $|\Psi(x)|^2$, as well.
    \end{itemize}
    \item Following de Broglie, we also associate waves with particles such as electrons.
    \begin{itemize}
        \item With the association of light as a particle, the particle wave duality leads to the appearance of probability.
    \end{itemize}
    \item What is the probability of finding the particle at the origin?
    \begin{align*}
        Pr &= \left| A\e[ik\cdot 0] \right|^2\\
        &= |A|^2\\
    \end{align*}
    \begin{itemize}
        \item Since the probability is not dependent on position, it is the same everywhere.
        \item We also run into issues \textbf{normalizing} this unbounded wavefunction.
        \item We know this particle's momentum exactly, but we know nothing about its position.
    \end{itemize}
    \item \textbf{Normalizing} (a wavefunction): Guaranteeing that the integral for the entire wavefunction is equal to 1.
    \item \textbf{Free particle}: A particle that does not have constraints on where it is more likely to be.
    \item Heisenberg's uncertainty relations are formalized in terms of matrix mechanics.
    \begin{itemize}
        \item We can Fourier transform the wave function of particle to convert it from a function of position to a function of momentum.
        \item The Fourier transform will yield one spike at $\hbar k$ and will be 0 everywhere else --- just like the Dirac delta function.
        \item Thus,
        \begin{equation*}
            \Psi(p) = \delta(p-\hbar k)
        \end{equation*}
    \end{itemize}
    \item Consider a Gaussian wave packet at $p=0$. Then
    \begin{equation*}
        \phi(p) = C\e[-\frac{p^2}{2(\Delta p)^2}]
    \end{equation*}
    \begin{itemize}
        \item $\Delta p$ is the standard deviation of the Gaussian/width of the distribution. It is a constant such that the probability drops to $1/\e$ of its maximum at $p=0$.
    \end{itemize}
    \item With the Fourier Transform of $\Psi(p)$, we obtain
    \begin{equation*}
        \Psi(x) = D\e[-\frac{(\Delta p)^2x^2}{2\hbar^2}]
    \end{equation*}
    \begin{itemize}
        \item Thus, a Gaussian quantum function produces a Gaussian position function via an FT as well, i.e.,
        \begin{equation*}
            \Psi(x) = D\e[-\frac{x^2}{2(\Delta x)^2}]
        \end{equation*}
    \end{itemize}
    \item Now if we set the last two equations equal to each other, we get
    \begin{align*}
        \frac{(\Delta p)^2}{2\hbar^2} &= \frac{1}{2(\Delta x)^2}\\
        (\Delta p)^2(\Delta x)^2 &= \hbar^2\\
        \Delta p\Delta x &= \hbar = \frac{h}{2\pi}
    \end{align*}
    \begin{itemize}
        \item This implies that the spread of the Gaussian in momentum times the spread of the Gaussian in position is a constant.
        \item If we make one Gaussian wave packet more specific, the other gets more spread out, and vice versa.
        \item Note that the above equality does \emph{satisfy} the Heisenberg uncertainty principle, but it is not it itself.
    \end{itemize}
\end{itemize}




\end{document}