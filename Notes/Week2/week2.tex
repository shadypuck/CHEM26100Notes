\documentclass[../notes.tex]{subfiles}

\pagestyle{main}
\renewcommand{\chaptermark}[1]{\markboth{\chaptername\ \thechapter\ (#1)}{}}
\setcounter{chapter}{1}

\begin{document}




\chapter{The Schr\"{o}dinger Equation}
\section{Particle-Wave Duality and Uncertainty Relations}
\begin{itemize}
    \item \marginnote{10/4:}Particle-wave duality (de Brogelie's original formulation):
    \begin{align*}
        \lambda\nu &= c&
        E &= h\nu&
        p &= \frac{h}{\lambda}
    \end{align*}
    \item \textbf{Angular frequency}: The quantity $\omega=2\pi r$.
    \item \textbf{Wavenumber}: The quantity $k=2\pi/\lambda$.
    \item We can create a symmetrical formulation of the de Broglie relation using these new quantities:
    \begin{align*}
        E &= \hbar\omega&
        p &= \hbar k
    \end{align*}
    \item What is the wave that we might associate with a de Broglie particle? Start with a completely general oscillator $A\e[iy]$ (are we also using complex coordinates to rigorously describe a circular Bohr orbit??), and substitute in the periodicity conditions from the classical wave equation, i.e., $kx-\omega t$ to get
    \begin{equation*}
        \psi(x) = A\e[i(kx-\omega t)]
    \end{equation*}
    \item Probability:
    \begin{itemize}
        \item By the classical wave power equation, such a wave might be associated with EM radiation hitting a surface with intensity $I=|\psi(x)|^2=\psi(x)\psi^*(x)$.
        \item As soon as we associate a particle (photon) with the wave, the intensity may be re-interpreted as the number of particles reaching the surface or the probability of a particle being at the surface.
        \item Thus, the probability of finding a particle at the surface becomes $|\psi(x)|^2$, as well.
    \end{itemize}
    \item Following de Broglie, we also associate waves with particles such as electrons.
    \begin{itemize}
        \item With the association of light as a particle, the particle wave duality leads to the appearance of probability.
    \end{itemize}
    \item What is the probability of finding the particle at the origin?
    \begin{align*}
        \Prob &= \left| A\e[ik\cdot 0] \right|^2\\
        &= |A|^2\\
    \end{align*}
    \begin{itemize}
        \item Since the probability is not dependent on position, it is the same everywhere.
        \item We also run into issues \textbf{normalizing} this unbounded wavefunction.
        \item We know this particle's momentum exactly, but we know nothing about its position.
    \end{itemize}
    \item \textbf{Normalizing} (a wavefunction): Guaranteeing that the integral for the entire wavefunction is equal to 1.
    \item \textbf{Free particle}: A particle that does not have constraints on where it is more likely to be.
    \item Heisenberg's uncertainty relations are formalized in terms of matrix mechanics.
    \begin{itemize}
        \item We can Fourier transform the wave function of particle to convert it from a function of position to a function of momentum.
        \item The Fourier transform will yield one spike at $\hbar k$ and will be 0 everywhere else --- just like the Dirac delta function.
        \item Thus,
        \begin{equation*}
            \psi(p) = \delta(p-\hbar k)
        \end{equation*}
    \end{itemize}
    \item Consider a Gaussian wave packet at $p=0$. Then
    \begin{equation*}
        \phi(p) = C\e[-\frac{p^2}{2(\Delta p)^2}]
    \end{equation*}
    \begin{itemize}
        \item $\Delta p$ is the standard deviation of the Gaussian/width of the distribution. It is a constant such that the probability drops to $1/\e$ of its maximum at $p=0$.
    \end{itemize}
    \item With the Fourier Transform of $\psi(p)$, we obtain
    \begin{equation*}
        \psi(x) = D\e[-\frac{(\Delta p)^2x^2}{2\hbar^2}]
    \end{equation*}
    \begin{itemize}
        \item Thus, a Gaussian quantum function produces a Gaussian position function via an FT as well, i.e.,
        \begin{equation*}
            \psi(x) = D\e[-\frac{x^2}{2(\Delta x)^2}]
        \end{equation*}
    \end{itemize}
    \item Now if we set the last two equations equal to each other, we get
    \begin{align*}
        \frac{(\Delta p)^2}{2\hbar^2} &= \frac{1}{2(\Delta x)^2}\\
        (\Delta p)^2(\Delta x)^2 &= \hbar^2\\
        \Delta p\Delta x &= \hbar = \frac{h}{2\pi}
    \end{align*}
    \begin{itemize}
        \item This implies that the spread of the Gaussian in momentum times the spread of the Gaussian in position is a constant.
        \item If we make one Gaussian wave packet more specific, the other gets more spread out, and vice versa.
        \item Note that the above equality does \emph{satisfy} the Heisenberg uncertainty principle, but it is not it itself.
    \end{itemize}
\end{itemize}



\section{The Schr\"{o}dinger Equation and Particle in a Box}
\begin{itemize}
    \item \marginnote{10/6:}Review:
    \begin{itemize}
        \item de Broglie describes an electron as a free particle.
        \begin{equation*}
            \psi(x) = A\e[ikx]
        \end{equation*}
        \item We can only observe the real part, but being able to access the complex part is important in quantum mechanics.
    \end{itemize}
    \item Schr\"{o}dinger was on vacation in the Swiss Alps with his mistress when he derived the wave equation.
    \begin{itemize}
        \item Schr\"{o}dinger realized that
        \begin{align*}
            \dv{\psi(x)}{x} &= Aik\e[ikx]\\
            -i\hbar\dv{\psi(x)}{x} &= Ap\e[ikx]\\
            &= p\psi(x)
        \end{align*}
        \item Let's introduce operators in quantum mechanics and let $\hat{p}$ be an operator that when it acts on $\psi(x)$, it gives the above. In other words,
        \begin{equation*}
            \hat{p} = -i\hbar\dv{x}
        \end{equation*}
        \item Thus,
        \begin{equation*}
            \hat{p}\psi(x) = p\psi(x)
        \end{equation*}
        \item But energy is more important than momentum, so let's introduce an energy operator $\hat{T}$ related to $\hat{p}$ by
        \begin{equation*}
            \hat{T} = \frac{\hat{p}^2}{2m}
            = \frac{-\hbar^2}{2m}\dv[2]{x}
        \end{equation*}
        since $E=mv^2/2=p^2/(2m)$.
        \item Thus, we have
        \begin{equation*}
            \hat{T}\psi(x) = \frac{p^2}{2m}\psi(x)
        \end{equation*}
        \item It follows from classical physics that the total energy operator $\hat{H}$ (the Hamiltonian) is the sum of the kinetic and potential energy operators, i.e., $\hat{H}=\hat{T}+\hat{V}$. Therefore, we must have
        \begin{equation*}
            \hat{H}\psi(x) = E\psi(x)
        \end{equation*}
        and that is the Schr\"{o}dinger equation.
    \end{itemize}
    \item The particle in a box is like a single electron in a one-dimensional chamber that runs from $-a$ to $a$ with $L=2a$ (Schr\"{o}dinger figured this out a few days later, still in the Swiss Alps).
    \begin{itemize}
        \item The walls are infinite and have infinite potential.
        \item We need the boundary condition, though, to be able to solve a differential equation like the Schr\"{o}dinger equation.
        \begin{itemize}
            \item Fortunately, we know that at $|x|=a$, we have $\psi(\pm a)=0$.
            \item Another important point is that $\dv*{\psi(x)}{x}$ at $a$ is discontinuous.
        \end{itemize}
        \item So we have that
        \begin{align*}
            -\frac{\hbar^2}{2m}\dv[2]{x}\psi_n(x) &= E_n\psi_n(x)\\
            \dv[2]{x}\psi(x) &= -k^2\psi(x)
        \end{align*}
        where
        \begin{equation*}
            k = \sqrt{\frac{2mE}{\hbar^2}}
        \end{equation*}
        \item The solution of the differential equation will be of the form
        \begin{equation*}
            \psi(x) = A\cos(kx)+B\sin(kx)
        \end{equation*}
        \item Boundary conditions 1 and 2, respectively:
        \begin{align*}
            0 &= \psi(a)&
                0 &= \psi(-a)\\
            &= A\cos(ka)+B\sin(ka)&
                &= A\cos(ka)-B\sin(ka)
        \end{align*}
        \item Adding/subtracting the two equations yields
        \begin{align*}
            A\cos(ka) &= 0&
            B\sin(ka) &= 0
        \end{align*}
        \item We satisfy these equations with either of 2 classes of nontrivial solutions (the trivial solution being $a=0$).
        \begin{enumerate}
            \item $B=0$ and $\cos(ka)=0$, i.e., $k_n=\frac{n\pi}{2a}$ for $n\in 2\N+1$.
            \item $A=0$ and $\sin(ka)=0$, i.e., $k_n=\frac{n\pi}{2a}$ for $n\in 2\N$.
        \end{enumerate}
        \item Thus, either
        \begin{equation*}
            \psi_n(x) = \frac{1}{\sqrt{a}}\cos\left( \frac{n\pi x}{2a} \right)
        \end{equation*}
        for $n\in 2\N+1$ are the \textbf{even solutions} (because cosine is an even function), and
        \begin{equation*}
            \psi_n(x) = \frac{1}{\sqrt{a}}\sin\left( \frac{n\pi x}{2a} \right)
        \end{equation*}
        for $n\in 2\N$ are the \textbf{odd solutions} (because sine is an odd function).
        \item Note that we derive the $1/\sqrt{a}$ coefficient by normalizing $\psi(x)$ with
        \begin{equation*}
            \int_{-a}^a|\psi(x)|^2\dd{x} = \int_{-a}^a\psi^*(x)\psi(x) = 1
        \end{equation*}
        \item The energies come out to
        \begin{equation*}
            E_n = \frac{\hbar^2k_n^2}{2m} = \frac{\hbar^2}{8m}\cdot\frac{\pi^2n^2}{a^2}
        \end{equation*}
        with the substitution $k_n=\frac{n\pi}{2a}$.
        \begin{itemize}
            \item Note that this means that the particle becomes more discrete the smaller the box gets (as uncertainty in position goes down, it acts more and more quantum mechanically).
        \end{itemize}
    \end{itemize}
\end{itemize}



\section{Potential Step}
\begin{itemize}
    \item \marginnote{10/8:}Particle in a box:
    \begin{itemize}
        \item For $n=1$, the potential is defined by one hump of a sine wave.
        \item For $n=2$, the potential is defined by two humps.
        \item The number of nodes is equal to the principal quantum number minus 1.
        \item We have
        \begin{equation*}
            E_n = \frac{\hbar^2}{8m}\frac{\pi^2n^2}{a^2}
        \end{equation*}
        for $n\in\N$.
        \item By the Heisenberg uncertainty relationship, we must have $E_1>0$. In other words, the \textbf{zero-point energy} arises from the UR.
        \item Trend wrt. $a$: As $a\to\infty$, all of the energies become degenerate.
        \item Trend wrt. $m$: As $m\to\infty$, $E_n\to 0$ as well.
        \begin{itemize}
            \item In other words, as $m\to\infty$, the particle behaves more classically!
            \item The zero-point energy also disappears as $a\to\infty$.
        \end{itemize}
    \end{itemize}
    \item \textbf{Zero-point energy}: The lowest possible energy a quantum mechanical system may have.
    \item All of that information comes from the Schr\"{o}dinger equation, so we now know much more than we used to.
    \item Free particle vs. particle in a box:
    \begin{itemize}
        \item For a free particle, we have $\psi(x)=\e[ikx]$. Boundary condition was a circle (as per the Bohr model).
        \item In the particle in a box, we weed out all of the free particle solutions that don't match the boundary conditions. And the only solutions that match the boundary conditions are the ones that have integers for the quantum number $n$.
        \item The only constraint is that you can retain more particles the bigger the box gets; this is why the particle gets more quantum mechanical as you shrink the box.
    \end{itemize}
    \item \textbf{Potential step}: Let the energy $E$ be 0 up until the origin, where it steps up to potential $V_0$.
    \begin{figure}[h!]
        \centering
        \begin{tikzpicture}[
            every node/.append style={black}
        ]
            \footnotesize
            \draw (-3,-0.5) -- node[pos=0.25,left]{$0$} node[left]{\small$E$} ++(0,2);
            \draw [grx,thick] (-2.8,0) -- node[above=1.1cm]{\small I} (0,0) node[below]{$0$} node[below=5mm]{\small$x$} -- node[right]{$V_0$} ++(0,1) -- node[above=1mm]{\small II} ++(2.8,0);
    
            \node [circle,fill,orx,inner sep=1.5pt,label={below:$E_0$},label={[xshift=1mm]above left:$\e[-]$}] at (-2,0.7) {}
                edge [->,dashed] ++(-0.5,0.7)
            ;
        \end{tikzpicture}
        \caption{Potential step.}
        \label{fig:potentialStep}
    \end{figure}
    \begin{itemize}
        \item We shoot a particle at a potential wall with energies varying from below the top to above the top.
    \end{itemize}
    \item In classical mechanics, we have
    \begin{equation*}
        E = \frac{p^2}{2m}+V
    \end{equation*}
    \begin{itemize}
        \item In region I, there's no potential, so the total energy is all kinetic. The particle is moving with momentum $p_\text{I}=\sqrt{2mE}$.
        \item In region II, the particle is moving with momentum $p_\text{II}=\sqrt{2m(E-V)}$.
        \begin{itemize}
            \item If $E_0<V$, the particle \emph{never} passes from region $\text{I}\to\text{II}$.
            \item If $E_0>V$, the particle \emph{always} passes from $\text{I}\to\text{II}$, but has less KE in II than I\footnote{Note that the classical resolution to the case $E=V_0$ is that the particle never has $E=V_0$; it always has energy $\epsilon$ above or $\epsilon$ below $V$. However, in some sense, there \emph{is} another answer: Classical mechanics is not an "accurate" reflection of reality, and this is a place where it shows. Indeed, we \emph{need} quantum mechanics to treat this case.}.
        \end{itemize}
    \end{itemize}
    \item Quantum particle motion:
    \begin{align*}
        -\frac{\hbar^2}{2m}\dv[2]{x}\psi(x)+V(x)\psi(x) &= E\psi(x)\\
        \dv[2]{x}\psi(x)+k^2\psi(x) &= 0
    \end{align*}
    where $k=\sqrt{2m(E-V)/\hbar^2}$.
    \begin{itemize}
        \item The total wave function will be the sum of the LCAOs that fit the boundary condition.
        \item Our general solution has two parts:
        \begin{align*}
            \psi_\text{I}(x)  &= A\e[i\alpha x]+B\e[-i\alpha x]&
            \psi_\text{II}(x) &= C\e[i\beta x]+D\e[-i\beta x]
        \end{align*}
        \item Two energy cases: $E>V_0$ and $E<V_0$.
        \item $E>V_0$:
        \begin{itemize}
            \item We must maintain the continuity of the $\psi(x)$ and $\dv*{\psi(x)}{x}$ at $x=0$. This yields
            \begin{align*}
                A+B &= C+D&
                i\alpha(A-B) &= i\beta(C-D)
            \end{align*}
            \item It follows that
            \begin{align*}
                A &= \frac{C(\alpha+\beta)}{2\alpha}+\frac{D(\alpha-\beta)}{2\alpha}&
                B &= \frac{C(\alpha-\beta)}{2\alpha}+\frac{D(\alpha+\beta)}{2\alpha}
            \end{align*}
            \item Assume that the particles only travel from left to right in II, i.e., $D=0$.
            \item The flux of the particle: The probability of the particle going left to right in region I is $|A|^2$. Thus, since the incident flux factors in the speed $v_\text{I}$ of the particle, the incident flux is $v_\text{I}|A|^2$. Similarly, the transmitted flux of the particle is $v_\text{II}|C|^2$.
            \item Consequently, the reflected flux of the particles is
            \begin{equation*}
                R = \frac{c|B|^2}{c|A|^2} = \frac{|B|^2}{|A|^2} = \frac{(\alpha-\beta)^2}{(\alpha+\beta)^2}
            \end{equation*}
            Note that the speed of the particle (the speed of light, $c$) is the same in both regions.
            \item Conclusion: There is a probability of reflection \emph{even when} $E_0>V_0$, disagreeing with classical mechanics.
            \item Fraction of transmitted particles:
            \begin{equation*}
                T = \frac{v_\text{II}}{v_\text{I}}\frac{|C|^2}{|A|^2} = \frac{4\alpha\beta}{(\alpha+\beta)^2}
            \end{equation*}
        \end{itemize}
        \item $E<V_0$:
        \begin{itemize}
            \item The continuity of $\psi(x)$ and $\psi'(x)$ at $x=0$ again gives us
            \begin{align*}
                A+B &= C+D&
                i\alpha(A-B) &= i\beta(C-D)
            \end{align*}
            \item But since we have
            \begin{equation*}
                \beta = \frac{\sqrt{2m(E-V_0)}}{\hbar}
            \end{equation*}
            and $E-V_0<0$, $\beta$ will be an imaginary number.
            \item Thus, we define $\beta_2$ to be the real number such that $\beta=i\beta_2$.
            \item Consequently, we may write
            \begin{equation*}
                R = \frac{|B|^2}{|A|^2} = \frac{|\alpha-\beta|^2}{|\alpha+\beta|^2} = \frac{|\alpha-i\beta_2|^2}{|\alpha+i\beta_2|^2} = \frac{\alpha^2+\beta_2^2}{\alpha^2+\beta_2^2} = 1
            \end{equation*}
            \item Conclusion: When the energy of the particle is less than the energy of the potential, even quantum mechanics predicts total reflection. However, there's still something subtle happening.
            \item Let's look at the wave function in region II:
            \begin{equation*}
                \psi_\text{II} = C\e[i\beta x] = C\e[i(i\beta_2)x] = C\e[-\beta_2x]
            \end{equation*}
            where $\beta_2>0$ by definition.
            \begin{figure}[H]
                \centering
                \begin{tikzpicture}[
                    every node/.append style={black}
                ]
                    \footnotesize
                    \draw (-3,-0.5) -- node[pos=0.25,left]{$0$} node[left]{\small$E$} ++(0,2);
                    \draw [grx,thick] (-2.8,0) -- (0,0) node[below]{$0$} node[below=5mm]{\small$x$} -- ++(0,1) -- ++(2.8,0);
            
                    \node [circle,fill,orx,inner sep=1.5pt,label={below:$E_0$},label={[xshift=1mm]above left:$\e[-]$}] at (-2,0.7) {}
                        edge [->,dashed] ++(-0.5,0.7)
                    ;
            
                    \draw [blx,semithick,densely dashed] plot[domain=0:2.2,smooth] (\x,{0.7*e^(-3*\x)});
                    \node (a) at (1.4,0.6) {$\e[-\beta_2x]$}
                        (a.south west) edge [->] (0.7,0.2)
                    ;
                \end{tikzpicture}
                \caption{Quantum tunneling.}
                \label{fig:quantumTunneling}
            \end{figure}
            \item Thus, even though the particle is reflected 100\%, it has some probability of going through the step, namely a probability that decays exponentially the farther you go into the wall.
            \item This is \textbf{quantum tunneling}.
            \item The particle can't ever get to $\infty$, so that's why $T=0$, but it can go into the wall for a little bit, just a sec.
        \end{itemize}
    \end{itemize}
\end{itemize}



\section{MathChapter B: Probability and Statistics}
\emph{From \textcite{bib:McQuarrieSimon}.}
\begin{itemize}
    \item \marginnote{10/10:}"Consider some experiment, such as the tossing of a coin or the rolling of a die, that has $n$ possible outcomes, each with probability $p_j$, where $j=1,2,\dots,n$" \parencite[63]{bib:McQuarrieSimon}.
    \item If the experiment is repeated indefinitely, we intuitively expect that for each $j=1,\dots,n$
    \begin{equation*}
        p_j = \lim_{N\to\infty}\frac{N_j}{N}
    \end{equation*}
    where $N_j$ is the number of times that the event $j$ occurs and $N$ is the total number of repetitions of the experiment.
    \item The fact that $0\leq N_j\leq N$ implies that $0\leq p_j\leq 1$ by the above condition.
    \item \textbf{Certainty}: An event $j$ such that $p_j=1$.
    \item \textbf{Impossibility}: An event $j$ such that $p_j=0$.
    \item \textbf{Normalization condition}: The result that
    \begin{equation*}
        \sum_{j=1}^np_j = 1
    \end{equation*}
    \begin{itemize}
        \item This follows from the fact that $\sum_{j=1}^nN_j=N$ and the above.
        \item The normalization condition expresses the idea that "the probability that some event occurs is a certainty" \parencite[64]{bib:McQuarrieSimon}.
    \end{itemize}
    \item \textbf{Average} (of $x$): The following quantity, where we associate some number $x_j$ with each outcome $j$. \emph{Also known as} \textbf{mean} (of $x$). \emph{Denoted by} $\bm{\prb{x}}$. \emph{Given by}
    \begin{equation*}
        \prb{x} = \sum_{j=1}^nx_jp_j = \sum_{j=1}^nx_jp(x_j)
    \end{equation*}
    \begin{figure}[h!]
        \centering
        \begin{tikzpicture}[
            every node/.append style={black}
        ]
            \footnotesize
            \draw (-3,0) -- (3,0);
            \draw [-stealth] (0,0) -- (0,2.5) node[above]{$p(x)$};
    
            \draw [rex,thick] (-2.5,0) node[below]{$x_1$} -- ++(0,1.5);
            \draw [rex,thick] (-1.6,0) node[below]{$x_2$} -- ++(0,1.2);
            \draw [rex,thick] (-1,0)   node[below]{$x_3$} -- ++(0,1.7);
            \draw [rex,thick] (1.5,0)  node[below]{$x_4$} -- ++(0,1.6);
        \end{tikzpicture}
        \caption{The discrete probability frequency function.}
        \label{fig:discreteProbabilityDensity}
    \end{figure}
    \begin{itemize}
        \item It is helpful to interpret a probability distribution like $p_j$ as a distribution of a unit mass along the $x$-axis in a discrete manner such that $p_j$ is the fraction of mass located at the point $x_j$.
        \item According to this interpretation, the average value of $x$ is the center of mass of this system.
    \end{itemize}
    \pagebreak
    \item \textbf{Second moment} (of the distribution $\{p_j\}$): The following quantity.
    \begin{equation*}
        \prb{x^2} = \sum_{j=1}^nx_j^2p_j
    \end{equation*}
    \begin{itemize}
        \item Note that $\prb{x^2}\neq\prb{x}^2$.
        \item Analogous to the moment of inertia.
    \end{itemize}
    \item The next quantity is physically more interesting than the second moment.
    \item \textbf{Second central moment}: The following quantity. \emph{Also known as} \textbf{variance}. \emph{Denoted by} $\bm{\sigma_x^2}$. \emph{Given by}
    \begin{equation*}
        \sigma_x^2 = \prb{(x-\prb{x})^2} = \sum_{j=1}^n(x_j-\prb{x})^2p_j
    \end{equation*}
    \begin{itemize}
        \item $\sigma_x^2\geq 0$ because it is a sum of positive terms.
        \item An alternate form of $\sigma_x^2$:
        \begin{align*}
            \sigma_x^2 &= \sum_{j=1}^n(x_j-\prb{x})^2p_j\\
            &= \sum_{j=1}^n(x_j^2-2\prb{x}x_j+\prb{x}^2)p_j\\
            &= \sum_{j=1}^nx_j^2p_j-2\prb{x}\sum_{j=1}^nx_jp_j+\prb{x}^2\sum_{j=1}^np_j\\
            &= \prb{x^2}-2\prb{x}\cdot\prb{x}+\prb{x}^2\cdot 1\\
            &= \prb{x^2}-\prb{x}^2
        \end{align*}
        \item If $\sigma_x^2=0$ or $\prb{x}^2=\prb{x^2}$, then we must have $x_j=\prb{x}$ for all $j$, i.e., the event is not really probabilistic because the event $j$ occurs on every trial.
    \end{itemize}
    \item \textbf{Standard deviation}: The positive square root of the variance. \emph{Denoted by} $\bm{\sigma_x}$.
    \item Both the standard deviation and the variance are measures of the spread of the distribution about its mean.
    \item We now step into continuous probability distributions.
    \item \textbf{Linear mass density}: The quantity $\rho(x)$ defined by
    \begin{equation*}
        \dd{m} = \rho(x)\dd{x}
    \end{equation*}
    where $\dd{m}$ is the fraction of the mass lying between $x$ and $x+\dd{x}$.
    \item It follows that the probability that, for example, a particle lies between positions $x$ and $x+\dd{x}$ in a box is
    \begin{equation*}
        \Prob(x,x+\dd{x}) = p(x)\dd{x}
    \end{equation*}
    \item Therefore,
    \begin{equation*}
        \Prob(a\leq x\leq b) = \int_a^bp(x)\dd{x}
    \end{equation*}
    \item Furthermore, the continuous normalization condition becomes
    \begin{equation*}
        \int_{-\infty}^\infty p(x)\dd{x} = 1
    \end{equation*}
    \item We may also analogously define
    \begin{align*}
        \prb{x} &= \int_{-\infty}^\infty xp(x)\dd{x}&
        \prb{x^2} &= \int_{-\infty}^\infty x^2p(x)\dd{x}&
        \sigma_x^2 &= \int_{-\infty}^\infty(x-\prb{x})^2p(x)\dd{x}
    \end{align*}
    \item \textbf{Gaussian distribution}: The most commonly occuring and the most important continuous probability distribution. \emph{Given by}
    \begin{equation*}
        p(x)\dd{x} = c\e[-x^2/2a^2]\dd{x}
    \end{equation*}
    \begin{itemize}
        \item Note that the normalization condition implies that
        \begin{equation*}
            c = \frac{1}{\sqrt{2\pi a^2}}
        \end{equation*}
        \item We can also prove that
        \begin{equation*}
            \sigma_x = a
        \end{equation*}
        \item Thus, the standard notation for a normalized Gaussian distribution function is
        \begin{equation*}
            p(x)\dd{x} = \frac{1}{\sqrt{2\pi\sigma_x^2}}\e[-x^2/2\sigma_x^2]\dd{x}
        \end{equation*}
        \item Note that as $\sigma_x$ gets smaller, the bell curves become narrower and taller, and vice versa as it gets larger.
        \item A more general form (one that accounts for a center at $x=\prb{x}$ as opposed to just $x=0$) is
        \begin{equation*}
            p(x)\dd{x} = \frac{1}{\sqrt{2\pi\sigma_x^2}}\e[-(x-\prb{x})^2/2\sigma_x^2]\dd{x}
        \end{equation*}
    \end{itemize}
\end{itemize}



\section{Chapter 3: The Schr\"{o}dinger Equation and a Particle In a Box}
\emph{From \textcite{bib:McQuarrieSimon}.}
\begin{itemize}
    \item \textbf{Schr\"{o}dinger equation}: The fundamental equation of quantum mechanics. A differential equation whose solution $\psi(x)$ describes a particle of mass $m$ moving in a potential field described by $V(x)$. \emph{Given by}
    \begin{equation*}
        -\frac{\hbar^2}{2m}\dv[2]{\psi}{x}+V(x)\psi(x) = E\psi(x)
    \end{equation*}
    \item \textbf{Wave function}: A solution to the Schr\"{o}dinger equation. A measure of the amplitude of the matter wave. \emph{Denoted by} $\bm{\psi(x)}$
    \item \textbf{Stationary-state wave function}: A solution to the time-independent Schr\"{o}dinger equation.
    \item \textbf{Particle in a box}: A system consisting of a free particle of mass $m$ that is restricted to lie along a one-dimensional interval of length $a$.
    \item \textbf{Spatial amplitude} (of a wave): The function $\psi(x)$ of position that serves as a coefficient to a time-dependent wave function.
    \item \textbf{Operator}: A symbol that tells you to do something to whatever follows the symbol. \emph{Denoted by} a capital letter with a carat over it.
    \item \textbf{Linear operator}: An operator $\hat{A}$ such that
    \begin{equation*}
        \hat{A}[c_1f_1(x)+c_2f_2(x)] = c_1\hat{A}f_1(x)+c_2\hat{A}f_2(x)
    \end{equation*}
    where $c_1,c_2$ are (possibly complex) constants.
    \item \textbf{Eigenvalue problem}: The problem of determining $\phi(x)$ and $a$ given $\hat{A}$ such that
    \begin{equation*}
        \hat{A}\phi(x) = a\phi(x)
    \end{equation*}
    \item \textbf{Eigenfunction}: The function $\phi(x)$ in an eigenvalue problem.
    \item \textbf{Eigenvalue}: The constant $a$ in an eigenvalue problem.
    \item \textbf{Hamiltonian operator}: The following operator. \emph{Denoted by} $\bm{\hat{H}}$. \emph{Given by}
    \begin{equation*}
        \hat{H} = -\frac{\hbar^2}{2m}\dv[2]{x}+V(x)
    \end{equation*}
    \item The substitution of the Hamiltonian operator into the Schr\"{o}dinger equation allows us to formulate the Schr\"{o}dinger equation as an eigenvalue problem.
    \begin{itemize}
        \item The wave function then becomes the eigenfunction and the energy, the eigenvalue of the Hamiltonian operator.
    \end{itemize}
    \item \textbf{Kinetic energy operator}: The following operator. \emph{Denoted by} $\bm{\hat{K}_x}$. \emph{Given by}
    \begin{equation*}
        \hat{K}_x = -\frac{\hbar^2}{2m}\dv[2]{x}
    \end{equation*}
    \begin{itemize}
        \item Defined by taking $V(x)=0$ in the Hamiltonian.
    \end{itemize}
    \item \textbf{Momentum operator}: The following operator. \emph{Denoted by} $\bm{\hat{P}_x}$.
    \begin{equation*}
        \hat{P}_x = -i\hbar\dv{x}
    \end{equation*}
    \begin{itemize}
        \item Defined by applying $K=p^2/2m$ to the kinetic energy operator to get
        \begin{equation*}
            \hat{P}_x^2 = -\hbar^2\dv[2]{x}
        \end{equation*}
        and then (noting that we define the square of an operator to be equivalent to applying the same operator successively) factoring the above into
        \begin{equation*}
            \hat{P}_x^2 = \hat{P}_x\hat{P}_x = -\hbar^2\dv[2]{x} = \left( -i\hbar\dv{x} \right)\left( -i\hbar\dv{x} \right)
        \end{equation*}
    \end{itemize}
    \item \textbf{Free particle}: A particle that experiences no potential energy, i.e., a particle for which $V(x)=0$.
    \item When solving the particle in a box, we say that $\psi(x)$ represents the the amplitude of the particle in some sense. More specifically, since the intensity of a wave is the square of the magnitude of the amplitude, we write that the "intensity of the particle" is proportional to $\psi^*(x)\psi(x)$.
    \begin{itemize}
        \item Born, a German physicist working in scattering theory, formalized this by saying that $\psi^*(x)\psi(x)\dd{x}$ is the "probability that the particle is located between $x$ and $x+\dd{x}$" \parencite[80]{bib:McQuarrieSimon}.
    \end{itemize}
    \item Schr\"{o}dinger's quantization of energy arises naturally from his equation and the boundary conditions, as opposed to having to be postulated as in Bohr's model.
    \item \textcite{bib:McQuarrieSimon} use the free-particle model to crudely explain the absorption spectrum of butadiene.
    \item \textbf{Normalized} (wave function): A wave function of the form
    \begin{equation*}
        \psi_n(x) = \sqrt{\frac{2}{a}}\sin\frac{n\pi x}{a}
    \end{equation*}
    \begin{itemize}
        \item "Because the Hamiltonian operator is a linear operator, if $\psi$ is a solution to $\hat{H}\psi=E\psi$, then any constant, say $A$, times $\psi$ is also a solution, and $A$ can always be chosen to produce a normalized solution to the Schr\"{o}dinger equation" \parencite[84]{bib:McQuarrieSimon}.
    \end{itemize}
    \item \textbf{Correspondence principle}: Quantum mechanical results and classical mechanical results tend to agree in the limit of large quantum numbers.
    \item Applying the statistical principles to the particle in a box, we can calculate that
    \begin{align*}
        \prb{x} &= \frac{a}{2}&
            \prb{x^2} &= \frac{a^2}{3}-\frac{a^2}{2n^2\pi^2}&
                \sigma_x^2 &= \prb{x^2}-\prb{x}^2&
                    \sigma_x &= \frac{a}{2\pi n}\sqrt{\frac{\pi^2n^2}{3}-2}\\
        &&
            &&
                &= \left( \frac{a}{2\pi n} \right)^2\left( \frac{\pi^2n^2}{3}-2 \right)
    \end{align*}
    \item Calculating the average energy or momentum:
    \begin{itemize}
        \item To calculate the average value of the physical quantity associated with an operator, we sandwich the operator between a wave function $\psi_n(x)$ and its complex conjugate $\psi_n^*(x)$.
        \item This will be formalized later, but for now, we assume that
        \begin{equation*}
            \prb{s} = \int\psi_n^*(x)\hat{S}\psi_n(x)\dd{x}
        \end{equation*}
        where $\hat{S}$ is the quantum-mechanical operator associated with the physical quantity $s$, and $\prb{s}$ is the average value of $s$ in the state described for the wave function.
        \item For example, the average momentum of a particle in a box in the state described by $\psi_n(x)$ is
        \begin{equation*}
            \prb{p} = \int_0^a\left[ \sqrt{\frac{2}{a}}\sin\frac{n\pi x}{a} \right]\left( -i\hbar\dv{x} \right)\left[ \sqrt{\frac{2}{a}}\sin\frac{n\pi x}{a} \right]\dd{x}
        \end{equation*}
    \end{itemize}
    \item Note that the average momentum of a particle in a box is zero.
    \item \marginnote{10/12:}Calculating the variance $\sigma_p^2$ of the momentum of a particle in a box.
    \begin{align*}
        \prb{p^2} &= \int\psi_n^*\hat{P}_x^2\psi_n(x)\dd{x}\\
        &= \frac{n^2\pi^2\hbar^2}{a^2}
    \end{align*}
    \begin{itemize}
        \item It follows that $\sigma_p=n\pi\hbar/a$.
    \end{itemize}
    \item \textbf{Root-mean-square momentum}: The square root of $\prb{p^2}$.
    \item "Because the variance $\sigma^2$, and hence the standard deviation $\sigma$, is a measure of the spread of a distribution about its mean value, we can interpret $\sigma$ as a measure of the uncertainty involved in any measurement" \parencite[89]{bib:McQuarrieSimon}.
    \begin{itemize}
        \item For the simple situation of a particle in a box, we can calculate \emph{exact} uncertainties in position and momentum $\sigma_x,\sigma_p$.
        \begin{itemize}
            \item We can see from these exact formulae that $\sigma_x$ is directly proportional to the length $a$ of the box, and $\sigma_p$ is inversely proportional to the length $a$ of the box.
            \item This means that as the box gets bigger, it becomes harder to know where the particle is but its momentum becomes more certain, and vice versa as the box gets smaller.
        \end{itemize}
        \item From the above, it is clear that $\sigma_x$ and $\sigma_p$ have a reciprocal relation.
        \item Indeed, taking the product $\sigma_x\sigma_p$ yields the \textbf{Heisenberg Uncertainty Principle}
        \begin{align*}
            \sigma_x\sigma_p &= \frac{\hbar}{2}\sqrt{\frac{\pi^2n^2}{3}-2}\\
            &> \frac{\hbar}{2}
        \end{align*}
    \end{itemize}
    \item \textbf{Free particle}: A particle that is allowed to range over the entire $x$-axis.
    \begin{itemize}
        \item "A free particle has a definite momentum, but its position is completely indefinite" \parencite[90]{bib:McQuarrieSimon}.
    \end{itemize}
    \item The Uncertainty Principle also says that the minimum product of the two uncertainties is on the order of Planck's constant.
    \item The particle in a three-dimensional box:
    \begin{itemize}
        \item If the box (a rectangular parallelepiped) has sides of length $a,b,c$, then the Schr\"{o}dinger equation for this case is
        \begin{equation*}
            -\frac{\hbar^2}{2m}\left( \pdv[2]{\psi}{x}+\pdv[2]{\psi}{y}+\pdv[2]{\psi}{z} \right) = E\psi(x,y,z)
        \end{equation*}
        where $0\leq x\leq a$, $0\leq y\leq b$, and $0\leq z\leq c$.
        \begin{itemize}
            \item An alternate form is
            \begin{equation*}
                -\frac{\hbar^2}{2m}\nabla^2\psi = E\psi
            \end{equation*}
        \end{itemize}
        \item Boundary conditions:
        \begin{itemize}
            \item $\psi(0,y,z)=\psi(a,y,z)=0$ for all $y,z$.
            \item $\psi(x,0,z)=\psi(x,b,z)=0$ for all $x,z$.
            \item $\psi(x,y,0)=\psi(x,y,c)=0$ for all $x,y$.
        \end{itemize}
        \item Invoke the method of separation of variables, i.e., suppose
        \begin{equation*}
            \psi(x,y,z) = X(x)Y(y)Z(z)
        \end{equation*}
        \item Then
        \begin{equation*}
            -\frac{\hbar^2}{2m}\frac{1}{X(x)}\dv[2]{X}{x}-\frac{\hbar^2}{2m}\frac{1}{Y(y)}\dv[2]{Y}{y}-\frac{\hbar^2}{2m}\frac{1}{Z(z)}\dv[2]{Z}{z} = E
        \end{equation*}
        \item It follows since each of the three terms contains only one of the variables and hence each of the terms can be varied independently that each term must equal a constant. The sum of the three constants will be $E$.
        \item But dividing the above equations into three smaller equations gives us cases entirely analogous to the one-dimensional particle in a box, meaning that
        \begin{align*}
            X(x) &= A_x\sin\frac{n_x\pi x}{a}&
            Y(y) &= A_y\sin\frac{n_y\pi y}{b}&
            Z(z) &= A_z\sin\frac{n_z\pi z}{c}
        \end{align*}
        for $n=1,2,3,\dots$.
        \item Therefore, the solution is
        \begin{equation*}
            \psi(x,y,z) = A_xA_yA_z\sin\frac{n_x\pi x}{a}\sin\frac{n_y\pi y}{b}\sin\frac{n_z\pi z}{c}
        \end{equation*}
        \item The normalization constant then turns out to be
        \begin{align*}
            1 &= \int_0^a\int_0^b\int_0^c\psi^*(x,y,z)\psi(x,y,z)\dd{x}\dd{y}\dd{z}\\
            A_xA_yA_z &= \sqrt{\frac{8}{abc}}
        \end{align*}
        \item We can now also obtain the following formula for the energies by plugging the full solution back into the original Schr\"{o}dinger equation.
        \begin{equation*}
            E_{n_xn_yn_z} = \frac{h^2}{8m}\left( \frac{n_x^2}{a^2}+\frac{n_y^2}{b^2}+\frac{n_z^2}{c^2} \right)
        \end{equation*}
        \item \textcite{bib:McQuarrieSimon} does calculations to show that $\prb{x}=(a/2,b/2,c/2)$ and $\prb{p}=0$.
        \item Note that we have the following 3D operators:
        \begin{itemize}
            \item Position operator:
            \begin{equation*}
                \hat{\mathbf{R}} = \hat{X}\mathbf{i}+\hat{Y}\mathbf{j}+\hat{Z}\mathbf{k}
            \end{equation*}
            \item Momentum operator:
            \begin{equation*}
                \hat{\mathbf{P}} = -i\hbar\left( \mathbf{i}\pdv{x}+\mathbf{j}\pdv{y}+\mathbf{k}\pdv{z} \right)
            \end{equation*}
        \end{itemize}
        \item Consider the special case where $a=b=c$. Then only one set of values $n_x,n_y,n_z$ corresponds to the lowest energy level, but \emph{three degenerate ones} correspond to the second energy level (211, 121, 112).
        \begin{itemize}
            \item Note that the degeneracy is introduced by the symmetry of the box and is lifted when the box becomes no longer symmetric.
        \end{itemize}
    \end{itemize}
    \item \textbf{Laplacian operator}: The operator $\nabla^2$.
    \item \textbf{Separable} (operator): An operator that is the sum of multiple variably independent terms.
\end{itemize}




\end{document}